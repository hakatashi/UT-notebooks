%% LyX 2.1.3 created this file.  For more info, see http://www.lyx.org/.
%% Do not edit unless you really know what you are doing.
\documentclass[english]{article}
\usepackage[T1]{fontenc}
\usepackage[utf8]{inputenc}
\usepackage{geometry}
\geometry{verbose,tmargin=2cm,bmargin=2cm,lmargin=1cm,rmargin=1cm}
\setlength{\parskip}{\smallskipamount}
\setlength{\parindent}{0pt}
\usepackage{verbatim}
\usepackage{amsmath}
\usepackage{graphicx}

\makeatletter
%%%%%%%%%%%%%%%%%%%%%%%%%%%%%% User specified LaTeX commands.
\usepackage[version=3]{mhchem}

\makeatother

\usepackage{babel}
\usepackage{listings}
\renewcommand{\lstlistingname}{Listing}

\begin{document}

\title{宇宙科学実習レポート課題: 銀河中心天体Sagittarius $A^{*}$の距離を求める}


\author{学籍番号 340728B 高橋光輝}

\maketitle

\section*{問2}

\begin{align*}
x & =r\cos\varphi\\
\dot{x} & =\dot{r}\cos\varphi-\dot{\varphi}r\sin\varphi\\
\ddot{x} & =\ddot{r}\cos\varphi-\dot{r}\dot{\varphi}\sin\varphi-\dot{r}\dot{\varphi}\sin\varphi-r\dot{\varphi}^{2}\cos\varphi-r\ddot{\varphi}\sin\varphi\\
 & =\left(\ddot{r}-r\dot{\varphi}^{2}\right)\cos\varphi-\left(2\dot{r}\dot{\varphi}+r\ddot{\varphi}\right)\sin\varphi
\end{align*}
\begin{align*}
y & =r\sin\varphi\\
\dot{y} & =\dot{r}\sin\varphi+\dot{\varphi}r\cos\varphi\\
\ddot{y} & =\ddot{r}\sin\varphi+\dot{r}\dot{\varphi}\cos\varphi+\dot{r}\dot{\varphi}\cos\varphi-r\dot{\varphi}^{2}\sin\varphi+r\ddot{\varphi}\cos\varphi\\
 & =\left(\ddot{r}-r\dot{\varphi}^{2}\right)\sin\varphi+\left(2\dot{r}\dot{\varphi}+r\ddot{\varphi}\right)\cos\varphi
\end{align*}


プリントの式(4),(5)より
\begin{align*}
\ddot{x} & =-\frac{GM}{r^{3}}x=-\frac{GM}{r^{2}}\cos\varphi\\
\ddot{y} & =-\frac{GM}{r^{3}}y=-\frac{GM}{r^{2}}\sin\varphi
\end{align*}
を代入して、
\begin{align*}
\left(\ddot{r}-r\dot{\varphi}^{2}\right)\cos\varphi-\left(2\dot{r}\dot{\varphi}+r\ddot{\varphi}\right)\sin\varphi & =-\frac{GM}{r^{2}}\cos\varphi\\
\left(\ddot{r}-r\dot{\varphi}^{2}\right)\sin\varphi+\left(2\dot{r}\dot{\varphi}+r\ddot{\varphi}\right)\cos\varphi & =-\frac{GM}{r^{2}}\sin\varphi
\end{align*}
\begin{align*}
\left(\ddot{r}-r\dot{\varphi}^{2}\right)-\left(2\dot{r}\dot{\varphi}+r\ddot{\varphi}\right)\tan\varphi & =-\frac{GM}{r^{2}}\\
\left(\ddot{r}-r\dot{\varphi}^{2}\right)+\left(2\dot{r}\dot{\varphi}+r\ddot{\varphi}\right)\tan\varphi & =-\frac{GM}{r^{2}}
\end{align*}
より
\[
\left(\ddot{r}-r\dot{\varphi}^{2}\right)-\left(2\dot{r}\dot{\varphi}+r\ddot{\varphi}\right)\tan\varphi=\left(\ddot{r}-r\dot{\varphi}^{2}\right)+\left(2\dot{r}\dot{\varphi}+r\ddot{\varphi}\right)\tan\varphi
\]
\[
2\dot{r}\dot{\varphi}+r\ddot{\varphi}=0
\]


\begin{align*}
-\frac{GM}{r^{2}} & =\left(\ddot{r}-r\dot{\varphi}^{2}\right)-\left(2\dot{r}\dot{\varphi}+r\ddot{\varphi}\right)\tan\varphi\\
 & =\ddot{r}-r\dot{\varphi}^{2}
\end{align*}


以上より、
\[
\begin{cases}
2\dot{r}\dot{\varphi}+r\ddot{\varphi}=0\\
\ddot{r}-r\dot{\varphi}^{2}=-\frac{GM}{r^{2}}
\end{cases}
\]



\section*{問4}

\[
\frac{\partial}{\partial t}\left(r^{2}\dot{\varphi}\right)=2r\dot{r}\dot{\varphi}+r^{2}\ddot{\varphi}=r\left(2\dot{r}\dot{\varphi}+r\ddot{\varphi}\right)=0
\]
より、$r^{2}\dot{\varphi}$は定数となる。よって
\[
l=r^{2}\dot{\varphi}
\]
とおくと、
\[
\frac{\partial\varphi}{\partial t}=\frac{l}{r^{2}}
\]
が得られる。さらに
\[
u=\frac{1}{r}
\]
と置くと、
\begin{align*}
r & =\frac{1}{u}\\
\frac{\partial^{2}r}{\partial u^{^{2}}} & =\frac{2}{u^{3}}
\end{align*}


ここで、
\begin{align*}
\ddot{r} & =\frac{\partial^{2}r}{\partial t^{2}}\\
 & =\frac{\partial^{2}\varphi}{\partial t^{2}}\frac{\partial^{2}u}{\partial\varphi^{2}}\frac{\partial^{2}r}{\partial u^{2}}\\
 & =l^{2}u^{4}\cdot\frac{\partial^{2}u}{\partial\varphi^{2}}\frac{2}{u^{3}}\\
 & =2l^{2}u\frac{\partial^{2}u}{\partial\varphi^{2}}
\end{align*}


これを式(8)に代入すると、
\begin{align*}
l^{2}u^{2}\frac{\partial^{2}u}{\partial\varphi^{2}}-r\dot{\varphi}^{2} & =-\frac{GM}{r^{2}}\\
l^{2}u^{2}\frac{\partial^{2}u}{\partial\varphi^{2}}-l^{2}u^{3} & =-GMu^{2}\\
\frac{\partial^{2}u}{\partial\varphi^{2}}-u & =-\frac{GM}{l^{2}}\\
\frac{\partial^{2}u}{\partial\varphi^{2}} & =u-\frac{GM}{l^{2}}
\end{align*}
となり二階微分方程式に帰着できる。

これを解いて、
\begin{align*}
u-\frac{GM}{l^{2}} & =\alpha\cos\left(\varphi+\beta\right)\\
\frac{1}{r} & =\alpha\cos\left(\varphi+\beta\right)+\frac{GM}{l^{2}}\\
r & =\frac{1}{\alpha\cos\left(\varphi+\beta\right)+\frac{GM}{l^{2}}}\\
 & =\frac{\frac{l^{2}}{GM}}{1+\frac{l^{2}}{GM}\alpha\cos\left(\varphi+\beta\right)}
\end{align*}


ここで$\frac{l^{2}}{GM}\alpha=\varepsilon,\beta=\alpha$とおくと、
\[
r=\frac{\frac{l^{2}}{GM}}{1+\varepsilon\cos\left(\varphi+\alpha\right)}
\]
となる。


\section{この表のデータ点をxy座標系から$\mathrm{SgrA^{*}}$を原点に取った極座標に直しなさい。角度の原点はどこにとってもよい。}

\begin{align*}
r & =\sqrt{x^{2}+y^{2}}\\
\varphi & =\mathrm{atan2}\left(x,y\right)
\end{align*}
より、以下の表を得た。

\includegraphics{C:/Users/hakatashi/OneDrive/Documents/Todai/astro/06/SgrA}


\section{S2の軌道を楕円軌道と仮定して、その長半径と離心率をフィッティングによって求めなさい。さらに、求めた最小二乗フィットの楕円とデータ点(誤差棒付き)をgnuplotでプロットしなさい。}

以下のpythonプログラムを使用してフィッティングを行った。

\lstinputlisting[breaklines=true,captionpos=b,frame=tb,language=Python,caption={fit.py}]{C:/Users/hakatashi/OneDrive/Documents/Todai/astro/06/fit.py}

ここでS2の軌道を
\[
f\left(x\right)=\frac{B}{1+A\cos\left(x+C\right)}
\]
としている。

これを実行したところ、以下の結果を得た。

\inputencoding{latin9}\begin{lstlisting}
> C:\Users\hakatashi\Anaconda\python fit.py
(array([-0.86188795,  0.0290979 ,  1.09635665]), 1)
\end{lstlisting}
\inputencoding{utf8}

よってS2の軌道を
\[
f\left(x\right)=\frac{2.91\times10^{-2}}{1-8.62\times10^{-1}\cdot\cos\left(x+1.10\right)}
\]
と見積もった。

このとき長半径$a$は、
\[
a=\frac{B}{1-A^{2}}=\frac{2.91\times10^{-2}}{1-\left(-8.62\times10^{-1}\right)^{2}}=1.13\times10^{-1}\mathrm{arcsec}
\]
離心率$\varepsilon$は、
\[
\varepsilon=A=-8.62\times10^{-1}
\]
となる。

これを以下のgnuplotプログラムでプロットしたところ、下の図を得た。

\verbatiminput{C:/Users/hakatashi/OneDrive/Documents/Todai/astro/06/sgrA.gnuplot}

\includegraphics{C:/Users/hakatashi/OneDrive/Documents/Todai/astro/06/S2}


\section{上のフィッティングの結果から、S2の軌道長半径の大きさを求めなさい。}

S2の軌道長半径$R$は、
\[
R=7.9\mathrm{kpc}\cdot1.13\times10^{-1}\mathrm{arcsec}\cdot\frac{2\pi}{60\cdot60\cdot360}=4.33\times10^{-6}\mathrm{kpc}=893\mathrm{AU}
\]



\section{データ点の位置から、およそ半周期だけ隔たっている二点を選び、これらの時刻の差から、公転周期を求めなさい。}

およそ長軸の端点にあるとみられる、$\mathrm{year}=1994.321,2002.334$の二点を選んだ。

これより公転周期$P$は、
\[
P=\left(2002.334-1994.321\right)\times2=16.026\mathrm{year}
\]



\section{ケプラーの第3法則を適用し、$\mathrm{SgrA^{*}}$の質量を求めなさい。}

$\mathrm{SgrA^{*}}$の質量を$M$、重力定数を$G$とすると、ケプラーの第3法則より、
\[
P=\frac{2\pi R^{\frac{3}{2}}}{\sqrt{GM}}
\]


ところで地球の軌道について、ケプラーの第3法則より
\begin{align*}
\left(1\mathrm{year}\right) & =\frac{2\pi\left(1\mathrm{AU}\right)^{\frac{3}{2}}}{\sqrt{GM_{\odot}}}\\
G & =\frac{4\pi^{2}}{M_{\odot}}
\end{align*}


これを代入すると、
\begin{align*}
P & =\frac{R^{\frac{3}{2}}}{\sqrt{\frac{M}{M_{\odot}}}}\\
\frac{M}{M_{\odot}} & =\frac{R^{3}}{P^{2}}\\
 & =\frac{893^{3}}{16.026^{2}}\\
 & =2.77\times10^{6}\\
M & =2.77\times10^{6}\cdot1.989\times10^{30}\\
 & =5.52\times10^{36}\mathrm{kg}
\end{align*}
となる。


\section{$\mathrm{SgrA^{*}}$のシュワルツシルト半径を求めて、S2の軌道長半径がその男倍くらいになるか計算しなさい。}

シュワルツシルト半径$r_{g}$は、
\begin{align*}
r_{g} & =\frac{2GM}{c^{2}}\\
 & =\frac{2\cdot6.67\times10^{-11}\cdot5.52\times10^{36}}{\left(3.00\times10^{8}\right)^{2}}\\
 & =8.18\times10^{9}\mathrm{m}
\end{align*}


S2の軌道長半径との半径比$\frac{R}{r_{g}}$は、
\begin{align*}
\frac{R}{r_{g}} & =\frac{893\cdot1.50\times10^{11}}{8.18\times10^{9}}\\
 & =1.64\times10^{4}\gg1
\end{align*}



\section{いま、仮に$\mathrm{SgrA^{*}}$が、太陽と同じ質量を持つ星の集団からなっているとします。そのKバンド絶対等級はどのくらいになりますか。}

$\mathrm{SgrA^{*}}$が、$\frac{M}{M_{\odot}}=2.77\times10^{6}$個のKバンド絶対等級$M_{K}=+3.28$の星の集団からなるものとする。予測されるその等級を$M_{\mathrm{SgrA^{*}}}$とすると、
\begin{align*}
M_{\mathrm{SgrA^{*}}} & =M_{K}-\log_{10}\frac{M}{M_{\odot}}\\
 & =+3.28-\log_{10}2.77\times10^{6}\\
 & =-3.16
\end{align*}
となり、おおむねS2の絶対等級と等しいにもかかわらず、$\mathrm{SgrA^{*}}$は観測されない。

以上より、$\mathrm{SgrA^{*}}$が巨大な質量を持ち、なおかつ外部に光を発しない天体、ブラックホールである可能性が高いことが伺える。
\end{document}
