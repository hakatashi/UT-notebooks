%% LyX 2.2.2 created this file.  For more info, see http://www.lyx.org/.
%% Do not edit unless you really know what you are doing.
\documentclass[english]{article}
\usepackage[T1]{fontenc}
\usepackage[utf8]{inputenc}
\usepackage[a5paper]{geometry}
\geometry{verbose,tmargin=2cm,bmargin=2cm,lmargin=1cm,rmargin=1cm}
\setlength{\parskip}{\smallskipamount}
\setlength{\parindent}{0pt}
\usepackage{amsmath}
\usepackage{amssymb}

\makeatletter
%%%%%%%%%%%%%%%%%%%%%%%%%%%%%% User specified LaTeX commands.
\usepackage[dvipdfmx]{hyperref}
\usepackage[dvipdfmx]{pxjahyper}

% http://tex.stackexchange.com/a/192428/116656
\AtBeginDocument{\let\origref\ref
   \renewcommand{\ref}[1]{(\origref{#1})}}

\makeatother

\usepackage{babel}
\begin{document}

\title{2017-A 電子回路II}

\author{教員: 入力: 高橋光輝}

\maketitle
\global\long\def\pd#1#2{\frac{\partial#1}{\partial#2}}
\global\long\def\d#1#2{\frac{\mathrm{d}#1}{\mathrm{d}#2}}
\global\long\def\pdd#1#2{\frac{\partial^{2}#1}{\partial#2^{2}}}
\global\long\def\dd#1#2{\frac{\mathrm{d}^{2}#1}{\mathrm{d}#2^{2}}}
\global\long\def\ddd#1#2{\frac{\mathrm{d}^{3}#1}{\mathrm{d}#2^{3}}}
\global\long\def\e{\mathrm{e}}
\global\long\def\i{\mathrm{i}}
\global\long\def\j{\mathrm{j}}
\global\long\def\grad{\operatorname{grad}}
\global\long\def\rot{\operatorname{rot}}
\global\long\def\opdiv{\operatorname{div}}
\global\long\def\diag{\operatorname{diag}}
\global\long\def\rank{\operatorname{rank}}
\global\long\def\prob{\operatorname{Prob}}
\global\long\def\cov{\operatorname{Cov}}
\global\long\def\when#1{\left.#1\right|}
\global\long\def\laplace#1{\mathcal{L}\left[#1\right]}


\section*{第4回}

\subsection*{フィードバック回路}

\begin{equation}
\begin{cases}
v_{in}-kHv_{1}=v_{1}\\
v_{out}=kv_{1}
\end{cases}\label{eq:1-2}
\end{equation}

\ref{eq:1-2}の両式から$v_{1}$を消去

\[
\begin{cases}
v_{in}=\left(1+kH\right)v_{1}\\
v_{out}=kv_{1}
\end{cases}\rightarrow v_{out}=\frac{k}{1+kH}v_{in}
\]

増幅率
\[
A=\frac{v_{out}}{v_{in}}=\frac{k}{1+kH}=\frac{1}{\frac{1}{k}+H}\simeq\frac{1}{H}\left(=H^{-1}\right)
\]


\paragraph{フィードバック回路のメリット}

その1: システムの増幅率が$k$に非依存

\[
\frac{\Delta A}{A}=\circ\frac{\Delta k}{k}+\square\frac{\Delta H}{H}
\]

というような感度解析を行いたい。

\[
\log\left(\frac{k}{1+kH}\right)=\log k-\log\left(HkH\right)\triangleq X
\]
とする。

$k$に対する感度は、
\begin{equation}
\pd Xk=\frac{1}{k}-\pd{}k\log\left(1+kH\right)=\frac{1}{k}-\frac{1}{1+kH}\cdot H=\frac{1+kH-kH}{k\left(1+kH\right)}=\frac{1}{k\left(1+kH\right)}\label{eq:2-1}
\end{equation}

$H$に対する感度は、
\begin{align}
\pd XH & =\pd{}H\log k-\pd{}H\log\left(1+kH\right)\nonumber \\
 & =0-\frac{1}{1+kH}\cdot k=-\frac{k}{1+kH}\label{2-2}
\end{align}

ところで
\begin{equation}
\pd{\log A}A=\frac{1}{A}\label{eq:2-3}
\end{equation}
である。

\begin{equation}
\pd{\log A}A\Delta A=\pd{\log A}k\Delta k+\pd{\log A}H\Delta H\label{eq:2-4}
\end{equation}

\[
\therefore\frac{\Delta A}{A}=\frac{\Delta k}{k\left(1+kH\right)}-\frac{k\Delta H}{1+kH}
\]

\[
\frac{\Delta A}{A}=\frac{1}{1+kH}\frac{\Delta k}{k}-\frac{kH}{1+kH}\frac{\Delta H}{H}
\]

より$\frac{1}{1+kH}$が$k$に対する感度、$-\frac{kH}{1+kH}$が$H$に対する感度となる。

$kH\gg1$なので、$\frac{1}{1+kH}$はほぼ$0$、$-\frac{kH}{1+kH}$はほぼ$-1$となる。

\subsubsection*{フィードバック回路の例}

\paragraph{演算増幅器 (OPAMP) を使ったフィードバック回路}

{[}図略: オペアンプ{]}

オペアンプ: 「差動」増幅 シングルエンド出力 $k\left(V_{+}-V_{-}\right)$

{[}図略: オペアンプを用いたフィードバック回路{]}

\[
\begin{cases}
V_{+}=v_{in}\\
V_{-}=\frac{R_{1}}{R_{1}+R_{2}}v_{out}\\
v_{out}=k\left(V_{+}-V_{-}\right)\left(1+kH\right)v_{out}=kv_{in}
\end{cases}
\]

\[
\therefore v_{out}=\frac{k}{1+kH}v_{in}\rightarrow\frac{R_{1}+R_{2}}{R_{1}}=1+\frac{R_{2}}{R_{1}}
\]

フィードバック回路のメリットその1: システムの増幅率が$k$に非依存

フィードバック回路のメリットその2: 出力端がノイズに強くなる

\[
kv_{1}+v_{n}=v_{out}
\]

\[
-Hv_{out}=v_{1}
\]

\[
-kHv_{out}+v_{n}=v_{out}
\]

\[
v_{n}=\left(1+kH\right)v_{out}
\]

\[
\therefore v_{out}=\frac{1}{1+kH}v_{n}
\]

c.f. ストレート回路では⋯⋯

\[
v_{out}=Av_{in}+v_{n}
\]

\[
v_{in}=0=v_{n}
\]

フィードバック回路のメリットその3: 入出力インピーダンスが理想的になる
\end{document}
