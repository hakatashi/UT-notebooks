%% LyX 2.2.2 created this file.  For more info, see http://www.lyx.org/.
%% Do not edit unless you really know what you are doing.
\documentclass[english]{article}
\usepackage[T1]{fontenc}
\usepackage[utf8]{inputenc}
\usepackage[a5paper]{geometry}
\geometry{verbose,tmargin=2cm,bmargin=2cm,lmargin=1cm,rmargin=1cm}
\setlength{\parskip}{\smallskipamount}
\setlength{\parindent}{0pt}
\usepackage{textcomp}
\usepackage{amstext}
\usepackage{amssymb}

\makeatletter

%%%%%%%%%%%%%%%%%%%%%%%%%%%%%% LyX specific LaTeX commands.
%% Because html converters don't know tabularnewline
\providecommand{\tabularnewline}{\\}

%%%%%%%%%%%%%%%%%%%%%%%%%%%%%% User specified LaTeX commands.
\usepackage[dvipdfmx]{hyperref}
\usepackage[dvipdfmx]{pxjahyper}

% http://tex.stackexchange.com/a/192428/116656
\AtBeginDocument{\let\origref\ref
   \renewcommand{\ref}[1]{(\origref{#1})}}

\makeatother

\usepackage{babel}
\begin{document}

\title{2017-A VLSI工学基礎}

\author{教員: 入力: 高橋光輝}

\maketitle
\global\long\def\pd#1#2{\frac{\partial#1}{\partial#2}}
\global\long\def\d#1#2{\frac{\mathrm{d}#1}{\mathrm{d}#2}}
\global\long\def\pdd#1#2{\frac{\partial^{2}#1}{\partial#2^{2}}}
\global\long\def\dd#1#2{\frac{\mathrm{d}^{2}#1}{\mathrm{d}#2^{2}}}
\global\long\def\ddd#1#2{\frac{\mathrm{d}^{3}#1}{\mathrm{d}#2^{3}}}
\global\long\def\e{\mathrm{e}}
\global\long\def\i{\mathrm{i}}
\global\long\def\j{\mathrm{j}}
\global\long\def\grad{\operatorname{grad}}
\global\long\def\rot{\operatorname{rot}}
\global\long\def\div{\operatorname{div}}
\global\long\def\diag{\operatorname{diag}}
\global\long\def\rank{\operatorname{rank}}
\global\long\def\prob{\operatorname{Prob}}
\global\long\def\cov{\operatorname{Cov}}
\global\long\def\when#1{\left.#1\right|}
\global\long\def\laplace#1{\mathcal{L}\left[#1\right]}


\section*{第1回}

\paragraph{教科書}
\begin{itemize}
\item 「集積回路設計」浅田著 コロナ社
\item (参) ``CMOS VLSI Design'' N.Weste., D.Harris Addison Wesley
\end{itemize}

\paragraph{講義}
\begin{enumerate}
\item MOSと論理回路
\item CMOSインバータ
\item 遅延モデル
\item LSI製造プロセス
\item 設計規則
\item 電力
\item 基本ゲート
\item メモリー
\item データパス
\item 配線
\item 回路シミュレーション
\item 安定性
\end{enumerate}

\paragraph{評価}

期末試験のみ (+出席点・毎回の課題による下駄)

\paragraph{MOSトランジスタ}

図VL1-1

↓旧

図VL1-2

↓新

図VL1-3
\begin{itemize}
\item 表面の凹凸が無い
\item 素子分離の距離が小さい
\end{itemize}

\paragraph{MOSの動作}
\begin{itemize}
\item $I_{D}=0\:\left(V_{GS}\leqq V_{TH}\right)$ 実際には$I_{D}=I_{DO}\e^{\frac{qV_{GS}}{KT}}$
\item $I_{D}=K_{p}\frac{w}{L}\left\{ \left(V_{GS}-V_{TH}\right)V_{DS}-\frac{V_{DS}^{2}}{2}\right\} \:\left(V_{TH}\leqq V_{GS}\leqq V_{DS}+V_{TH}\right)$
\item $I_{D}=\frac{K_{p}}{2}\frac{w}{L}\left(V_{GS}-V_{TH}\right)^{2}\:\left(V_{GS}-V_{TH}\geqq V_{DS}\right)$
\end{itemize}

\paragraph{CMOS論理}

例 インバータ

・記号

図VL1-4

・真理値表

\begin{tabular}{|c|c|}
\hline 
A & Y\tabularnewline
\hline 
\hline 
0 & 1\tabularnewline
\hline 
1 & 0\tabularnewline
\hline 
\end{tabular}

・トランジスタレベル回路図

図VL1-5

・CMOS論理: 一般化

図VL1-6

\begin{tabular}{|c|c|c|}
\hline 
Y & $P_{U}=\text{OFF}$ & $P_{U}=\text{ON}$\tabularnewline
\hline 
\hline 
$P_{D}=\text{OFF}$ & HighZ & $Y=1$\tabularnewline
\hline 
$P_{D}=\text{ON}$ & $Y=0$ & \texttimes{}\tabularnewline
\hline 
\end{tabular}
\begin{itemize}
\item PMOS
\begin{itemize}
\item 入力$A=0$: PMOS: ON
\item 入力$A=1$: PMOS: OFF
\end{itemize}
\item NMOS
\begin{itemize}
\item 入力$A=0$: NMOS: OFF
\item 入力$A=1$: NMOS: ON
\end{itemize}
\end{itemize}

\paragraph{MOSの直並列接続}

・PMOS

図VL1-7

\begin{tabular}{|c|c|c|c|}
\hline 
 &  &  & \tabularnewline
\hline 
\hline 
0 (ON) & 0 (ON) & 1 (OFF) & 1 (OFF)\tabularnewline
\hline 
0 (ON) & 1 (OFF) & 0 (ON) & 1 (OFF)\tabularnewline
\hline 
ON & OFF & OFF & OFF\tabularnewline
\hline 
\end{tabular}

図VL1-8

\begin{tabular}{|c|c|c|c|}
\hline 
 &  &  & \tabularnewline
\hline 
\hline 
0 (ON) & 0 (ON) & 1 (OFF) & 1 (OFF)\tabularnewline
\hline 
0 (ON) & 1 (OFF) & 0 (ON) & 1 (OFF)\tabularnewline
\hline 
ON & ON & ON & OFF\tabularnewline
\hline 
\end{tabular}

・NMOS

図VL1-9

\begin{tabular}{|c|c|c|c|}
\hline 
 &  &  & \tabularnewline
\hline 
\hline 
0 (OFF) & 0 (OFF) & 1 (ON) & 1 (ON)\tabularnewline
\hline 
0 (OFF) & 1 (ON) & 0 (OFF) & 1 (ON)\tabularnewline
\hline 
OFF & OFF & OFF & ON\tabularnewline
\hline 
\end{tabular}

図VL1-10

\begin{tabular}{|c|c|c|c|}
\hline 
 &  &  & \tabularnewline
\hline 
\hline 
0 (OFF) & 0 (OFF) & 1 (ON) & 1 (ON)\tabularnewline
\hline 
0 (OFF) & 1 (ON) & 0 (OFF) & 1 (ON)\tabularnewline
\hline 
OFF & ON & ON & ON\tabularnewline
\hline 
\end{tabular}

PMOSの直列はNMOSの並列に、PMOSの並列はNMOSの直列に対応する。

\paragraph{複合ゲート}

\[
Y=\overline{AB+CD}
\]

Nネットワーク $AB+CD$

図VL1-11

$Y=\overline{AB+CD}$の複合ゲートでの実現

図VL1-12

\paragraph{複雑な論理}

4入力NAND

図VL1-13

\paragraph{宿題}

次のトランジスタレベル回路図を書け。
\begin{itemize}
\item 4入力NOR
\item $Y=\overline{A\cdot B\cdot C+D}$
\item $Y=\overline{\left(AB+C\right)\cdot D}$
\item $Y=\overline{AB+C\left(A+B\right)}$
\end{itemize}
ただし、最もトランジスタ数が少なくなるCMOS回路として。
\end{document}
