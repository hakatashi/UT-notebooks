%% LyX 2.2.2 created this file.  For more info, see http://www.lyx.org/.
%% Do not edit unless you really know what you are doing.
\documentclass[english]{article}
\usepackage[T1]{fontenc}
\usepackage[utf8]{inputenc}
\usepackage[a5paper]{geometry}
\geometry{verbose,tmargin=2cm,bmargin=2cm,lmargin=1cm,rmargin=1cm}
\setlength{\parskip}{\smallskipamount}
\setlength{\parindent}{0pt}
\usepackage{textcomp}
\usepackage{amsmath}

\makeatletter
%%%%%%%%%%%%%%%%%%%%%%%%%%%%%% User specified LaTeX commands.
\usepackage[dvipdfmx]{hyperref}
\usepackage[dvipdfmx]{pxjahyper}

% http://tex.stackexchange.com/a/192428/116656
\AtBeginDocument{\let\origref\ref
   \renewcommand{\ref}[1]{(\origref{#1})}}

\makeatother

\usepackage{babel}
\begin{document}

\title{2017-A 光電子工学}

\author{教員: 入力: 高橋光輝}

\maketitle
\global\long\def\pd#1#2{\frac{\partial#1}{\partial#2}}
\global\long\def\d#1#2{\frac{\mathrm{d}#1}{\mathrm{d}#2}}
\global\long\def\pdd#1#2{\frac{\partial^{2}#1}{\partial#2^{2}}}
\global\long\def\dd#1#2{\frac{\mathrm{d}^{2}#1}{\mathrm{d}#2^{2}}}
\global\long\def\ddd#1#2{\frac{\mathrm{d}^{3}#1}{\mathrm{d}#2^{3}}}
\global\long\def\e{\mathrm{e}}
\global\long\def\i{\mathrm{i}}
\global\long\def\j{\mathrm{j}}
\global\long\def\grad{\operatorname{grad}}
\global\long\def\rot{\operatorname{rot}}
\global\long\def\div{\operatorname{div}}
\global\long\def\diag{\operatorname{diag}}
\global\long\def\rank{\operatorname{rank}}
\global\long\def\prob{\operatorname{Prob}}
\global\long\def\cov{\operatorname{Cov}}
\global\long\def\when#1{\left.#1\right|}
\global\long\def\laplace#1{\mathcal{L}\left[#1\right]}


\section*{光電子工学I (前半)}

\subsection*{第1回}
\begin{itemize}
\item 科学・工学の先端。様々な応用。
\item 電磁波
\end{itemize}

\section{光電子工学I (前半)}

\subsection{複素振幅と解析信号}

正弦波

\[
S\left(t\right)=S_{0}\cos\left(\omega t-\phi_{0}\right)=\Re A\e^{-\i\omega t}=\frac{1}{2}\left[A\e^{-\i\omega t}+A^{*}\e^{\i\omega t}\right]
\]

$S_{0}$: 振幅

$\omega$: 角周波数

$\phi_{0}$: 位相

$A$: 複素振幅 $A=S_{0}\e^{\i\phi_{0}}$

\paragraph{任意の信号}

様々な$\omega$の正弦波の和

\[
\begin{cases}
s\left(t\right)=\frac{1}{2\pi}\int\hat{s}\left(\omega\right)\e^{-\i\omega t}\mathrm{d}\omega\\
\hat{s}\left(\omega\right)=\int s\left(t\right)\e^{\i\omega t}\mathrm{d}t
\end{cases}
\]

\[
s\left(t\right)=\Re\left[A\left(t\right)\e^{-\i\omega_{0}t}\right]=\Re.\varepsilon\left(t\right)
\]

$A\left(t\right)$: 複素振幅

$\omega_{0}$: 中心周波数

\subsection{ビートと群遅延}
\begin{itemize}
\item 2つの周波数の正弦波の和→強度が差周波で振動 (ビート)
\end{itemize}
\[
\varepsilon\left(t\right)=A_{1}\e^{-\i\omega_{1}t}+A_{2}\e^{-\i\omega_{2}t}
\]

\[
\left|\varepsilon\left(t\right)\right|^{2}=\varepsilon\left(t\right)\varepsilon^{*}\left(t\right)=\underbrace{\left|A_{1}\right|^{2}+\left|A_{2}\right|^{2}}_{\text{強度の時間平均}}+2\left|A_{1}\right|\left|A_{2}\right|\cos\left(\underbrace{-\left(\omega_{1}-\omega_{2}\right)t}_{\text{強度の時間変化}}+\underbrace{\phi_{1}-\phi_{2}}_{\text{位相}}\right)
\]

強度が最大になる時間$\tau=\frac{\phi_{1}-\phi_{2}}{\omega_{1}-\omega_{2}}$:
群遅延

\subsection{パルス波形と群遅延分散}

ガウシアンパルス

\begin{align*}
A\left(t\right) & =A_{0}\exp\left(-\frac{1}{2}\left(\frac{t}{T_{0}}\right)^{2}\right)\\
\left|A\left(t\right)\right|^{2} & =\left|A_{0}\right|^{2}\exp\left(-\left(\frac{t}{T_{0}}\right)^{2}\right)\\
\xrightarrow{\mathcal{F}}\hat{A}\left(\omega\right) & =\sqrt{2\pi}A_{0}T_{0}\exp\left(-\frac{\left(\omega T_{0}\right)^{2}}{2}\right)
\end{align*}

図光1-1

$\Delta t$: 半値全幅 (full-width half maximum: FWHM)

\[
\Delta t\Delta f=\left(\text{一定}\right)\qquad\left(\text{ガウシアンでは0.44}\right)
\]

位相が、時間に対して1次関数的に変化すると?

図光1-2

\[
\mathcal{F}\left[\underbrace{A\left(t\right)\e^{-\i\Omega t}}_{A'\left(t\right)}\right]=A\left(\omega-\Omega\right)
\]

$\phi\left(t\right)=\arg A'\left(t\right)=\arg A_{0}-\Omega t$

$\Omega=-\d{\phi\left(t\right)}t$: 瞬時周波数

位相が、周波数に対して一次関数的に変化すると?

図光1-3

\[
\mathcal{F}^{-1}\left[\hat{A}\left(\omega\right)\e^{-\i\omega\tau}\right]=A\left(t-\tau\right)
\]

\[
\hat{\phi}\left(\omega\right)=\arg\hat{A}\left(\omega\right)\e^{\i\omega\tau}=\arg A_{0}+\omega t
\]

$\tau=\d{\hat{\phi}\left(\omega\right)}{\omega}$: 群遅延

位相が、時間に対して2次関数的に変化すると?

\[
A''\left(t\right)=A\left(t\right)\exp\left(-\frac{\i C}{2}\left(\frac{t}{T_{0}}\right)^{2}\right)=A_{0}\exp\left(-\frac{1+\i C}{2}\left(\frac{t}{T_{0}}\right)^{2}\right)
\]

$\phi''\left(t\right)=\arg A''\left(t\right)=\arg A_{0}-\frac{C}{2}\left(\frac{t}{T_{0}}\right)^{2}\rightarrow\Omega=-\d{\phi^{2}\left(t\right)}t=\frac{C}{T_{0}^{2}}t$

図光1-4

フーリエ変換: $T_{0}\rightarrow\frac{T_{0}}{\sqrt{1+\i C}}$

\[
\hat{A}''\left(\omega\right)=\frac{\sqrt{2\pi}A_{0}T_{0}}{\sqrt{1+\i C}}\exp\left(-\frac{\left(\omega T_{0}\right)^{2}}{1+\i C}\right)
\]

スペクトル幅: $\sqrt{1+C^{2}}$倍

\[
\d{\hat{\phi}''\left(\omega\right)}{\omega}=\frac{CT_{0}^{2}}{1+C^{2}}\omega
\]

群遅延が$\omega$に依存 (\textbf{群遅延分散})

\subsection{位相速度と群速度}

\begin{align*}
\varepsilon\left(z,t\right) & =A_{0}\exp.\i\left(kz-\omega t\right)\\
\varepsilon\left(z,t+\Delta t\right) & =A_{0}\exp.\i\left(k\left(z-\frac{\omega}{k}\Delta t\right)-\omega t\right)
\end{align*}

$k=\frac{2\pi}{\lambda}$: 波数

$\frac{\omega}{k}$を位相速度と呼ぶ。

ビートの場合: 周波数$\omega_{1},\omega_{2}$・波数$k_{1},k_{2}$→群遅延$\tau=\frac{k_{1}z-k_{2}z}{\omega_{1}-\omega_{2}}$・ビートの速度$\frac{z}{\tau}=\frac{\omega_{1}-\omega_{2}}{k_{1}-k_{2}}$

パルスの場合: $\tau=\d{k\left(\omega\right)z}{\omega}$→パルスの速度$v_{g}=\left(\d k{\omega}\right)^{-1}$

※物質中で、$v_{g}$の$\omega$依存性 (群速度分散) があると、伝搬後にパルスが広がる。

\subsection{スペクトル干渉とパルス列}

図光1-5

\section*{第2回}

\section{?}

\subsection{?}

\subsection{2つの平面波の干渉}

\begin{align*}
A\left(\boldsymbol{r}\right) & =A_{1}\exp\left(\i\boldsymbol{k}_{1}\cdot\boldsymbol{r}\right)+A_{2}\exp\left(\i\boldsymbol{k}_{2}\cdot\boldsymbol{r}\right)\\
\left|A\left(\boldsymbol{r}\right)\right|^{2} & =\left|A_{1}\right|^{2}+\left|A_{2}\right|^{2}+2\Re A_{1}A_{2}^{*}\exp\left(\i\left(\boldsymbol{k}_{1}-\boldsymbol{k}_{2}\right)\cdot\boldsymbol{r}\right)
\end{align*}

ここで
\[
2\Re A_{1}A_{2}^{*}\exp\left(\i\left(\boldsymbol{k}_{1}-\boldsymbol{k}_{2}\right)\cdot\boldsymbol{r}\right)=2\left|A_{1}\right|\left|A_{2}\right|\cos\left(\left(\boldsymbol{k}_{1}-\boldsymbol{k}_{2}\right)\cdot\boldsymbol{r}+\phi_{1}-\phi_{2}\right)
\]


\subsection{ガウシアンビーム}

$z=0$における複素振幅

\begin{align*}
A\left(x,y,0\right) & =A_{0}\exp\left(-\frac{1}{2}\left(\frac{x}{W_{0}}\right)^{2}\right)\exp\left(-\frac{1}{2}\left(\frac{y}{W_{0}}\right)^{2}\right)\\
\tilde{A}\left(k_{x},k_{y}\right) & =F_{xy}\left[A\left(x,y,0\right)\right]=\iint A\left(x,y\right)\e^{-\i\left(k_{x}x+k_{y}y\right)}\mathrm{d}x\mathrm{d}y\\
A\left(x,y,0\right) & =F_{xy}^{-1}\left[\tilde{A}\left(k_{x},k_{y}\right)\right]=\left(\frac{1}{2\pi}\right)^{2}\iint\tilde{A}\left(k_{x},k_{y}\right)\e^{\i\left(k_{x}x+k_{y}y\right)}\mathrm{d}k_{x}\mathrm{d}k_{y}\\
A\left(x,y,z\right) & =\left(\frac{1}{2\pi}\right)^{2}\iint\tilde{A}\left(k_{x},k_{y}\right)\e^{\i\left(k_{x}x+k_{y}y+k_{z}z\right)}\mathrm{d}k_{x}\mathrm{d}k_{y}\\
 & =F_{xy}^{-1}\left[\tilde{A}\left(k_{x},k_{y}\right)\e^{\i k_{z}z}\right]
\end{align*}

ただし
\[
k_{z}=k-\frac{k_{x}^{2}+k_{y}^{2}}{2k}
\]

波長が長く($k$が小さい)、ビームが短いほど、$\boldsymbol{k}$の広がりが大きく、早く広がる。(回折)

$\boldsymbol{k}$が長い($\lambda\rightarrow0$)と、光は広がらない。真っ直ぐ進む(光線)

\[
A\left(x,y,z\right)=\underbrace{a\left(z\right)\exp\left(\i kz\right)}_{x=y=0\text{における複素振幅}}\underbrace{\exp\left(-\frac{1}{2}\frac{x^{2}+y^{2}}{\underbrace{W_{0}^{2}\left(z\right)}_{\text{ビームの幅}}}\right)}_{\text{ビームの空間強度}}\exp\left(\frac{\i k}{2}\frac{x^{2}+y^{2}}{\underbrace{R\left(z\right)}_{\text{波面の曲率半径}}}\right)
\]


\section*{第3回}

\section{回折}

伝搬とともに光が形を変える

\subsection{角スペクトル法}

図光3-1

$A_{1}\left(x,y\right)$を平面波に分解

\begin{align*}
\hat{A}_{1}\left(k_{x},k_{y}\right) & =F_{xy}\left[A\left(x,y\right)\right]\\
\hat{A}_{2}\left(k_{x},k_{y}\right) & =\hat{A}_{1}\left(k_{x},k_{y}\right)\exp\left(\i\left(k-\frac{k_{x}^{2}+k_{y}^{2}}{2k}\right)z\right)\\
\rightarrow A_{2}\left(x,y\right) & =F^{-1}\left[\hat{A}_{2}\left(k_{x},k_{y}\right)\right]
\end{align*}

※$\i\left(k-\frac{k_{x}^{2}+k_{y}^{2}}{2k}\right)=\sqrt{k^{2}-k_{x}^{2}-k_{y}^{2}}=k_{z}$

\subsection{フレネル回折積分}

実空間で表したもの

\[
A_{2}\left(x,y\right)=A_{1}\left(x,y\right)\Asterisk H\left(x,y\right)
\]

$H\left(x,y\right)\leftarrow F^{-1}\left[\exp\i\left(k-\frac{k_{x}^{2}+k_{y}^{2}}{k}\right)z\right]$

$H\left(x,y\right)$の導出: ガウシアンビームの解$A\left(x,y,z\right)$に対し、$2\pi A_{0}W_{0}^{2}=1$を保ったまま$W_{0}\rightarrow0$

\[
\hat{A}\left(k_{x},k_{y}\right)=A_{0}\sqrt{2\pi}W_{0}\exp\left(-\frac{1}{2}\left(k_{0}W_{0}\right)^{2}\right)\sqrt{2\pi}W_{0}\exp\left(-\frac{1}{2}\left(k_{0}W_{0}\right)\right)\rightarrow1
\]

\[
A\left(x,y,z\right)\rightarrow\frac{1}{\i\lambda z}\exp\left(\i k\left(z+\frac{x^{2}+y^{2}}{2z}\right)\right)\equiv H\left(x,y\right)
\]

$H\left(x,y\right)$の意味: $H\left(x,y\right)\sim\frac{1}{\i\lambda r}\e^{\i kr}$
(球面波)

図光3-2
\begin{itemize}
\item $\frac{1}{z}\left(\frac{1}{r}\right)$で減衰: エネルギー保存
\item $\frac{1}{\lambda}$に比例。$\lambda$が大きいほど早く回折
\end{itemize}
\begin{align*}
A_{2}\left(x,y\right) & =\iint A_{1}\left(X,Y\right)H\left(x-X,y-Y\right)\mathrm{d}X\mathrm{d}Y\\
 & =\frac{\e^{\i kz}}{\i\lambda z}\iint A_{1}\left(X,Y\right)\exp\frac{\i k\left(\left(x-X\right)^{2}+\left(y-Y\right)^{2}\right)}{2z}\mathrm{d}X\mathrm{d}Y
\end{align*}

$A_{2}$は$A_{1}$に球面波を畳み込んだもの。

もし、$A_{1}\left(x,y\right)=1$とすると?

\begin{align*}
\hat{A}_{1}\left(k_{x},k_{y}\right) & =\left(2\pi\right)^{2}\delta\left(k_{x}\right)\delta\left(k_{x}\right)\\
\hat{A}_{2}\left(k_{x},k_{y}\right) & =\hat{A}_{1}\left(k_{x},k_{y}\right)\exp\left(\i\left(k-\frac{k_{x}^{2}+k_{y}^{2}}{2k}\right)z\right)\\
 & =\left(2\pi\right)^{2}\delta\left(k_{x}\right)\delta\left(x_{y}\right)\exp\i kz\\
\hat{A}_{2}\left(x,y\right) & =\exp\i kz\leftarrow\text{平面波}
\end{align*}

球面波を一様に平面で畳み込むと、平面波。

図3-3

これをホイヘンスの原理という。

\subsection{フラウンホーファー回折}

図光3-4

十分遠方では$\left|A_{2}\left(x,y\right)\right|^{2}$は$\left|\hat{A}_{1}\left(k_{x},k_{y}\right)\right|^{2}$と相似。

\begin{align*}
A_{x}\left(x,y\right)= & \frac{\exp\i kz}{\i\lambda z}\iint A_{1}\left(X,Y\right)\exp\frac{\i k\left(\left(x-X\right)^{2}+\left(y-Y\right)^{2}\right)}{2z}\mathrm{d}X\mathrm{d}Y\\
= & \frac{\exp\i kz}{\i\lambda z}\exp\frac{\i k\left(x^{2}+y^{2}\right)}{2z}\\
 & \times\iint\underbrace{A_{1}\left(X,Y\right)\exp\frac{\i k\left(X^{2}+Y^{2}\right)}{2z}\exp\frac{-ik}{z}\left(xX+yY\right)}_{\hat{A}_{1}\left(\frac{k}{z}x,\frac{k}{z}y\right)}\mathrm{d}X\mathrm{d}Y
\end{align*}

\[
\left|A_{2}\left(x,y\right)\right|^{2}\propto\left|A_{1}\left(\frac{k}{z}x,\frac{k}{z}y\right)\right|^{2}
\]


\subsection{開口による回折}

図光3-5

ダブルスリット

図光3-6

円形開口による回折

図光3-7

$\frac{J_{1}\left(r\right)}{r}$: 円盤の2次元フーリエ変換

$k_{r}=\sqrt{k_{x}^{2}+k_{y}^{2}}$の関数

図光3-8

ベッセル関数: 
\[
J_{n}\left(m\right)=\int_{-\pi}^{\pi}\e^{\i m\sin t}\e^{-\i nt}\mathrm{d}t
\]

位相変調のフーリエ展開級数

図光3-9

\section{?}

\section{?}

\section{?}

\section{時間領域の光学}

光・周波数依存性
\begin{itemize}
\item 色
\item 光ファイバ通信・波長多重
\item 光計測技術
\end{itemize}
インパルス応答
\begin{itemize}
\item 物質由来 (電子の動き)
\item 構造由来 (光路の工夫) ←今日の話
\end{itemize}

\subsection{マイケルソン干渉計}

インパルス応答

\[
h\left(t\right)=\frac{1}{2}\left(\delta\left(t\right)+\delta\left(t-\tau\right)\right)
\]

周波数応答
\[
H\left(\omega\right)=\frac{1}{2}\left(1+\e^{\i\omega\tau}\right)
\]

パワー透過率
\[
\left|H\left(\omega\right)\right|^{2}=\frac{1}{2}\left(1+\cos\omega\tau\right)
\]


\subsection{ファブリ・ペローチ干渉計}

\subsection{フーリエ分光}

\subsection{フーリエ分光}

マイケルソン干渉計を使った、光のスペクトルの計測法

光強度 $I\left(t,\tau\right)=\left|\frac{1}{2}\left(\varepsilon\left(t\right)+\varepsilon\left(t-\tau\right)\right)\right|^{2}$

\paragraph{単色光の時}

\begin{align*}
\varepsilon\left(t\right) & =\varepsilon_{0}\exp\left(-\i\omega t\right)\\
I\left(t,\tau\right) & =\frac{1}{2}\left|\varepsilon_{0}\right|^{2}\left(1+\cos\omega\tau\right)
\end{align*}

($I$は$t$によらない)

\paragraph{広帯域の時}

\begin{align*}
I\left(\tau\right) & \equiv\int\left|\frac{1}{2}\left(\varepsilon\left(t\right)+\varepsilon\left(t-\tau\right)\right)\right|^{2}\mathrm{d}t\\
F\left[I\left(\tau\right)\right] & =c\delta\left(\omega\right)+\frac{1}{4}\left|\hat{\varepsilon}\left(\omega\right)\right|^{2}+\frac{1}{4}\left|\hat{\varepsilon}\left(-\omega\right)\right|^{2}
\end{align*}


\paragraph{試験について}
\begin{itemize}
\item 次回以降の授業は種村先生が担当
\item 試験は1月30日午前中に実施
\item 内容は「言葉の説明問題」「応用問題」
\item 持ち込みなし
\item 覚えていないと解けない問題は出さない
\item ビート・ガウシアンパルス・二光速干渉・ガウシアン近似・フレネル回折積分・ホイヘンスの原理・フラウンホーファー回折・4f光学系・レンツ・フーリエ変換を2回行う話・屈折と反射・干渉計・フーリエ分光・回折格子
\item 習った内容の確認を行う
\end{itemize}

\end{document}
