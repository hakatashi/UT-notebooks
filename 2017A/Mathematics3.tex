%% LyX 2.2.2 created this file.  For more info, see http://www.lyx.org/.
%% Do not edit unless you really know what you are doing.
\documentclass[english]{article}
\usepackage[T1]{fontenc}
\usepackage[utf8]{inputenc}
\usepackage[a5paper]{geometry}
\geometry{verbose,tmargin=2cm,bmargin=2cm,lmargin=1cm,rmargin=1cm}
\setlength{\parskip}{\smallskipamount}
\setlength{\parindent}{0pt}
\usepackage{textcomp}
\usepackage{amsmath}

\makeatletter
%%%%%%%%%%%%%%%%%%%%%%%%%%%%%% User specified LaTeX commands.
\usepackage[dvipdfmx]{hyperref}
\usepackage[dvipdfmx]{pxjahyper}

% http://tex.stackexchange.com/a/192428/116656
\AtBeginDocument{\let\origref\ref
   \renewcommand{\ref}[1]{(\origref{#1})}}

\makeatother

\usepackage{babel}
\begin{document}

\title{2017-A 数学3}

\author{教員: 入力: 高橋光輝}

\maketitle
\global\long\def\pd#1#2{\frac{\partial#1}{\partial#2}}
\global\long\def\d#1#2{\frac{\mathrm{d}#1}{\mathrm{d}#2}}
\global\long\def\pdd#1#2{\frac{\partial^{2}#1}{\partial#2^{2}}}
\global\long\def\dd#1#2{\frac{\mathrm{d}^{2}#1}{\mathrm{d}#2^{2}}}
\global\long\def\ddd#1#2{\frac{\mathrm{d}^{3}#1}{\mathrm{d}#2^{3}}}
\global\long\def\e{\mathrm{e}}
\global\long\def\i{\mathrm{i}}
\global\long\def\j{\mathrm{j}}
\global\long\def\grad{\operatorname{grad}}
\global\long\def\rot{\operatorname{rot}}
\global\long\def\opdiv{\operatorname{div}}
\global\long\def\diag{\operatorname{diag}}
\global\long\def\rank{\operatorname{rank}}
\global\long\def\prob{\operatorname{Prob}}
\global\long\def\cov{\operatorname{Cov}}
\global\long\def\when#1{\left.#1\right|}
\global\long\def\laplace#1{\mathcal{L}\left[#1\right]}


\section*{第4回}

(図略: 曲面・赤道・緯線・経線・接平面)

緯度$u$、経度$v$とする。

$\boldsymbol{r}\left(u+\Delta u,v+\Delta v\right)$と$\boldsymbol{r}\left(u,v\right)$の間の距離の自乗$\left(\Delta\boldsymbol{r}\right)^{2}$

\[
\left(\Delta\boldsymbol{r}\right)^{2}=I=E\left(\Delta u\right)^{2}+2F\Delta u\Delta v+F\left(\Delta v\right)^{2}
\]

点$P$における接ベクトル$\boldsymbol{r}_{u},\boldsymbol{r}_{v}$

\[
\boldsymbol{e}=\frac{\boldsymbol{r}_{u}\times\boldsymbol{r}_{v}}{\left|\boldsymbol{r}_{u}\times\boldsymbol{r}_{v}\right|}
\]

\[
\boldsymbol{I}=-\mathrm{d}\boldsymbol{r}\cdot\mathrm{d}\boldsymbol{e}=\mathrm{d}^{2}\boldsymbol{r}\cdot\boldsymbol{e}=\mathrm{d}\left(\mathrm{d}\boldsymbol{r}\boldsymbol{e}\right)-\mathrm{d}\boldsymbol{r}\cdot\mathrm{d}\boldsymbol{e}
\]

曲面上の曲率を求める

\[
\boldsymbol{r}=\boldsymbol{r}\left(u\left(s\right),v\left(s\right)\right)
\]

\[
\boldsymbol{r}''=\dd{\boldsymbol{r}}s
\]

($s$は孤長定数)

\[
\boldsymbol{r}''=\boldsymbol{k}_{g}+\boldsymbol{k}_{m}
\]

$\boldsymbol{k}_{g}$: 接平面内のベクトル (測地的曲率ベクトル)

$\boldsymbol{k}_{m}$: 接平面に垂直 (法曲率ベクトル, normal curvature vecture)

平面の時$\boldsymbol{k}_{g}$のみ

\[
\kappa_{n}=\left|\boldsymbol{k}_{n}\right|
\]
と定めると、
\[
\boldsymbol{k}_{n}=\kappa_{n}\boldsymbol{e}
\]
と表される。

\[
\kappa_{n}=\boldsymbol{k}_{n}\cdot\boldsymbol{e}=\left(\boldsymbol{r}''-\boldsymbol{k}_{g}\right)\cdot\boldsymbol{e}=\boldsymbol{r}''\cdot\boldsymbol{e}-\boldsymbol{k}_{g}\cdot\boldsymbol{e}
\]

\begin{align*}
\kappa_{n} & =\boldsymbol{r}''\cdot\boldsymbol{e}=\left(\boldsymbol{r}'\cdot\boldsymbol{e}\right)-\boldsymbol{r}'\cdot\boldsymbol{e}'\\
 & =-\left(\pd{\boldsymbol{r}}u\d us+\pd{\boldsymbol{r}}v\d vs\right)\cdot\left(\pd{\boldsymbol{e}}u\d us+\pd{\boldsymbol{e}}v\d vs\right)\\
 & =\underbrace{-\boldsymbol{r}_{u}\cdot\boldsymbol{e}_{u}}_{L}\left(\d us\right)^{2}\underbrace{-\left(\boldsymbol{r}_{u}\cdot\boldsymbol{e}_{v}+\boldsymbol{r}_{v}\boldsymbol{e}_{u}\right)}_{2M}\d us\d vs\underbrace{-\boldsymbol{r}_{v}\cdot\boldsymbol{e}_{v}}_{N}\left(\d vs\right)^{2}\\
 & =L\left(\d us\right)^{2}+2M\d us\d vs+N\left(\d vs\right)^{2}\\
 & =\frac{\text{II}}{\left(\mathrm{d}s\right)^{2}}
\end{align*}
と第II基本形式を用いて表される。

接ベクトル$\boldsymbol{n}=\xi\boldsymbol{r}_{u}+\eta\boldsymbol{r}_{v}$に対して
\[
II\left(\boldsymbol{w},\boldsymbol{n}\right)=L\xi^{2}+2M\xi\eta+N\eta^{2}
\]
と定めよう

$\left|\boldsymbol{n}\right|^{2}=1$の単位円上で$\kappa_{n}$の極地を求めよう

\begin{align*}
\left|\boldsymbol{n}\right|^{2} & =\left(\xi\boldsymbol{r}_{u}+\eta\boldsymbol{r}_{v}\right)^{2}\\
 & =\boldsymbol{r}_{u}\cdot\boldsymbol{r}_{u}\xi^{2}+2\boldsymbol{r}_{u}\cdot\boldsymbol{r}_{v}\xi\eta+\boldsymbol{r}_{v}\cdot\boldsymbol{r}_{v}\eta^{2}\\
 & =E\xi^{2}+2F\xi\eta+G\eta^{2}
\end{align*}
て表される。ただし
\[
\begin{cases}
E=\boldsymbol{r}_{u}\cdot\boldsymbol{r}_{u}\\
F=\boldsymbol{r}_{u}\cdot\boldsymbol{r}_{v}\\
G=\boldsymbol{r}_{v}\cdot\boldsymbol{r}_{v}
\end{cases}
\]

(図略: トーラス)

\[
\lambda=\frac{II\left(\boldsymbol{u},\boldsymbol{n}\right)}{\left|\boldsymbol{n}\right|^{2}}
\]
の$\xi,\eta$にかんする極地を求めれば良い。($\pd{\lambda}{\xi}=0,\pd{\lambda}{\eta}=0$となる$\lambda$)

$II\left(\boldsymbol{n},\boldsymbol{u}\right)-\lambda\left|\boldsymbol{n}\right|^{2}=0$を$\xi$と$\eta$で微分して0いなるものを求める。

ただし
\[
\begin{cases}
L=-\boldsymbol{r}_{u}\cdot\boldsymbol{e}_{u}\\
M=-\boldsymbol{r}_{u}\cdot\boldsymbol{e}_{v}=-\boldsymbol{r}_{v}\cdot\boldsymbol{e}_{u}\\
N=-\boldsymbol{r}_{v}\cdot\boldsymbol{e}_{v}
\end{cases}
\]

\[
A=II\left(\boldsymbol{n},\boldsymbol{n}\right)-\lambda\left|\boldsymbol{n}\right|^{2}
\]

\begin{align*}
\pd A{\xi} & =2L\xi+2M\eta-\lambda\left(2E\xi+2F\eta\right)=0\\
\pd A{\eta} & =2M\xi+2N\eta-\lambda\left(2F\xi+2G\eta\right)=0
\end{align*}

\begin{align*}
\left(\begin{array}{cc}
L & M\\
M & N
\end{array}\right)\left(\begin{array}{c}
\xi\\
\eta
\end{array}\right)-\left(\begin{array}{cc}
E & F\\
F & G
\end{array}\right)\left(\begin{array}{c}
\xi\\
\eta
\end{array}\right) & =\left(\begin{array}{c}
0\\
0
\end{array}\right)\\
 & =\left(\begin{array}{cc}
L-\lambda E & M-\lambda F\\
M-\lambda F & N-\lambda G
\end{array}\right)\left(\begin{array}{c}
\xi\\
\eta
\end{array}\right)
\end{align*}

\begin{align*}
\det\left(\begin{array}{cc}
L-\lambda E & M-\lambda F\\
M-\lambda F & N-\lambda G
\end{array}\right) & =0\\
 & =\left(L-\lambda E\right)\left(N-\lambda G\right)-\left(M-\lambda F\right)^{2}\\
 & =\lambda^{2}\left(EG-F^{2}\right)-\lambda\left(EN-2FM+GL\right)+\left(LN-M^{2}\right)
\end{align*}

法曲率の極地を$\kappa_{1},\kappa_{2}$(主曲率と呼ぶ) は
\begin{align*}
\kappa_{1}+\kappa_{2} & =\frac{EN+GL-2FM}{EG-F^{2}}\\
\kappa_{1}\kappa_{2} & =\frac{LN-M^{2}}{EG-F^{2}}
\end{align*}

$K=\kappa_{1}\kappa_{2}$: ガウス曲率 (Gaussian curvature)

$H=\frac{1}{2}\left(\kappa_{1}+\kappa_{2}\right)$: 平均曲率 (mean curvature)

\[
\left(\boldsymbol{n}\right)^{2}=\left(\xi\boldsymbol{r}_{u}+\eta\boldsymbol{r}_{v}\right)^{2}
\]

\begin{align*}
\kappa_{n} & =\frac{II}{\left(\mathrm{d}s\right)^{2}}\\
 & =II\left(\boldsymbol{n},\boldsymbol{n}\right)
\end{align*}

但し、$\left|\boldsymbol{n}\right|=1$のとき。

一般には
\[
\frac{II\left(\boldsymbol{n},\boldsymbol{n}\right)}{\left|\boldsymbol{n}\right|^{2}}
\]

$H=0$の曲面を極小曲面とよぶ。局所的な変化で面積最小となる曲面。

$\kappa_{n}$の極地$\kappa_{1},\kappa_{2}$を主曲率 (principal curvature)
と呼ぶ。

$II\left(\boldsymbol{n}_{1},\boldsymbol{n}_{1}\right)=\kappa_{1},II\left(\boldsymbol{n}_{2},\boldsymbol{n}_{2}\right)=\kappa_{2}$となる方向$\boldsymbol{n}_{1},\boldsymbol{n}_{2}$を主方向
(principal directions) とよぶ。

2つの主方向は直交する。

\[
\left(\begin{array}{cc}
L & M\\
M & N
\end{array}\right)\left(\begin{array}{c}
\xi_{\gamma}\\
\eta_{\gamma}
\end{array}\right)=\kappa_{\gamma}\left(\begin{array}{cc}
E & F\\
F & G
\end{array}\right)\left(\begin{array}{c}
\xi_{\gamma}\\
\eta_{\gamma}
\end{array}\right)
\]
が成立。

$\left(\begin{array}{c}
\xi_{\gamma}\\
\eta_{\gamma}
\end{array}\right)=\boldsymbol{u}_{\gamma}$とおいて、
\begin{align*}
^{t}\boldsymbol{u}_{1}\left(\kappa_{2}I\boldsymbol{u}_{2}\right) & =^{t}\boldsymbol{u}_{1}I\boldsymbol{u}_{2}\\
 & =^{t}\boldsymbol{v}_{2}\cdot II\boldsymbol{u}_{1}={}^{t}\boldsymbol{v}_{2}\kappa I\boldsymbol{u}_{1}
\end{align*}

\[
\kappa_{2}\cdot{}^{t}\boldsymbol{u}_{1}I\boldsymbol{u}_{2}=\kappa{}^{t}\boldsymbol{v}_{2}I\boldsymbol{u}_{1}
\]

$\kappa_{1}\neq\kappa_{2}$のとき $^{t}\boldsymbol{v}_{2}I\boldsymbol{u}_{1}=0$

$\kappa_{n}=\boldsymbol{r}''\cdot\boldsymbol{e}$は正と負の値が可能 ($\boldsymbol{e}$の向きによって)

半径$a$の場合
\[
\kappa_{1}=\kappa_{2}=\frac{1}{a}
\]

\[
K=\frac{1}{a^{2}}>0
\]

$z=x^{2}+y^{2}$ $\boldsymbol{r}=0$の点は鞍点 (saddle point)
\begin{itemize}
\item ガウス曲率$K>0$となる点: 楕円点 (曲面は凸)
\item ガウス曲率$K<0$となる点: 双曲点 (曲面は鞍状)
\item ガウス曲率$K=0$となる点: 放物点
\end{itemize}
曲面点の点$P$において$\boldsymbol{r}_{u},\boldsymbol{r}_{v},\boldsymbol{e}$は1次独立なので、$\boldsymbol{r}_{uu}=\pdd ru,\boldsymbol{r}_{uv},\boldsymbol{r}_{vv}$は
\begin{align*}
\boldsymbol{r}_{uu} & =\Gamma_{uu}^{u}\boldsymbol{r}_{u}+\Gamma_{uu}^{v}\boldsymbol{r}_{v}+L\boldsymbol{e}\\
\boldsymbol{r}_{uv} & =\Gamma_{uv}^{u}\boldsymbol{r}_{u}+\Gamma_{uv}^{v}\boldsymbol{r}_{v}+M\boldsymbol{e}\\
\boldsymbol{r}_{vv} & =\Gamma_{vv}^{u}\boldsymbol{r}_{u}+\Gamma_{vv}^{v}\boldsymbol{r}_{v}+N\boldsymbol{e}
\end{align*}

※$L=-\boldsymbol{r}_{u}\cdot\boldsymbol{e}_{u}=\boldsymbol{r}_{uu}\cdot\boldsymbol{e}$:
ガウス式

$\Gamma_{ab}^{c}$: クリストッフェル (Christoffel) の記号

\begin{align*}
\boldsymbol{e}_{u} & =\frac{FM-GL}{EG-F^{2}}\boldsymbol{r}_{u}+\frac{FL-EM}{EG-F^{2}}\boldsymbol{r}_{v}\\
\boldsymbol{e}_{v} & =\frac{FN-GM}{EG-F^{2}}\boldsymbol{r}_{u}+\frac{FM-EN}{EG-F^{2}}\boldsymbol{r}_{v}
\end{align*}
: Weingartenの式

$\boldsymbol{e}\cdot\boldsymbol{e}=1$より$\boldsymbol{e}_{u}\cdot\boldsymbol{e}=0\Rightarrow\boldsymbol{e}_{u}=A\boldsymbol{r}_{u}+B\boldsymbol{r}_{v}$と表される。
($\boldsymbol{e}_{u}\bot\boldsymbol{e}$)

\begin{align*}
\boldsymbol{e}_{u}\cdot\boldsymbol{r}_{u} & =-\boldsymbol{r}_{uu}\cdot\boldsymbol{e}=-L\\
 & =\left(A\boldsymbol{r}_{u}+B\boldsymbol{r}_{v}\right)\cdot\boldsymbol{r}_{u}=A\underbrace{\boldsymbol{r}_{u}\cdot\boldsymbol{r}_{u}}_{E}+B\underbrace{\boldsymbol{r}_{u}\cdot\boldsymbol{r}_{v}}_{F}=AE+BF
\end{align*}

\begin{align*}
\boldsymbol{e}_{u}\cdot\boldsymbol{r}_{v} & =-\boldsymbol{r}_{vu}\cdot\boldsymbol{e}=-M\\
 & =\left(A\boldsymbol{r}_{u}+B\boldsymbol{r}_{v}\right)\cdot\boldsymbol{r}_{v}\\
 & =AF+BG
\end{align*}

\[
\begin{cases}
AE+BF=-L\\
AF+BG=-M
\end{cases}
\]
は、
\[
\left(\begin{array}{cc}
E & F\\
F & G
\end{array}\right)\left(\begin{array}{c}
A\\
B
\end{array}\right)=\left(\begin{array}{c}
-L\\
-M
\end{array}\right)
\]
となる。

\begin{align*}
\left(\begin{array}{c}
A\\
B
\end{array}\right) & =\left(\begin{array}{cc}
E & F\\
F & G
\end{array}\right)^{-1}\left(\begin{array}{c}
-L\\
-M
\end{array}\right)\\
 & =\frac{1}{EG-F^{2}}\left(\begin{array}{cc}
G & -F\\
-F & E
\end{array}\right)\left(\begin{array}{c}
-L\\
-M
\end{array}\right)=\frac{1}{EG-F^{2}}\left(\begin{array}{c}
FM-GL\\
FL-EM
\end{array}\right)
\end{align*}
よりWeingartenの式の1番目が確認できる。

\paragraph{Gaussの球面表示}

曲面の各点$P$に法ベクトル$\boldsymbol{e}$を対応させる。

点$P$における接平面の面積$\left|\boldsymbol{r}_{u}\times\boldsymbol{r}_{v}\right|\Delta u\Delta v$と単位球面上の面積$\left|\boldsymbol{e}_{u}\times\boldsymbol{e}_{v}\right|\Delta u\Delta v$の比はガウス曲率になる。

\begin{align*}
\boldsymbol{e}_{u}\times\boldsymbol{e}_{v} & =\left(\alpha\boldsymbol{r}_{u}+\beta\boldsymbol{r}_{v}\right)\times\left(\gamma\boldsymbol{r}_{u}+\delta\boldsymbol{r}_{v}\right)\\
 & =\alpha\delta\boldsymbol{r}_{u}\times\boldsymbol{r}_{v}+\beta\gamma\boldsymbol{r}_{v}\times\boldsymbol{r}_{u}\\
 & =\left(\alpha\delta-\beta\gamma\right)\boldsymbol{r}_{u}\times\boldsymbol{r}_{v}
\end{align*}

\end{document}
