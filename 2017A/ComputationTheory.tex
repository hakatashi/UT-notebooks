%% LyX 2.2.2 created this file.  For more info, see http://www.lyx.org/.
%% Do not edit unless you really know what you are doing.
\documentclass[english]{article}
\usepackage[T1]{fontenc}
\usepackage[utf8]{inputenc}
\usepackage[a5paper]{geometry}
\geometry{verbose,tmargin=2cm,bmargin=2cm,lmargin=1cm,rmargin=1cm}
\setlength{\parskip}{\smallskipamount}
\setlength{\parindent}{0pt}
\usepackage{textcomp}
\usepackage{amsmath}
\usepackage{amssymb}

\makeatletter

%%%%%%%%%%%%%%%%%%%%%%%%%%%%%% LyX specific LaTeX commands.
%% Because html converters don't know tabularnewline
\providecommand{\tabularnewline}{\\}

%%%%%%%%%%%%%%%%%%%%%%%%%%%%%% User specified LaTeX commands.
\usepackage[dvipdfmx]{hyperref}
\usepackage[dvipdfmx]{pxjahyper}

% http://tex.stackexchange.com/a/192428/116656
\AtBeginDocument{\let\origref\ref
   \renewcommand{\ref}[1]{(\origref{#1})}}

\makeatother

\usepackage{babel}
\begin{document}

\title{2017-A 計算論}

\author{教員: 入力: 高橋光輝}

\maketitle
\global\long\def\pd#1#2{\frac{\partial#1}{\partial#2}}
\global\long\def\d#1#2{\frac{\mathrm{d}#1}{\mathrm{d}#2}}
\global\long\def\pdd#1#2{\frac{\partial^{2}#1}{\partial#2^{2}}}
\global\long\def\dd#1#2{\frac{\mathrm{d}^{2}#1}{\mathrm{d}#2^{2}}}
\global\long\def\ddd#1#2{\frac{\mathrm{d}^{3}#1}{\mathrm{d}#2^{3}}}
\global\long\def\e{\mathrm{e}}
\global\long\def\i{\mathrm{i}}
\global\long\def\j{\mathrm{j}}
\global\long\def\grad{\operatorname{grad}}
\global\long\def\rot{\operatorname{rot}}
\global\long\def\opdiv{\operatorname{div}}
\global\long\def\diag{\operatorname{diag}}
\global\long\def\rank{\operatorname{rank}}
\global\long\def\prob{\operatorname{Prob}}
\global\long\def\cov{\operatorname{Cov}}
\global\long\def\when#1{\left.#1\right|}
\global\long\def\laplace#1{\mathcal{L}\left[#1\right]}


\section*{第1回}

有限オートマトンは、有限個の状態と遷移動作から成る。

図計1-1

$w=0011$を与える→受理される

$w=010$を与える→受理されない

\paragraph{決定性有限状態オートマトン (DFA: deterministic finite automaton)}

状態$Q$、入力記号$\Sigma$、受理状態$F\subseteq Q$が与えられたおき、DFAは$M=\left(Q,\Sigma,\delta,q_{0},F\right)$で与えられる。
\begin{enumerate}
\item $Q$は状態の有限集合 ($Q=\left\{ q_{1},q_{2},q_{3}\right\} $)
\item $\Sigma$はアルファベットの有限集合 ($\Sigma=\left\{ 0,1\right\} $)
\item $\delta$: $Q\times\Sigma\rightarrow Q$の遷移関数 ($\delta\left(q_{1},0\right)=q_{2}$)

\begin{tabular}{|c|c|c|}
\hline 
 & 0 & 1\tabularnewline
\hline 
\hline 
$q_{1}$ & $q_{1}$ & $q_{2}$\tabularnewline
\hline 
$q_{2}$ & $q_{3}$ & $q_{2}$\tabularnewline
\hline 
$q_{3}$ & $q_{2}$ & $q_{2}$\tabularnewline
\hline 
\end{tabular}
\item $q_{0}\in Q$は開始状態 ($q_{0}=q_{1}$)
\item $F\subseteq Q$は受理状態集合 ($F=\left\{ q_{2}\right\} $)
\end{enumerate}
DFA $A$が受理する文字列の集合を$A$の言語といい、$L\left(A\right)$で表す。$L\left(A\right)=\left\{ w|w\text{は少なくとも1つの1を含み、最後に現れる1の後ろに偶数個の0を含む}\right\} $

あるDFAで受理される言語を正規言語という。

図計1-2

\paragraph{非決定性有限状態オートマトン (NFA: nondeterministic finite automaton)}

DFA: 「状態」と「入力文字」に対して「次の状態」を与える

NFA: 「状態」と「入力文字又は空列」に対して「取りうる次の状態集合」を与える。

$P\left(Q\right)$で$Q$のべき集合を表すとする。

例 $Q=\left\{ A,B,C\right\} ,P\left(Q\right)=\left\{ \emptyset,\left\{ A\right\} ,\left\{ B\right\} ,\left\{ C\right\} ,\left\{ A,B\right\} ,\left\{ A,C\right\} ,\left\{ B,C\right\} ,\left\{ A,B,C\right\} \right\} $

$\Sigma_{\varepsilon}=\Sigma\cup\left\{ \varepsilon\right\} $とする。NFAの遷移関数は$\delta:Q\times\Sigma_{\varepsilon}\rightarrow P\left(Q\right)$

図計1-3

NFAは$\left(Q,\Sigma,\delta,q_{0},F\right)$で定義される
\begin{enumerate}
\item $Q$
\item $\Sigma$
\item $\delta$: $Q\times\Sigma_{\varepsilon}\rightarrow P\left(Q\right)$の遷移関数
\item $q_{0}$
\item $F$
\end{enumerate}
入力が終了した時で、1つでも受理状態にあれば、文字列はNFAに受理される。

\begin{tabular}{|c|c|c|c|}
\hline 
 & 0 & 1 & $\varepsilon$\tabularnewline
\hline 
\hline 
$q_{1}$ & $\left\{ q_{1}\right\} $ & $\left\{ q_{1},q_{2}\right\} $ & $\emptyset$\tabularnewline
\hline 
$q_{2}$ & $\left\{ q_{3}\right\} $ & $\emptyset$ & $\left\{ q_{3}\right\} $\tabularnewline
\hline 
$q_{3}$ & $\emptyset$ & $\left\{ q_{4}\right\} $ & $\emptyset$\tabularnewline
\hline 
$q_{4}$ & $\left\{ q_{4}\right\} $ & $\left\{ q_{4}\right\} $ & $\emptyset$\tabularnewline
\hline 
\end{tabular}

\paragraph{NFAとDFAの等価性}

DFAとNFAは同じ言語を受理する。

全てのNFAがDFAに変換できることを示せば良い。

NFAの$k$状態に対して、そのべき集合を考えれば良い。

図計1-4

\paragraph{正規演算 (regular operation)}

$A,B$を正規言語とする。以下の3つを正規演算という。
\begin{enumerate}
\item 和集合演算 $A\cup B=\left\{ x|x\in A\:\text{or}\:x\in B\right\} $
\item 連結演算 $A\circ B=\left\{ xy|x\in A\:\text{and}\:x\in B\right\} $
\item スター演算 $A^{*}=\left\{ x_{1}x_{2}\cdots x_{k}|x\geq0\:\text{and}\:x_{i}\in A\right\} $
\end{enumerate}
$A=\left\{ \text{cat},\text{dog}\right\} ,B=\left\{ \text{car},\text{bus}\right\} $

$A\cup B=\left\{ \text{cat},\text{dog},\text{car},\text{bus}\right\} $

$A\circ B=\left\{ \text{catcar},\text{catbus},\text{dogcar},\text{dogbus}\right\} $

$A^{*}=\left\{ \varepsilon,\text{cat},\text{dog},\text{catcat},\text{catdog},\text{dogcat},\text{dogdog},\cdots\right\} $

正規言語は正規演算に対して閉じている。

(「実数は四則に対して閉じている」「整数は四則に対して閉じていない」)

\paragraph{和集合の証明}

図計1-5

\paragraph{連結演算の証明}

図計1-6

\paragraph{スター演算の証明}

図計1-7

\paragraph{正規表現}

言語を記述する表現。正規表現は機能的に定義される。

正規表現は以下の場合
\begin{enumerate}
\item アルファベットに属する$a$
\item $\varepsilon$
\item $\emptyset$
\item $R_{1}$と$R_{2}$を正規表現として$\left(R_{1}\cup R_{2}\right)$
\item $\left(R_{1}\circ R_{2}\right)$
\item $\left(R_{1}^{*}\right)$
\end{enumerate}
言語は、正規表現で表される時、かつその時に限り正規言語である。g

\section*{第3回}

\paragraph{例}

$B=\left\{ 0^{n}1^{n}|n\geq0\right\} $に対してポンピング補題を適用する

ある$s\in B$がポンプ出来ないことを示す。

ポンプ可能と仮定する (ポンピング長を$p$とする)

$s=0^{p}1^{p}$を考える。$s\in B$である。$s=xyz$に分解できるはず。
\begin{enumerate}
\item $y=0$のみの場合。$xy^{2}z$は0が1より多いので、$xy^{2}z\notin B$
\item $y=1$のみの場合。$xy^{2}z$は1が1より多いので、$xy^{2}z\notin B$
\item $y$が0、1を含む。$xy^{2}z\left(=0\cdots010111\cdots1\right)$は1を0の前に持つので、$xy^{2}z\notin B$
\item ポンピング補題を満たさない。$B$は非正規。
\end{enumerate}

\paragraph{非正規言語でポンピング補題を満たす例}

\[
E=\left\{ ab^{j}c^{j}|j\geq0\right\} \cup\left\{ a^{i}a^{j}a^{k}|i,j,k\geq0\text{かつ}i\neq1\right\} 
\]

$E$はポンプ可能である。$F=L\left(ab^{*}c^{*}\right)$を考える。$F$は正規言語。

$E\cap F=\left\{ ab^{j}c^{j}|j\geq0\right\} $は正規言語ではない。

正規言語の積集合は正規言語なので、$E$は正規言語でない。

\paragraph{文脈自由文法 (CFG: context free grammar)}

\begin{tabular}{|c|c|c|}
\hline 
 & 出力 & 受け取る\tabularnewline
\hline 
\hline 
正規言語 & 正規表現 & DFA\tabularnewline
\hline 
文脈自由言語 & CFG & PDA\tabularnewline
\hline 
\end{tabular}

※PDAはプッシュダウン・オートマトンの略。

CFGは正規表現よりも複雑な言語を生成可能である。

CFGは人の離す言語を解析するために考案されたモデル。

CFGは4つの変数$\left(V,\Sigma,R,S\right)$によって定義される。
\begin{itemize}
\item $V$は変数の有限集合 (非終端記号とも言う)
\item $\Sigma$は終端記号の有限集合。$\Sigma\cap V=\oslash$
\item $R$は書き換え規則の有限集合
\item $S\in V$は開始記号
\end{itemize}
書き換え規則は$A\rightarrow\alpha\left(A\in V,\alpha\in\left(\Sigma\cup V\right)^{*}\right)$で表される。

書き換え規則$R$を開始記号に適用することで文 (文字列) が精製される。

例えば$\alpha_{1}A\alpha_{2}\left(\alpha_{1},\alpha_{2}\in\left(\Sigma\cup V\right)^{*},A\in V\right)$に上の規則を適用すると$\alpha_{1}\alpha\alpha_{2}$になる。

$\alpha_{1}\xrightarrow[\text{書き換え規則を適用}]{}\alpha_{2}\rightarrow\cdots\rightarrow\alpha_{n}$の解き、$\alpha_{n}$は$\alpha_{1}$から導出
(derive) されるという。

この導出過程は構文解析木と呼ばれる木構造で表せる。

集合$\left\{ w\in\Sigma^{*}|S\text{から導出される}w\right\} $を文法の言語という。

\paragraph{例}

\begin{align*}
V & =\left\{ \left\langle \text{expr}\right\rangle ,\left\langle \text{op2}\right\rangle ,\left\langle \text{op1}\right\rangle ,\left\langle \text{var}\right\rangle ,\left\langle \text{const}\right\rangle \right\} \\
\Sigma & =\left\{ +,-,\times,\div,\sin,\cos,\exp,\log,x,C\right\} \\
S & =\left\langle \text{expr}\right\rangle 
\end{align*}

$R$:
\begin{align*}
\left\langle \text{expr}\right\rangle  & \rightarrow\left\langle \text{op2}\right\rangle \left\langle \text{expr}\right\rangle \left\langle \text{expr}\right\rangle |\left\langle \text{op1}\right\rangle \left\langle \text{expr}\right\rangle |\left\langle \text{var}\right\rangle \left\langle \text{const}\right\rangle \\
\left\langle \text{op2}\right\rangle  & \rightarrow+|-|\times|\div\\
\left\langle \text{op1}\right\rangle  & \rightarrow\sin|\cos|\exp|\log\\
\left\langle \text{var}\right\rangle  & \rightarrow x\\
\left\langle \text{const}\right\rangle  & \rightarrow C
\end{align*}

\begin{align*}
\left\langle \text{expr}\right\rangle  & \rightarrow\left\langle \text{op2}\right\rangle \left\langle \text{expr}\right\rangle \left\langle \text{expr}\right\rangle \\
 & \rightarrow+\left\langle \text{expr}\right\rangle \left\langle \text{expr}\right\rangle \\
 & \rightarrow+\left\langle \text{op2}\right\rangle \left\langle \text{expr}\right\rangle \left\langle \text{expr}\right\rangle \left\langle \text{expr}\right\rangle \\
 & \rightarrow++\left\langle \text{expr}\right\rangle \left\langle \text{expr}\right\rangle \left\langle \text{expr}\right\rangle \\
 & \rightarrow++\left\langle \text{op1}\right\rangle \left\langle \text{expr}\right\rangle \left\langle \text{expr}\right\rangle \left\langle \text{expr}\right\rangle \\
 & \rightarrow++\log\left\langle \text{expr}\right\rangle \left\langle \text{expr}\right\rangle \left\langle \text{expr}\right\rangle \\
 & \rightarrow++\log\left\langle \text{var}\right\rangle \left\langle \text{expr}\right\rangle \left\langle \text{expr}\right\rangle \\
 & \rightarrow++\log x\left\langle \text{expr}\right\rangle \left\langle \text{expr}\right\rangle \\
 & \rightarrow++\log x\left\langle \text{var}\right\rangle \left\langle \text{expr}\right\rangle \\
 & \rightarrow++\log xx\left\langle \text{expr}\right\rangle \\
 & \rightarrow++\log xx\left\langle \text{const}\right\rangle \\
 & \rightarrow++\log xxC
\end{align*}

※$+\left(+\left(\log\left(x\right),x\right),C\right)\rightarrow\log x+x+C$

構文解析木 (図略)

正規表現では$\left\{ 0^{n}1^{n}|n\geq0\right\} $を生成できなかった。

CFGでは$E\rightarrow0E1|\varepsilon$とし、$S=E,\Sigma=\left\{ 0,1\right\} $とすることで生成できる。

\paragraph{$DFA\subsetneq CFG$ (正規言語$\subsetneq$文脈自由言語)}

任意の正規言語に対して、それを受理するDFAから$L$を生成するCFGを構成できる。

$L$を受理するDFAを$M$とする。$M$の各繊維規則$\delta\left(q_{i},a\right)=q_{j}$に対してCFGの書き換え規則$q_{i}\rightarrow aq_{j}$を作成する。$q_{j}$が受理状態の時$q_{j}\rightarrow\varepsilon$を追加する。$M$の開始状態が$q_{0}$なら、CFGの開始変数を$q_{0}$とする。

\paragraph{最左導出}

文字列の導出において、最も左の変数に書き換え規則を適用する導出を最左導出という。

CFGから生成されるすべての文字列は最左導出可能。

文字列$w$が2つの最左導出を持つ場合、その文法は曖昧であるという。

曖昧な例

\[
\left\langle \text{expr}\right\rangle \rightarrow\left\langle \text{expr}\right\rangle \div\left\langle \text{expr}\right\rangle |\left\langle \text{expr}\right\rangle \times\left\langle \text{expr}\right\rangle |\left\langle \text{expr}\right\rangle |a
\]

この時$a\div a\times a$は2つ以上の最左導出をもつ。

\paragraph{チョムスキー標準形 (CNF: Chomsxy normal form)}

全ての書き換え規則が$B,C\in V$ (但し開始記号は含まない), $a\in\Sigma$に対して
\[
A\rightarrow BC,A\rightarrow a
\]
となっている時CNFであると言う。$S\rightarrow\varepsilon$を加えても良い。
\begin{itemize}
\item $\left\langle \text{expr}\right\rangle \rightarrow a$: ◯
\item $\left\langle \text{expr}\right\rangle \rightarrow\left\langle \text{op1}\right\rangle \left\langle expr\right\rangle $:
◯
\item $\left\langle \text{op1}\right\rangle \left\langle \text{expr}\right\rangle \rightarrow\left\langle \text{op2}\right\rangle \left\langle \text{expr2}\right\rangle $:
\texttimes{}
\end{itemize}
任意の文法はCNFに変換可能である。

\paragraph{グライバッハ標準形 (GNF: Greibach \textasciitilde{})}

全ての書き換え規則が、$B$が開始記号意外の変数からなる文字列 (空列を含む), $a\in\Sigma$に対して
\[
a\rightarrow aB
\]
となっている時GNFであるという。$S\rightarrow\varepsilon$を加えてもよい。
\end{document}
