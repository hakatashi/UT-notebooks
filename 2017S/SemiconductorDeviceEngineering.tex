%% LyX 2.2.2 created this file.  For more info, see http://www.lyx.org/.
%% Do not edit unless you really know what you are doing.
\documentclass[english]{article}
\usepackage[T1]{fontenc}
\usepackage[utf8]{inputenc}
\usepackage[a5paper]{geometry}
\geometry{verbose,tmargin=2cm,bmargin=2cm,lmargin=1cm,rmargin=1cm}
\setlength{\parskip}{\smallskipamount}
\setlength{\parindent}{0pt}
\usepackage{amsmath}
\usepackage{amssymb}

\makeatletter
%%%%%%%%%%%%%%%%%%%%%%%%%%%%%% User specified LaTeX commands.
\usepackage[dvipdfmx]{hyperref}
\usepackage[dvipdfmx]{pxjahyper}

% http://tex.stackexchange.com/a/192428/116656
\AtBeginDocument{\let\origref\ref
   \renewcommand{\ref}[1]{(\origref{#1})}}

\makeatother

\usepackage{babel}
\begin{document}

\title{2017-S 半導体デバイス工学}

\author{教員: 入力: 高橋光輝}

\maketitle
\global\long\def\pd#1#2{\frac{\partial#1}{\partial#2}}
\global\long\def\d#1#2{\frac{\mathrm{d}#1}{\mathrm{d}#2}}
\global\long\def\pdd#1#2{\frac{\partial^{2}#1}{\partial#2^{2}}}
\global\long\def\dd#1#2{\frac{\mathrm{d}^{2}#1}{\mathrm{d}#2^{2}}}
\global\long\def\ddd#1#2{\frac{\mathrm{d}^{3}#1}{\mathrm{d}#2^{3}}}
\global\long\def\e{\mathrm{e}}
\global\long\def\i{\mathrm{i}}
\global\long\def\j{\mathrm{j}}
\global\long\def\grad{\operatorname{grad}}
\global\long\def\rot{\operatorname{rot}}
\global\long\def\div{\operatorname{div}}
\global\long\def\diag{\operatorname{diag}}
\global\long\def\rank{\operatorname{rank}}
\global\long\def\prob{\operatorname{Prob}}
\global\long\def\cov{\operatorname{Cov}}
\global\long\def\when#1{\left.#1\right|}
\global\long\def\laplace#1{\mathcal{L}\left[#1\right]}


\section*{第1回}

\paragraph{参考書}
\begin{itemize}
\item 「半導体デバイス入門」柴田直
\item 「半導体デバイス」S.M.ジィー 産業図書
\end{itemize}

\section{半導体}

\paragraph{結晶中の電子}

de Broglie波長 $\lambda=\frac{h}{p}$ ($p$: 運動量)

有効質量$m^{*}$の自由電子

\paragraph{バンド構造}

図半工1-2

図半工1-3

図半工1-4

\paragraph{フェルミ構造}

図半工1-5

\[
f\left(E\right)=\frac{1}{1+\exp\left(\frac{E-E_{F}}{kT}\right)}
\]


\paragraph{真性半導体}

図半工1-6

$E_{i}$: 申請フェルミ準位$\sim\frac{E_{c}-E_{v}}{2}$

$n=p=n_{i}$: 真性キャリア濃度

$S_{i}:$$n_{i}=10^{10}\mathrm{cm^{-3}}$ (室温)

\paragraph{N型半導体}

Siの場合: P, Asを添加

図半工1-7

P, As: ドナー

$N_{D}$: $10^{15}\sim10^{20}\mathrm{cm^{-3}}$

$n\sim N_{D}$

\paragraph{P型半導体}

Bをdoping

図半工1-1

\paragraph{質量作用の法則}

\[
n\cdot p=n_{i}^{2}
\]

N型 (ex. Si(R.T.))

\begin{align*}
n & =N_{D}=10^{20}\mathrm{cm^{-1}}\quad\left(\text{多数キャリア}\right)\\
p & =\frac{n_{i}^{2}}{N_{D}}=1\mathrm{cm^{-3}}\quad\left(\text{少数キャリア}\right)
\end{align*}


\paragraph{キャリア濃度}

\begin{align*}
n & =N_{c}\exp\left(\frac{E_{F}-E_{c}}{kT}\right)=n_{i}\exp\left(\frac{E_{p}-E_{i}}{kT}\right)\\
 & =2\left(\frac{m_{e}kT}{2\pi\hbar^{2}}\right)^{\frac{3}{2}}\quad\left(\text{実効状態密度}\right)\\
p & =N_{V}\exp\left(\frac{E_{v}-E_{F}}{kT}\right)=n_{i}\exp\left(\frac{E_{i}-E_{F}}{kT}\right)
\end{align*}

\begin{align*}
E_{F}-E_{i} & =kT\ln\left(\frac{n}{n_{i}}\right)\\
E_{F}-E_{i} & =-kT\ln\left(\frac{p}{n_{i}}\right)
\end{align*}


\section{半導体中の電気伝導}

電流の媒体: 電子、ホール

電流の機構: ドリフト-拡散

\begin{align*}
J_{n} & =q\mu_{n}n\vec{\varepsilon}+qD_{n}\d nx\\
J_{p} & =q\mu_{p}p\vec{\varepsilon}-qD_{p}\d px
\end{align*}

\[
\mu=\frac{q\tau}{m}
\]

$\tau$: 飽和時間

アインシュタインの関係式

\[
D=\left(\frac{kT}{q}\right)\mu
\]


\subsection{キャリアの生成・再結合}

図半工1-8

\subsection{連続の式}

\begin{align*}
\pd nt & =\frac{1}{q}\pd{J_{n}}x+\left(g-r\right)\\
\pd pt & =-\frac{1}{q}\pd{J_{p}}x-\left(g-r\right)
\end{align*}

\begin{align*}
\pd nt & =\mu_{n}\pd{\left(n\vec{\varepsilon}\right)}x+D_{n}\pdd nx+\left(g-r\right)\\
\pd pt & =-\mu_{p}\pd{\left(p\vec{\varepsilon}\right)}x+D_{p}\pdd px+\left(g-r\right)
\end{align*}

電界$\vec{\varepsilon}=-\pd{}x\phi$ ($\phi$: 電位(静電ポテンシャル))、$\pdd{}x\phi=-\frac{\rho}{\varepsilon}$($\rho$:
電荷密度、$\varepsilon$: 誘電率)

\[
\rho=q\left(p-n+N_{D}-N_{A}\right)
\]


\subsection{半導体中の静電ポテンシャル}

図半工1-9

図半工1-10

\[
\phi=-\frac{E_{i}}{q}
\]


\subsection{熱平衡時のフェルミ準位}

\[
\d{E_{F}}x=0
\]


\subsection{擬フェルミ準位}

\begin{align*}
n & =n_{i}\exp\left(\frac{E_{Fn}-E_{i}}{kT}\right)\\
p & =n_{i}\exp\left(\frac{E_{i}-E_{Fp}}{kT}\right)
\end{align*}

\begin{align*}
J_{n} & =\mu_{n}n\d{E_{Fn}}x\\
J_{p} & =\mu_{n}p\d{E_{Fp}}x
\end{align*}


\section*{第3回}

図半工3-1

\[
\frac{n'}{n}=\e^{-\frac{q\left(V_{bi}-V\right)}{kT}}
\]

\begin{align*}
n_{p0} & =n_{n0}\exp\left(-\frac{qV_{bi}}{kT}\right)\\
p_{n0} & =p_{p0}\exp\left(-\frac{qV_{bi}}{kT}\right)
\end{align*}

\begin{align*}
n_{p}\left(l_{p}\right) & =n_{n}\left(-l_{n}\right)\exp\left(-\frac{q\left(V_{bi}-V\right)}{kT}\right)\\
p_{n}\left(-l_{n}\right) & =p_{p}\left(l_{p}\right)\exp\left(-\frac{q\left(V_{bi}-V\right)}{kT}\right)
\end{align*}

$\text{注入された少数キャリア}\ll\text{多数キャリア}$なので、
\begin{align*}
n_{p}\left(l_{p}\right) & =n_{n0}\exp\left\{ -\frac{q\left(V_{bi}-V\right)}{kT}\right\} =n_{p0}\exp\left\{ \frac{qV}{kT}\right\} \\
p_{n}\left(-l_{n}\right) & =P_{n0}\exp\left\{ \frac{qV}{kT}\right\} 
\end{align*}

図半工3-2

連続の式
\[
\pd{n_{p}}t=\mu_{n}\pd{\left(n\vec{\varepsilon}\right)}x+D_{n}\pdd{n_{p}}x+\left(g-r\right)
\]

中世領域では$\vec{\varepsilon}=0$を仮定

定常状態
\[
\pdd{n_{p}}x+\frac{g-r}{D_{n}}=0
\]

図半工3-3

\[
U=r-g=\frac{n_{p}-n_{p0}}{\tau_{n}}
\]

\[
\therefore\pdd{n_{p}}x-\frac{n_{p}-n_{p0}}{\tau_{n}}=0
\]

$x=l_{p}$で$n_{p}=n_{p0}\exp\left\{ +\frac{qV}{kT}\right\} $

$x=\infty$で$n_{p}=n_{p0}$

拡張長$L_{n}=\sqrt{D_{n}\tau_{n}}$

\begin{align*}
n_{p}-n_{p0} & =\left[n_{p0}\e^{\frac{qV}{kT}}-n_{p0}\right]\e^{-\frac{x-l_{p}}{L_{n}}}\\
p_{n}-p_{n0} & =\left[p_{n0}\e^{\frac{qV}{kT}}-p_{n0}\right]\e^{\frac{x+l_{n}}{L_{p}}}
\end{align*}

\begin{align*}
J_{n} & =qD_{n}\pd{n_{p}}x=-\frac{qD_{n}n_{p0}}{L_{n}}\left(\e^{\frac{qV}{kT}}-1\right)\e^{-\frac{x-l_{p}}{L_{n}}}\\
J_{p} & =-\frac{qD_{p}p_{n0}}{L_{p}}\left(\e^{\frac{qV}{kT}}-1\right)\e^{\frac{x+l_{n}}{L_{p}}}\\
J & =-J_{p}\left(-l_{n}\right)-J_{n}\left(l_{p}\right)\\
 & =J_{s}\left\{ \e^{\frac{qV}{kT}}-1\right\} \\
J_{s} & =\frac{qD_{n}n_{p0}}{L_{n}}+\frac{qD_{p}p_{n0}}{L_{p}}
\end{align*}

$J_{s}$: 飽和電流密度

$n_{p}\left(l_{p}\right)=n_{p0}\exp\left\{ \frac{qV}{kT}\right\} \rightarrow\sim0,V<0$の時

室温$\frac{kT}{q}\sim26\mathrm{mV}$

ex. $V=-0.2V$
\[
\exp\left\{ \frac{qV}{kT}\right\} \sim5\times10^{4}
\]

図半工3-4

\begin{align*}
J_{n}\left(l_{p}\right) & =\frac{qD_{n}n_{p0}}{L_{n}}\\
J_{p}\left(-l_{n}\right) & =\frac{qD_{p}p_{n0}}{L_{p}}
\end{align*}

\[
\therefore J=-J_{s}
\]

\[
J=J_{s}\left(\e^{\frac{qV}{kT}}-1\right)
\]

図半工3-5

図半工3-6

\section{?}

\subsection{?}

\subsection{?}

\subsection{擬フェルミ準位}

図半工3-7

図半工3-8

図半工3-9

図半工3-10

\subsection{接合容量}

図半工3-11

図半工3-12

平行平板コンデンサ

接合容量(単位面積)

\[
C=\d Q{V_{R}}=\frac{\varepsilon_{s}}{W}
\]

$W$: $V_{R}$に依存

\subsection{実際のPN接合ダイオード}

図半工3-13
\end{document}
