%% LyX 2.2.2 created this file.  For more info, see http://www.lyx.org/.
%% Do not edit unless you really know what you are doing.
\documentclass[english]{article}
\usepackage[T1]{fontenc}
\usepackage[utf8]{inputenc}
\usepackage[a5paper]{geometry}
\geometry{verbose,tmargin=2cm,bmargin=2cm,lmargin=1cm,rmargin=1cm}
\setlength{\parskip}{\smallskipamount}
\setlength{\parindent}{0pt}
\usepackage{textcomp}
\usepackage{amsmath}

\makeatletter

%%%%%%%%%%%%%%%%%%%%%%%%%%%%%% LyX specific LaTeX commands.
%% Because html converters don't know tabularnewline
\providecommand{\tabularnewline}{\\}

%%%%%%%%%%%%%%%%%%%%%%%%%%%%%% User specified LaTeX commands.
\usepackage[dvipdfmx]{hyperref}
\usepackage[dvipdfmx]{pxjahyper}

% http://tex.stackexchange.com/a/192428/116656
\AtBeginDocument{\let\origref\ref
   \renewcommand{\ref}[1]{(\origref{#1})}}

\makeatother

\usepackage{babel}
\begin{document}

\title{2017-S 電気回路理論第二}

\author{教員: 山﨑俊彦 入力: 高橋光輝}

\maketitle
\global\long\def\pd#1#2{\frac{\partial#1}{\partial#2}}
\global\long\def\d#1#2{\frac{\mathrm{d}#1}{\mathrm{d}#2}}
\global\long\def\pdd#1#2{\frac{\partial^{2}#1}{\partial#2^{2}}}
\global\long\def\dd#1#2{\frac{\mathrm{d}^{2}#1}{\mathrm{d}#2^{2}}}
\global\long\def\ddd#1#2{\frac{\mathrm{d}^{3}#1}{\mathrm{d}#2^{3}}}
\global\long\def\e{\mathrm{e}}
\global\long\def\i{\mathrm{i}}
\global\long\def\j{\mathrm{j}}
\global\long\def\grad{\operatorname{grad}}
\global\long\def\rot{\operatorname{rot}}
\global\long\def\div{\operatorname{div}}
\global\long\def\diag{\operatorname{diag}}
\global\long\def\rank{\operatorname{rank}}
\global\long\def\prob{\operatorname{Prob}}
\global\long\def\cov{\operatorname{Cov}}
\global\long\def\when#1{\left.#1\right|}
\global\long\def\laplace#1{\mathcal{L}\left[#1\right]}
\global\long\def\invlaplace#1{\mathcal{L}^{-1}\left[#1\right]}


\section*{第1回}

\section{回路網の性質}

\subsection{回路網関数}

S(ラプラス変換後)平面で回路を記述して解析

\subsubsection{回路網関数とは}

線形回路において、初期条件が0ならば
\[
G\left(s\right)=H\left(s\right)\cdot F\left(s\right)
\]

$G\left(s\right)$: 出力

$H\left(s\right)$: 回路網関数

$F\left(s\right)$: 出力

一般的には、$G\left(s\right)=H\left(F\left(s\right)\right)$

\paragraph{例1: 1端子対回路(2端子回路)}

図電回1-1

\[
V\left(s\right)=Z\left(s\right)\cdot I\left(s\right)
\]

$Z\left(s\right)$: インピーダンス回路

\[
I\left(s\right)=Y\left(s\right)\cdot V\left(s\right)
\]

$Y\left(s\right)$: アドミッタンス関数

まとめてイミタンス関数と呼ぶ。

\paragraph{例2: 4端子回路}

図電回1-2

\[
\left[\begin{array}{c}
V_{1}\\
V_{2}
\end{array}\right]=\left[\begin{array}{cc}
Z_{11} & Z_{12}\\
Z_{21} & Z_{22}
\end{array}\right]\left[\begin{array}{c}
I_{1}\\
I_{2}
\end{array}\right]
\]

$Z_{11},Z_{21}$:駆動点関数

$Z_{12},Z_{22}$: 伝達関数

\subsubsection{集中定数回路の回路網関数}

\[
H\left(s\right)=\frac{P\left(s\right)}{Q\left(s\right)}=\frac{b_{n}S^{n}+b_{n}+S^{n-1}+\cdots+b_{0}}{a_{m}S^{m}+a_{m-1}S^{m-1}+\cdots+a_{0}}
\]
と書ける←$S$の実係数有理関数

\paragraph{例}

図電回1-3

\[
H\left(s\right)=\frac{V_{out}\left(s\right)}{V_{in}\left(s\right)}=\frac{1}{sCR+1}
\]


\subsubsection{極と零点}

\[
H\left(s\right)=\frac{P\left(s\right)}{Q\left(s\right)}=H\frac{\left(S-S_{o1}\right)\cdots\left(S-S_{on}\right)}{\left(S-S_{p1}\right)\cdots\left(S-S_{pn}\right)}
\]

$H=\frac{b_{m}}{a_{n}}$: scale factor (尺度函数)

零点: 分子を0にするような$S=S_{o1},S_{o2},\cdots S_{on}$

極: 分母を0にするような$S=S_{p1},S_{p2},\cdots,S_{pm}$

$r$重根→$r$位の極

解は実数or共役複素数の形に限る

\paragraph{例}

図電回1-4

\paragraph{例}

図電回1-5

\paragraph{無限遠点$S\rightarrow\infty$の扱い}

$n>m$: $H\left(s\right)=HS^{n-m}$…$n-m$位の零点

$n<m$: $H\left(s\right)=H\frac{1}{S^{m}-n}$…$m-n$位の極

$n=m$: $H\left(s\right)=H$

\subsubsection{回路の応答}

応答 $g\left(t\right)=\invlaplace{G\left(s\right)}=\invlaplace{H\left(s\right)F\left(s\right)}$

入力 $f\left(t\right)=\delta\left(t\right)$ (インパルス信号) → $F\left(s\right)=1$
とすると

\paragraph{例}

図電回2-1

$m$位の極(m-th order of polen

\[
\frac{Km}{\left(S-S_{pi}\right)^{m}}\xrightarrow{\mathcal{L}^{-1}}\frac{Km}{\left(m-1\right)!}t^{m-1}\e^{S_{pi}t}
\]

図電回2-2

\subsubsection{回路全体の安定性}

($t\rightarrow\infty$のときの固有応答)

\begin{tabular}{|c|c|c|c|}
\hline 
極 & 左半面内 & 虚軸上 & 右半面内\tabularnewline
\hline 
\hline 
1位 & 減衰 & 振動 & 発散\tabularnewline
\hline 
2位以上 & 減衰 & 発散 & 発散\tabularnewline
\hline 
\end{tabular}
\begin{itemize}
\item 回路網関数の曲が全て左半面内
\begin{itemize}
\item 応答は全て減衰→狭義安定
\end{itemize}
\item 左反面内+虚軸上(1位)
\begin{itemize}
\item 応答は持続→広義安定
\end{itemize}
\item 一つでも右半面内にある or 虚軸上で2位以上→不安定
\end{itemize}
(注)広義安定な回路でも、外部入力の虚軸上に1位の極があると合わせて2位→共振、発散

\subsection{回路網の安定性}

\subsubsection{安定な回路の駆動点イミタンス関数 ($Z_{11},Z_{22},Y_{11},Y_{22}$)}

広義安定の条件
\begin{enumerate}
\item 極と零点が右反面にない
\begin{itemize}
\item 零点→電圧と電流を入れ替えても成り立つから
\end{itemize}
\item 虚軸上の極、零点は高々1位
\item 分母と分子の次数さは高々1次
\begin{itemize}
\item $S\rightarrow\infty\left(\omega\rightarrow\infty\right)$とすると、回路はL,C,Rのいずれかと等価
\end{itemize}
\end{enumerate}

\subsubsection{安定な回路の伝達関数(transfer function) ($Z_{12},Z_{21},Y_{12},Y_{21}$)}

広義安定
\begin{enumerate}
\item 極は右半面内にない
\item 虚軸上の極は高々1位
\item $H\left(s\right)=\frac{n\text{次}}{m\text{次}}$としたとき、$n\leq m+1$
\end{enumerate}
逆数に対する制約がないので、駆動点イミタンス関数よりもゆるい。

\subsection{回路網関数の周波数特性}

\subsubsection{極・零点と周波数特性の関係}

\[
H\left(s\right)=H\frac{\left(S-S_{o1}\right)\cdots\left(S-S_{om}\right)}{\left(S-S_{p1}\right)\cdots\left(S-S_{pn}\right)}
\]
とすると、

\paragraph{振幅特性}

\[
H\left(\j\omega\right)=\left|H\right|\frac{d_{o1}\cdots d_{om}}{d_{p1}\cdots d_{pn}}
\]

$d_{om}$: 虚軸上$S\left(\j\omega\right)$と$S_{om}$との距離

図電回2-3

\[
\log\left|H\left(\j\omega\right)\right|=\log\left|H\right|+\sum_{i}\log d_{oi}-\sum_{j}\log d_{pj}
\]


\subsubsection{1次と2次の伝達関数}

(1) 1次伝達関数 $H\left(s\right)=\frac{b_{1}S+b_{0}}{S+a_{0}}$

①LP型

\[
H\left(s\right)=\frac{b_{0}}{S+a_{0}}
\]

図電回2-4

図電回2-5

②HP型

図電回2-6

図電回2-7

③AP型

図電回2-8

図電回2-9

図電回2-10
\end{document}
