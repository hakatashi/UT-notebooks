%% LyX 2.2.2 created this file.  For more info, see http://www.lyx.org/.
%% Do not edit unless you really know what you are doing.
\documentclass[english]{article}
\usepackage[T1]{fontenc}
\usepackage[utf8]{inputenc}
\usepackage[a5paper]{geometry}
\geometry{verbose,tmargin=2cm,bmargin=2cm,lmargin=1cm,rmargin=1cm}
\setlength{\parskip}{\smallskipamount}
\setlength{\parindent}{0pt}
\usepackage{textcomp}

\makeatletter
%%%%%%%%%%%%%%%%%%%%%%%%%%%%%% User specified LaTeX commands.
\usepackage[dvipdfmx]{hyperref}
\usepackage[dvipdfmx]{pxjahyper}

% http://tex.stackexchange.com/a/192428/116656
\AtBeginDocument{\let\origref\ref
   \renewcommand{\ref}[1]{(\origref{#1})}}

\makeatother

\usepackage{babel}
\begin{document}

\title{2017-S 電気機器}

\author{教員: 入力: 高橋光輝}

\maketitle
\global\long\def\pd#1#2{\frac{\partial#1}{\partial#2}}
\global\long\def\d#1#2{\frac{\mathrm{d}#1}{\mathrm{d}#2}}
\global\long\def\pdd#1#2{\frac{\partial^{2}#1}{\partial#2^{2}}}
\global\long\def\dd#1#2{\frac{\mathrm{d}^{2}#1}{\mathrm{d}#2^{2}}}
\global\long\def\ddd#1#2{\frac{\mathrm{d}^{3}#1}{\mathrm{d}#2^{3}}}
\global\long\def\e{\mathrm{e}}
\global\long\def\i{\mathrm{i}}
\global\long\def\j{\mathrm{j}}
\global\long\def\grad{\operatorname{grad}}
\global\long\def\rot{\operatorname{rot}}
\global\long\def\div{\operatorname{div}}
\global\long\def\diag{\operatorname{diag}}
\global\long\def\rank{\operatorname{rank}}
\global\long\def\prob{\operatorname{Prob}}
\global\long\def\cov{\operatorname{Cov}}
\global\long\def\when#1{\left.#1\right|}
\global\long\def\laplace#1{\mathcal{L}\left[#1\right]}


\section*{第2回}

工作機械、ロボット、運動制御、交通、OA機器

1873 motorとgeneratorは同じもの

1879 電気鉄道

1980年代 Power electronicsの実用化→交流可変速駆動

\paragraph{VVVF}

Variable Voltage Varable Frequency AC drive

(古典)力学→

電磁気家化学→

電気回路 交流回路理論(簡単な) 複素数

電気機器

→パワーエレクトロニクス (半導体電力変換)→モーションコントロール

発変電工学→電力系統工学 (Power System)

\paragraph{電気機器の種類}
\begin{itemize}
\item 静止期(stomd still machine)
\begin{itemize}
\item 変圧器 (Transformer)
\item 変換器 (Powerelectronic Cenverter)
\end{itemize}
\item 回転機 (Rotary Machine, Linear machine)
\begin{itemize}
\item DC machine 直流機
\item AC machine 交流機
\begin{itemize}
\item 誘導機 Induction Machine, Asynchronous Machine
\item 同期機 Synchronous Machine
\end{itemize}
\end{itemize}
\end{itemize}

\paragraph{物理の基礎知識の復習}

SI単位 m(長さ), kg(質量), sec(時間), A(電流)
\end{document}
