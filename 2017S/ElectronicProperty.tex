%% LyX 2.2.2 created this file.  For more info, see http://www.lyx.org/.
%% Do not edit unless you really know what you are doing.
\documentclass[english]{article}
\usepackage[T1]{fontenc}
\usepackage[utf8]{inputenc}
\usepackage[a5paper]{geometry}
\geometry{verbose,tmargin=2cm,bmargin=2cm,lmargin=1cm,rmargin=1cm}
\setlength{\parskip}{\smallskipamount}
\setlength{\parindent}{0pt}
\usepackage{textcomp}
\usepackage{amsmath}
\usepackage{amssymb}

\makeatletter
%%%%%%%%%%%%%%%%%%%%%%%%%%%%%% User specified LaTeX commands.
\usepackage[dvipdfmx]{hyperref}
\usepackage[dvipdfmx]{pxjahyper}

% http://tex.stackexchange.com/a/192428/116656
\AtBeginDocument{\let\origref\ref
   \renewcommand{\ref}[1]{(\origref{#1})}}

\makeatother

\usepackage{babel}
\begin{document}

\title{2017-S 電子物性基礎}

\author{教員: 入力: 高橋光輝}

\maketitle
\global\long\def\pd#1#2{\frac{\partial#1}{\partial#2}}
\global\long\def\d#1#2{\frac{\mathrm{d}#1}{\mathrm{d}#2}}
\global\long\def\pdd#1#2{\frac{\partial^{2}#1}{\partial#2^{2}}}
\global\long\def\dd#1#2{\frac{\mathrm{d}^{2}#1}{\mathrm{d}#2^{2}}}
\global\long\def\ddd#1#2{\frac{\mathrm{d}^{3}#1}{\mathrm{d}#2^{3}}}
\global\long\def\e{\mathrm{e}}
\global\long\def\i{\mathrm{i}}
\global\long\def\j{\mathrm{j}}
\global\long\def\grad{\operatorname{grad}}
\global\long\def\rot{\operatorname{rot}}
\global\long\def\div{\operatorname{div}}
\global\long\def\diag{\operatorname{diag}}
\global\long\def\rank{\operatorname{rank}}
\global\long\def\prob{\operatorname{Prob}}
\global\long\def\cov{\operatorname{Cov}}
\global\long\def\when#1{\left.#1\right|}
\global\long\def\laplace#1{\mathcal{L}\left[#1\right]}


\section*{第1回}

\paragraph{結晶のポイント}
\begin{enumerate}
\item 結晶構造
\begin{enumerate}
\item 周期的格子配列
\item 周期ポテンシャル
\end{enumerate}
→物性決まる
\item 結晶構造を形成する結合力
\begin{enumerate}
\item イオン結合→イオン結晶→決定因子: 電荷、半径比
\item 共有結合→共有性結晶→軌道の方位
\item 金属結合→金属結晶→玉のパッキング
\item ファンデルワールス結合→(分子性結晶)→双極子、結合様態
\item 水素結合
\end{enumerate}
\end{enumerate}

\paragraph{原子核の電子へ量子力学を導入}
\begin{enumerate}
\item 中心力力場の電子
\end{enumerate}
水素原子: 原子核+電子1つ 2体問題、原子核+電子2つ以上 多体問題

→電子近似

\[
\left\{ -\frac{\hbar^{2}}{2m}\left(\pdd{}x+\pdd{}y+\pdd{}z\right)+V\left(r\right)\right\} \psi\left(r\right)=E\psi\left(r\right)
\]

直交座標+極座標

\[
\pd{}x=\pd rx\pd{}r+\pd{\theta}x\pd{}{\theta}+\pd{\phi}x\pd{}{\phi}
\]

\[
r=\sqrt{x^{2}+y^{2}+z^{2}},\theta=\tan^{-1}\frac{\sqrt{x^{2}+y^{2}}}{z},\phi=\tan^{-1}\frac{y}{x}
\]

→(3.5)式

\[
V\left(r\right)=-\frac{1}{4\pi\varepsilon_{0}}\frac{Q^{2}}{r}
\]

原子核は固定。断熱近似

\[
\Psi\left(r,\theta,\phi\right)=R\left(r\right)\cdot Y\left(\theta,\phi\right)
\]

$R\left(r\right)$: 動径関数

$Y\left(\theta,\phi\right)$: 角関数

\section*{第2回}

:innocent:

\section*{第3回}

\paragraph{前回}
\begin{enumerate}
\item 結晶構造の分類
\begin{itemize}
\item 晶系: 7
\begin{itemize}
\item 長さ、確度
\end{itemize}
\item 空間格子(ブラーべ格子): 14
\begin{itemize}
\item 対称操作
\end{itemize}
\item 点群: 32
\begin{itemize}
\item 回転、鏡映、反転
\end{itemize}
\item 空間群: 230
\begin{itemize}
\item 点群+並進操作
\end{itemize}
\item 結晶面、方向の表現
\end{itemize}
\item 実格子(直接)と逆格子(フーリエ空間)
\item 逆格子により、
\begin{itemize}
\item 面間距離→点-点距離
\item Bra的→Ewal球
\end{itemize}
\end{enumerate}
物性を決める因子: 秩序因子

\paragraph{誘電特性}
\begin{itemize}
\item 誘電体とは (巨視的、微視的)
\item 誘電率とは (電気的、物理的)
\item 誘電率の起源
\end{itemize}
スケーリング則の限界

\[
C=\frac{\varepsilon_{0}\varepsilon_{n}\uparrow S\downarrow}{l\downarrow}
\]

\begin{itemize}
\item $\varepsilon_{r}$の値
\begin{itemize}
\item SiO2: 3.9
\end{itemize}
\end{itemize}
1. 誘電分極

図電性3-1

図電性3-2

\[
E=\frac{\sigma}{\varepsilon_{0}}
\]

\[
E=\frac{V}{d}
\]

真電荷$\sigma=\sigma'-P$

\[
E=\frac{\sigma'-P}{\varepsilon_{0}}
\]

ここで$\frac{\sigma'}{\sigma}=\varepsilon_{r}$とする。$P=\left(\varepsilon_{r}-1\right)\varepsilon_{0}E$

$D$: 電束密度

$E$: 電界

$P$: 分極

\begin{align*}
D & =\varepsilon_{0}E+P\\
 & =\varepsilon_{0}\varepsilon_{r}E\rightarrow\varepsilon_{r}\propto\d DE\\
 & =\varepsilon_{0}\left(1+\chi\right)E
\end{align*}

$\varepsilon_{r}$: 比誘電率

$\varepsilon_{0}$: 真空の誘電率 $8.85\times10^{-12}\mathrm{F/m}$

$\chi$: 非電気感受率

\paragraph{$\varepsilon_{r}$の物理的意味}

単位電荷のもとで、単位体積中に蓄えられるエネルギーの大きさを示す。

\paragraph{内部電解(局所)}

Lorentsの級の考え方

\[
E_{i}+E_{1}+E_{2}
\]

$E$: 外部からの電が

$E_{1}$: 玉の外部にある物質がAにつくる電解

$E_{2}$: 玉の内部にある物質がAにつくる電解 等方性より$E_{2}=0$

$\mathrm{d}S$上にある電荷
\begin{align*}
\mathrm{d}\sigma & =-P\cos\theta\mathrm{d}S\\
 & =-P\cos\theta\cdot2\pi r^{2}\sin\theta\cdot\mathrm{d}\theta
\end{align*}

中心の単位電荷に与える力
\[
\mathrm{d}E=\frac{\mathrm{d}\sigma}{4\pi\varepsilon_{0}r^{2}}=\frac{P\cos\theta\sin\theta\mathrm{d}\theta}{2\varepsilon_{0}}
\]

\[
E_{i}=\int_{0}^{\pi}\mathrm{d}E\cos\theta=\int_{0}^{\pi}\frac{P\cos^{2}\theta\sin\theta}{2\varepsilon_{0}}\mathrm{d}\theta=\frac{P}{3\varepsilon_{0}}
\]

\[
\Rightarrow E_{i}=E+\frac{P}{3\varepsilon_{0}}
\]

ローレンツの内部電界

\paragraph{巨視的-微視点}

\[
P=\varepsilon_{0}\left(\varepsilon_{r}-1\right)E
\]

\[
P=N\cdot\alpha\cdot E_{i}
\]

$\alpha$: 分極率

ローレンツ電解 $E_{r}=E+\frac{P}{3\varepsilon_{0}}$

\[
\Rightarrow\frac{\varepsilon_{r}-1}{\varepsilon_{r}+2}=\frac{\alpha N}{3\varepsilon_{0}}
\]


\paragraph{4. 誘電性発現の起源}
\begin{enumerate}
\item 電子分極 $P_{E}$
\item イオン電極 $P_{I}$
\item 双極子配向分極 $P_{O}$
\end{enumerate}
\[
P=P_{E}+P_{I}+P_{O}
\]


\paragraph{5. 誘電分散}
\begin{itemize}
\item 緩和分散 (配向)
\item 共鳴分散 (イオン、電子)
\end{itemize}
\begin{align*}
\varepsilon_{r}^{*}-1 & =\frac{\varepsilon_{r0}-1}{1+\j\omega\tau}\\
 & =\left(\varepsilon_{r0}-1\right)\left\{ \frac{1}{1+\omega^{2}\tau^{2}}-\i\frac{\omega\tau}{1+\omega^{2}\tau^{2}}\right\} 
\end{align*}

\[
\varepsilon_{r}^{*}=\varepsilon_{r0}'-\i\varepsilon_{r0}''
\]

\[
\varepsilon_{r0}'\left(\omega\right)=1+\frac{\varepsilon_{r0}-1}{1+\omega^{2}\tau^{2}}
\]

\[
\varepsilon_{r0}''=\left(\varepsilon_{r0}-1\right)\frac{\omega\tau}{1+\omega^{2}\tau^{2}}
\]

図電性3-3

\[
\left(\varepsilon'-\frac{\varepsilon_{r0}+1}{2}\right)^{2}+\varepsilon''^{2}=\frac{\left(\varepsilon_{r0}-1\right)^{2}}{8}
\]

図電性3-4

\paragraph{共鳴分散}

粒子$m$

振動電解により単振動

運動方程式: 1次元調和振動子

\[
\dd Xt+\gamma\d Xt+\omega_{0}^{2}X=\frac{\varepsilon}{m}E_{0}\e^{\i\omega t}
\]

$\gamma$: 制動項

$\omega_{0}$: 共鳴周波数

\begin{align*}
\varepsilon_{r}' & =1+\frac{nq^{2}}{\varepsilon_{0}m}\frac{\omega_{0}^{2}-\omega^{2}}{\left(\omega_{0}^{2}-\omega^{2}\right)^{2}+\gamma^{2}\omega^{2}}\\
\varepsilon_{r}'' & =\frac{nq^{2}}{\varepsilon_{0}m}\frac{\gamma\omega}{\left(\omega_{0}^{2}+\omega^{2}\right)^{2}+\gamma^{2}\omega^{2}}
\end{align*}

静電界$\omega=0$
\end{document}
