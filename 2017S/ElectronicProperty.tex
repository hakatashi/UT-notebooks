%% LyX 2.2.2 created this file.  For more info, see http://www.lyx.org/.
%% Do not edit unless you really know what you are doing.
\documentclass[english]{article}
\usepackage[T1]{fontenc}
\usepackage[utf8]{inputenc}
\usepackage[a5paper]{geometry}
\geometry{verbose,tmargin=2cm,bmargin=2cm,lmargin=1cm,rmargin=1cm}
\setlength{\parskip}{\smallskipamount}
\setlength{\parindent}{0pt}
\usepackage{textcomp}
\usepackage{amssymb}

\makeatletter
%%%%%%%%%%%%%%%%%%%%%%%%%%%%%% User specified LaTeX commands.
\usepackage[dvipdfmx]{hyperref}
\usepackage[dvipdfmx]{pxjahyper}

% http://tex.stackexchange.com/a/192428/116656
\AtBeginDocument{\let\origref\ref
   \renewcommand{\ref}[1]{(\origref{#1})}}

\makeatother

\usepackage{babel}
\begin{document}

\title{2017-S 電子物性基礎}

\author{教員: 入力: 高橋光輝}

\maketitle
\global\long\def\pd#1#2{\frac{\partial#1}{\partial#2}}
\global\long\def\d#1#2{\frac{\mathrm{d}#1}{\mathrm{d}#2}}
\global\long\def\pdd#1#2{\frac{\partial^{2}#1}{\partial#2^{2}}}
\global\long\def\dd#1#2{\frac{\mathrm{d}^{2}#1}{\mathrm{d}#2^{2}}}
\global\long\def\ddd#1#2{\frac{\mathrm{d}^{3}#1}{\mathrm{d}#2^{3}}}
\global\long\def\e{\mathrm{e}}
\global\long\def\i{\mathrm{i}}
\global\long\def\j{\mathrm{j}}
\global\long\def\grad{\operatorname{grad}}
\global\long\def\rot{\operatorname{rot}}
\global\long\def\div{\operatorname{div}}
\global\long\def\diag{\operatorname{diag}}
\global\long\def\rank{\operatorname{rank}}
\global\long\def\prob{\operatorname{Prob}}
\global\long\def\cov{\operatorname{Cov}}
\global\long\def\when#1{\left.#1\right|}
\global\long\def\laplace#1{\mathcal{L}\left[#1\right]}


\section*{第1回}

\paragraph{結晶のポイント}
\begin{enumerate}
\item 結晶構造
\begin{enumerate}
\item 周期的格子配列
\item 周期ポテンシャル
\end{enumerate}
→物性決まる
\item 結晶構造を形成する結合力
\begin{enumerate}
\item イオン結合→イオン結晶→決定因子: 電荷、半径比
\item 共有結合→共有性結晶→軌道の方位
\item 金属結合→金属結晶→玉のパッキング
\item ファンデルワールス結合→(分子性結晶)→双極子、結合様態
\item 水素結合
\end{enumerate}
\end{enumerate}

\paragraph{原子核の電子へ量子力学を導入}
\begin{enumerate}
\item 中心力力場の電子
\end{enumerate}
水素原子: 原子核+電子1つ 2体問題、原子核+電子2つ以上 多体問題

→電子近似

\[
\left\{ -\frac{\hbar^{2}}{2m}\left(\pdd{}x+\pdd{}y+\pdd{}z\right)+V\left(r\right)\right\} \psi\left(r\right)=E\psi\left(r\right)
\]

直交座標+極座標

\[
\pd{}x=\pd rx\pd{}r+\pd{\theta}x\pd{}{\theta}+\pd{\phi}x\pd{}{\phi}
\]

\[
r=\sqrt{x^{2}+y^{2}+z^{2}},\theta=\tan^{-1}\frac{\sqrt{x^{2}+y^{2}}}{z},\phi=\tan^{-1}\frac{y}{x}
\]

→(3.5)式

\[
V\left(r\right)=-\frac{1}{4\pi\varepsilon_{0}}\frac{Q^{2}}{r}
\]

原子核は固定。断熱近似

\[
\Psi\left(r,\theta,\phi\right)=R\left(r\right)\cdot Y\left(\theta,\phi\right)
\]

$R\left(r\right)$: 動径関数

$Y\left(\theta,\phi\right)$: 角関数
\end{document}
