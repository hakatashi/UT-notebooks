%% LyX 2.2.2 created this file.  For more info, see http://www.lyx.org/.
%% Do not edit unless you really know what you are doing.
\documentclass[english]{article}
\usepackage[T1]{fontenc}
\usepackage[utf8]{inputenc}
\usepackage[a5paper]{geometry}
\geometry{verbose,tmargin=2cm,bmargin=2cm,lmargin=1cm,rmargin=1cm}
\setlength{\parskip}{\smallskipamount}
\setlength{\parindent}{0pt}
\usepackage{textcomp}
\usepackage{amsmath}
\usepackage{amssymb}

\makeatletter
%%%%%%%%%%%%%%%%%%%%%%%%%%%%%% User specified LaTeX commands.
\usepackage[dvipdfmx]{hyperref}
\usepackage[dvipdfmx]{pxjahyper}

% http://tex.stackexchange.com/a/192428/116656
\AtBeginDocument{\let\origref\ref
   \renewcommand{\ref}[1]{(\origref{#1})}}

\makeatother

\usepackage{babel}
\begin{document}

\title{2017-S 制御工学}

\author{教員: 入力: 高橋光輝}

\maketitle
\global\long\def\pd#1#2{\frac{\partial#1}{\partial#2}}
\global\long\def\d#1#2{\frac{\mathrm{d}#1}{\mathrm{d}#2}}
\global\long\def\pdd#1#2{\frac{\partial^{2}#1}{\partial#2^{2}}}
\global\long\def\dd#1#2{\frac{\mathrm{d}^{2}#1}{\mathrm{d}#2^{2}}}
\global\long\def\ddd#1#2{\frac{\mathrm{d}^{3}#1}{\mathrm{d}#2^{3}}}
\global\long\def\e{\mathrm{e}}
\global\long\def\i{\mathrm{i}}
\global\long\def\j{\mathrm{j}}
\global\long\def\grad{\operatorname{grad}}
\global\long\def\rot{\operatorname{rot}}
\global\long\def\div{\operatorname{div}}
\global\long\def\diag{\operatorname{diag}}
\global\long\def\rank{\operatorname{rank}}
\global\long\def\prob{\operatorname{Prob}}
\global\long\def\cov{\operatorname{Cov}}
\global\long\def\when#1{\left.#1\right|}
\global\long\def\laplace#1{\mathcal{L}\left[#1\right]}


\section*{第1回}

図制1

図制2

\[
\begin{cases}
v_{f}=A\left(v_{0}-v\right)\\
v=Kv_{f}-r_{a}i=KA\left(v_{0}-v\right)-r_{a}i
\end{cases}
\]

\[
\therefore v=\frac{KA}{1+KA}v_{0}-\frac{r_{a}}{1+KA}i
\]
 

図制3

結果を見て原因に戻している=フィードバック

図制4

\section*{第2回}

\section{?}

\section{システムの動特性表現}
\begin{enumerate}
\item 信号伝達の状態遷移
\item 動作点周りの線形化
\item 線形系の表現
\end{enumerate}
1.~ 2タンク系

図制2-1

$Q_{0}$: 出力できる制御入力

$h_{2}$: 制御量

図制2-2

\begin{align*}
A_{1}\d{h_{1}}t & =Q_{0}-Q_{1}\\
 & =Q_{0}-k_{1}\sqrt{h_{1}}\\
A_{2}\d{h_{2}}t & =Q_{1}-Q_{2}\\
 & =k_{1}\sqrt{h_{1}}-k_{2}\sqrt{h_{2}}\left(\text{非線形システム}\right)
\end{align*}

2. ~

図制2-3

\begin{align*}
Q_{0} & =Q_{00}+\Delta Q_{0}\\
h_{1} & =h_{10}+\Delta h_{1}\\
h_{2} & =h_{20}+\Delta h_{2}
\end{align*}

\begin{align*}
A_{1}\d{h_{1}}t & =Q_{0}-k_{1}\sqrt{h_{1}}\\
 & =Q_{00}+\Delta Q_{0}-k_{1}\sqrt{h_{10}+\Delta h_{1}}\\
 & =Q_{00}+\Delta Q_{0}-k_{1}\sqrt{h_{10}}\frac{\Delta h_{1}}{2h_{10}}
\end{align*}

\[
\sqrt{h_{10}+\Delta h_{1}}=\sqrt{h_{10}}\sqrt{1+\frac{\Delta h_{1}}{h_{10}}}
\]

\begin{align*}
\sqrt{1+\varepsilon} & \doteqdot1+\frac{1}{2}\varepsilon+\left(\qquad\right)\varepsilon^{2}+\cdots\\
^{n}\sqrt{1+\varepsilon} & \doteqdot1+\frac{1}{n}\varepsilon+\left(\qquad\right)\varepsilon^{2}+\cdots
\end{align*}

平衡点なので、

\[
\d{h_{1}}t=0\rightarrow Q_{00}-k_{1}\sqrt{h_{10}}=0
\]

\[
\d{}t\Delta h_{1}=\Delta Q_{0}-\frac{k_{1}}{2\sqrt{h_{10}}}\Delta h_{1}
\]

同様にして、
\[
A_{2}\d{\Delta h_{2}}t=\frac{k_{1}}{1\sqrt{h_{10}}}\Delta h_{1}-\frac{k_{2}}{2\sqrt{h_{20}}}\Delta h_{2}
\]

図制2-4

一般に
\[
\dot{\mathbf{x}}=\mathbf{f}\left(\mathbf{x},\mathbf{u}\right)
\]

\[
\begin{cases}
\dot{\mathbf{x}_{1}}=f_{1}\left(\mathbf{x},\mathbf{u}\right)\\
\dot{\mathbf{x}_{2}}=f_{2}\left(\mathbf{x},\mathbf{u}\right)
\end{cases}
\]

\[
\begin{cases}
\mathbf{x}=\binom{x_{1}}{x_{2}}=\binom{h_{1}}{h_{2}}\\
\mathbf{u}=U_{1}=Q_{0}
\end{cases}
\]

\[
\begin{cases}
\mathbf{x}=\mathbf{x}_{0}+\Delta\mathbf{x}\\
\mathbf{u}=\mathbf{u}_{0}+\Delta\mathbf{u}
\end{cases}
\]

\[
\Delta\dot{\mathbf{x}}=A\Delta\mathbf{x}+B\Delta\mathbf{u}
\]

\begin{align*}
\rightarrow A & =\left[\pd{\mathbf{f}}{\mathbf{x}}\right]_{\mathbf{x}_{0},\mathbf{u}_{0}}=\left[\begin{array}{cc}
\pd{f_{1}}{x_{1}} & \pd{f_{1}}{x_{2}}\\
\pd{f_{2}}{x_{1}} & \pd{f_{2}}{x_{2}}
\end{array}\right]_{\mathbf{x}_{0},\mathbf{u}_{0}}\\
B & =\left[\pd{\mathbf{f}}{\mathbf{u}}\right]_{\mathbf{x}_{0},\mathbf{u}_{0}}=\left[\begin{array}{c}
\pd{f_{1}}{u_{1}}\\
\pd{f_{2}}{u_{1}}
\end{array}\right]_{\mathbf{x}_{0},\mathbf{u}_{0}}
\end{align*}

図制2-5

図中Cの設計のために、
\[
\Delta\dot{\mathbf{x}}=A\Delta\mathbf{x}+B\Delta\mathbf{u}
\]
という、線形化システムを相手にする。

3. ~

\[
\Delta Q_{0}\rightarrow A_{1}\d{\Delta h_{1}}t+\frac{k_{1}}{2\sqrt{h_{10}}}\Delta h_{1}=\Delta Q_{1}\rightarrow\Delta h_{1}
\]

\begin{enumerate}
\item 微分方程式
\item $G\left(s\right)$ 伝達関数
\item $g\left(t\right)$ インパルス応答
\item $G\left(\j\omega\right)$ 周波数応答
\end{enumerate}
\[
G\left(s\right)=\int_{0}^{\infty}g\left(t\right)\e^{-st}\mathrm{d}t
\]

\[
G\left(\j\omega\right)=\when{G\left(s\right)}_{s=\j\omega}
\]

ステップ応答

図制2-6

たたみこみ積分

図制2-7

\begin{align*}
y\left(t\right) & =\int_{0}^{t}x\left(\tau\right)g\left(t-\tau\right)\mathrm{d}\tau\\
\xrightarrow{\mathcal{L}} & Y\left(s\right)=G\left(s\right)X\left(s\right)
\end{align*}

インパルス応答

図制2-8

図制2-9

\begin{align*}
y\left(t\right) & =\sum x\left(\tau\right)\cdot g\left(t-\tau\right)\\
 & \rightarrow\int_{0}^{t}x\left(\tau\right)g\left(t-\tau\right)\mathrm{d}\tau
\end{align*}


\subsection*{2.5 }

図制2-10

\begin{align*}
y & =G_{1}x\\
z & =G_{2}y=G_{2}G_{1}x
\end{align*}

図制2-11

\[
y=\left(G_{1}+G_{2}\right)x
\]

図制2-12

$r\rightarrow y$の伝達関数

\[
\frac{y}{r}=\frac{G}{1+GH}
\]

\[
\because\left(r-Hy\right)G=y
\]

\[
rG-GHy=y
\]

\[
rG=\left(1+GH\right)y
\]

図制2-13

\subsection*{2.6 特性の計測}

(同定: identification)

図制2-14

図制2-15

図制2-16

\paragraph{例 $J$の固定}

図制2-17

\[
J\cdot\d{\omega}t=T
\]

\[
M\cdot\alpha=f
\]


\section*{第4回}

\begin{align*}
F\left(s\right) & =K_{0}\frac{\left(s-z_{1}\right)\left(s-z_{2}\right)\cdots\left(s-z_{k}\right)}{\left(s-p_{1}\right)\left(s-p_{2}\right)\cdots\left(s-p_{k}\right)}\\
 & =K_{0}\frac{\prod x_{i}}{\prod y_{i}}\e^{\j\left(\sum\phi_{i}\sum\varphi_{i}\right)}
\end{align*}

\[
\begin{cases}
s-z_{i}=x_{i}\e^{\j\phi_{i}}\\
s-p_{i}=y_{i}\e^{\j\varphi_{i}}
\end{cases}
\]

図制4-2

\paragraph{安定度}

図制4-1

安定度: どのくらい安定か、不安定化するまでの余裕
\begin{itemize}
\item $\theta_{m}$: 位相余裕 phase margin
\item $g_{m}$ ゲイン余裕 gain margin
\end{itemize}
$KP\left(s\right)H\left(s\right)$を調べて、$\frac{KP\left(s\right)}{1+KP\left(s\right)H\left(s\right)}$の安定/不安定を論じた。(all
or nothing)

あと$\theta_{m}$だけ位相が送れると不安定

あと$g_{m}$だけゲインが大きくなると不安定

図制4-3

どれも、$\left(-1,0\right)$ or $\left(0\mathrm{dB},-180^{\circ}\right)$からの距離

図制4-4

\paragraph{ロバスト制御 ($H\infty$制御の入り口)}

図制4-4

図制4-5
\begin{enumerate}
\item $r=\min\left|1+G\left(\j\omega\right)\right|$を指標に(→代わりに$\theta_{m},g_{m}$)

\[
S\left(\j\omega\right)\triangleq\frac{1}{1+G\left(\j\omega\right)}
\]

($S\left(s\right)\triangleq\frac{1}{1+G\left(s\right)}$: 感度関数)$\rightarrow\left\Vert S\left(s\right)\right\Vert _{\infty}\rightarrow\min$

$H\infty$制御の感度最適化問題
\item $\Delta$の変動に対してシステムが安定であるためには、
\begin{align*}
\left|1+G\left(\j\omega\right)\right| & >\left|\Delta\left(\j\omega\right)G\left(\j\omega\right)\right|\\
\left|\frac{G\left(\j\omega\right)}{1+G\left(\j\omega\right)}\right| & <\left|\frac{1}{\Delta\left(\j\omega\right)}\right|
\end{align*}

$\rightarrow\left\Vert T\left(s\right)\right\Vert _{\infty}<\gamma$
なるべく小さくしたい
\end{enumerate}
\begin{align*}
T\left(\j\omega\right) & =\frac{G\left(\j\omega\right)}{1+G\left(\j\omega\right)}\\
\rightarrow T\left(s\right) & =\frac{G\left(s\right)}{1+G\left(s\right)}
\end{align*}

相補感度関数

\begin{align*}
\left\Vert S\left(s\right)\right\Vert _{\infty} & \rightarrow\text{小}\\
\left\Vert T\left(s\right)\right\Vert _{\infty} & \rightarrow\text{小}
\end{align*}
but
\[
S\left(s\right)+T\left(s\right)=1
\]

\[
\left\Vert \begin{array}{c}
W_{1}\left(s\right)S\left(s\right)\\
W_{2}\left(s\right)T\left(s\right)
\end{array}\right\Vert _{\infty}<\gamma
\]

混合感度関数

$\gamma\rightarrow\text{iteration}$

\begin{align*}
\left\Vert x\right\Vert _{2} & =\sqrt[2]{\sum x_{i}^{2}}\\
\left\Vert x\right\Vert _{4} & =\sqrt[4]{\sum x_{i}^{2}}\\
\left\Vert x\right\Vert _{\infty} & =\sqrt[\infty]{\sum x_{i}^{2}}\\
 & =\max x_{i}
\end{align*}

\end{document}
