%% LyX 2.2.2 created this file.  For more info, see http://www.lyx.org/.
%% Do not edit unless you really know what you are doing.
\documentclass[english]{article}
\usepackage[T1]{fontenc}
\usepackage[utf8]{inputenc}
\usepackage[a5paper]{geometry}
\geometry{verbose,tmargin=2cm,bmargin=2cm,lmargin=1cm,rmargin=1cm}
\setlength{\parskip}{\smallskipamount}
\setlength{\parindent}{0pt}
\usepackage{textcomp}
\usepackage{amsmath}

\makeatletter
%%%%%%%%%%%%%%%%%%%%%%%%%%%%%% User specified LaTeX commands.
\usepackage[dvipdfmx]{hyperref}
\usepackage[dvipdfmx]{pxjahyper}

% http://tex.stackexchange.com/a/192428/116656
\AtBeginDocument{\let\origref\ref
   \renewcommand{\ref}[1]{(\origref{#1})}}

\makeatother

\usepackage{babel}
\begin{document}

\title{2017-S 電子回路I}

\author{教員: 山﨑俊彦 入力: 高橋光輝}

\maketitle
\global\long\def\pd#1#2{\frac{\partial#1}{\partial#2}}
\global\long\def\d#1#2{\frac{\mathrm{d}#1}{\mathrm{d}#2}}
\global\long\def\pdd#1#2{\frac{\partial^{2}#1}{\partial#2^{2}}}
\global\long\def\dd#1#2{\frac{\mathrm{d}^{2}#1}{\mathrm{d}#2^{2}}}
\global\long\def\ddd#1#2{\frac{\mathrm{d}^{3}#1}{\mathrm{d}#2^{3}}}
\global\long\def\e{\mathrm{e}}
\global\long\def\i{\mathrm{i}}
\global\long\def\j{\mathrm{j}}
\global\long\def\grad{\operatorname{grad}}
\global\long\def\rot{\operatorname{rot}}
\global\long\def\div{\operatorname{div}}
\global\long\def\diag{\operatorname{diag}}
\global\long\def\rank{\operatorname{rank}}
\global\long\def\prob{\operatorname{Prob}}
\global\long\def\cov{\operatorname{Cov}}
\global\long\def\when#1{\left.#1\right|}
\global\long\def\laplace#1{\mathcal{L}\left[#1\right]}


\section*{第1回}

\paragraph{電子回路I}

\paragraph{目標}
\begin{itemize}
\item 電子回路(アナログ回路)をトランジスタレベルで動作を理解する
\item 前提
\begin{itemize}
\item トランジスタ=MOLFETの動作
\item 簡単な数学(特にフーリエ・ラプラス変換)
\begin{itemize}
\item →非線形素子のトランジスタを線形化して理解
\end{itemize}
\end{itemize}
\end{itemize}

\paragraph{講義}
\begin{itemize}
\item 出席: 摂ります
\item 宿題: 出します
\item 評価: 期末試験のみ
\end{itemize}

\paragraph{教科書}

「MOSによる電子回路基礎」数理工学社 池田著

→サポート情報を見る

\paragraph{参考書}

「アナログCMOS集積回路」B.Razavi著 黒田訳 丸善出版

→演習編あり

メールアドレス: {[}secret{]}

ハッシュタグ: \#UTEEAnalogCircuits

\paragraph{講義の流れ}
\begin{enumerate}
\item イントロ、トランジスタの小信号特性
\item ノース接地回路
\item ゲート接地回路、ソースフォロワー
\item カスゴート回路
\item 差動増幅回路
\item カレントミラー
\item MOSトランジスタの構造と動作
\item 回路の周波数応答と安定性
\item フィードバック
\item オペアンプ
\end{enumerate}

\paragraph{ディジタルとアナログ}

回路はディジタルとアナログ
\begin{itemize}
\item ディジタル
\begin{itemize}
\item 回路規模: 大
\item 動作速度: 遅
\item 精度: 高
\end{itemize}
\item アナログ
\begin{itemize}
\item 回路規模: 小
\item 動作速度: 速
\item 精度: 低
\end{itemize}
\end{itemize}
自然界の信号は全てアナログ

信号処理はディジタルが有利

この2つを相互に変換するのがA/D, D/A変換→アンプ(増幅回路)

\paragraph{MOSトラン上位スタとバイポーラトランジスタ}
\begin{itemize}
\item 3端子素子
\item MOS: ゲート(ソース間=基盤)の電圧で、ドレイン-ソース間の電流を制御
\item バイポーラ: ベース-エミッタ間の電流で、コレクタ-エミッタ単の電流を制御
\end{itemize}

\paragraph{MOS (NMOS)}

図回I1-1

\paragraph{PMOS}

図回I1-2

\paragraph{バイポーラ}

図回I1-3

バイポーラトランジスタはコレクタとエミッタは非対称

\paragraph{NMOSとPMOS}

図回I1-4

\paragraph{MOSトランジスタの動作}

図回I1-5

\paragraph{弱反転}

$V_{GS}<V_{TH}$ 
\[
I_{D}=\left(\frac{\omega}{L}\right)I_{DO}\e^{\frac{qV_{GS}}{kT}}
\]


\paragraph{強反転}

飽和領域 $V_{TH}<V_{GS}<V_{TH}+V_{DS}$

\[
I_{D}=\mu C_{OX}\frac{\omega}{L}\frac{\left(V_{GS}-V_{TH}\right)^{2}}{2}
\]

線形領域 $V_{TH}+V_{DS}<V_{GS}$
\[
I_{D}=\mu C_{OX}\frac{\omega}{L}\left\{ \left(V_{GS}-V_{TH}\right)V_{DS}+\frac{V_{DS}^{2}}{2}\right\} 
\]

図回I1-6

\[
I_{D}'=I_{D}\left(1+\lambda V_{DS}\right)
\]

$\lambda$: チャネル長変調係数

NMOS: $\lambda_{n}=0.1$

PMOS: $\lambda_{p}=0.2$

\paragraph{バイアスと小信号解析}

少信号解析→線形近似

\[
y=f'\left(x_{0}\right)\Delta x+y_{0}
\]

\[
\Delta y=f'\left(x_{0}\right)\Delta x
\]

傾き(=ゲイン)のみを考える→小信号解析

\[
I_{D}=I_{D0}\left(V_{GS}-V_{TH}\right)^{2}
\]

\[
I_{D0}=\frac{1}{2}\mu C_{OX}\frac{\omega}{L}
\]

\begin{align*}
g_{m} & =\pd{I_{D}}{V_{GS}}=2I_{D0}\left(V_{GS}-V_{TH}\right)\\
 & \rightarrow\Delta I_{D}=g_{m}\cdot V_{GS}
\end{align*}

\[
\text{相互コンダクタンス}=\frac{2I_{D}}{V_{GS}-V_{TH}}=\sqrt{\frac{\cdots}{I_{D}}}
\]

図回I1-8

MOSトランジスタ信号等価回路

変化分は小文字で表す

\begin{align*}
\Delta V_{GS} & \rightarrow v_{gs}\rightarrow i_{d}=g_{m}v_{gs}\\
\Delta I_{D} & \rightarrow i_{d}
\end{align*}


\paragraph{チャネル長変調効果の小信号モデル}

\begin{align*}
I_{D} & =f\left(V_{DS}\right)\\
 & =\pd{I_{D}}{V_{DS}}=\frac{I}{V_{D}}
\end{align*}

チャンネル接続

図回I1-9

\[
I_{D}=f\left(V_{GS}\cdot V_{DS}\right)
\]

\[
I_{D}=\when{\pd f{V_{GS}}}_{V_{GS}}\Delta V_{GS}+\when{\pd f{V_{DS}}}\Delta V_{DS}
\]

\end{document}
