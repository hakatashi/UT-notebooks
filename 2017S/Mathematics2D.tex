%% LyX 2.2.2 created this file.  For more info, see http://www.lyx.org/.
%% Do not edit unless you really know what you are doing.
\documentclass[english]{article}
\usepackage[T1]{fontenc}
\usepackage[utf8]{inputenc}
\usepackage[a5paper]{geometry}
\geometry{verbose,tmargin=2cm,bmargin=2cm,lmargin=1cm,rmargin=1cm}
\setlength{\parskip}{\smallskipamount}
\setlength{\parindent}{0pt}
\usepackage{amsmath}

\makeatletter
%%%%%%%%%%%%%%%%%%%%%%%%%%%%%% User specified LaTeX commands.
\usepackage[dvipdfmx]{hyperref}
\usepackage[dvipdfmx]{pxjahyper}

% http://tex.stackexchange.com/a/192428/116656
\AtBeginDocument{\let\origref\ref
   \renewcommand{\ref}[1]{(\origref{#1})}}

\makeatother

\usepackage{babel}
\begin{document}

\title{2017-S 数学2D}

\author{教員: 入力: 高橋光輝}

\maketitle
\global\long\def\pd#1#2{\frac{\partial#1}{\partial#2}}
\global\long\def\d#1#2{\frac{\mathrm{d}#1}{\mathrm{d}#2}}
\global\long\def\pdd#1#2{\frac{\partial^{2}#1}{\partial#2^{2}}}
\global\long\def\dd#1#2{\frac{\mathrm{d}^{2}#1}{\mathrm{d}#2^{2}}}
\global\long\def\ddd#1#2{\frac{\mathrm{d}^{3}#1}{\mathrm{d}#2^{3}}}
\global\long\def\e{\mathrm{e}}
\global\long\def\i{\mathrm{i}}
\global\long\def\j{\mathrm{j}}
\global\long\def\grad{\operatorname{grad}}
\global\long\def\rot{\operatorname{rot}}
\global\long\def\div{\operatorname{div}}
\global\long\def\diag{\operatorname{diag}}
\global\long\def\rank{\operatorname{rank}}
\global\long\def\prob{\operatorname{Prob}}
\global\long\def\cov{\operatorname{Cov}}
\global\long\def\when#1{\left.#1\right|}
\global\long\def\laplace#1{\mathcal{L}\left[#1\right]}
\global\long\def\invlaplace#1{\mathcal{L}^{-1}\left[#1\right]}
\global\long\def\combination#1#2{_{#1}\mathrm{C}_{#2}}
\global\long\def\permutation#1#2{_{#1}\mathrm{P}_{#2}}


\section*{第2回}

\paragraph{複素関数が微分可能な条件}

\[
\lim_{\Delta z\rightarrow0}\frac{f\left(z+\Delta z\right)-f\left(z\right)}{\Delta z}
\]
が存在。

\paragraph{例)}

1. $f\left(z\right)=z^{n}$

\begin{align*}
\frac{f\left(z+\Delta z\right)-f\left(z\right)}{\Delta z} & =\frac{1}{\Delta z}\left[\sum_{m=0}^{n}z^{m}\left(\Delta z\right)^{n-m}\combination nm-z^{n}\right]\\
 & =\frac{1}{\Delta z}\left[\sum_{m=0}^{n-1}z^{m}\left(\Delta z\right)^{n-m}\combination nm\right]\\
\xrightarrow[\Delta z\rightarrow0]{}\combination m{n-1}z^{n-1} & =nz^{n-1}
\end{align*}

2. $f\left(z\right)=\overline{z}$

\[
\frac{\overline{z+\Delta z}-\overline{z}}{\Delta z}=\frac{\overline{\Delta z}}{\Delta z}=\frac{\left|\Delta z\right|\e^{-\i\theta}}{\left|\Delta z\right|\e^{\i\theta}}=\e^{-\i2\theta}
\]

極限が存在しない。

$\lim_{\Delta z\rightarrow0}\frac{f\left(z+\Delta z\right)-f\left(z\right)}{\Delta z}$が存在⇔コーシーリーマンの関係式

\[
f\left(z\right)=u\left(x,y\right)+\i v\left(x,y\right)\qquad\left(z=x+\i y\right)
\]

\[
\pd ux=\pd vy,\mathrm{and}\pd uv=-\pd vu
\]

\end{document}
