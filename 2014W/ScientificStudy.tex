%% LyX 2.2.2 created this file.  For more info, see http://www.lyx.org/.
%% Do not edit unless you really know what you are doing.
\documentclass[english]{article}
\usepackage[T1]{fontenc}
\usepackage[utf8]{inputenc}
\setlength{\parskip}{\smallskipamount}
\setlength{\parindent}{0pt}
\usepackage{babel}
\begin{document}

\title{科学技術基礎論講義ノート}

\maketitle
「科学技術基礎論」という単語には2つの意味がある。
\begin{enumerate}
\item 科学技術の基礎についての議論 (philosophy of sciense)
\item 科学技術論の基礎 (sciense and technology study)
\end{enumerate}
いまから数十年前には科学技術が世の中を良い物にするという一方的な信念があった。カントの時代以来の哲学のテーマの一つは、自然科学はなぜこんなにも正確な結論を得られるのか、自然科学と他の学問の違いは何かという探求であった。これがある時代の科学技術基礎論であった。

ところが1970年代に公害問題が発生するようになってから、科学技術の負の側面を意識した議論がされるように鳴ってきた。これが科学技術基礎論の第二の意味である。

当議論で扱う「科学技術基礎論」はこの2番目の意味を中心としつつ、1番目の意味についても適宜触れていく形となる。

したがって、当講義においては概ね以下の内容にそって講義を進めていく。
\begin{enumerate}
\item 科学方法論
\item トランスサイエンス論…科学にできることとできないことを規定しようという試み
\item 科学技術リスク論
\end{enumerate}
このうち第3の「科学技術リスク論」について学ぶ際には自著『サイエンティフィックリテラシー論』を読んで欲しい。また、授業で使用するプリントは当講義のECCの「講義用WWWサーバ」に提出されたアドレスに送付するのでWWWサーバー上のフォームから必ず登録して欲しい。

評価方法はそのHPにも記載しているが、出席3割、その他7割である。その他についてだが「最終授業時の試験」「定期試験」「レポート」のなかから投票によって最も。なお定期試験においては持ち込みは全て可とする。レポートや定期試験の論述については、似たような文章が見られた際はその中で相互に点数を分割する。レポートの分量はおおむね400字詰めで3.5~4枚程度である。

出席点の取り方は「出席課題」と呼ばれる課題を何回か提出してもらい30点とする。これはまじめに書けば必ず10点がもらえる。なお特殊な事情がある方は直接相談に来れば「特別課題」というものを提出することができる。

同時に発言点が存在し、授業中に何らかの発言をした場合、後日どういう発言したのかをメールで報告すれば一回につき3点が加算される。

以上のシステムによって得た得点に処理を施し、優が全体の33\%、平均点が75点になるように変換する。これが最終成績となる。
\end{document}
