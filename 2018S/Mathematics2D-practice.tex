%% LyX 2.3.0rc1 created this file.  For more info, see http://www.lyx.org/.
%% Do not edit unless you really know what you are doing.
\documentclass[english]{article}
\usepackage[T1]{fontenc}
\usepackage[utf8]{inputenc}
\usepackage[a5paper]{geometry}
\geometry{verbose,tmargin=2cm,bmargin=2cm,lmargin=1cm,rmargin=1cm}
\setlength{\parskip}{\smallskipamount}
\setlength{\parindent}{0pt}
\usepackage{babel}
\usepackage{amsmath}
\usepackage[unicode=true,
 bookmarks=false,
 breaklinks=false,pdfborder={0 0 1},backref=section,colorlinks=false]
 {hyperref}
\hypersetup{
 dvipdfmx}

\makeatletter
%%%%%%%%%%%%%%%%%%%%%%%%%%%%%% User specified LaTeX commands.
\usepackage[dvipdfmx]{pxjahyper}


% http://tex.stackexchange.com/a/192428/116656
\AtBeginDocument{\let\origref\ref
   \renewcommand{\ref}[1]{(\origref{#1})}}



\usepackage{babel}

\makeatother

\begin{document}

\title{2018-S 数学2D演習}

\author{教員: 入力: 高橋光輝}

\maketitle
\global\long\def\pd#1#2{\frac{\partial#1}{\partial#2}}
 \global\long\def\d#1#2{\frac{\mathrm{d}#1}{\mathrm{d}#2}}
 \global\long\def\pdd#1#2{\frac{\partial^{2}#1}{\partial#2^{2}}}
 \global\long\def\dd#1#2{\frac{\mathrm{d}^{2}#1}{\mathrm{d}#2^{2}}}
 \global\long\def\ddd#1#2{\frac{\mathrm{d}^{3}#1}{\mathrm{d}#2^{3}}}
 \global\long\def\e{\mathrm{e}}
 \global\long\def\i{\mathrm{i}}
 \global\long\def\j{\mathrm{j}}
 \global\long\def\grad{\operatorname{grad}}
 \global\long\def\rot{\operatorname{rot}}
 \global\long\def\div{\operatorname{div}}
 \global\long\def\diag{\operatorname{diag}}
 \global\long\def\rank{\operatorname{rank}}
 \global\long\def\prob{\operatorname{Prob}}
 \global\long\def\cov{\operatorname{Cov}}
 \global\long\def\when#1{\left.#1\right|}
 \global\long\def\laplace#1{\mathcal{L}\left[#1\right]}
 \global\long\def\invlaplace#1{\mathcal{L}^{-1}\left[#1\right]}
 \global\long\def\combination#1#2{_{#1}\mathrm{C}_{#2}}
 \global\long\def\permutation#1#2{_{#1}\mathrm{P}_{#2}}


\section*{第1回}

\paragraph{{[}1{]} (1)}

\begin{align*}
\exp\left(\i\alpha\right)\exp\left(\i\beta\right) & =\left(\cos\alpha+\i\sin\alpha\right)\left(\cos\beta+\i\sin\beta\right)\\
 & =\left(\cos\alpha\cos\beta-\sin\alpha\sin\beta\right)+\i\left(\cos\alpha\sin\beta+\sin\alpha\sin\beta\right)\\
 & =\cos\left(\alpha+\beta\right)+\i\sin\left(\alpha+\beta\right)\\
 & =\exp\left(\i\alpha+\i\beta\right)
\end{align*}

よって$\exp\left(\i\alpha\right)\exp\left(\i\beta\right)=\exp\left(\i\alpha+\i\beta\right)$

\paragraph{(2)}

$\alpha=\beta=\theta$として、(1)より
\[
\exp\left(\i\theta\right)\exp\left(\i\theta\right)=\exp\left(2\i\theta\right)
\]
\begin{align*}
\left(\text{左辺}\right) & =\left(\sin\theta+\i\sin\theta\right)^{2}\\
 & =\left(\cos^{2}\theta-\sin^{2}\theta+2\i\sin\theta\cos\theta\right)
\end{align*}
\[
\left(\text{右辺}\right)=\cos2\theta+\i\sin2\theta
\]

両辺の実部と虚部を比較

\begin{align*}
\cos2\theta & =\cos^{2}\theta-\sin^{2}\theta\\
\sin2\theta & =2\sin\theta\cos\theta
\end{align*}

3倍角も同様

\paragraph{(3)}

省略

\paragraph{(4-1)}

$r$は$z$の絶対値、$\theta$は偏角

\paragraph{(4-2)}

\[
r=\sqrt{1^{2}+1^{2}}=\sqrt{2}
\]
\[
\theta=\arctan\left(\frac{1}{1}\right)=\frac{\pi}{2}
\]

\[
1+\i=\sqrt{2}\exp\left(\frac{\pi}{2}\right)
\]


\paragraph{(4-3)}

省略

\paragraph{(5)}

省略

\paragraph{{[}2{]}(1)}

$\Delta\alpha\beta\gamma$が正三角形$\Leftrightarrow$$\frac{\alpha-\gamma}{\beta-\gamma}=\frac{1\pm\sqrt{3}}{2}$

まず$\Leftarrow$を示す。

\[
\left|\frac{\alpha-\gamma}{\beta-\gamma}\right|=\left|\frac{1\pm\sqrt{3}\i}{2}\right|=1
\]
より
\[
\left|\alpha-\gamma\right|=\left|\beta-\gamma\right|
\]
なので、辺$\gamma\beta$と辺$\gamma\alpha$は等しく、

\[
\arg\left(\frac{\alpha-\gamma}{\beta-\gamma}\right)=\pm\frac{\pi}{3}
\]
なので、
\[
\left|\angle\beta\gamma\alpha\right|=\frac{\pi}{3}
\]
したがって$\Delta\alpha\beta\gamma$は頂角$\frac{\pi}{3}$のに底辺三角形、すなわち正三角形である。

逆に、$\Delta\alpha\beta\gamma$が正三角形のときは、確変が等しく、度の角も$\frac{\pi}{3}$なので、点どうしの関係として
\[
\frac{\alpha-\gamma}{\beta-\gamma}=\frac{1\pm\sqrt{3}\i}{2}
\]
を満たす。

\paragraph{(2)}

\begin{align*}
 & \alpha^{2}+\beta^{2}+\gamma^{2}-\alpha\beta-\beta\gamma-\gamma\alpha=0\\
\Rightarrow & \alpha^{2}-2\alpha\gamma+\gamma^{2}+\alpha\gamma+\beta^{2}-2\beta\gamma+\gamma^{2}+\beta\gamma-\gamma^{2}-\alpha\beta=0\\
\Rightarrow & \left(\alpha-\gamma\right)^{2}+\left(\beta-\gamma\right)^{2}+\alpha\gamma-\gamma^{2}+\beta\gamma-\alpha\beta=0\\
\Rightarrow & \left(\alpha-\gamma\right)^{2}+\left(\beta-\gamma\right)^{2}+\gamma\left(\alpha-\gamma\right)-\beta\left(\alpha-\gamma\right)=0\\
\Rightarrow & \left(\alpha-\gamma\right)^{2}+\left(\beta-\gamma\right)^{2}-\left(\alpha-\gamma\right)\left(\beta-\gamma\right)=0\\
\Rightarrow & \left(\alpha-\gamma-\left(\frac{1+\sqrt{3}\i}{2}\right)\left(\beta-\gamma\right)\right)\left(\alpha-\gamma-\left(\frac{1-\sqrt{3}\i}{2}\right)\left(\beta-\gamma\right)\right)=0\\
\Rightarrow & \alpha-\gamma=\frac{1+\sqrt{3}\i}{2}\left(\beta-\gamma\right)\\
\Rightarrow & \frac{\alpha-\gamma}{\beta-\gamma}=\frac{1\pm\sqrt{3}\i}{2}
\end{align*}


\paragraph{{[}3{]}(1)}

\[
\left(\text{与式}\right)=\sum_{n=0}^{\infty}\frac{z^{n}}{n!}
\]

d'Alambertの判定法より、$a_{n}=\frac{1}{n!}$であるので、
\[
r=\lim_{n\rightarrow\infty}\left|\frac{\left(n+1\right)!}{n!}\right|=\infty
\]


\paragraph{(2)}

同様に$a_{n}=\left(-1\right)^{n+1}\frac{1}{n}$として、
\[
r=\lim_{n\rightarrow\infty}\left|\frac{\left(-1\right)^{n+1}}{n}\times\frac{n+1}{\left(-1\right)^{n}}\right|=1
\]


\paragraph{(3)}

同様に$a_{n}=\frac{\alpha\left(\alpha-1\right)\cdots\left(\alpha-n\right)}{n!}$として、
\begin{align*}
r & =\lim_{n\rightarrow\infty}\left|\frac{\alpha\left(\alpha-1\right)\cdots\left(\alpha-\left(n+1\right)\right)}{\left(n+1\right)!}\times\frac{n!}{\alpha\left(\alpha-1\right)\cdots\left(\alpha-n\right)}\right|\\
 & =\left|\frac{\alpha-\left(n+1\right)}{n+1}\right|=1
\end{align*}

\end{document}
