%% LyX 2.2.2 created this file.  For more info, see http://www.lyx.org/.
%% Do not edit unless you really know what you are doing.
\documentclass{report}
\usepackage[T1]{fontenc}
\usepackage{geometry}
\geometry{verbose}
\setcounter{secnumdepth}{3}
\setcounter{tocdepth}{3}
\setlength{\parskip}{\smallskipamount}
\setlength{\parindent}{0pt}
\usepackage[japanese]{babel}
\usepackage{amsmath}
\usepackage{amsthm}

\makeatletter

%%%%%%%%%%%%%%%%%%%%%%%%%%%%%% LyX specific LaTeX commands.

\newcommand*\LyXbar{\rule[0.585ex]{1.2em}{0.25pt}}

%%%%%%%%%%%%%%%%%%%%%%%%%%%%%% Textclass specific LaTeX commands.
\numberwithin{equation}{section}
\numberwithin{figure}{section}

\makeatother

\begin{document}

\chapter{第一回 13/10/07}

哲学の古典的な区分の方法として、「論理学」「自然学」「倫理学」という三種類に分類するものがある。これらは対象とする物事による分類である。まず論理学とは、現在の世界のあり方にかかわらず妥当し、普遍的に成り立つ、ロゴスに関する学問のことであり、``理性の真理''を対象とする。さらに範囲を広げ、おおよそ自然というもの全体を対象とする学問が、自然学である。これは英語の物理学physicsとほぼ同義である。これは``自然の事実''、``自然の法則的秩序''を対象とし探求する学問である。対象となる自然には様々な法則があり、その一番外側に存在する物理法則(全宇宙がそれに従っている必然的な法則)、さらに生物における「快・苦の原理」(生物を必然的に縛る、生存のための原理)などで構成される。

そして、本題となる倫理学はこのいずれとも異なり、人間というものがあって初めて確立しうる学問である。自然学と倫理学を分けるものは、人間は単なる生物ではないという主観であるといえる。つまり、ここでしばしば注目されるのは「``規範''とはなにか?」ということである。ここにおける規範と法則の違いは、(直感的には、)法則というのは必ず従わざるをえない、例外を許さない必然的なもの(法則的・必然的・決定論的宇宙観と親和的なもの)であるということである。物体が上に落下するなど、自然現象に反する減少は絶対に起こらない。それに対して、規範とは、従うべきとはされても従わないことがありうるものである。(規範に対する違反がありうる・「自由」という前提が存在する)

このように、自然に対する見方は、重層的である。一番外側には、宇宙のあり方にかかわらない論理的必然性を探求する視線、その内側には、特定の宇宙のあり方に依存する物理的な必然性、そして、物理的な事象では収まらない動物的自然の諸原理(食欲や性欲といったものに突き動かされて行動する原理など)。そしてそれらの制約に縛られた上で、その内側に、人間の空間が存在するのか、人間に特異な原理が存在するのか、という視点が存在する。主体的な行為の自由に基づいた道徳的判断の対象となる物である。例えば、自由な行為の選択の余地があった上で、何らかの器物の損壊を犯したとすると、其れは道徳的判断の対象になる、ということである。

これらの問題を考える上で、どれほど我々の定式が通用するのか、これらの原理には根拠があるのか、それとも無根拠なのか、ということを探求するのが倫理学であるといえる。結局、倫理学における根本的な問題ハ、「なぜ『悪いこと』をしてはいけないのか?(why
be moral?)」(違反の規範性)ということである。「盗みを働いてはならない」「カンインしてはいけない」「嘘をついてはいけない」というった、古典的な「カイ」の妥当性に対しても深く切り込んでいく。此れに対して、物理学的、自然学的、進化論的な観点を採って、例えば「種の存続」を目的として「道徳的な法則」を発明することもできる(原因からのアプローチ)が、それはひとまず置いて、この問の「なぜ」という点について考えてゆく。

ここでは、道徳判断の曖昧な性格というものについて考えてみる。ここで、道徳的判断とはどういうものか。まずひとつには、「感情の表明」との類似性があげられる。悪事を発見したときに思考を媒介することなくこみ上げてくる「悪いことだ」「許せない」という感情の表明が道徳的判断となることがある。だが、感情というものが常に主観的であるのに対して、道徳判断の場合は、第三者の視点・公的妥当性が要求される。よって、感情に依拠し切ることなく、「なぜ悪いことなのか」という吟味を必要とする。

さらに、道徳的判断には「知覚判断」との類似性が挙げられる。たとえば、現実の物体がどのような物理的事象を持っているのかという判断には人間の知覚機関がどのような反応性を持っているか落ちうことに依存する(反応依存的性質)。道徳的判断も同様に、人間がこのような生活を営んでいるから、ある種の行為は「悪い」ことであるという判断が働く、というような依存関係が存在する。ただし、道徳判断においては、判断の食い違いが生じた場合への対応の差が存在しうる。道徳的判断に置いては、同じ状態を知覚したときの(知覚的な)認識の違いについて言語的コミュニケーションで解決しようとする試みは行われない。

そして、道徳的判断には「法的判断」との類似性が存在する。ただし、法的判断においては、判断に際して根拠となるような一般的な原則が共有されているとかんがえられるが、道徳的判断においては、必ずしても明示的な形で原則が共有されているとは限らない・文化的歴史的なバックボーンにおいて曖昧に共有されているという考えがなされる。

以上を踏まえると、道徳的判断というのは根拠となるものが簡単には見つからず、非常に曖昧なものであるといえる。倫理学とは(人間社会の中で共有されているだろうと考えられる)(''物理的自然''や''動物的自然''を超越した)''人間的自然''という曖昧な根拠を調べる学問である。これらは、たとえば古代ギリシャにおいては「法(ノモス)は自然(ピュシス)とどのように関連しているのか。どのように違うのか」というような問題として考えられた。

この授業では、特に西洋の思想の歴史に準拠して進めてゆくが、もちろん東洋的なものについても重要な連関があれば折にふれてゆく。教科書にそって授業は進めてゆくので、最初の3,40ページ程度を予め予習してもらえれば良い。成績は、出席を取らず期末試験によって決定する。

\chapter{第二回 13/10/15}

教科書p.17において、問題としての倫理が発生するのは、「自分の利益と他人の利益がぶつかるときである」というように述べてある。自分にとって「良いこと」が必ずしも「良いこと」ではなくなり、自分にとって「悪いこと」が「悪いこと」でなくなるときに、倫理以前の道徳的問題の原型が発生する。

また、教科書p.27「してはいけないことというのは存在しうるのか」ということがこの本の根本的な問題提起である。できることに対して「してはいけない」と禁止を課すことの意味を考えるのが倫理学の目的である。そしてそこには「道徳以前の自然な我々の生活における良し悪しの意味と道徳という観念が持ち込まれた後の良し悪しの意味の逆転」が存在するのではないかというのが永井氏の問題提起である。

教科書p.15においてこの書はソクラテスの話から始めている。今回は哲学史を汲んでこの辺りの詳しい解説を加える。

{[}プリント1{]}

「悪いと知っていて悪いことをする人はいない」という観念はプラトン著の『プロタゴラス』に出てくるものである。ソフィストのプロタゴラスを相手にソクラテスが弁論を行った際の(初期)対話篇である。初期対話篇というのはプラトンがアカデメイアを作ったBC387以前のごくごく初期の対話篇のことである。それゆえソクラテスの思想がありのままに残っておりソクラテスの思想を知るのに適している資料である。

プリント(3)に件の文章が登場するが、ここで善悪という言葉がどのような文脈で用いられているのか考察する。

快と善、苦と悪はどのように関連しているのか、さらに知っているということがそれにどう関連するのか、ということがここで述べられている。常識においては、人はしばしば悪いことを悪いと知っていながらやってしまうと考えられる。(アリストテレスはこれを人間の心の弱さと捉えてアクラシア(akrasia,
無抑制・意志の弱さ)と表現した。)これに対してソクラテスは、その「知っている」と言うのは正しく知っているとはいえず、それは無知である、とし、そこから「人間は悪いと知って悪いことをすることがない」というテーゼを立てた。

プリント(1)を踏まえると、ソクラテスは、それが周囲にどう影響するのかを考えない限りは、楽しみは良いことであり、苦しみは悪いことであると述べている。これはソクラテス初期のストレートな「快楽主義(hedonism)」である。プロタゴラスの高尚な快楽と下卑た快楽に差をつけようとする態度に対して、ソクラテスは、快楽は、純粋にそれだけを取ってみれば善であるといえるとしている。

さらにソクラテスはこのスタンスに、知識というものを絡めてくる。もしただ単に快楽が善となるのでないならば、ある種の「知性主義」が介入していることになる、とソクラテスは述べた。常識的な知識観ではまた、知識というものはいざというときに役に立たず、一時の快楽や恐怖に負けてしまうようなものであるとされるが、これにソクラテスは反抗した。ソクラテスの問いは、「いわゆる『快楽に負けて、何が最善のことかを知りながらもそれを行わない』という状態は何を意味するか」ということである。これに対してソクラテスは、「快楽に負けて悪いことと知りながら悪を行う」という際の「悪」とは、快楽をもたらすゆえに悪とされるのではなく、それが後に(自分に)より大きな苦痛をもたらすゆえに悪とされるのもであるとし、それを選んでしまうのは、後に起こる大きな苦痛を真の意味で知らないからだ、とした。そして苦痛や快楽の大小を評価するための手段として「計量の技術」が必要とされるとした。「快楽に負けて」と人が言うのはこの計量を正しく行うことができず、「無知」によって誤った選択を行ってしまうということである。

すなわち、人間の本姓においては、快楽主義こそが善悪の基準となるものであり、それに正しく導くのが「知識」であるとソクラテスは結論づけた。

これを踏まえて考えると、教科書p.17以降の議論は、必ずしもこのソクラテスの学説をストレートに覆そうとするものではないことが分かる。教科書で問題とされているのは、自分にとっての善と他の人にとっての善との問題であり、ソクラテスの(ある意味自己中心的な)自分にとっての善と自分にとっての悪の観念とは大きく異なるものである。まとめると、永井氏が問題としているのは道徳以前の自然的な善さである「自分にとっての善いこと」と道徳的な善さである「他人にとっての善いこと」の対立関係である。この教科書ではこの対立関係を踏まえた上で西洋哲学史を一通りなぞってゆく。

\chapter{第三回 13/10/21}

前回の授業では、本筋とはやや関係ないものの、本書で取り上げられているソクラテスの議論に感じた疑問に注釈を加えてみた。

今回はプラトンの中期対話篇である『国家』の初め3分の1程度で取り上げられている議論がどのようなものなのかについて触れていく。

本書で一貫して取り上げられているのは、道徳外での良さ悪さというものと道徳内での善悪というものが逆転する関係にある場合があるということである。このことを価値の享受という観点から表現売ると、自分にとっての良し悪しと他人にとっての良し悪しが拮抗している状態と言い換えることができる。前回取り上げたソクラテスの話は、この枠組みからやや外れて、善悪というものは完全に自分にとっての良し悪しそのものであるという考え方に基づいたものであった。すなわち、永井氏の言うような一時的な自分にとっての良し悪しと長期的な観点から見た時の自分にとっての良し悪しが対立する場合のことを考えることはあっても、他人にとってというような話はソクラテスの議論の中では出てこない。そのような話は、むしろプラトンの『国家』のなかで取り上げられているものである。

補足プリントの(1)は、イーリスクラウンスの『断片』に記録されたヘラクレイトスの言葉である。ここでは、\textbf{「自分固有のもの(idion)」}と\textbf{「公共・共通のもの(koinon)」}という対について繰り返し述べられており、特に人間は共通のものである共同体(koinonia)・言語(ロゴス)・法律(ノモス)に従う必要があるということにこだわる様子が伺える。

プリント(2)に引用したとおり、ハンナ・アーレントの『人間の条件』ではこの公私の別がギリシャ社会における根本問題であったと分析されている。共同的なものは家族や個人が作るような自然的な生活(phylē)と真っ向から対立するものであるという意識がギリシャ社会にはあり、ロゴスによって結ばれた人間の共同体の中で価値を持つのは活動と言論である。必要だとか役に立つだとか言いう価値観はすべて個人的なidionの世界での話であって、これらを排除し公共性に基づいた社会を構築することが人間的な営みであるという考え方がギリシャ世界特有のものであり、これをローマ人も継承しているとした。これはギリシャ思想を理解する上で非常に重要な考え方である。

この話は教科書p29以降に対応する。『国家』篇というのは''Politeia''の慣例訳であるが、本来は「市民共同体としてのpolisのありかた」という意味である。これをキケロガラテン語に訳したのが''De
Re Publica(Republic,公事論)''であり、『国家』という訳はここに由来するのだが、本来個々ではpolisのあるべき姿について議論していることに注意して欲しい。ここでの議論で必要となったのは「正義」(正しいこと、dikaion)とはなにか、ということである。

『国家』篇の中でトラシュマコスが強者の利益こそが正しいことであり、同時に正しいことは弱者にとっての不利益であるとしたのに対して、ソクラテスは、強者の技術とは非支配者の利益のためのものであるとした。

\chapter{第四回 13/10/28}

前回は、人間には自分固有のものであるイディオン(idion)と、共通なものであるコイノン(koinon)の二つの対立が根底に有り、イディオンの束縛を離れコイノンに従いい切る事こそが美しい生き方であるという考え方がギリシャ時代からローマ時代にかけて存在したという話をした。コイノンに従うことは言論(logos)や法律(nomos)に従うことが含まれており、イディオンに従うことは自然(physis)的な結合に基づき家庭(oikos)に従うことが含まれる。

これを踏まえてプラトンの『国家』篇を見てみると、(コイノンに属する)ポリスの良いあり方が論じられており、トラシュマコスへの反論という形で話が進められていた。

前回の授業の内容を再び整理する。まず前提として「正義・道徳的規範というのは誰にとってのものなのか」という対立、そしてそれが誰によって定められたのかという問いについての対立が背景に存在している。トラシュマコスはこれに対して、道徳(法律)とは「強者が」「強者の利益のために」定めたものであり、それに反する弱者が不正な人間と言われるのみであると主張し、言い換えると、道徳(法律)とは弱者が強者に奉仕する場であり、その本質は滅私・利他でしかないとした。これに対して(国家篇での)ソクラテスは、道徳とは「強者が」「自己都合を離れて弱者のために」定めたものであるとし、これを「医者が医術を用いるのは自身のためではなく患者のためである」という例を引き出して説明した。

これをイディオンとコイノンの対立の構図に当てはめてみると、道徳の本質はコイノンに基づくものであるとされているが、実際には政治的権力・発言力に影響されるコイノンはまやかしであり、真の道徳とはイディオンに基づくものであるとした。対するソクラテスの主張は、「自己の利益を離れる」ということが道徳であり、これはまさしく公共に奉仕するというコイノンの行為に他ならないということだと捉えることができる。

関連して参考までに、トゥキディデスの『歴史』という史学上でも重要な資料で書かれていることを述べる。アテナイの使節団がメロス島に趣き会談を行い、中立を保持するメロス島の住民に対して自らに与しないことを怒ったアテナイの使節団が脅しを掛ける場面が存在する(第五巻104-105)。ここでメロス棟の住民が述べたのが、「たとえ力で劣っていても我々は正義に悖った行為をしていないのだから神の助けという点においては我々は決してあなた方に劣っていない」ということである。これに対するアテナイ使節団の応答は「我々は強者が勝利するという自然の法則に従って突き進んでいるのみであり、決して正義に悖った行為ではない」(「優者常勝の人道こそ天則」)というものである。このアテナイ使節団の自己正当化の理論において「自己都合をはなれている」という点を抽出したのがソクラテスの理論であり、「強者の利益にもとづいている」という点を抽出したのがトラシュマコスの理論であるといえる。教科書p.30付近の話は以上のように修正されるべきであるように思う。

教科書では続いてグラウコンが「ギュゲスの指輪」の思考実験に基づきトラシュマコスの意見を養護する理論を展開する様子が書かれている。もしギュゲスの指輪を手に入れたならば誰しもが「法」に縛られることなく自らの欲望を達成するはずであり、故にこれが人間本来のあり方である。グラウコンは誰もが自然本来のあり方においては利己主義者であるが、弱者たちの自己保全のための相互協定契約であるところの「法」によって縛られている。すなわち「法」に自らすすんで従うものはおらず、全ての人間は仕方なしに「法」に従っているのみであるとした。このように自然本来における善悪とは、自らの利益・損害と一致している。これより、正義とは人間の利益・損害との中間の妥協策としての「法」そのものであるといえるとした。

先の「正義・道徳的規範というのは誰が誰のために定めたものなのかという見解に対する対立」という構図にグラウコンを当てはめてみると、グラウコンの主張は、道徳とは「弱者たちが」「弱者たちの自己保全のために」定めた約束であり、トラシュマコスやソクラテスの主張と異なり強者という項目が消えることになる。ある意味ではグラウコスはソクラテスの「自己利益を離れる」という理論に対する異なる根拠を与えたといえ、同時にコイノンの概念である法(nomos)をイディオンである自然(physis)とのつながりの中で説明し得た考え方であるとも言える。

この後教科書p.34においてグラウコンに対するソクラテスの反論が述べられている。まずソクラテスはグラウコンの唱えたギュゲスの指輪を手に入れた時の人間の行動に対して疑問を投げかけ、人間は自然本来において利己主義者であるというようなことはないというように主張した。正義に叶う生き方こそが本来の幸福であり、正義にかなわない生き方は本来望ましい生き方ではなく、道徳と幸福の関係を内的な関係で捉えた。

教科書p.35-36にかけて、プラトン=ソクラテスの反論である3つの話が取り上げられている。1つ目は、国政のあり方になぞらえての議論、2つ目は、魂の三分説、3つ目は快楽の本性についての議論である。

\chapter{第五回 13/11/05}

先週問いかけられたグラウコンの「極端に不正な人と極端に正しい人はどちらがより幸せなのか」という問いは『国家』第一巻で提示されたものだが、これにい対する回答は第九巻で与えられている。すなわち『国家』という文献はその内容の殆どをこの問題に対する理由付けに費やされたものである。

グラウコンの問いに対する「ソクラテス」の応答は3つの方法によってなされている。一つに、国政との類比を行い、独裁制よりも王政・優秀者支配性のほうが優れていることを述べた。二つに、魂の三区分説を唱え、魂の理知的な部分は気概的な部分・欲望的な部分よりも上位にあるとみなし、欲望に囚われた状態は決して幸せな状態でないと述べた。三つに、快楽論の検知から、快楽には真の快楽と偽の快楽があるとし、偽の快楽に突き動かされるのは真の快楽の姿を知らないからだとした。

しかしこの回答が具体的にグラウコンのどのような問題に対する回答となっているのかを考えると、必ずしも教科書の記述は正しくないと思われる。一番のポイントはギュゲスの物語位は「人間の自然」を表現しえているのか?という点である。実のところソクラテスは『国家』第九巻において「善の内に自ら目覚める欲望」の存在を否定していない。同時にその議論を「幸福論」の観点から留保し、放埒な人を孔の空いた瓶にたとえ、
\begin{itemize}
\item 穴のあい高め~放埒な人

\begin{itemize}
\item 必要を超えて、際限なく求める渇望
\end{itemize}
\item 人間の中の「壊れた部分」\LyXbar 欲望充足の制限装置

\begin{itemize}
\item 「有りもしないもの」を追求してやまない苦痛
\end{itemize}
\item 自己自身を「試合」できていることの快

\begin{itemize}
\item 「理知的部分」による統御

\begin{itemize}
\item 気概的部分や欲望の部分について真の快をもたらす
\end{itemize}
\item 「身体の健康」に否定された「魂の健康」
\end{itemize}
\item 永井氏の反問

\begin{itemize}
\item プラトンの議論は「社会の要請」の密輸入の上に成り立つ「護教論」に過ぎないのではないか?

\begin{itemize}
\item その本当の根拠を隠蔽した上で「自分自身の幸福」のためには理知的で思慮分別の生き方付が望ましい、等と嘘をついているのではないか?
\end{itemize}
\item 自分自身の実感から出発せざるを得ない議論の可能性と限界
\end{itemize}
\end{itemize}

\end{document}
