%% LyX 2.0.6 created this file.  For more info, see http://www.lyx.org/.
%% Do not edit unless you really know what you are doing.
\documentclass[english]{report}
\usepackage[T1]{fontenc}
\usepackage[utf8]{inputenc}
\setcounter{secnumdepth}{3}
\setcounter{tocdepth}{3}
\setlength{\parskip}{\smallskipamount}
\setlength{\parindent}{0pt}
\usepackage{babel}

\usepackage{xltxtra}
\setmainfont{A-OTF-RyuminPro-Light.otf}

\begin{document}

\title{構造化学レポートNo.1}


\author{高橋光輝 学籍番号: 340728B}


\date{10/25/13}

\maketitle

\paragraph{波動性と粒子性}

電子が粒子であることを示す実験事実: 検出器を用いて電子を分離し観測できる

電子が波動であることを示す実験事実: バイプリズムを用いた干渉実験により電子同士の干渉が観測される

光子が粒子であることを示す実験事実: 光電効果において光の強度ではなく波長に応じた電子が発振される

光子が波動であることを示す実験事実: ヤングの実験により光同士の干渉効果が観測される


\paragraph{単位の変換}

1. $\lambda=656\mathrm{nm}$より、$\nu=\frac{c}{\lambda}=4.57\times10^{14}\mbox{\ensuremath{\left[\mathrm{s^{-1}}\right]}}$

\begin{eqnarray*}
E & = & h\nu\\
 & = & 6.63\times10^{-34}\left[\mathrm{Js}\right]\times4.57\times10^{14}\left[\mathrm{s^{-1}}\right]\\
 & = & 3.04\times10^{-19}\left[\mathrm{J}\right]\\
 & = & 1.90\left[\mathrm{eV}\right]
\end{eqnarray*}


また
\begin{eqnarray*}
E & = & \frac{1}{\lambda}\\
 & = & 1.524\times10^{6}\left[\mathrm{m^{-1}}\right]\\
 & = & 1.524\times10^{4}\left[\mathrm{cm^{-1}}\right]
\end{eqnarray*}


$620\sim750\mathrm{nm}$に含まれ、赤色である。

2. $f=800\left[\mathrm{MHz}\right]$のとき
\begin{eqnarray*}
\lambda & = & \frac{c}{f}\\
 & = & \frac{3\times10^{8}\left[\mathrm{m/s}\right]}{8\times10^{8}\left[\mathrm{Hz}\right]}\\
 & = & 0.37\left[\mathrm{m}\right]
\end{eqnarray*}
$f=1.5\left[\mathrm{GHz}\right]$のとき
\begin{eqnarray*}
\lambda & = & \frac{c}{f}\\
 & = & \frac{3\times10^{8}\left[\mathrm{m/s}\right]}{1.5\times10^{9}\left[\mathrm{Hz}\right]}\\
 & = & 0.2\left[\mathrm{m}\right]
\end{eqnarray*}



\paragraph{電子の波動性}
\end{document}
