%% LyX 2.2.2 created this file.  For more info, see http://www.lyx.org/.
%% Do not edit unless you really know what you are doing.
\documentclass[english]{article}
\usepackage[T1]{fontenc}
\usepackage[utf8]{inputenc}
\usepackage[a5paper]{geometry}
\geometry{verbose,tmargin=2cm,bmargin=2cm,lmargin=1cm,rmargin=1cm}
\setlength{\parskip}{\smallskipamount}
\setlength{\parindent}{0pt}
\usepackage{textcomp}
\usepackage{amsmath}
\usepackage{amssymb}
\usepackage{stmaryrd}
\usepackage{graphicx}
\usepackage{esint}

\makeatletter
%%%%%%%%%%%%%%%%%%%%%%%%%%%%%% User specified LaTeX commands.
\usepackage[dvipdfmx]{hyperref}
\usepackage[dvipdfmx]{pxjahyper}

\makeatother

\usepackage{babel}
\begin{document}

\title{2016-A 電気磁気学 }

\author{教員: 小野/種村 入力: 高橋光輝}

\maketitle
\global\long\def\pd#1#2{\frac{\partial#1}{\partial#2}}
\global\long\def\d#1#2{\frac{\mathrm{d}#1}{\mathrm{d}#2}}
\global\long\def\pdd#1#2{\frac{\partial^{2}#1}{\partial#2^{2}}}
\global\long\def\dd#1#2{\frac{\mathrm{d}^{2}#1}{\mathrm{d}#2^{2}}}
\global\long\def\e{\mathrm{e}}
\global\long\def\i{\mathrm{i}}
\global\long\def\j{\mathrm{j}}
\global\long\def\grad{\operatorname{grad}}
\global\long\def\rot{\operatorname{rot}}
\global\long\def\div{\operatorname{div}}
\global\long\def\diag{\operatorname{diag}}
\global\long\def\rank{\operatorname{rank}}
\global\long\def\prob{\operatorname{Prob}}
\global\long\def\cov{\operatorname{Cov}}
\global\long\def\when#1{\left.#1\right|}


\section*{第1回}

\section{第1章 電荷と静電界}

\subsection{クーロンの法則}
\begin{center}
\includegraphics{images/Electromagnetics12/1-1}
\par\end{center}

\begin{align*}
F & =\frac{1}{4\pi\varepsilon_{0}}\frac{Q_{1}Q_{2}}{r^{2}}\left[\mathrm{N}\right]\\
 & =9\times10^{9}\frac{Q_{1}Q_{2}}{r^{2}}
\end{align*}

クーロンの法則

$\varepsilon_{0}$: 真空の誘電率 $8.854\times10^{-12}$

$Q$の単位: $\left[\mathrm{C}\right]=\left[\mathrm{A\cdot s}\right]$

\paragraph{例題}

$\pm1\left[\mathrm{C}\right]$の電荷を1m離れた場所においた時に働く力は?

\begin{align*}
F & =9\times10^{9}\left[\mathrm{N}\right]\\
 & =9\times10^{8}\left[\mathrm{kgw}\right]
\end{align*}


\paragraph{重ね合わせの理}
\begin{center}
\includegraphics{images/Electromagnetics12/1-2}
\par\end{center}

\[
\mathbb{F}=\mathbb{F}_{1}+\mathbb{F}_{2}
\]


\paragraph{練習問題}
\begin{center}
\includegraphics{images/Electromagnetics12/1-3}
\par\end{center}

\begin{align*}
F_{1} & =\frac{1}{4\pi\varepsilon_{0}}\frac{qQ}{a^{2}+y^{2}}\\
F_{1,x} & =F_{1}\cos\theta\\
 & =\frac{1}{4\pi\varepsilon_{0}}\frac{qQ}{a^{2}+y^{2}}\frac{a}{\sqrt{a^{2}+y^{2}}}\\
\mathbb{F} & =\left(-\frac{qQa}{2\pi\varepsilon_{0}\left(a^{2}+y^{2}\right)^{\frac{3}{2}}},0\right)
\end{align*}

点電荷しか扱わない場合はクーロンの法則で十分だが、実際には電荷は連続に分布している。

\subsection{電界(電場) $E$}

電荷そのものではなく、電荷が作り出す電場から力を受けていると考える。
\begin{center}
\includegraphics{images/Electromagnetics12/1-4}
\par\end{center}

\[
F=\frac{1}{4\pi\varepsilon_{0}}\frac{q_{1}q_{2}}{r^{2}}=q_{1}E_{2}
\]
\[
E_{2}=\frac{1}{4\pi\varepsilon_{0}}\frac{q_{2}}{r^{2}}
\]

このような$E_{2}$を、$q_{2}$が作る電気の場、電界と考える。

例: 先ほどの問題で$-Q$と$+Q$が$\left(0,y\right)$に作る$\mathbb{E}$は?

\[
\mathbb{E}=\left(-\frac{Qa}{2\pi\varepsilon_{0}\left(a^{2}+y^{2}\right)^{\frac{3}{2}}},0\right)
\]

$\mathbb{E}=\mathbb{E}_{1}+\mathbb{E}_{2}$が成立する。

なお、静電気においては「自己力」(電荷自身が作り出す電場の影響)は考えない。

\paragraph{電気力線}
\begin{center}
\includegraphics{images/Electromagnetics12/1-5}
\par\end{center}
\begin{itemize}
\item $\mathbb{E}$の大きさ$\propto$力線の密度
\item $+q$から出て$-q$に入るのみ
\item 電荷がないところでは発生しない
\item 電荷がないところで消滅しない

\begin{itemize}
\item 無限遠で消滅する場合、無限遠に対応する電荷が存在するものと仮定する。
\end{itemize}
\item 交差しない
\end{itemize}

\paragraph{電荷から出る力線の本数は?}
\begin{center}
\includegraphics{images/Electromagnetics12/1-6}
\par\end{center}

\[
E=\frac{1}{4\pi\varepsilon_{0}}\frac{q}{r^{2}}
\]

線密度は
\[
\frac{N}{4\pi r^{2}}
\]

よって
\begin{align*}
\frac{1}{4\pi\varepsilon_{0}}\frac{q}{r^{2}} & =\frac{N}{4\pi r^{2}}\\
N & =\frac{q}{\varepsilon_{0}}
\end{align*}


\subsection{電位 ($q$の位置エネルギー)}
\begin{center}
\includegraphics{images/Electromagnetics12/1-7}
\par\end{center}

$+q$を$A\rightarrow B$に準静的に運ぶ。
\begin{center}
\includegraphics{images/Electromagnetics12/1-8}
\par\end{center}

\[
\mathrm{d}w=-q\mathbb{E}\cdot\mathrm{d}\boldsymbol{s}
\]

\begin{align*}
W_{A\rightarrow B} & =\int_{A}^{B}\mathrm{d}w\\
 & =-\int_{A}^{B}q\boldsymbol{E}\cdot\mathrm{d}\boldsymbol{s}
\end{align*}

例 点電荷$q_{1}$が作る$\boldsymbol{E}$

\[
\boldsymbol{E}=\frac{1}{4\pi\varepsilon_{0}}\frac{q_{1}}{r^{2}}\hat{\boldsymbol{r}}
\]

\begin{center}
\includegraphics{images/Electromagnetics12/1-9}
\par\end{center}

\begin{align*}
W_{A\rightarrow B} & =-\frac{qq_{1}}{4\pi\varepsilon_{0}}\int_{A}^{B}\frac{\hat{\boldsymbol{r}}\mathrm{d}\boldsymbol{s}}{r^{2}}\\
 & =-\frac{qq_{1}}{4\pi\varepsilon_{0}}\int_{A}^{B}\frac{\mathrm{d}r}{r^{2}}\\
 & =\frac{qq_{1}}{4\pi\varepsilon_{0}}\left(\frac{1}{r_{B}}-\frac{1}{r_{A}}\right)
\end{align*}

\begin{center}
\includegraphics{images/Electromagnetics12/1-10}
\par\end{center}

この式は経路によらず、始点と終点にのみ左右される。→保存力

つまり、$q$に比例する位置エネルギーを定義できる。

\[
qV_{B}=qV_{A}+W_{A\rightarrow B}
\]
\begin{align*}
W_{AB} & =q\left(V_{B}-V_{A}\right)\\
 & =qV_{BA}
\end{align*}

$V_{BA}$が、ABの電位差$\left[\mathrm{V}\right]$となる。

\[
V_{BA}=-\int_{A}^{B}\boldsymbol{E}\mathrm{d}\boldsymbol{s}
\]

電圧を微分すると電界となり、電界を微分すると電圧となる。

$V=0\left[\mathrm{V}\right]$の基準
\begin{itemize}
\item 無限遠で$V=0$
\item 地球(earth)で$V=0$

$V$も重ね合わせ可能
\end{itemize}

\paragraph{演習問題回答}

問1

\[
F_{1}=-\frac{q_{1}q_{2}}{4\pi\varepsilon_{0}a^{2}}-\frac{q_{1}q_{3}}{4\pi\varepsilon_{0}\left(2a\right)^{2}}=-\frac{q_{1}\left(4q_{2}+q_{3}\right)}{16\pi\varepsilon_{0}a^{2}}
\]
\[
F_{2}=\frac{q_{1}q_{2}}{4\pi\varepsilon_{0}a^{2}}-\frac{q_{1}q_{3}}{4\pi\varepsilon_{0}a^{2}}=\frac{q_{2}\left(q_{1}-q_{3}\right)}{4\pi\varepsilon_{0}a^{2}}
\]
\[
F_{1}=\cdots=-\frac{q_{3}\left(q_{1}+4q_{2}\right)}{16\pi\varepsilon_{0}a^{2}}
\]

$F_{1}=F_{2}=F_{3}$より
\[
q_{1}=q_{3}
\]
\begin{align*}
q_{1}+4q_{2} & =0\\
q_{2} & =-\frac{q_{1}}{4}
\end{align*}

よって
\[
q_{1}:q_{2}:q_{3}=4:-1:4
\]

問2

図略

\begin{align*}
\mathrm{d}q & =\sigma\mathrm{d}S\\
 & =\sigma a^{2}\sin\theta\mathrm{d}\theta\mathrm{d}\varphi
\end{align*}
 余弦定理より
\begin{align*}
c^{2} & =a^{2}+a^{2}-2a\cdot a\cos\theta\\
 & =2a^{2}\left(1-\cos\theta\right)
\end{align*}

\begin{align*}
\mathrm{d}V & =\frac{1}{4\pi\varepsilon_{0}}\frac{\sigma a^{2}\sin\theta}{\sqrt{2}a\sqrt{1-\cos\theta}}\mathrm{d}\theta\mathrm{d}\varphi\\
 & =\frac{1}{4\pi\varepsilon_{0}}\frac{\sigma a^{2}\cdot2\sin\frac{\theta}{2}\cos\frac{\theta}{2}}{\sqrt{2}a\sqrt{2}\sin\frac{\theta}{2}}\mathrm{d}\theta\mathrm{d}\varphi\\
 & =\frac{1}{4\pi\varepsilon_{0}}\sigma a\cos\frac{\theta}{2}\mathrm{d}\theta\mathrm{d}\varphi
\end{align*}
\begin{align*}
V & =\oint_{S}\mathrm{d}V\\
 & =\int_{0}^{\pi}\int_{0}^{2\pi}\frac{\sigma a}{4\pi\varepsilon_{0}}\cos\frac{\theta}{2}\mathrm{d}\theta\mathrm{d}\varphi\\
 & =\int_{0}^{\pi}\frac{\sigma a}{2\varepsilon_{0}}\cos\frac{\theta}{2}\mathrm{d}\theta\\
 & =\frac{\sigma a}{\varepsilon_{0}}\left[\sin\frac{\theta}{2}\right]_{0}^{\pi}\\
 & =\frac{\sigma a}{\varepsilon_{0}}\\
 & =\frac{Q}{4\pi\varepsilon_{0}a}
\end{align*}

問3
\begin{center}
\includegraphics{images/Electromagnetics12/1-11}
\par\end{center}

\[
\mathrm{d}E=\frac{1}{4\pi\varepsilon_{0}}\frac{\rho\mathrm{d}s}{a^{2}+h^{2}}
\]
\begin{align*}
\mathrm{d}E_{z} & =\mathrm{d}E\cdot\frac{h}{\sqrt{a^{2}+h^{2}}}\\
 & =\frac{\rho h\mathrm{d}s}{4\pi\varepsilon_{0}\left(a^{2}+h^{2}\right)^{\frac{3}{2}}}
\end{align*}
\begin{align*}
E & =\oint_{c}\mathrm{d}E_{z}\\
 & =\frac{\rho h}{4\pi\varepsilon_{0}\left(a^{2}+h^{2}\right)^{\frac{3}{2}}}\oint_{c}\mathrm{d}s\\
 & =\frac{\rho ha}{2\varepsilon_{0}\left(a^{2}+h^{2}\right)^{\frac{3}{2}}}
\end{align*}
\[
\mathrm{d}s=a\mathrm{d}\theta
\]
\[
\oint_{c}\mathrm{d}s=\int_{0}^{2\pi}a\mathrm{d}\theta=2\pi a
\]

(2)

\begin{align*}
V & =-\int_{\infty}^{p}\boldsymbol{E}\cdot\mathrm{d}\boldsymbol{s}\\
 & =-\int_{\infty}^{h}\frac{\rho az}{2\varepsilon_{0}\left(z^{2}+a^{2}\right)^{\frac{3}{2}}}\mathrm{d}z
\end{align*}

$t=\sqrt{z^{2}+a^{2}}$とおいて$\d tz=\frac{z}{\sqrt{z^{2}+a^{2}}}=\frac{z}{t}$

\begin{align*}
V & =-\int=\frac{\rho a}{2\varepsilon_{0}t^{3}}t\mathrm{d}t\\
 & =\frac{\rho a}{2\varepsilon_{0}}\left[\frac{1}{t}\right]\\
 & =\frac{\rho a}{2\varepsilon_{0}\sqrt{h^{2}+a^{2}}}
\end{align*}

問4

O34kg 16g/mol→2500mol\texttimes 8

C12kg 12g/mol→1000mol\texttimes 6

H6kg 1g/mol→6000mol\texttimes 1

N3kg 14g/mol→200mol\texttimes 7

以下板書電1

\rule[0.5ex]{1\columnwidth}{1pt}

\section*{第2回}

\paragraph{等電位面}

ex. 点電荷→球面 $\frac{1}{4\pi\varepsilon_{0}}\frac{q}{r}$

電界(電気力線)と直交
\begin{center}
\includegraphics{images/Electromagnetics12/2-1}
\par\end{center}

\[
\boldsymbol{E}=\grad\boldsymbol{V}
\]

$\grad$はgradientであり、要するに微分である。
\begin{center}
\includegraphics{images/Electromagnetics12/2-2}
\par\end{center}

多変数関数の微分
\begin{center}
\includegraphics{images/Electromagnetics12/2-3}
\par\end{center}

\[
\text{等}V\text{面}\perp\grad V
\]

\begin{align*}
\boldsymbol{E} & =\left(E_{x},E_{y},E_{z}\right)\\
 & =\left(\pd Vx,\pd Vy,\pd Vz\right)
\end{align*}

\[
V_{BA}=-\int_{A}^{B}\boldsymbol{E}\mathrm{d}q
\]

$\boldsymbol{E}$を積分すると$V$になり、$V$を微分すると$\boldsymbol{E}$になる。

\paragraph{ナブラ演算子}

\[
\nabla=\left(\pd{}x,\pd{}y,\pd{}z\right)
\]

\begin{align*}
\grad V & =\nabla V\\
 & =\left(\pd Vx,\pd Vy,\pd Vz\right)
\end{align*}


\paragraph{例1.6}
\begin{center}
\includegraphics{images/Electromagnetics12/2-4}
\par\end{center}

クーロン
\[
\boldsymbol{E}=\frac{1}{4\pi\varepsilon_{0}}\frac{Q}{r^{2}}\hat{\boldsymbol{r}}
\]

\[
\boldsymbol{E}=-\grad V,V=\frac{1}{4\pi\varepsilon_{0}}\frac{Q}{r}\Rightarrow\boldsymbol{E}?
\]

\begin{align*}
E_{x} & =-\pd Vx\\
 & =-\pd{}x\left(\frac{1}{4\pi\varepsilon_{0}}\frac{Q}{\sqrt{x^{2}+y^{2}+z^{2}}}\right)\\
 & =\frac{1}{4\pi\varepsilon_{0}}\frac{Qx}{\left(x^{2}+y^{2}+z^{2}\right)^{\frac{3}{2}}}\\
 & =\frac{Qx}{4\pi\varepsilon_{0}r^{3}}
\end{align*}

同様にして
\begin{align*}
\boldsymbol{E} & =\frac{Q}{4\pi\varepsilon_{0}}\left(x,y,z\right)\\
 & =\frac{Q}{4\pi\varepsilon_{0}r^{3}}\boldsymbol{r}\\
 & =\frac{Q}{4\pi\varepsilon_{0}r^{2}}\hat{\boldsymbol{r}}
\end{align*}

別解: $\left(r,\theta,\varphi\right)$で解く。

\begin{align*}
\boldsymbol{E} & =\left(E_{r},E_{\theta},E_{\varphi}\right)\\
 & =-\grad V\\
 & =-\left(\pd Vr,\frac{1}{r}\pd V{\theta},\frac{1}{r\sin\theta}\pd V{\varphi}\right)\\
 & =-\left(-\pd Vr,0,0\right)\\
 & =\left(\frac{1}{4\pi\varepsilon_{0}}\frac{Q}{r^{2}},0,0\right)
\end{align*}


\subsection{ガウスの法則}
\begin{center}
\includegraphics{images/Electromagnetics12/2-5}
\par\end{center}

閉曲面$S$内で発生した$E$力線の本数=$S$を外向きに貫く$E$力線の本数=$Q$本

$\mathrm{d}S$近傍
\begin{center}
\includegraphics{images/Electromagnetics12/2-6}
\par\end{center}

\begin{align*}
\mathrm{d}S\text{を貫く本数} & =E\cdot\mathrm{d}S_{n}\\
 & =E\cos\theta\frac{\mathrm{d}S_{n}}{\cos\theta}\\
 & =E_{n}\mathrm{d}S
\end{align*}

\[
S\text{を貫く本数}=\oint_{S}E_{n}\mathrm{d}S
\]
\[
\therefore\oint_{S}E_{n}\mathrm{d}S=\frac{Q}{\varepsilon_{0}}
\]

これをガウスの法則という。

\paragraph{例1.7}
\begin{center}
\includegraphics{images/Electromagnetics12/2-7}
\par\end{center}
\begin{enumerate}
\item $r>r_{0}$の場合

半径$r$の球$S$→$S$上で$E=\text{一定},E_{n}=E$

\begin{align*}
\oint E_{n}\mathrm{d}S & =E\oint_{S}\mathrm{d}S\\
 & =4\pi r^{2}E=\frac{Q}{\varepsilon_{0}}
\end{align*}

\[
E=\frac{Q}{4\pi\varepsilon_{0}r^{2}}
\]

\item $r\leqq r_{0}$の場合

\[
\oint_{S}E_{n}\mathrm{d}S=4\pi r^{2}E=\frac{q}{\varepsilon_{0}}
\]

$q=0$より$E=0$
\end{enumerate}

\paragraph{微分形}

微小体積$\Delta v$に適用
\begin{center}
\includegraphics{images/Electromagnetics12/2-8}
\par\end{center}

電荷は連続的に分布し$\rho\left(x,y,z\right)\left[\mathrm{c/m^{2}}\right]$

\begin{align*}
\int_{S_{1}+S_{2}}E_{n}\mathrm{d}S & =-E_{x}\Delta y\Delta z\left(E_{x}+\pd{E_{x}}x\Delta x\right)\Delta y\Delta z\\
 & =\pd{E_{x}}x\Delta x\Delta y\Delta z\\
 & =\pd{E_{x}}x\Delta v
\end{align*}

同様にして
\[
\oint_{S}E_{n}\mathrm{d}S=\left(\pd{E_{x}}x,\pd{E_{y}}y,\pd{E_{z}}z\right)\Delta v
\]

また
\[
\oint_{S}E_{n}\mathrm{d}S=\frac{1}{\varepsilon_{0}}\rho\Delta v
\]

よって
\[
\pd{E_{x}}x+\pd{E_{y}}y+\pd{E_{z}}z=\frac{\rho}{\varepsilon_{0}}
\]
\[
\mathrm{div}\boldsymbol{E}=\nabla\cdot\boldsymbol{E}
\]

$\mathrm{div}$: divergence, 湧き出し, 発散

\[
\mathrm{div}\boldsymbol{E}=\frac{\rho}{\varepsilon_{0}}
\]

ガウスの法則 微分形

\subsection{ポアソン・ラプラスの式}

\[
\mathrm{div}\boldsymbol{E}=\frac{\rho}{\varepsilon_{0}},\boldsymbol{E}=-\grad V
\]
\[
\div\grad V=-\frac{\rho}{\varepsilon_{0}}
\]
\[
\nabla\cdot\left(\nabla V\right)=-\frac{\rho}{\varepsilon_{0}}
\]

$\nabla^{2}V=-\frac{\rho}{\varepsilon_{0}}$: ポアソンの式

\[
\pdd Vx+\pdd Vy+\pdd Vz=-\frac{\rho}{\varepsilon_{0}}
\]

$\rho$の分布から$V$を求めることができる。

$\rho=0$のとき

$\nabla^{2}V=0$: ラプラスの式

\subsection{静電界の保存則}
\begin{center}
\includegraphics{images/Electromagnetics12/2-9}
\par\end{center}

\begin{align*}
\oint_{C}\boldsymbol{E}\cdot\mathrm{d}\boldsymbol{s} & =\int_{AC_{1}}^{B}\boldsymbol{E}\cdot\mathrm{d}\boldsymbol{s}+\int_{BC_{2}}^{A}\boldsymbol{E}\cdot\mathrm{d}\boldsymbol{s}\\
 & =\int_{AC_{1}}^{B}-\int_{BC_{2}}^{A}\\
 & =0
\end{align*}

渦なしの条件

\paragraph{微分形}
\begin{center}
\includegraphics{images/Electromagnetics12/2-10}
\par\end{center}

\[
\oint_{c}\boldsymbol{E}\mathrm{d}\boldsymbol{s}=E_{x}\left(x,y-\frac{\Delta y}{2}\right)\Delta x+E_{y}\left(x+\frac{\Delta x}{2},y\right)\Delta y-E_{x}\left(x,y+\frac{\Delta y}{2}\right)\Delta x-E_{y}\left(x+\frac{\Delta x}{2},y\right)\Delta y
\]

\[
E_{x}\left(x,y-\frac{\Delta y}{2}\right)\Delta x-E_{x}\left(x,y+\frac{\Delta y}{2}\right)\Delta x=-\pd{E_{x}}y\Delta y
\]

\[
E_{y}\left(x+\frac{\Delta x}{2},y\right)\Delta y-E_{y}\left(x+\frac{\Delta x}{2},y\right)\Delta y=\pd{E_{y}}x\Delta x
\]

\[
\therefore\oint_{c}\boldsymbol{E}\cdot\mathrm{d}\boldsymbol{s}=\left(\pd{E_{x}}x-\pd{E_{y}}y\right)\Delta x\Delta y
\]

左辺=0より$\left(\rot\boldsymbol{E}\right)_{z}$

$\therefore\rot\boldsymbol{E}=0$: 微分形

\[
\rot\boldsymbol{E}=-\pd{\boldsymbol{B}}t\left(=0\right)
\]

$\rot\boldsymbol{E}=\nabla\times\boldsymbol{E}$

\paragraph{前回課題(1)}

$\mathrm{d}S$にある$\mathrm{d}q=\sigma\mathrm{d}S$が点Pにつくる$\mathrm{d}E$

\[
\mathrm{d}E=\frac{1}{4\pi\varepsilon_{0}}\frac{\sigma\mathrm{d}S}{r^{2}+h^{2}}
\]

\begin{align*}
E & =\int_{S}\mathrm{d}E_{z}\\
 & =\int_{b}^{a}\int_{0}^{2\pi}\frac{1}{4\pi\varepsilon_{0}}\frac{\sigma\mathrm{d}S}{r^{2}+h^{2}}\cdot\frac{h}{\sqrt{r^{2}+h^{2}}}\\
 & =\frac{\sigma h}{4\pi\varepsilon_{0}}\int_{b}^{a}\int_{0}^{2\pi}\frac{r}{\left(r^{2}+h^{2}\right)^{\frac{3}{2}}}\mathrm{d}\theta\mathrm{d}r\\
 & =\frac{\sigma h}{2\varepsilon_{0}}\int_{b}^{a}\frac{r}{\left(r^{2}+h^{2}\right)^{\frac{3}{2}}}\mathrm{d}r\\
 & =\frac{\sigma h}{2\varepsilon_{0}}\left(\frac{1}{\sqrt{b^{2}+h^{2}}}-\frac{1}{\sqrt{a^{2}+h^{2}}}\right)
\end{align*}

(向きは$z$方向)

\paragraph{前回課題(2)}

\[
W_{P\rightarrow O}=q\left(V_{O}-V_{P}\right)
\]

\begin{align*}
V_{P} & =-\int_{\infty}^{h}E\mathrm{d}s\\
 & =-\frac{\sigma}{2\varepsilon_{0}}\int_{\infty}^{h}\left(\frac{z}{\sqrt{z^{2}+b^{2}}}-\frac{z}{\sqrt{z^{2}+a^{2}}}\right)\mathrm{d}z\\
 & =\frac{\sigma}{2\varepsilon_{0}}\left[\sqrt{z^{2}+a^{2}}-\sqrt{z^{2}+b^{2}}\right]_{\infty}^{h}\\
 & =\frac{\sigma}{2\varepsilon_{0}}\left[\frac{a^{2}-b^{2}}{\sqrt{z^{2}+a^{2}}+\sqrt{z^{2}+b^{2}}}\right]_{\infty}^{h}\\
 & =\frac{\sigma}{2\varepsilon_{0}}\left(\frac{a^{2}-b^{2}}{\sqrt{h^{2}+a^{2}}+\sqrt{h^{2}+b^{2}}}\right)
\end{align*}

\[
V_{O}=V_{P}|_{h=0}=\frac{\sigma}{2\varepsilon_{0}}\left(a-b\right)
\]

\begin{align*}
W_{P\rightarrow O} & =q\left(V_{O}-V_{P}\right)\\
 & =\frac{\sigma}{2\varepsilon_{0}}\left(a-b\right)\left(1-\frac{a+b}{\sqrt{h^{2}+a^{2}}+\sqrt{h^{2}+b^{2}}}\right)
\end{align*}


\paragraph{演習問題問1}

(1)
\begin{center}
\includegraphics{images/Electromagnetics12/2-11}
\par\end{center}

\[
\begin{cases}
z=r\cos\theta\\
\sqrt{x^{2}+y^{2}+z^{2}}=r
\end{cases}
\]

\begin{align*}
V & =\frac{aq}{2\pi\varepsilon_{0}}\frac{r\cos\theta}{r^{3}}\\
 & =\frac{aq}{2\pi\varepsilon_{0}}\frac{\cos\theta}{r^{2}}
\end{align*}

(2) 極座標では
\[
\nabla=\left(\pd{}r,\frac{1}{r}\pd{}{\theta},\frac{1}{r\sin\theta}\pd{}{\varphi}\right)
\]
より
\begin{align*}
\left(E_{r},E_{\theta},E_{\varphi}\right) & =-\nabla V\\
 & =\left(\frac{aq}{\pi\varepsilon_{0}}\frac{\cos\theta}{r^{3}},\frac{aq}{2\pi\varepsilon_{0}}\frac{\sin\theta}{r^{3}},0\right)
\end{align*}

(3) 極座標なので、
\begin{align*}
\div\boldsymbol{E} & =\frac{1}{r^{3}}\pd{\left(r^{2}E_{r}\right)}r+\frac{1}{r\sin\theta}\pd{\left(\sin\theta E_{\theta}\right)}{\theta}+\frac{1}{r\sin\theta}\pd{E_{\varphi}}{\varphi}\\
 & =\frac{1}{r^{2}}\left(-\frac{aq}{\pi\varepsilon_{0}}\frac{\cos\theta}{r^{2}}\right)+\frac{1}{r\sin\theta}\left(\frac{aq}{2\pi\varepsilon_{0}}\frac{2\sin\theta\cos\theta}{r^{3}}\right)\\
 & =0
\end{align*}


\paragraph{演習問題問2}

\[
\nabla^{2}V=-\frac{\rho}{\varepsilon_{0}}
\]

$\pdd Vy=\pdd Vz=0$より、
\[
\nabla^{2}V=\pdd Vx=-\frac{\rho}{\varepsilon_{0}}
\]

\[
\pd Vx=-\frac{\rho}{\varepsilon_{0}}x+C_{1}
\]

\[
V=-\frac{\rho}{2\varepsilon_{0}}x^{2}+C_{1}x+C_{2}
\]

$V\left(0\right)=0$より$C_{2}=0$

$V\left(d\right)=V_{0}$より
\[
-\frac{\rho}{2\varepsilon_{0}}d^{2}+C_{1}d=V_{0}
\]

\[
C_{1}=\frac{V_{0}}{d}+\frac{\rho d}{2\varepsilon_{0}}
\]

\[
\therefore V=-\frac{\rho}{2\varepsilon_{0}}x^{2}+\left(\frac{V_{0}}{d}+\frac{\rho d}{2\varepsilon_{0}}\right)x
\]

\begin{align*}
E_{x} & =-\left(\grad V\right)_{x}\\
 & =-\pd Vx\\
 & =\frac{\rho}{\varepsilon_{0}}x-\frac{V_{0}}{d}-\frac{\rho d}{2\varepsilon_{0}}
\end{align*}


\paragraph{演習問題問3}

$V\left(a\right)=0$

\[
\oint_{S}E_{n}\mathrm{d}S=\begin{cases}
\frac{1}{\varepsilon_{0}}\pi\left(a^{2}-b^{2}\right)h\rho & \left(r>a\right)\\
\frac{1}{\varepsilon_{0}}\pi\left(r^{2}-b^{2}\right)h\rho & \left(b<r<a\right)\\
0 & \left(r<b\right)
\end{cases}
\]

左辺は
\begin{align*}
\oint_{S}E_{n}\mathrm{d}S & =E\oint_{S}\mathrm{d}S\\
 & =2\pi rhE
\end{align*}

$\therefore$
\[
E=\begin{cases}
\frac{\left(a^{2}-b^{2}\right)\rho}{2\varepsilon_{0}r} & \left(r>a\right)\\
\frac{\left(r^{2}-b^{2}\right)\rho}{2\varepsilon_{0}r} & \left(b<r<a\right)\\
0 & \left(r<b\right)
\end{cases}
\]

$V$を求める。
\begin{enumerate}
\item $r>a$のとき

\begin{align*}
V & =\int_{a}^{r}E\left(r'\right)\mathrm{d}r'\\
 & =-\frac{\left(a^{2}-b^{2}\right)\rho}{2\varepsilon_{0}}\left[\ln r'\right]_{a}^{r}\\
 & =-\frac{\left(a^{2}-b^{2}\right)\rho}{2\varepsilon_{0}}\ln\frac{r}{a}
\end{align*}

\item $b<r<a$のとき

\begin{align*}
V & =-\int_{a}^{r}E\left(r'\right)\mathrm{d}r'\\
 & =\frac{\rho}{2\varepsilon_{0}}\left[\frac{r'^{2}}{2}-b^{2}\ln r'\right]_{r}^{a}\\
 & =\frac{\rho}{4\varepsilon_{0}}\left(a^{2}-r^{2}-2b^{2}\ln\frac{a}{r}\right)
\end{align*}

\item $r<b$のとき

\begin{align*}
V & =-\int_{a}^{r}E\left(r'\right)\mathrm{d}r'\\
 & =-\int_{a}^{b}E\left(r'\right)\mathrm{d}r'-\int_{b}^{r}E\left(r'\right)\mathrm{d}r'\\
 & =\frac{\rho}{4\varepsilon_{0}}\left(a^{2}-b^{2}-2b^{2}\ln\frac{a}{b}\right)
\end{align*}

\end{enumerate}

\section*{第3回}

\section{導体}

導体⇔(誘電体=絶縁体)
\begin{center}
\includegraphics{images/Electromagnetics12/3-1}
\par\end{center}

$K=0$:
\begin{itemize}
\item $E_{\text{in}}=0$→導体は等電位 ($\boldsymbol{E}=-\grad V$)
\item $\text{表面}\perp\boldsymbol{E}$

\begin{itemize}
\item \begin{center}
\includegraphics{images/Electromagnetics12/3-2}
\par\end{center}
\end{itemize}
\end{itemize}

\subsection{導体表面の電荷密度}
\begin{center}
\includegraphics{images/Electromagnetics12/3-3}
\par\end{center}

\[
\oint E_{n}\mathrm{d}S=\frac{q}{\varepsilon_{0}}
\]

\[
ES=\frac{\sigma S}{\varepsilon_{0}}
\]

\[
\sigma=\varepsilon_{0}E
\]


\paragraph{例}
\begin{center}
\includegraphics{images/Electromagnetics12/3-4}
\par\end{center}

\begin{align*}
4\pi a^{2}E & =\frac{Q}{\varepsilon_{0}}\\
 & =\frac{1}{\varepsilon_{0}}4\pi a^{2}\sigma
\end{align*}

\[
E=\frac{\sigma}{\varepsilon_{0}}
\]

\begin{center}
\includegraphics{images/Electromagnetics12/3-5}
\par\end{center}

\[
E\cdot2S=\frac{\sigma S}{\varepsilon_{0}}
\]

\[
E=\frac{\sigma}{2\varepsilon_{0}}
\]

\begin{center}
\includegraphics{images/Electromagnetics12/3-6}
\par\end{center}

\subsection{大地面}

地球の電位は常に0である。
\begin{center}
\includegraphics{images/Electromagnetics12/3-7}
\par\end{center}

\begin{center}
\includegraphics{images/Electromagnetics12/3-8}
\par\end{center}

\paragraph{影像法}
\begin{center}
\includegraphics{images/Electromagnetics12/3-9}
\par\end{center}

例2.2
\begin{center}
\includegraphics{images/Electromagnetics12/3-10}
\par\end{center}

\begin{center}
\includegraphics{images/Electromagnetics12/3-11}
\par\end{center}

\begin{align*}
\sigma & =-\varepsilon_{0}E\\
 & =-\frac{hq}{2\pi\left(h^{2}+r^{2}\right)^{\frac{3}{2}}}
\end{align*}

\begin{center}
\includegraphics{images/Electromagnetics12/3-12}
\par\end{center}

\begin{align*}
\int\sigma\mathrm{d}S & =-\int_{0}^{\infty}\frac{hq}{2\pi\left(h^{2}+g^{2}\right)^{\frac{3}{2}}}\cdot2\pi r\mathrm{d}r\\
 & =-hq\int_{0}^{\infty}\frac{r}{\left(h^{2}+g^{2}\right)^{\frac{3}{2}}}\mathrm{d}r\\
 & =-hq\left[\frac{1}{\sqrt{h^{2}+r^{2}}}\right]_{0}^{\infty}=-q
\end{align*}


\subsection{同心球面}
\begin{center}
\includegraphics{images/Electromagnetics12/3-13}
\par\end{center}
\begin{enumerate}
\item $r<r_{1}$のとき

\[
\int E\mathrm{d}S=0
\]
\[
E=0
\]

\item $r_{1}<r<r_{2}$のとき

\[
\underbrace{\int E\mathrm{d}S}_{=4\pi r^{2}E}=\frac{Q_{1}}{\varepsilon_{0}}
\]
\[
\therefore E=\frac{Q}{4\pi\varepsilon_{0}r^{2}}
\]

\item $r>r_{2}$のとき

\[
\int E\mathrm{d}S=\frac{Q_{1}+Q_{2}}{\varepsilon_{0}}
\]
\[
E=\frac{Q_{1}+Q_{2}}{4\pi\varepsilon_{0}r^{2}}
\]

\end{enumerate}
$V$は?

\[
V=-\int E\cdot\mathrm{d}s
\]

\begin{align*}
V_{2} & =-\int_{\infty}^{r_{2}}E\mathrm{d}r\\
 & =-\int_{\infty}^{r_{2}}\frac{Q_{1}+Q_{2}}{4\pi\varepsilon_{0}r^{2}}\mathrm{d}r\\
 & =\frac{Q_{1}+Q_{2}}{4\pi\varepsilon_{0}r_{2}}
\end{align*}

\begin{align*}
V_{1} & =V_{2}-\int_{r_{2}}^{r_{1}}E\mathrm{d}r\\
 & =V_{2}-\int_{r_{2}}^{r_{1}}\frac{Q}{4\pi\varepsilon_{0}r^{2}}\mathrm{d}r\\
 & =\frac{1}{4\pi\varepsilon_{0}}\left(\frac{Q_{1}}{r_{1}}+\frac{Q_{2}}{r_{2}}\right)
\end{align*}

ここで外球を接地すると?

$V_{2}=0$なので$Q_{1}+Q_{2}=0$ $\therefore Q_{2}=-Q_{1}$

\[
V_{1}=\frac{Q_{1}}{4\pi\varepsilon_{0}}\left(\frac{1}{r_{1}}-\frac{1}{r_{2}}\right)
\]


\paragraph{中空導体}
\begin{center}
\includegraphics{images/Electromagnetics12/3-14}
\par\end{center}

静電シールド

\subsection{静電容量}
\begin{center}
\includegraphics{images/Electromagnetics12/3-15}
\par\end{center}

$\mathrm{C=\frac{Q}{V}}$: 静電容量{[}F{]}

非接地導体\texttimes 2のとき、
\begin{center}
\includegraphics{images/Electromagnetics12/3-16}
\par\end{center}

\begin{align*}
V & =Ed\\
 & =\frac{\sigma}{\varepsilon_{0}}d\\
 & =\frac{d}{\varepsilon_{0}S}Q
\end{align*}

\begin{center}
\includegraphics{images/Electromagnetics12/3-17}
\par\end{center}

\[
4\pi r^{2}E=\frac{Q}{\varepsilon_{0}}
\]

\[
E=\frac{Q}{4\pi\varepsilon_{0}r^{2}}
\]

\begin{align*}
V & =-\int_{b}^{a}\frac{Q}{4\pi\varepsilon_{0}r^{2}}\mathrm{d}r\\
 & =\frac{Q}{4\pi\varepsilon_{0}}\left(\frac{1}{a}-\frac{1}{b}\right)
\end{align*}

\begin{align*}
C & =\frac{Q}{V}\\
 & =\frac{4\pi\varepsilon_{0}}{\frac{1}{a}-\frac{1}{b}}
\end{align*}

\begin{center}
\includegraphics{images/Electromagnetics12/3-18}
\par\end{center}

$C=\frac{Q}{V}$、非接地導体\texttimes 1

\[
4\pi r^{2}E=\frac{Q}{\varepsilon_{0}}
\]

\[
E=\frac{Q}{4\pi\varepsilon_{0}r^{2}}
\]

\begin{align*}
V & =-\int_{\infty}^{a}\frac{Q}{4\pi\varepsilon_{0}r^{2}}\mathrm{d}r\\
 & =\frac{Q}{4\pi\varepsilon_{0}a}
\end{align*}

\begin{align*}
C & =\frac{Q}{V}\\
 & =4\pi\varepsilon_{0}a
\end{align*}


\subsection{電界のエネルギー}
\begin{center}
\includegraphics{images/Electromagnetics12/3-19}
\par\end{center}

\[
W=QV
\]

エネルギー

\begin{align*}
U & =W\\
 & =\frac{1}{2}CV^{2}\\
 & =\frac{1}{2}QV
\end{align*}

\begin{center}
\includegraphics{images/Electromagnetics12/3-20}
\par\end{center}

\[
W=\int\mathrm{d}W=\int_{0}^{Q}\frac{q}{c}\mathrm{d}q=\frac{Q^{2}}{2c}
\]


\paragraph{例題2-2}
\begin{center}
\includegraphics{images/Electromagnetics12/3-21}
\par\end{center}

くっつく前は
\[
C=4\pi\varepsilon_{0}r
\]
\[
U=2\cdot\frac{+Q^{2}}{2C}=\frac{Q^{2}}{4\pi\varepsilon_{0}r}
\]

くっついた後は
\[
U'=\frac{\left(2Q\right)^{2}}{2C'}=\frac{Q}{2\pi\varepsilon_{0}\sqrt[3]{2}r}
\]

\[
\frac{U'}{U}=\frac{4}{2\sqrt[3]{2}}=1.59
\]


\subsection{導体に働く力}
\begin{center}
\includegraphics{images/Electromagnetics12/3-22}
\par\end{center}

$\Delta x$の移動に要する仕事
\[
-F_{x}\Delta x
\]

$F_{x}$: $F$の$x$方向成分(外力)

$Q=\text{一定}$→エネルギー増分

\[
\Delta W_{e}=-F_{x}\Delta x
\]

\[
F_{x}=-\left(\frac{\Delta W_{e}}{\Delta x}\right)_{Q}\rightarrow-\left(\pd{W_{e}}x\right)_{Q}
\]

\begin{center}
\includegraphics{images/Electromagnetics12/3-23}
\par\end{center}

\begin{align*}
W_{e} & =\frac{Q^{2}}{2C}\\
 & =\frac{Q^{2}x}{2\varepsilon_{0}S}
\end{align*}

$C=\frac{\varepsilon_{0}S}{x},Q=CV$

\begin{align*}
F & =-\left(\pd{W_{e}}x\right)_{Q}\\
 & =\frac{-Q^{2}}{2\varepsilon_{0}S}\\
 & =-\frac{\varepsilon_{0}S}{2}\left(\frac{V}{x}\right)^{2}
\end{align*}

\begin{center}
\includegraphics{images/Electromagnetics12/3-24}
\par\end{center}

$V$一定: $x\rightarrow x+\Delta x\Rightarrow Q\rightarrow Q+\Delta Q$

電源が$V\Delta Q$の仕事をする。

\[
\begin{cases}
\Delta W_{e}=-F\Delta x+V\Delta Q\\
\Delta W_{e}=\frac{1}{2}V\Delta Q\left(\because W_{e}=\frac{1}{2}VQ\right)
\end{cases}
\]

\[
\therefore F\Delta x=\frac{1}{2}V\Delta Q=\Delta W_{e}
\]

\[
F=\left(\pd{W_{e}}x\right)_{V}
\]


\paragraph{前回課題(1)}
\begin{center}
\includegraphics{images/Electromagnetics12/3-25}
\par\end{center}

\[
\rho\left(r\right)=\frac{\rho_{0}}{r^{4}}
\]

\[
\frac{1}{r^{2}}\pd{}r\left(r^{2}\pd Vr\right)=-\frac{\rho_{0}}{\varepsilon_{0}r^{4}}
\]

\[
\pd{}r\left(r^{2}\pd Vr\right)=-\frac{\rho_{0}}{\varepsilon_{0}r^{2}}
\]

\[
r^{2}\pd Vr=\frac{\rho_{0}}{\varepsilon_{0}r}+C_{1}
\]

\[
\pd Vr=\frac{\rho_{0}}{\varepsilon_{0}r^{3}}+\frac{C_{1}}{r^{2}}
\]

\[
V=-\frac{\rho_{0}}{2\varepsilon_{0}r^{2}}-\frac{C_{1}}{r}+C_{2}
\]

$V\left(\infty\right)=0$より$C_{2}=0$

$V\left(a\right)=0$より$C_{1}=-\frac{\rho_{0}}{2\varepsilon_{0}a}$

\[
\therefore V=\frac{\rho_{0}}{2\varepsilon_{0}r}\left(\frac{1}{a}-\frac{1}{r}\right)
\]


\paragraph{前回課題(2)}

\[
\int E\mathrm{d}S=\frac{1}{\varepsilon_{0}}\left(q+\int_{a}^{r}\frac{\rho_{0}}{r^{4}}4\pi r^{3}\mathrm{d}r\right)
\]

\[
4\pi r^{2}E=\frac{1}{\varepsilon_{0}}\left(q+4\pi\rho_{0}\left(\frac{1}{a}-\frac{1}{r}\right)\right)
\]
\[
E=\frac{q}{4\pi\varepsilon_{0}r^{2}}+\frac{\rho_{0}}{\varepsilon_{0}r^{2}}\left(\frac{1}{a}-\frac{1}{r}\right)
\]

\begin{align*}
V_{a} & =-\int_{\infty}^{a}E\mathrm{d}r\\
 & =\left[\frac{\rho_{0}}{\varepsilon_{0}}\left(\frac{1}{ar}-\frac{1}{2r^{2}}\right)+\frac{q}{4\pi\varepsilon_{0}r}\right]_{\infty}^{a}\\
 & =\frac{\rho_{0}}{2\varepsilon_{0}a^{2}}+\frac{q}{4\pi\varepsilon_{0}a}
\end{align*}

$V_{a}=0$より、
\[
q=-\frac{2\pi\rho_{0}}{a}
\]


\paragraph{演習課題問1}
\begin{center}
\includegraphics{images/Electromagnetics12/3-26}
\par\end{center}

\begin{align*}
F & =\frac{q_{1}}{4\pi\varepsilon_{0}}\left(\frac{q_{2}}{a^{2}}+\frac{q_{1}}{\left(2a\right)^{2}}+\frac{q_{2}}{\left(3a\right)^{2}}\right)\\
 & =\frac{q_{1}}{4\pi\varepsilon_{0}a^{2}}\left(\frac{q_{1}}{4}+\frac{10q_{2}}{9}\right)
\end{align*}

\begin{align*}
V & =\frac{1}{4\pi\varepsilon_{0}}\left(\frac{q_{2}}{a}-\frac{q_{1}}{2a}-\frac{q_{2}}{3a}\right)\\
 & =\frac{1}{4\pi\varepsilon_{0}a}\left(-\frac{q_{1}}{2}+\frac{2q_{2}}{3}\right)\\
V' & =\frac{1}{4\pi\varepsilon_{0}}\left(\frac{q_{2}}{\sqrt{2}a}-\frac{q_{1}}{2a}-\frac{q_{2}}{\sqrt{10}a}\right)\\
 & =\frac{1}{4\pi\varepsilon_{0}a}\left(-\frac{q_{1}}{2}+\frac{\left(\sqrt{5}-1\right)q_{1}}{\sqrt{10}}\right)
\end{align*}

\begin{align*}
W & =\Delta U\\
 & =q_{1}\left(V'-V\right)\\
 & =\frac{q_{1}q_{2}}{4\pi\varepsilon_{0}a}\left(\frac{\sqrt{5}-1}{\sqrt{10}}-\frac{2}{3}\right)
\end{align*}


\paragraph{演習問題問2(1)}
\begin{center}
\includegraphics{images/Electromagnetics12/3-27}
\par\end{center}

\[
\int E\mathrm{d}S=\frac{q}{\varepsilon_{0}}
\]

\[
4\pi r^{2}E=\frac{q}{\varepsilon_{0}}
\]

\[
E=\frac{q}{4\pi\varepsilon_{0}r^{2}}
\]

\begin{align*}
V & =-\int_{b}^{a}E\mathrm{d}r\\
 & =\frac{q}{4\pi\varepsilon_{0}}\left(\frac{1}{a}-\frac{1}{b}\right)
\end{align*}

\[
C=\frac{q}{V}=\frac{4\pi\varepsilon_{0}}{\frac{1}{a}-\frac{1}{b}}=\frac{4\pi\varepsilon_{0}ab}{b-a}
\]


\paragraph{演習問題問2(2)}

非接地\texttimes 1
\begin{center}
\includegraphics{images/Electromagnetics12/3-28}
\par\end{center}

\[
\begin{cases}
E_{1}=\frac{-q'}{4\pi\varepsilon_{0}r^{2}}\\
E_{2}=\frac{q-q'}{4\pi\varepsilon_{0}r^{2}}
\end{cases}
\]

\begin{align*}
V_{b} & =-\int_{a}^{b}E_{1}\mathrm{d}r\\
 & =-\int_{a}^{b}\frac{-q'}{4\pi\varepsilon_{0}r^{2}}\mathrm{d}r\\
 & =\frac{q'}{4\pi\varepsilon_{0}}\left(\frac{1}{a}-\frac{1}{b}\right)
\end{align*}

\begin{align*}
V_{b} & =-\int_{\infty}^{b}E_{2}\mathrm{d}r\\
 & =-\int_{\infty}^{b}\frac{q-q'}{4\pi\varepsilon_{0}r^{2}}\mathrm{d}r\\
 & =\frac{q-q'}{4\pi\varepsilon_{0}b}
\end{align*}

上の2つの式は等しいので、
\[
\frac{q'}{q}=\frac{a}{b}
\]

$q'$を$V_{b}$の式に代入

\[
V_{b}=\frac{aq}{4\pi\varepsilon_{0}b}\left(\frac{1}{a}-\frac{1}{b}\right)
\]

\begin{align*}
C & =\frac{q}{V_{0}}\\
 & =\frac{4\pi\varepsilon_{0}b^{2}}{b-a}
\end{align*}


\paragraph{演習問題問2(3)}

\begin{align*}
f & =\frac{1}{4\pi b^{2}}F\\
 & =-\frac{1}{4\pi b^{2}}\left(\pd{W_{e}}b\right)_{q}\\
 & =-\frac{1}{4\pi b^{2}}\left(\pd{}b\frac{q^{2}}{2C}\right)_{q}\\
 & =-\frac{q^{2}}{8\pi b^{2}}\left(\pd{}b\frac{1}{4\pi\varepsilon_{0}}\left(\frac{1}{a}-\frac{1}{b}\right)\right)\\
 & =\frac{-q^{2}}{32\pi^{2}\varepsilon_{0}b^{2}}\frac{1}{b^{2}}\\
 & =\frac{-q^{2}}{32\pi^{2}\varepsilon_{0}b^{4}}
\end{align*}

別解

Maxwell応力
\[
\sigma=\frac{-q}{4\pi b^{2}}
\]

\begin{align*}
E & =\frac{\sigma}{\varepsilon_{0}}\\
 & =\frac{-q}{4\pi\varepsilon_{0}b^{2}}
\end{align*}

\begin{align*}
f & =\frac{1}{2}\varepsilon_{0}E^{2}\left(r=b\right)\\
 & =\frac{\varepsilon_{0}}{2}\frac{q^{2}}{16\pi^{2}\varepsilon_{0}^{2}b^{4}}
\end{align*}


\paragraph{演習問題問3}
\begin{center}
\includegraphics{images/Electromagnetics12/3-30}
\par\end{center}

球電荷の半径を$r\rightarrow r+\mathrm{d}r$とするのに必要な仕事$\mathrm{d}w$

\[
\mathrm{d}w=V_{1}\mathrm{d}q=\frac{q_{r}}{4\pi\varepsilon_{0}r}\mathrm{d}q
\]

\[
q_{r}=\frac{4}{3}\pi r^{3}\rho
\]

\[
\therefore q_{r}=\frac{r^{3}}{a^{3}Q}
\]

および

\begin{align*}
\mathrm{d}q & =4\pi r^{2}\mathrm{d}r\cdot\rho\\
 & =\frac{3Qr^{2}}{a^{3}}\mathrm{d}r
\end{align*}

より、
\[
\mathrm{d}w=\frac{3Q^{2}r^{4}}{4\pi\varepsilon_{0}a^{3}}\mathrm{d}r
\]

よって、$q:0\rightarrow Q$に必要な$W$は、
\begin{align*}
W & =\int\mathrm{d}w=\int_{0}^{a}\frac{3Q^{2}r^{4}}{4\pi\varepsilon_{0}a^{6}}\mathrm{d}r\\
 & =\frac{3Q^{2}}{20\pi\varepsilon_{0}a}
\end{align*}


\section*{第4回}

\paragraph{中間試験 教室変}

743→761

\section{誘電体}

\subsection{誘電体とは}

=絶縁体

電気を通さない物質⇔導体

\paragraph{分極}
\begin{center}
\includegraphics{images/Electromagnetics12/4-1}
\par\end{center}

\begin{center}
\includegraphics{images/Electromagnetics12/4-2}
\par\end{center}
\begin{itemize}
\item 電子分極: 原子核と電子雲のずれ
\item イオン分極: 材料中のイオンの移動
\item 分極電荷:
\begin{center}
\includegraphics{images/Electromagnetics12/4-3}
\par\end{center}

$Q_{F}$があるから$Q_{P}$がある。
\[
Q_{F}>Q_{P}
\]

\end{itemize}

\paragraph{双極子モーメント$\mathbf{p}$}
\begin{center}
\includegraphics{images/Electromagnetics12/4-4}
\par\end{center}

\[
\boldsymbol{p}=q\boldsymbol{T}\left[\mathrm{C\cdot m}\right]
\]


\paragraph{面と分極}
\begin{center}
\includegraphics{images/Electromagnetics12/4-5}
\par\end{center}

$p$が$S$を跨ぐ確率
\[
\frac{\ell}{L}
\]

$S$を通過する電荷量
\begin{align*}
Q & =nSL\cdot\frac{\ell}{L}\cdot q\\
 & =nL\ell q
\end{align*}

$n$: 密度

\[
\therefore\sigma_{P}=\frac{Q}{S}=n\ell q=np=P
\]

$P$: 分極

\[
\boldsymbol{p}=n\left\langle p\right\rangle 
\]

分極ベクトル

$\left\langle \boldsymbol{p}\right\rangle $: $\boldsymbol{p}$のベクトル平均

\subsection{誘電体を含むガウスの法則}
\begin{center}
\includegraphics{images/Electromagnetics12/4-6}
\par\end{center}

$\sigma_{P}=P$付近で$\mathrm{d}S$

\[
Q_{P}=\int_{S}\boldsymbol{p}\cdot\boldsymbol{n}\mathrm{d}S\left(-\int_{S}p_{n}\mathrm{d}S\right)
\]

$\boldsymbol{n}$: $\mathrm{d}S$の単位法線ベクトル

$\boldsymbol{p}\cdot\boldsymbol{n}$: 面$\mathrm{d}S$を貫く$\sigma_{P}$

$Q_{F}$: 真電荷とも言う。

\[
\oint_{S}\boldsymbol{E}\cdot\boldsymbol{n}\mathrm{d}S=\frac{Q_{F}+Q_{P}}{\varepsilon_{0}}
\]

\[
\oint_{S}\left(\varepsilon_{0}\boldsymbol{E}+\boldsymbol{p}\right)\cdot\boldsymbol{n}\mathrm{d}S=Q_{F}
\]

$\left(\varepsilon_{0}\boldsymbol{E}+\boldsymbol{p}\right)=\boldsymbol{D}$:
電束密度$\left[\mathrm{C/m^{2}}\right]$

\[
\oint_{S}\boldsymbol{D}\cdot\boldsymbol{n}\mathrm{d}S=Q_{F}
\]

$\boldsymbol{D}$: $Q_{F}$と関係した量

$\boldsymbol{E}$: $Q_{F}$と$Q_{P}$と関係した量

\[
V_{BA}=-\int_{A}^{B}\boldsymbol{E}\cdot\mathrm{d}q
\]

\[
\oint\boldsymbol{E}\cdot\mathrm{d}\boldsymbol{s}=0
\]

が成立する。

\paragraph{例}
\begin{center}
\includegraphics{images/Electromagnetics12/4-7}
\par\end{center}

\[
-D_{1\bot}\Delta S+D_{2\bot}\Delta S=0
\]

\[
D_{1\bot}=D_{2\bot}
\]

\begin{center}
\includegraphics{images/Electromagnetics12/4-8}
\par\end{center}

\begin{align*}
\oint\boldsymbol{E}\cdot\mathrm{d}\boldsymbol{s} & =-E_{1\sslash}\Delta\ell+E_{2\sslash}\Delta\ell\\
 & =0
\end{align*}


\paragraph{微分系}

\[
\div\boldsymbol{D}=\rho_{F}
\]

\[
\div\left(\varepsilon_{0}\boldsymbol{E}+\boldsymbol{p}\right)=\rho_{F}
\]

\[
\div\left(\varepsilon_{0}\boldsymbol{E}\right)=\rho_{F}+\rho_{P}
\]

\[
\div\boldsymbol{p}=-\rho_{P}
\]

\[
\boldsymbol{E}=-\grad V
\]

\[
\rot\boldsymbol{E}=\boldsymbol{0}
\]


\subsection{誘電率}

常誘電体: $\boldsymbol{p}\propto\boldsymbol{E}$

$\boldsymbol{p}=x_{\varepsilon}\boldsymbol{E}$

$x_{\varepsilon}$: 電気感受率(真空$x_{e}=0$)

\begin{align*}
\boldsymbol{D} & =\varepsilon_{0}\boldsymbol{E}+\boldsymbol{p}\\
 & =\left(1+x_{e}\right)\varepsilon_{0}E\\
 & =\varepsilon\boldsymbol{E}
\end{align*}

$\varepsilon$: $\varepsilon_{r}\varepsilon_{0}$: 誘電率

\paragraph{$\varepsilon_{r}$の値}
\begin{itemize}
\item 空気 1.00059
\item 紙 2\textasciitilde{}3
\item 石英 4
\item 水 80
\item BTO 21000\textasciitilde{}5000
\end{itemize}
\begin{center}
\includegraphics{images/Electromagnetics12/4-9}
\par\end{center}

\[
\oint_{S}D_{n}\mathrm{d}S=D_{0}\Delta S=\sigma_{F}\Delta S
\]

\[
D_{_{0}}=\sigma_{F}
\]

\[
-D_{0}S+D_{1}\Delta S=0
\]

\[
D_{1}=D_{0}
\]

\[
E_{0}=\frac{D_{0}}{\varepsilon_{0}}=\frac{\sigma_{F}}{\varepsilon_{0}}
\]

\[
E_{1}=\frac{D_{1}}{\varepsilon}=\frac{\sigma_{F}}{\varepsilon}
\]

\begin{align*}
V & =E_{0}\left(d_{0}-d\right)+E_{1}d\\
 & =\sigma_{F}\left(\frac{d_{0}-d}{\varepsilon_{0}}+\frac{d}{\varepsilon}\right)
\end{align*}

\begin{align*}
C & =\frac{Q_{F}}{V}\\
 & =\frac{\sigma_{F}S}{V}\\
 & =\frac{S}{\frac{d_{0}-d}{\varepsilon_{0}}+\frac{d}{\varepsilon}}
\end{align*}

\begin{align*}
O_{P} & =-\int\boldsymbol{p}\cdot\boldsymbol{n}\cdot\mathrm{d}S\\
 & P\Delta S
\end{align*}

\[
\sigma_{P}=\frac{Q_{P}}{\Delta S}
\]

$d=d_{0}$のとき、
\[
C=\frac{\varepsilon S}{d}
\]

$d=0$のときの$\frac{\varepsilon}{\varepsilon_{0}}$

電磁図4-10
\[
P=\left(Q_{P}\Delta S\right)
\]

\begin{center}
\includegraphics{images/Electromagnetics12/4-11}
\par\end{center}

\[
V=-\int\boldsymbol{E}\cdot\mathrm{d}\boldsymbol{s}
\]

\[
D=\varepsilon_{0}E+P
\]

\begin{center}
\includegraphics{images/Electromagnetics12/4-12}
\par\end{center}

$D_{L}$: $D_{1}\cos\theta_{1}=D_{2}\cos\theta_{2}$

$E_{\sslash}$: $E_{1}\sin\theta_{1}=E_{2}\sin\theta_{2}$

\[
\underbrace{\frac{E_{1}}{D_{1}}}_{\frac{1}{\varepsilon_{1}}}\tan\theta_{1}=\underbrace{\frac{E_{2}}{D_{2}}}_{\frac{1}{\varepsilon_{2}}}\tan\theta_{2}
\]

\[
\therefore\frac{\tan\theta_{1}}{\tan\theta_{2}}=\frac{\varepsilon_{1}}{\varepsilon_{2}}
\]

\begin{center}
\includegraphics{images/Electromagnetics12/4-13}
\par\end{center}

\paragraph{前回課題(1)}

\[
\oint_{S}E_{n}\mathrm{d}S=E\cdot2\pi r?=\frac{Q}{\varepsilon_{0}}
\]

\[
E=\frac{Q}{2\pi\varepsilon_{0}Lr}
\]

\begin{align*}
V & =-\int_{b}^{a}\frac{Q}{2\pi\varepsilon_{0}Lr}\mathrm{d}r\\
 & =\frac{Q}{2\pi\varepsilon_{0}L}\ln\frac{b}{a}
\end{align*}


\paragraph{前回課題(2)}

図略(プリントと同じ)

\[
C=\frac{2\pi\varepsilon_{0}\left(L-x\right)}{\ln\frac{b}{a}}
\]

\begin{align*}
F & =\left(\pd Wx\right)_{V}\\
 & =\pd{}x\left(\frac{1}{2}CV_{0}^{2}\right)\\
 & =\pd{}x\left(\frac{\pi\varepsilon_{0}V_{0}^{2}\left(L-x\right)}{\ln\frac{b}{a}}\right)\\
 & =-\frac{\pi\varepsilon_{0}V_{0}^{2}}{\ln\frac{b}{a}}
\end{align*}

マイナスなので、$x$が減少する方向。

\paragraph{前回課題(3)}

\begin{align*}
F & =\left(\pd Wb\right)_{V}\\
 & =\pd{}b\left(\frac{1}{2}CV_{0}^{2}\right)\\
 & =\pd{}b\left(\frac{\pi\varepsilon_{0}LV_{0}^{2}}{\ln\frac{b}{a}}\right)\\
 & =-\frac{\pi\varepsilon_{0}LV_{0}^{2}}{b\left(\ln\frac{b}{a}\right)^{2}}
\end{align*}

\begin{align*}
f & =\frac{F}{2\pi bL}\\
 & =-\frac{\varepsilon_{0}V_{0}^{2}}{2b^{2}\left(\ln\frac{b}{a}\right)^{2}}
\end{align*}


\paragraph{前回課題(4)}

図略(プリントと同じ)

SWを開いたとき、内側の電荷$Q=CV_{0}$

$Q_{0}$を与えると、内側の電荷$Q+Q_{0}$

\[
E=\frac{Q+Q_{0}}{2\pi\varepsilon_{0}rL}
\]

\begin{align*}
V & =\frac{Q+Q_{0}}{C}=V_{0}+\frac{Q_{0}}{C}\\
 & =V_{0}+\frac{\ln\frac{b}{a}}{2\pi\varepsilon_{0}L}Q_{0}
\end{align*}


\paragraph{演習問題問1}

\[
\oint_{S}D\mathrm{d}S=Q_{F}
\]

\[
2\pi rD=Q
\]

\[
D=\frac{Q}{2\pi r}
\]
\[
E=\begin{cases}
\frac{Q}{2\pi\varepsilon_{1}r} & \left(a<r<c\right)\\
\frac{Q}{2\pi\varepsilon_{2}r} & \left(c<r<b\right)
\end{cases}
\]

\begin{align*}
V & =-\int_{b}^{c}\frac{Q}{2\pi\varepsilon_{2}r}\mathrm{d}r-\int_{c}^{a}\frac{Q}{2\pi\varepsilon_{1}r}\mathrm{d}r\\
 & =\frac{Q}{2\pi}\left(\frac{1}{\varepsilon_{2}}\ln\frac{b}{c}+\frac{1}{\varepsilon_{1}}\ln\frac{c}{a}\right)
\end{align*}

\[
C=\frac{Q}{V}=\frac{2\pi}{\frac{1}{\varepsilon_{2}}\ln\frac{b}{c}+\frac{1}{\varepsilon_{1}}\ln\frac{c}{a}}
\]


\paragraph{演習問題問2(1)}
\begin{center}
\includegraphics{images/Electromagnetics12/4-15}
\par\end{center}

$D_{\bot}$が保存

\[
\varepsilon_{0}E_{1}=2\varepsilon_{0}E_{2}
\]

\begin{align*}
V & =E_{1}\frac{2}{3}d+E_{2}\frac{1}{3}d\\
 & =\frac{5}{3}dE_{2}
\end{align*}

\[
\therefore E_{2}=\frac{3V}{5d},E_{1}=\frac{6V}{5d}
\]


\paragraph{演習問題問2(2)}
\begin{center}
\includegraphics{images/Electromagnetics12/4-16}
\par\end{center}

\begin{align*}
V & =E\frac{2}{3}d+3E\frac{1}{3}d\\
 & =\frac{5}{3}dE
\end{align*}

\[
E=\frac{3V}{5d}
\]

\begin{align*}
\oint_{S}D\mathrm{d}S & =-\varepsilon_{0}E_{0}\Delta S+2\varepsilon_{0}\cdot3E\Delta S\\
 & =5\varepsilon_{0}E\Delta S\\
 & =\sigma_{F}\Delta S
\end{align*}

\begin{align*}
\sigma_{F} & =5\varepsilon_{0}E\\
 & =\frac{3\varepsilon_{0}V}{d}
\end{align*}

\begin{center}
\includegraphics{images/Electromagnetics12/4-17}
\par\end{center}

\begin{align*}
\sigma_{P} & =-P\\
 & =-\left(D-\varepsilon_{0}\cdot3E\right)\\
 & =-3\varepsilon_{0}E\\
 & =-\frac{9\varepsilon_{0}V}{5d}
\end{align*}


\paragraph{演習問題問3}
\begin{center}
\includegraphics{images/Electromagnetics12/4-18}
\par\end{center}

\begin{align*}
V & =\frac{q}{4\pi\varepsilon_{0}r_{1}}-\frac{q}{4\pi\varepsilon_{0}r_{2}}\\
 & =\frac{q}{4\pi\varepsilon_{0}}\left(\frac{1}{r_{1}}-\frac{1}{r_{2}}\right)
\end{align*}

余弦定理より、
\[
r_{1}^{2}=r^{2}+\left(\frac{\ell}{2}\right)^{2}-r\ell\cos\theta
\]

\begin{align*}
\frac{1}{r_{1}} & =\frac{1}{r\sqrt{1-\frac{\ell}{r}\cos\theta+\left(\frac{\ell}{2}\right)^{2}}}\\
 & \simeq\frac{1}{r}\left(1-\frac{l}{r}\cos\theta\right)^{-\frac{1}{2}}\\
 & \simeq\frac{1}{r}\left(1+\frac{\ell}{2r}\cos\theta\right)
\end{align*}

$r\ll1$のとき$\left(1+r\right)^{a}\simeq1+ar$を用いた。

同様に、
\[
\frac{1}{r_{2}}\simeq\frac{1}{r}\left(1-\frac{\ell}{2r}\cos\theta\right)
\]

\[
\therefore V=\frac{q}{4\pi\varepsilon_{0}}\frac{\ell}{r^{2}}\cos\theta
\]

$\boldsymbol{p}\cdot\hat{\boldsymbol{r}}=q\ell\cos\theta$より、
\[
V=\frac{\boldsymbol{p}\cdot\hat{\boldsymbol{r}}}{4\pi\varepsilon_{0}r^{2}}
\]


\section*{第5回}

\section{電界のエネルギー}

電界のエネルギー密度

\[
w_{t}=\frac{1}{2}\boldsymbol{E}\cdot\boldsymbol{D}\left[\mathrm{J/m^{3}}\right]
\]

\begin{center}
\includegraphics{images/Electromagnetics12/5-1}
\par\end{center}

\begin{center}
\includegraphics{images/Electromagnetics12/5-2}
\par\end{center}

\[
W_{e}=\int_{0}^{Q_{F}}V\left(q_{F}\right)\mathrm{d}q_{F}
\]

\[
V=-\int_{\ell}E\mathrm{d}\ell
\]

\[
q_{F}=\oint_{S}D\mathrm{d}S
\]

\[
\Rightarrow q_{F}+\mathrm{d}q_{F}=\oint_{S}\left(D+\mathrm{d}D\right)\mathrm{d}S
\]

\[
\Rightarrow\mathrm{d}q_{F}=\oint_{S}\mathrm{d}D\mathrm{d}S
\]

\begin{align*}
\therefore W_{e} & =\int\left(\int_{\ell}E\mathrm{d}\ell\right)\left(\oint_{S}\mathrm{d}D\mathrm{d}S\right)\\
 & =\int_{0}^{D}\oint_{L}\int_{\ell}E\mathrm{d}\ell\mathrm{d}S\mathrm{d}D\\
 & =\int_{v}\left(\int_{0}^{D}E\mathrm{d}D\right)\mathrm{d}v\\
 & =\int_{0}^{D}\boldsymbol{E}\cdot\mathrm{d}\boldsymbol{D}=\frac{1}{2}\boldsymbol{E}\cdot\boldsymbol{D}
\end{align*}
: 電界のエネルギー密度である。

$\boldsymbol{D}=\varepsilon\boldsymbol{E}$のときは、
\begin{align*}
W_{e} & =\int_{0}^{D}\frac{D}{\varepsilon}\mathrm{d}D\\
 & =\frac{v}{2\varepsilon}=\frac{1}{2}ED=\frac{1}{2}\varepsilon E^{2}
\end{align*}


\section{誘電体に働く力}

\paragraph{仮想変位}
\begin{center}
\includegraphics{images/Electromagnetics12/5-3}
\par\end{center}

\[
F=-\left(\pd{W_{e}}{\ell}\right)_{Q_{F}}
\]

$C=\frac{S}{\frac{\ell_{0}-\ell}{\ell_{0}}+\frac{\ell}{\varepsilon}}$より、
\begin{align*}
F & =-\pd{}{\ell}\left(\frac{Q_{F}^{2}}{2C}\right)\\
 & =-\pd{}{\ell}\left\{ \frac{Q_{F}^{2}}{2S}\left(\frac{\ell_{0}-\ell}{\varepsilon_{0}}+\frac{\ell}{\varepsilon}\right)\right\} \\
 & =\frac{\varepsilon-\varepsilon_{0}}{2}\frac{W}{d}\left(\frac{Q_{F}}{C}\right)^{2}\:\left(>0\right)
\end{align*}


\paragraph{Maxwellの応力}

電気力線の圧力$=\frac{1}{2}ED\mathrm{\left[N=m^{2}\right]}$
\begin{center}
\includegraphics{images/Electromagnetics12/5-4}
\par\end{center}

\[
E=\frac{V}{d}=\frac{Q_{F}}{Cd}
\]

\[
D_{0}=\frac{\varepsilon_{0}Q_{F}}{Cd}
\]

\[
D_{1}=\frac{\varepsilon Q_{F}}{Cd}
\]

\begin{align*}
F & =\frac{1}{2}E\left(D_{1}-D_{0}\right)wd\\
 & =\frac{1}{2}\left(\frac{Q_{F}}{Cd}\right)^{2}\left(\varepsilon-\varepsilon_{0}\right)wd\\
 & =\frac{\varepsilon-\varepsilon_{0}}{2}\frac{W}{d}\left(\frac{Q_{F}}{C}\right)^{2}
\end{align*}


\paragraph{例3.4}
\begin{center}
\includegraphics{images/Electromagnetics12/5-5}
\par\end{center}

\[
F=-\left(\pd{W_{e}}d\right)_{Q_{F}}
\]

$C=\frac{S}{\frac{d_{0}-d}{\varepsilon_{0}}+\frac{d}{\varepsilon}}$より、
\[
F=\frac{1}{2}\left(\frac{1}{\varepsilon_{0}}-\frac{1}{\varepsilon}\right)\frac{Q_{F}^{2}}{S}
\]

\[
D=\sigma_{F}=\frac{Q_{F}}{S}
\]

\[
E_{0}=\frac{D}{\varepsilon_{0}}=\frac{Q_{F}}{\varepsilon_{0}S}
\]

\[
E_{1}=\frac{Q_{F}}{\varepsilon S}
\]

\begin{align*}
F & =Sf\\
 & =S\left(\frac{1}{2}E_{0}D-\frac{1}{2}E_{1}D\right)\\
 & =\frac{S}{2}\left(\frac{Q_{F}}{S}\right)^{2}\left(\frac{1}{\varepsilon_{0}}-\frac{1}{\varepsilon}\right)
\end{align*}


\paragraph{端効果}

教科書 p.54 Fig 3.14
\begin{center}
\includegraphics{images/Electromagnetics12/5-6}
\par\end{center}

\section*{第5章 電流}

\subsection*{1 電流}

ある麺を1s辺りに通過する電荷量

\[
I\left[\mathrm{A}\right]=\left[C/s\right]
\]

\begin{center}
\includegraphics{images/Electromagnetics12/5-7}
\par\end{center}

\paragraph{電流密度 $i$}
\begin{center}
\includegraphics{images/Electromagnetics12/5-8}
\par\end{center}

$i=\frac{I}{S}$ $i\bot S$ $\left[\mathrm{A/m^{2}}\right]$

$\boldsymbol{i}$: 電流密度ベクトル
\begin{center}
\includegraphics{images/Electromagnetics12/5-9}
\par\end{center}

$S$を通過する電流

\[
I=\int_{S}\boldsymbol{i}\cdot\boldsymbol{n}\mathrm{d}S
\]

$\Delta S$の電流$\Delta I=\boldsymbol{i}\cdot\boldsymbol{n}\Delta S$
\begin{center}
\includegraphics{images/Electromagnetics12/5-10}
\par\end{center}

\begin{align*}
\Delta I & =iS\cos\theta\\
 & =i\cos\theta S\\
 & =i_{n}S\\
 & =\boldsymbol{i}\cdot\boldsymbol{n}\Delta S
\end{align*}


\subsection*{2 電荷の保存則}
\begin{center}
\includegraphics{images/Electromagnetics12/5-11}
\par\end{center}

\[
-\d Qt=I=\oint_{S}\boldsymbol{i}\cdot\boldsymbol{n}\mathrm{d}S
\]

電荷は不生不成。

\[
Q=\int\rho\mathrm{d}v
\]
(ただし$v$は$S$内の体積)より、
\[
-\d Qt=-\d{}t\int\rho\mathrm{d}v=-\int\left(\pd{\rho}t\right)\mathrm{d}v
\]

ガウスの定理

\[
\oint_{S}\boldsymbol{X}\cdot\boldsymbol{n}\mathrm{d}S=\int_{v}\div\boldsymbol{X}\mathrm{d}v
\]
より、
\[
\oint_{S}\boldsymbol{i}\cdot n\mathrm{d}S=\int\div\boldsymbol{i}\mathrm{d}v
\]

\[
\therefore-\int\left(\pd{\rho}t\right)\mathrm{d}v=\int\div\boldsymbol{i}\mathrm{d}v
\]

\[
\therefore\div\boldsymbol{i}=-\left(\pd{\rho}t\right)
\]

これが電荷の保存則の微分形である。

\paragraph{ガウスの定理}
\begin{center}
\includegraphics{images/Electromagnetics12/5-12}
\par\end{center}

$\div\boldsymbol{X}$(ベクトル$\boldsymbol{X}$の発散)の和

\[
\int\div\boldsymbol{X}\mathrm{d}v=\oint_{S}\boldsymbol{i}\cdot\boldsymbol{n}\mathrm{d}S
\]

\begin{center}
\includegraphics{images/Electromagnetics12/5-13}
\par\end{center}

\subsection*{3 物質と電流}
\begin{center}
\includegraphics{images/Electromagnetics12/5-14}
\par\end{center}

$\boldsymbol{i}=\sigma\boldsymbol{E}$: オームの法則 ($\sigma$: 導電率)

または$\boldsymbol{E}=\kappa\boldsymbol{i}$ ($\kappa$: 抵抗率/比抵抗($=\frac{1}{\sigma}$))

$V=RI$の$R$は?

\[
V=E\ell,I=iS,i=\sigma E
\]

\[
E\ell=R\sigma ES
\]

\[
\therefore R=\frac{\ell}{\sigma S}=\frac{\Delta\ell}{S}\left[\mathrm{\Omega}\right]
\]

\[
G=\frac{1}{R}=\frac{S}{\kappa\ell}\left[\mathrm{S}\right]
\]

コンダクタンスである。

\paragraph{表5.1}
\begin{itemize}
\item 銅: $1.7\times10^{-8}$
\item 鉄: $8.7\times10^{-8}$
\item 水銀: $1\times10^{-6}$
\item 大地: $100\sim5000$
\item 塩化ビニル: $\sim10^{12}$
\item PET: $\sim10^{17}$
\item 石英ガラス: $10^{16}$
\end{itemize}

\paragraph{接地抵抗}
\begin{center}
\includegraphics{images/Electromagnetics12/5-15}
\par\end{center}

\begin{center}
\includegraphics{images/Electromagnetics12/5-16}
\par\end{center}

$R=\frac{\ell}{\sigma S}$より、
\[
\mathrm{d}R=\frac{\mathrm{d}r}{\sigma\cdot2\pi r^{2}}
\]

\[
R=\int_{a}^{\infty}\frac{\mathrm{d}r}{\sigma\cdot2\pi r^{2}}=\frac{1}{2\pi\sigma a}
\]


\paragraph{前回課題問1}
\begin{center}
\includegraphics{images/Electromagnetics12/5-17}
\par\end{center}

\begin{align*}
\varepsilon\left(r\right) & =\frac{\varepsilon_{2}-\varepsilon_{1}}{b-a}\left(r-a\right)+\varepsilon_{1}\\
 & =C_{1}+C_{2}r
\end{align*}

ただし
\[
C_{1}=\frac{\varepsilon_{1}b-\varepsilon_{2}a}{b-a}
\]
\[
C_{2}=\frac{\varepsilon_{2}-\varepsilon_{1}}{b-a}
\]

$Q$を与える。

\[
\oint_{S}E\mathrm{d}S=\frac{q}{\varepsilon}
\]

\[
2\pi rE=\frac{Q}{\varepsilon}
\]

\begin{align*}
V & =-\int_{b}^{a}E\mathrm{d}r\\
 & =-\int_{b}^{a}\frac{Q}{2\pi\varepsilon r}\mathrm{d}r\\
 & =-\frac{Q}{2\pi}\int_{b}^{a}\frac{1}{C_{1}}\left(\frac{1}{r}-\frac{C_{2}}{C_{1}+C_{2}r}\right)\mathrm{d}r\\
 & =-\frac{Q}{2\pi C_{1}}\left[\ln r-\ln\left(C_{1}+C_{2}r\right)\right]_{b}^{a}\\
 & =\frac{Q}{2\pi C_{1}}\left[\ln\frac{C_{1}+C_{2}r}{r}\right]_{b}^{a}\\
 & =\frac{Q}{2\pi C_{1}}\ln\left(\frac{C_{1}+C_{2}a}{C_{1}+C_{2}b}\frac{b}{a}\right)\\
 & =\frac{Q}{2\pi C_{1}}\ln\frac{\varepsilon_{1}b}{\varepsilon_{2}a}
\end{align*}

\begin{align*}
C & =\frac{Q}{V}\\
 & =\frac{2\pi}{\ln\left(\frac{\varepsilon_{1}b}{\varepsilon_{2}a}\right)}\frac{\varepsilon_{1}b-\varepsilon_{2}a}{b-a}
\end{align*}


\paragraph{前回課題問2}
\begin{center}
\includegraphics{images/Electromagnetics12/5-18}
\par\end{center}

\[
\begin{cases}
4\pi r^{2}D_{1}=Q\\
4\pi r^{2}D_{2}=Q+4\pi\left(2a\right)^{2}\sigma
\end{cases}
\]

\[
\therefore\begin{cases}
E_{1}=\frac{D_{1}}{\varepsilon_{1}}=\frac{Q}{4\pi\varepsilon_{1}r^{2}}\\
E_{2}=\frac{D_{2}}{\varepsilon_{2}}=\frac{Q+16\pi a^{2}\sigma}{4\pi\varepsilon_{2}r^{2}}
\end{cases}
\]

\begin{align*}
V_{1} & =\int_{2a}^{a}E_{1}\mathrm{d}r\\
 & =-\int_{2a}^{a}\frac{Q}{4\pi\varepsilon_{1}r^{2}}\mathrm{d}r\\
 & =\frac{Q}{8\pi\varepsilon_{1}a}
\end{align*}

\begin{align*}
V_{2} & =-\int_{3a}^{2a}\frac{Q+16\pi a^{2}\sigma}{4\pi\varepsilon_{2}r^{2}}\mathrm{d}r\\
 & =\frac{Q+16\pi a^{2}\sigma}{24\pi\varepsilon_{2}a}
\end{align*}

$V_{1}=V_{2}$より、
\[
3\varepsilon_{2}Q=\varepsilon_{1}\left(Q+16\pi a^{2}\sigma\right)
\]

\[
Q=\frac{16\pi\varepsilon_{1}\sigma a^{2}}{3\varepsilon_{2}-\varepsilon_{1}}
\]


\paragraph{演習問題ヒント}

問1(2) $U=\int_{v}\frac{1}{2}\varepsilon E^{2}\mathrm{d}v$

問2(3) (2)の解を用いて⋯⋯

問2 「十分長い時間$V$を印加」=定常状態

\paragraph{演習問題問1(1)}
\begin{center}
\includegraphics{images/Electromagnetics12/5-19}
\par\end{center}

\[
E=\frac{Q}{4\pi\varepsilon r^{2}}
\]

\begin{align*}
V & =-\int_{\infty}^{a}E\mathrm{d}r\\
 & =-\int_{\infty}^{a}\frac{Q}{2\pi\varepsilon r^{2}}\mathrm{d}r\\
 & =\frac{Q}{4\pi\varepsilon a}
\end{align*}

\[
C=\frac{Q}{V}=4\pi\varepsilon a
\]

\[
\mathrm{d}R=\frac{\kappa\mathrm{d}r}{4\pi r^{2}}
\]

\begin{align*}
R & =\int\mathrm{d}R\\
 & =\int_{a}^{\infty}\frac{\kappa\mathrm{d}r}{4\pi r^{2}}\\
 & =\frac{\kappa}{4\pi a}
\end{align*}


\paragraph{演習問題問1(2)}

\[
U=\frac{Q^{2}}{2C}=\frac{Q^{2}}{8\pi\varepsilon a}
\]

\begin{align*}
U & =\int\frac{1}{2}\varepsilon E^{2}\mathrm{d}v\\
 & =\int_{a}^{\infty}\frac{\varepsilon}{2}\frac{Q^{2}}{16\pi^{2}\varepsilon^{2}r^{4}}4\pi r^{2}\mathrm{d}r\\
 & =\frac{Q^{2}}{8\pi\varepsilon a}
\end{align*}


\paragraph{演習問題問1(3)}

\begin{align*}
\d qt & =-I=-\frac{V}{R}=-\frac{4\pi a}{\kappa}\frac{q}{4\pi\varepsilon a}\\
 & =-\frac{q}{\varepsilon\kappa}
\end{align*}
\[
\therefore q=Q\e^{-\frac{t}{\varepsilon\kappa}}
\]


\paragraph{演習問題問2(1)}
\begin{center}
\includegraphics{images/Electromagnetics12/5-20}
\par\end{center}

\begin{center}
\includegraphics{images/Electromagnetics12/5-21}
\par\end{center}

$r$と$r+\mathrm{d}r$の間の$\mathrm{d}R$

\[
\mathrm{d}R=\frac{\kappa\mathrm{d}r}{2\pi Lr}\:\text{or}\:\frac{2\kappa\mathrm{d}r}{2\pi Lr}
\]

\begin{align*}
R & =\int_{a}^{2a}\frac{\kappa\mathrm{d}r}{2\pi Lr}+\int_{2a}^{3a}\frac{2\pi\mathrm{d}r}{2\pi Lr}\\
 & =\frac{\kappa}{2\pi L}\left(\ln2+2\ln\frac{3}{2}\right)\\
 & =\frac{\kappa}{2\pi L}\ln\frac{9}{2}
\end{align*}


\paragraph{演習問題問2(2)}

定常状態→定常電流

\[
I=\frac{V}{R}=\frac{2\pi LV}{\kappa\ln\frac{9}{2}}
\]

\[
i=\frac{I}{2\pi Lr}=\frac{V}{\kappa r\ln\frac{9}{2}}
\]


\paragraph{演習問題問2(3)}

ガウスの法則を使うと失敗する。

$a<r<2a$で$E_{1}$、$2a<r<3a$で$E_{2}$とすると、

\begin{align*}
E_{1} & =\kappa i\\
 & =\frac{V}{r\ln\frac{9}{2}}
\end{align*}

\begin{align*}
E_{2} & =2\kappa i\\
 & =\frac{2V}{r\ln\frac{9}{2}}
\end{align*}


\paragraph{演習問題問2(4)}
\begin{center}
\includegraphics{images/Electromagnetics12/5-22}
\par\end{center}

\[
\int D\mathrm{d}S=q
\]

\[
-D_{1}\Delta S+D_{2}\Delta S=\sigma\Delta S
\]

\[
D_{2}-D_{1}=\sigma
\]

求める$\sigma$は、
\begin{align*}
\sigma & =D_{2}\left(2a\right)-D_{1}\left(2a\right)\\
 & =\varepsilon\left[E_{2}\left(2a\right)-E_{1}\left(2a\right)\right]\\
 & =\frac{\varepsilon V}{2a\ln\frac{9}{2}}
\end{align*}


\paragraph{演習問題問2(5)}

\begin{align*}
f & =\frac{1}{2}\varepsilon E_{2}^{2}-\frac{1}{2}\varepsilon E_{1}^{2}\\
 & =\frac{1}{2}\varepsilon\left(E_{2}^{2}-E_{1}^{2}\right)_{r=2a}\\
 & =\frac{3\varepsilon V^{2}}{8a^{2}\left(\ln\frac{9}{2}\right)^{2}}
\end{align*}


\section*{第6回}

中間試験3限761

\paragraph{定常電流}

\[
\begin{cases}
\div\boldsymbol{i}=0\\
\boldsymbol{i}=\sigma\boldsymbol{E}\\
\rot\boldsymbol{E}=\boldsymbol{O}
\end{cases}
\]

$\rho_{F}=0$中の誘電体内の$\boldsymbol{D}$と$\boldsymbol{F}$
\[
\begin{cases}
\div\boldsymbol{D}=0\\
\boldsymbol{D}=\varepsilon\boldsymbol{E}\\
\rot\boldsymbol{E}=\boldsymbol{O}
\end{cases}
\]
と、$q\Leftrightarrow\boldsymbol{D},\varepsilon\Leftrightarrow\sigma$ど対応している。

図電磁6-1

\[
\therefore R=\frac{V}{I},C=\frac{Q}{V}
\]

\[
\Rightarrow RC=\frac{Q}{I}=\frac{\oint_{S}\boldsymbol{D}\cdot\boldsymbol{n}\mathrm{d}S}{\oint_{S}\boldsymbol{i}\cdot\boldsymbol{n}\mathrm{d}S}=\frac{\oint_{S}\varepsilon\boldsymbol{E}\cdot\boldsymbol{n}\mathrm{d}S}{\oint_{S}\sigma\boldsymbol{E}\cdot\boldsymbol{n}\mathrm{d}S}=\frac{\varepsilon}{\sigma}
\]

図電磁6-2

\[
R=\frac{1}{2\pi a\sigma}
\]

図電磁6-3

$C$は?

\[
\oint_{S}D\cdot\mathrm{d}S=Q
\]

\[
2\pi r^{2}D=Q
\]

\[
D=\frac{Q}{2\pi r^{2}}
\]

\[
E=\frac{Q}{2\pi\varepsilon r^{2}}
\]

\begin{align*}
V & =-\int_{\infty}^{a}E\mathrm{d}r\\
 & =\frac{Q}{2\pi\varepsilon a}
\end{align*}

\[
C=\frac{Q}{V}=2\pi\varepsilon a
\]

\[
R=\frac{\varepsilon}{\sigma C}=\frac{1}{2\pi\sigma a}
\]


\paragraph{ジュール熱 抵抗の発熱}

\[
p_{J}=\boldsymbol{E}\cdot\boldsymbol{i}=\frac{i^{2}}{\sigma}=\kappa i^{2}\left(\mathrm{W/m^{2}}\right)
\]

図電磁6-4

\begin{align*}
P_{J} & =p_{J}Sl=\frac{i^{2}}{\sigma}Sl\\
 & =IEl\\
 & =IV\left[\mathrm{W}\right]
\end{align*}


\subsection*{4 起電力と電気回路}

\paragraph{キルヒホッフの法則}

第1 $\sum I=0\leftrightarrow\underbrace{\div}_{\text{わきだし}}\boldsymbol{i}=0$

図電磁6-5

第2 $\sum V_{S}=\sum RI\leftrightarrow\rot\boldsymbol{E}=\boldsymbol{O}\leftrightarrow\oint_{C}\boldsymbol{E}\cdot\mathrm{d}\boldsymbol{s}=0$

図電磁6-6

\[
V_{S}=R_{1}I+R_{2}I
\]


\subsection*{5 非定常的な電界}

基本式
\[
\begin{cases}
\div\boldsymbol{D}=\rho_{F}\\
\rot\boldsymbol{E}=\boldsymbol{O}\\
\div\boldsymbol{i}_{F}=-\pd{\rho_{F}}t\\
\boldsymbol{D}=\varepsilon\boldsymbol{E}\\
\boldsymbol{i}=\sigma\boldsymbol{E}
\end{cases}
\]

図電磁6-7

(1)(3)式より、
\[
\div\left(\boldsymbol{i}_{F}+\pd{\boldsymbol{D}}t\right)=0
\]

$\pd{\boldsymbol{D}}t$: 電流とみなす→変位電流

\[
\pd Dt=\varepsilon\pd Et=\frac{\varepsilon}{d}\pd Vt=\underbrace{\frac{\varepsilon}{dC}}_{\frac{1}{S}}\underbrace{\pd Qt}_{I}=i
\]


\paragraph{電界の緩和}

図電磁6-8

\[
\frac{Q}{C}=RI=-R\d Qt
\]

\[
-\d Qt\Rightarrow Q=Q_{0}\e^{-\frac{t}{\tau}}
\]

$\tau=RC$

図電磁6-9

何らかの原因で分離→緩和して元に戻る速度

$\tau=\frac{\varepsilon}{\sigma}$: 緩和時間

図電磁6-10

\[
\begin{cases}
\sigma_{F}=D=\varepsilon E=\varepsilon\frac{i}{\sigma}\\
-i=\pd{\sigma_{F}}t
\end{cases}
\]

\[
\pd{\sigma_{F}}t=-\frac{\sigma}{\varepsilon}\sigma_{F}
\]
 

\[
\therefore\sigma_{F}=\sigma_{F}\circ\e^{-\frac{\sigma}{\varepsilon}t}
\]

\[
\tau=\frac{\varepsilon}{\sigma}
\]


\section{第4章 静電界の解法}

\paragraph{基本式}

\[
\nabla^{2}V=-\frac{\rho}{\varepsilon}+\text{境界条件}
\]

図電磁6-11

\[
\rightarrow V\rightarrow\boldsymbol{E}=-\grad V\rightarrow\boldsymbol{E}
\]

この他、いくつかの解法
\begin{enumerate}
\item 影像法

導体球-点電荷

図電磁6-12

\[
z=\frac{r^{2}}{d},Q'=-\frac{r}{d}Q
\]

導体球の電位を$V$にしたら?

球の中心に$Q''$

\[
V=\frac{Q''}{4\pi\varepsilon_{0}r}
\]

誘電体界面-点電荷

図電磁6-14

I内の$E,V$

図電磁6-15

II内の$E,V$

図電磁6-16
\item 電荷重畳法

図電磁6-13

$Q_{1}$: $V$が満たされる

$-Q_{1}$: $V-0$が満たされる

$Q_{2}$: $V$が満たされる

$-Q_{2}$: $V-0$が満たされる

$Q_{3}$: $V$が満たされる

⋯⋯を無限に繰り返す。

\paragraph{電荷重畳法(数値計算)}

図電磁6-17

$\pm Q_{1},\pm Q_{2}$とおく。(仮想電荷)

検査点2つ: $p_{1},p_{2}$←$V$になるように$Q_{1},Q_{2}$を計算
\item 差分法(コンピュータ)

図電磁6-18

$\pdd{\phi}x+\pdd{\phi}y=0$: ラプラス

\[
\left(\pdd{\phi}x\right)_{\phi=\phi_{0}}\simeq\frac{\left(\phi_{3}-\phi_{0}\right)/h-\left(\phi_{0}-\phi_{1}\right)/h}{h}
\]

$\pdd{\phi}y$も同様

図電磁6-19

→ラプラスに代入

\[
\therefore\phi_{1}+\phi_{2}+\phi_{3}+\phi_{4}-4\phi_{0}=0
\]

\item 有限要素法

図電磁6-20

各要素のエネルギーの和

\[
\sum\frac{1}{2}\varepsilon E^{2}\Delta v
\]

\end{enumerate}

\paragraph{前回課題(1)}

図略(プリントと同じ)

\[
E_{I}=E_{II}=\frac{V}{d}
\]


\paragraph{前回課題(2)}

\[
\rho_{I}=D_{I}=\varepsilon_{0}E_{I}=\frac{\varepsilon_{0}V}{d}
\]

\[
\rho_{II}=D_{II}=\varepsilon E_{II}=\frac{\varepsilon V}{d}
\]

\begin{align*}
C & =\frac{\varepsilon_{0}dw}{d}+\frac{\varepsilon lw}{d}\\
 & =\frac{\left(\varepsilon+\varepsilon_{0}\right)lw}{d}
\end{align*}


\paragraph{前回課題(3)}

\begin{align*}
\rho_{P} & =\left|P\right|=\left(\varepsilon-\varepsilon_{0}\right)E_{II}\\
 & =\frac{\left(\varepsilon-\varepsilon_{0}\right)V}{d}
\end{align*}


\paragraph{前回課題(4)}

\[
i=\sigma E_{II}=\frac{\sigma V}{d}
\]

\[
I=ilw=\frac{lw\sigma V}{d}
\]


\paragraph{前回課題(5)}

\[
\begin{cases}
\d Qt=-I=-\frac{lw\sigma V}{d}\\
V=\frac{Q}{C}=\frac{d}{\left(\varepsilon+\varepsilon_{0}\right)lw}Q
\end{cases}
\]

\[
\d Qt=-\frac{\sigma}{\varepsilon+\varepsilon_{0}}Q
\]

\[
\therefore Q=Q_{0}\e^{-\frac{\sigma}{\varepsilon+\varepsilon_{0}}t}
\]

(2)より
\begin{align*}
Q_{0} & =\left(\rho_{I}+\rho_{II}\right)lw\\
 & =\frac{\left(\varepsilon+\varepsilon_{0}\right)lwV}{d}
\end{align*}

図電磁6-21

\[
\frac{Q}{C}=RI
\]

\[
\d Qt=-I=-\frac{Q}{CR}
\]

\[
R=\frac{l}{\sigma S}=\frac{d}{\sigma lw}
\]

\begin{align*}
C & =\frac{\varepsilon_{0}lw}{d}+\frac{\varepsilon lw}{d}\\
 & =\frac{\left(\varepsilon+\varepsilon_{0}\right)lw}{d}
\end{align*}

\[
CR=\frac{\varepsilon+\varepsilon_{0}}{\sigma}
\]

\begin{align*}
Q & =Q_{0}\e^{-\frac{t}{RC}}\\
 & =Q_{0}\e^{-\frac{\sigma}{\varepsilon+\varepsilon_{0}}t}
\end{align*}


\paragraph{演習問題問1}

\[
i=nev
\]

\begin{align*}
v & =\frac{i}{ne}=\frac{\frac{I}{S}}{ne}=\frac{\frac{10}{0.02}}{8.5\times10^{22}\times1.6\times10^{-19}}\\
 & =0.037\mathrm{cm/s}
\end{align*}


\paragraph{演習問題問2(1)}

図電磁6-22

\[
V_{a}=\frac{Q}{4\pi\varepsilon a}-\frac{Q}{4\pi\varepsilon d}=\frac{Q}{4\pi\varepsilon}\left(\frac{1}{a}-\frac{1}{d}\right)
\]

同様に
\[
V_{b}=\frac{Q}{4\pi\varepsilon d}-\frac{Q}{4\pi\varepsilon b}=\frac{Q}{4\pi\varepsilon}\left(\frac{1}{d}-\frac{1}{b}\right)
\]

\[
V=V_{a}-V_{b}=\frac{Q}{4\pi\varepsilon}\left(\frac{1}{a}+\frac{1}{b}-\frac{2}{d}\right)\simeq\frac{Q}{4\pi\varepsilon}\left(\frac{1}{a}+\frac{1}{b}\right)
\]

\[
C=\frac{Q}{V}=\frac{4\pi\varepsilon}{\frac{1}{a}+\frac{1}{b}}=\frac{4\pi\varepsilon ab}{a+b}
\]


\paragraph{演習問題問2(2)}

\[
R=\frac{\varepsilon}{\sigma c}=\frac{a+b}{4\pi\sigma ab}
\]


\paragraph{演習問題問3(1)}

$E_{\sslash},D_{\perp}$

図電磁6-23

右向きを正とする。

\begin{align*}
D_{1\perp} & =\frac{Q}{4\pi r^{2}}\cos\theta-\frac{Q'}{4\pi r^{2}}\cos\theta\\
 & =\frac{Q-Q'}{4\pi r^{2}}\cos\theta\\
 & =\frac{Q}{4\pi r^{2}}\frac{2\varepsilon_{2}}{\varepsilon_{1}+\varepsilon_{2}}\cos\theta
\end{align*}

\begin{align*}
E_{1\sslash} & =\frac{Q+Q'}{4\pi\varepsilon_{1}r^{2}}\sin\theta\\
 & =\frac{Q}{4\pi\varepsilon_{1}r^{2}}\frac{2\varepsilon_{1}}{\varepsilon_{1}+\varepsilon_{2}}\sin\theta
\end{align*}

図電磁6-24

\begin{align*}
D_{2\perp} & =\frac{Q''}{4\pi r^{2}}\cos\theta\\
 & =\frac{Q}{4\pi r^{2}}\frac{2\varepsilon_{2}}{\varepsilon_{1}+\varepsilon_{2}}\cos\theta\\
 & =D_{1\perp}
\end{align*}

\begin{align*}
E_{2\sslash} & =\frac{Q''}{4\pi\varepsilon_{2}r^{2}}\sin\theta\\
 & =\frac{Q}{4\pi\varepsilon_{0}r^{2}}\frac{2\varepsilon_{2}}{\varepsilon_{1}+\varepsilon_{2}}\sin\theta\\
 & =E_{1\sslash}
\end{align*}


\paragraph{演習問題問3(2)}

図電磁6-25

$Q_{1}$のみ考えると

図電磁6-26

\[
F_{1}=\frac{Q_{1}Q_{1}'}{4\pi\varepsilon_{1}\left(2a\right)^{2}}=\frac{Q_{1}^{2}}{16\pi\varepsilon_{1}a^{2}}\frac{\varepsilon_{1}-\varepsilon_{2}}{\varepsilon_{1}+\varepsilon_{2}}
\]

$Q_{2}$から受ける力

図電磁6-27

\[
F_{2}=\frac{Q_{1}Q_{2}''}{4\pi\varepsilon_{1}\left(2a\right)^{2}}=\frac{Q_{1}Q_{2}}{16\pi\varepsilon_{1}a^{2}}\frac{2\varepsilon_{1}}{\varepsilon_{1}+\varepsilon_{2}}
\]

\begin{align*}
\therefore F & =F_{1}+F_{2}\\
 & =\frac{Q_{1}}{16\pi\varepsilon_{1}a^{2}}\left(\frac{\varepsilon_{1}-\varepsilon_{2}}{\varepsilon_{1}+\varepsilon_{2}}Q_{1}+\frac{2\varepsilon_{1}}{\varepsilon_{1}+\varepsilon_{2}}Q_{2}\right)
\end{align*}

\end{document}
