%% LyX 2.1.3 created this file.  For more info, see http://www.lyx.org/.
%% Do not edit unless you really know what you are doing.
\documentclass[english]{article}
\usepackage[T1]{fontenc}
\usepackage[utf8]{inputenc}
\usepackage[a5paper]{geometry}
\geometry{verbose,tmargin=2cm,bmargin=2cm,lmargin=1cm,rmargin=1cm}
\setlength{\parskip}{\smallskipamount}
\setlength{\parindent}{0pt}
\usepackage{textcomp}
\usepackage{amsmath}
\usepackage{amssymb}
\usepackage{esint}

\makeatletter
%%%%%%%%%%%%%%%%%%%%%%%%%%%%%% User specified LaTeX commands.
\usepackage[version=3]{mhchem}

\usepackage[dvipdfmx]{hyperref}
\usepackage[dvipdfmx]{pxjahyper}

\makeatother

\usepackage{babel}
\begin{document}

\title{2016-A 電気磁気学 }


\author{教員: 小野/種村 入力: 高橋光輝}

\maketitle
\global\long\def\pd#1#2{\frac{\partial#1}{\partial#2}}
\global\long\def\d#1#2{\frac{\mathrm{d}#1}{\mathrm{d}#2}}
\global\long\def\pdd#1#2{\frac{\partial^{2}#1}{\partial#2^{2}}}
\global\long\def\dd#1#2{\frac{\mathrm{d}^{2}#1}{\mathrm{d}#2^{2}}}
\global\long\def\e{\mathrm{e}}
\global\long\def\i{\mathrm{i}}
\global\long\def\j{\mathrm{j}}
\global\long\def\grad{\mathrm{grad}}
\global\long\def\rot{\mathrm{rot}}
\global\long\def\div{\mathrm{div}}



\section*{第1回}


\section{第1章 電荷と静電界}


\subsection{クーロンの法則}

図電1

\begin{align*}
F & =\frac{1}{4\pi\varepsilon_{0}}\frac{Q_{1}Q_{2}}{r^{2}}\left[\mathrm{N}\right]\\
 & =9\times10^{9}\frac{Q_{1}Q_{2}}{r^{2}}
\end{align*}


クーロンの法則

$\varepsilon_{0}$: 真空の誘電率 $8.854\times10^{-12}$

$Q$の単位: $\left[\mathrm{C}\right]=\left[\mathrm{A\cdot s}\right]$


\paragraph{例題}

$\pm1\left[\mathrm{C}\right]$の電荷を1m離れた場所においた時に働く力は?

\begin{align*}
F & =9\times10^{9}\left[\mathrm{N}\right]\\
 & =9\times10^{8}\left[\mathrm{kgw}\right]
\end{align*}



\paragraph{重ね合わせの理}

図電2

\[
\mathbb{F}=\mathbb{F}_{1}+\mathbb{F}_{2}
\]



\paragraph{練習問題}

図電3

\begin{align*}
F_{1} & =\frac{1}{4\pi\varepsilon_{0}}\frac{qQ}{a^{2}+y^{2}}\\
F_{1,x} & =F_{1}\cos\theta\\
 & =\frac{1}{4\pi\varepsilon_{0}}\frac{qQ}{a^{2}+y^{2}}\frac{a}{\sqrt{a^{2}+y^{2}}}\\
\mathbb{F} & =\left(-\frac{qQa}{2\pi\varepsilon_{0}\left(a^{2}+y^{2}\right)^{\frac{3}{2}}},0\right)
\end{align*}


点電荷しか扱わない場合はクーロンの法則で十分だが、実際には電荷は連続に分布している。


\subsection{電界(電場) $E$}

電荷そのものではなく、電荷が作り出す電場から力を受けていると考える。

図電4

\[
F=\frac{1}{4\pi\varepsilon_{0}}\frac{q_{1}q_{2}}{r^{2}}=q_{1}E_{2}
\]
\[
E_{2}=\frac{1}{4\pi\varepsilon_{0}}\frac{q_{2}}{r^{2}}
\]


このような$E_{2}$を、$q_{2}$が作る電気の場、電界と考える。

例: 先ほどの問題で$-Q$と$+Q$が$\left(0,y\right)$に作る$\mathbb{E}$は?

\[
\mathbb{E}=\left(-\frac{Qa}{2\pi\varepsilon_{0}\left(a^{2}+y^{2}\right)^{\frac{3}{2}}},0\right)
\]


$\mathbb{E}=\mathbb{E}_{1}+\mathbb{E}_{2}$が成立する。

なお、静電気においては「自己力」(電荷自身が作り出す電場の影響)は考えない。


\paragraph{電気力線}

図電5
\begin{itemize}
\item $\mathbb{E}$の大きさ$\propto$力線の密度
\item $+q$から出て$-q$に入るのみ
\item 電荷がないところでは発生しない
\item 電荷がないところで消滅しない

\begin{itemize}
\item 無限遠で消滅する場合、無限遠に対応する電荷が存在するものと仮定する。
\end{itemize}
\item 交差しない
\end{itemize}

\paragraph{電荷から出る力線の本数は?}

図電6

\[
E=\frac{1}{4\pi\varepsilon_{0}}\frac{q}{r^{2}}
\]


線密度は
\[
\frac{N}{4\pi r^{2}}
\]


よって
\begin{align*}
\frac{1}{4\pi\varepsilon_{0}}\frac{q}{r^{2}} & =\frac{N}{4\pi r^{2}}\\
N & =\frac{q}{\varepsilon_{0}}
\end{align*}



\subsection{電位 ($q$の位置エネルギー)}

図電7

$+q$を$A\rightarrow B$に準静的に運ぶ。

図電8

\[
\mathrm{d}w=-q\mathbb{E}\cdot\mathrm{d}\boldsymbol{s}
\]


\begin{align*}
W_{A\rightarrow B} & =\int_{A}^{B}\mathrm{d}w\\
 & =-\int_{A}^{B}q\boldsymbol{E}\cdot\mathrm{d}\boldsymbol{s}
\end{align*}


例 点電荷$q_{1}$が作る$\boldsymbol{E}$

\[
\boldsymbol{E}=\frac{1}{4\pi\varepsilon_{0}}\frac{q_{1}}{r^{2}}\hat{\boldsymbol{r}}
\]


図電9

\begin{align*}
W_{A\rightarrow B} & =-\frac{qq_{1}}{4\pi\varepsilon_{0}}\int_{A}^{B}\frac{\hat{\boldsymbol{r}}\mathrm{d}\boldsymbol{s}}{r^{2}}\\
 & =-\frac{qq_{1}}{4\pi\varepsilon_{0}}\int_{A}^{B}\frac{\mathrm{d}r}{r^{2}}\\
 & =\frac{qq_{1}}{4\pi\varepsilon_{0}}\left(\frac{1}{r_{B}}-\frac{1}{r_{A}}\right)
\end{align*}


図電10

この式は経路によらず、始点と終点にのみ左右される。→保存力

つまり、$q$に比例する位置エネルギーを定義できる。

\[
qV_{B}=qV_{A}+W_{A\rightarrow B}
\]
\begin{align*}
W_{AB} & =q\left(V_{B}-V_{A}\right)\\
 & =qV_{BA}
\end{align*}


$V_{BA}$が、ABの電位差$\left[\mathrm{V}\right]$となる。

\[
V_{BA}=-\int_{A}^{B}\boldsymbol{E}\mathrm{d}\boldsymbol{s}
\]


電圧を微分すると電界となり、電界を微分すると電圧となる。

$V=0\left[\mathrm{V}\right]$の基準
\begin{itemize}
\item 無限遠で$V=0$
\item 地球(earth)で$V=0$


$V$も重ね合わせ可能

\end{itemize}

\paragraph{演習問題回答}

問1

\[
F_{1}=-\frac{q_{1}q_{2}}{4\pi\varepsilon_{0}a^{2}}-\frac{q_{1}q_{3}}{4\pi\varepsilon_{0}\left(2a\right)^{2}}=-\frac{q_{1}\left(4q_{2}+q_{3}\right)}{16\pi\varepsilon_{0}a^{2}}
\]
\[
F_{2}=\frac{q_{1}q_{2}}{4\pi\varepsilon_{0}a^{2}}-\frac{q_{1}q_{3}}{4\pi\varepsilon_{0}a^{2}}=\frac{q_{2}\left(q_{1}-q_{3}\right)}{4\pi\varepsilon_{0}a^{2}}
\]
\[
F_{1}=\cdots=-\frac{q_{3}\left(q_{1}+4q_{2}\right)}{16\pi\varepsilon_{0}a^{2}}
\]


$F_{1}=F_{2}=F_{3}$より
\[
q_{1}=q_{3}
\]
\begin{align*}
q_{1}+4q_{2} & =0\\
q_{2} & =-\frac{q_{1}}{4}
\end{align*}


よって
\[
q_{1}:q_{2}:q_{3}=4:-1:4
\]


問2

図略

\begin{align*}
\mathrm{d}q & =\sigma\mathrm{d}S\\
 & =\sigma a^{2}\sin\theta\mathrm{d}\theta\mathrm{d}\varphi
\end{align*}
 余弦定理より
\begin{align*}
c^{2} & =a^{2}+a^{2}-2a\cdot a\cos\theta\\
 & =2a^{2}\left(1-\cos\theta\right)
\end{align*}


\begin{align*}
\mathrm{d}V & =\frac{1}{4\pi\varepsilon_{0}}\frac{\sigma a^{2}\sin\theta}{\sqrt{2}a\sqrt{1-\cos\theta}}\mathrm{d}\theta\mathrm{d}\varphi\\
 & =\frac{1}{4\pi\varepsilon_{0}}\frac{\sigma a^{2}\cdot2\sin\frac{\theta}{2}\cos\frac{\theta}{2}}{\sqrt{2}a\sqrt{2}\sin\frac{\theta}{2}}\mathrm{d}\theta\mathrm{d}\varphi\\
 & =\frac{1}{4\pi\varepsilon_{0}}\sigma a\cos\frac{\theta}{2}\mathrm{d}\theta\mathrm{d}\varphi
\end{align*}
\begin{align*}
V & =\oint_{S}\mathrm{d}V\\
 & =\int_{0}^{\pi}\int_{0}^{2\pi}\frac{\sigma a}{4\pi\varepsilon_{0}}\cos\frac{\theta}{2}\mathrm{d}\theta\mathrm{d}\varphi\\
 & =\int_{0}^{\pi}\frac{\sigma a}{2\varepsilon_{0}}\cos\frac{\theta}{2}\mathrm{d}\theta\\
 & =\frac{\sigma a}{\varepsilon_{0}}\left[\sin\frac{\theta}{2}\right]_{0}^{\pi}\\
 & =\frac{\sigma a}{\varepsilon_{0}}\\
 & =\frac{Q}{4\pi\varepsilon_{0}a}
\end{align*}


問3

図電11

\[
\mathrm{d}E=\frac{1}{4\pi\varepsilon_{0}}\frac{\rho\mathrm{d}s}{a^{2}+h^{2}}
\]
\begin{align*}
\mathrm{d}E_{z} & =\mathrm{d}E\cdot\frac{h}{\sqrt{a^{2}+h^{2}}}\\
 & =\frac{\rho h\mathrm{d}s}{4\pi\varepsilon_{0}\left(a^{2}+h^{2}\right)^{\frac{3}{2}}}
\end{align*}
\begin{align*}
E & =\oint_{c}\mathrm{d}E_{z}\\
 & =\frac{\rho h}{4\pi\varepsilon_{0}\left(a^{2}+h^{2}\right)^{\frac{3}{2}}}\oint_{c}\mathrm{d}s\\
 & =\frac{\rho ha}{2\varepsilon_{0}\left(a^{2}+h^{2}\right)^{\frac{3}{2}}}
\end{align*}
\[
\mathrm{d}s=a\mathrm{d}\theta
\]
\[
\oint_{c}\mathrm{d}s=\int_{0}^{2\pi}a\mathrm{d}\theta=2\pi a
\]


(2)

\begin{align*}
V & =-\int_{\infty}^{p}\boldsymbol{E}\cdot\mathrm{d}\boldsymbol{s}\\
 & =-\int_{\infty}^{h}\frac{\rho az}{2\varepsilon_{0}\left(z^{2}+a^{2}\right)^{\frac{3}{2}}}\mathrm{d}z
\end{align*}


$t=\sqrt{z^{2}+a^{2}}$とおいて$\d tz=\frac{z}{\sqrt{z^{2}+a^{2}}}=\frac{z}{t}$

\begin{align*}
V & =-\int=\frac{\rho a}{2\varepsilon_{0}t^{3}}t\mathrm{d}t\\
 & =\frac{\rho a}{2\varepsilon_{0}}\left[\frac{1}{t}\right]\\
 & =\frac{\rho a}{2\varepsilon_{0}\sqrt{h^{2}+a^{2}}}
\end{align*}


問4

O34kg 16g/mol→2500mol\texttimes 8

C12kg 12g/mol→1000mol\texttimes 6

H6kg 1g/mol→6000mol\texttimes 1

N3kg 14g/mol→200mol\texttimes 7

以下板書電1

\rule[0.5ex]{1\columnwidth}{1pt}


\section*{第2回}


\paragraph{等電位面}

ex. 点電荷→球面 $\frac{1}{4\pi\varepsilon_{0}}\frac{q}{r}$

電界(電気力線)と直交

図電磁2-1

\[
\boldsymbol{E}=\grad\boldsymbol{V}
\]


$\grad$はgradientであり、要するに微分である。

図電磁2-2

多変数関数の微分

図電磁2-3

\[
\text{等}V\text{面}\perp\grad V
\]


\begin{align*}
\boldsymbol{E} & =\left(E_{x},E_{y},E_{z}\right)\\
 & =\left(\pd Vx,\pd Vy,\pd Vz\right)
\end{align*}


\[
V_{BA}=-\int_{A}^{B}\boldsymbol{E}\mathrm{d}q
\]


$\boldsymbol{E}$を積分すると$V$になり、$V$を微分すると$\boldsymbol{E}$になる。


\paragraph{ナブラ演算子}

\[
\nabla=\left(\pd{}x,\pd{}y,\pd{}z\right)
\]


\begin{align*}
\grad V & =\nabla V\\
 & =\left(\pd Vx,\pd Vy,\pd Vz\right)
\end{align*}



\paragraph{例1.6}

図電磁2-4

クーロン
\[
\boldsymbol{E}=\frac{1}{4\pi\varepsilon_{0}}\frac{Q}{r^{2}}\hat{\boldsymbol{r}}
\]


\[
\boldsymbol{E}=-\grad V,V=\frac{1}{4\pi\varepsilon_{0}}\frac{Q}{r}\Rightarrow\boldsymbol{E}?
\]


\begin{align*}
E_{x} & =-\pd Vx\\
 & =-\pd{}x\left(\frac{1}{4\pi\varepsilon_{0}}\frac{Q}{\sqrt{x^{2}+y^{2}+z^{2}}}\right)\\
 & =\frac{1}{4\pi\varepsilon_{0}}\frac{Qx}{\left(x^{2}+y^{2}+z^{2}\right)^{\frac{3}{2}}}\\
 & =\frac{Qx}{4\pi\varepsilon_{0}r^{3}}
\end{align*}


同様にして
\begin{align*}
\boldsymbol{E} & =\frac{Q}{4\pi\varepsilon_{0}}\left(x,y,z\right)\\
 & =\frac{Q}{4\pi\varepsilon_{0}r^{3}}\boldsymbol{r}\\
 & =\frac{Q}{4\pi\varepsilon_{0}r^{2}}\hat{\boldsymbol{r}}
\end{align*}


別解: $\left(r,\theta,\varphi\right)$で解く。

\begin{align*}
\boldsymbol{E} & =\left(E_{r},E_{\theta},E_{\varphi}\right)\\
 & =-\grad V\\
 & =-\left(\pd Vr,\frac{1}{r}\pd V{\theta},\frac{1}{r\sin\theta}\pd V{\varphi}\right)\\
 & =-\left(-\pd Vr,0,0\right)\\
 & =\left(\frac{1}{4\pi\varepsilon_{0}}\frac{Q}{r^{2}},0,0\right)
\end{align*}



\subsection{ガウスの法則}

図電磁2-5

閉曲面$S$内で発生した$E$力線の本数=$S$を外向きに貫く$E$力線の本数=$Q$本

$\mathrm{d}S$近傍 図電磁2-6

\begin{align*}
\mathrm{d}S\text{を貫く本数} & =E\cdot\mathrm{d}S_{n}\\
 & =E\cos\theta\frac{\mathrm{d}S_{n}}{\cos\theta}\\
 & =E_{n}\mathrm{d}S
\end{align*}


\[
S\text{を貫く本数}=\oint_{S}E_{n}\mathrm{d}S
\]
\[
\therefore\oint_{S}E_{n}\mathrm{d}S=\frac{Q}{\varepsilon_{0}}
\]


これをガウスの法則という。


\paragraph{例1.7}

図電磁2-7
\begin{enumerate}
\item $r>r_{0}$の場合


半径$r$の球$S$→$S$上で$E=\text{一定},E_{n}=E$


\begin{align*}
\oint E_{n}\mathrm{d}S & =E\oint_{S}\mathrm{d}S\\
 & =4\pi r^{2}E=\frac{Q}{\varepsilon_{0}}
\end{align*}



\[
E=\frac{Q}{4\pi\varepsilon_{0}r^{2}}
\]


\item $r\leqq r_{0}$の場合


\[
\oint_{S}E_{n}\mathrm{d}S=4\pi r^{2}E=\frac{q}{\varepsilon_{0}}
\]



$q=0$より$E=0$

\end{enumerate}

\paragraph{微分形}

微小体積$\Delta v$に適用

図電磁2-8

電荷は連続的に分布し$\rho\left(x,y,z\right)\left[\mathrm{c/m^{2}}\right]$

\begin{align*}
\int_{S_{1}+S_{2}}E_{n}\mathrm{d}S & =-E_{x}\Delta y\Delta z\left(E_{x}+\pd{E_{x}}x\Delta x\right)\Delta y\Delta z\\
 & =\pd{E_{x}}x\Delta x\Delta y\Delta z\\
 & =\pd{E_{x}}x\Delta v
\end{align*}


同様にして
\[
\oint_{S}E_{n}\mathrm{d}S=\left(\pd{E_{x}}x,\pd{E_{y}}y,\pd{E_{z}}z\right)\Delta v
\]


また
\[
\oint_{S}E_{n}\mathrm{d}S=\frac{1}{\varepsilon_{0}}\rho\Delta v
\]


よって
\[
\pd{E_{x}}x+\pd{E_{y}}y+\pd{E_{z}}z=\frac{\rho}{\varepsilon_{0}}
\]
\[
\mathrm{div}\boldsymbol{E}=\nabla\cdot\boldsymbol{E}
\]


$\mathrm{div}$: divergence, 湧き出し, 発散

\[
\mathrm{div}\boldsymbol{E}=\frac{\rho}{\varepsilon_{0}}
\]


ガウスの法則 微分形


\subsection{ポアソン・ラプラスの式}

\[
\mathrm{div}\boldsymbol{E}=\frac{\rho}{\varepsilon_{0}},\boldsymbol{E}=-\grad V
\]
\[
\div\grad V=-\frac{\rho}{\varepsilon_{0}}
\]
\[
\nabla\cdot\left(\nabla V\right)=-\frac{\rho}{\varepsilon_{0}}
\]


$\nabla^{2}V=-\frac{\rho}{\varepsilon_{0}}$: ポアソンの式

\[
\pdd Vx+\pdd Vy+\pdd Vz=-\frac{\rho}{\varepsilon_{0}}
\]


$\rho$の分布から$V$を求めることができる。

$\rho=0$のとき

$\nabla^{2}V=0$: ラプラスの式


\subsection{静電界の保存則}

図電磁2-9

\begin{align*}
\oint_{C}\boldsymbol{E}\cdot\mathrm{d}\boldsymbol{s} & =\int_{AC_{1}}^{B}\boldsymbol{E}\cdot\mathrm{d}\boldsymbol{s}+\int_{BC_{2}}^{A}\boldsymbol{E}\cdot\mathrm{d}\boldsymbol{s}\\
 & =\int_{AC_{1}}^{B}-\int_{BC_{2}}^{A}\\
 & =0
\end{align*}


渦なしの条件


\paragraph{微分形}

図電磁2-10

\[
\oint_{c}\boldsymbol{E}\mathrm{d}\boldsymbol{s}=E_{x}\left(x,y-\frac{\Delta y}{2}\right)\Delta x+E_{y}\left(x+\frac{\Delta x}{2},y\right)\Delta y-E_{x}\left(x,y+\frac{\Delta y}{2}\right)\Delta x-E_{y}\left(x+\frac{\Delta x}{2},y\right)\Delta y
\]


\[
E_{x}\left(x,y-\frac{\Delta y}{2}\right)\Delta x-E_{x}\left(x,y+\frac{\Delta y}{2}\right)\Delta x=-\pd{E_{x}}y\Delta y
\]


\[
E_{y}\left(x+\frac{\Delta x}{2},y\right)\Delta y-E_{y}\left(x+\frac{\Delta x}{2},y\right)\Delta y=\pd{E_{y}}x\Delta x
\]


\[
\therefore\oint_{c}\boldsymbol{E}\cdot\mathrm{d}\boldsymbol{s}=\left(\pd{E_{x}}x-\pd{E_{y}}y\right)\Delta x\Delta y
\]


左辺=0より$\left(\rot\boldsymbol{E}\right)_{z}$

$\therefore\rot\boldsymbol{E}=0$: 微分形

\[
\rot\boldsymbol{E}=-\pd{\boldsymbol{B}}t\left(=0\right)
\]


$\rot\boldsymbol{E}=\nabla\times\boldsymbol{E}$


\paragraph{前回課題(1)}

$\mathrm{d}S$にある$\mathrm{d}q=\sigma\mathrm{d}S$が点Pにつくる$\mathrm{d}E$

\[
\mathrm{d}E=\frac{1}{4\pi\varepsilon_{0}}\frac{\sigma\mathrm{d}S}{r^{2}+h^{2}}
\]


\begin{align*}
E & =\int_{S}\mathrm{d}E_{z}\\
 & =\int_{b}^{a}\int_{0}^{2\pi}\frac{1}{4\pi\varepsilon_{0}}\frac{\sigma\mathrm{d}S}{r^{2}+h^{2}}\cdot\frac{h}{\sqrt{r^{2}+h^{2}}}\\
 & =\frac{\sigma h}{4\pi\varepsilon_{0}}\int_{b}^{a}\int_{0}^{2\pi}\frac{r}{\left(r^{2}+h^{2}\right)^{\frac{3}{2}}}\mathrm{d}\theta\mathrm{d}r\\
 & =\frac{\sigma h}{2\varepsilon_{0}}\int_{b}^{a}\frac{r}{\left(r^{2}+h^{2}\right)^{\frac{3}{2}}}\mathrm{d}r\\
 & =\frac{\sigma h}{2\varepsilon_{0}}\left(\frac{1}{\sqrt{b^{2}+h^{2}}}-\frac{1}{\sqrt{a^{2}+h^{2}}}\right)
\end{align*}


(向きは$z$方向)


\paragraph{前回課題(2)}

\[
W_{P\rightarrow O}=q\left(V_{O}-V_{P}\right)
\]


\begin{align*}
V_{P} & =-\int_{\infty}^{h}E\mathrm{d}s\\
 & =-\frac{\sigma}{2\varepsilon_{0}}\int_{\infty}^{h}\left(\frac{z}{\sqrt{z^{2}+b^{2}}}-\frac{z}{\sqrt{z^{2}+a^{2}}}\right)\mathrm{d}z\\
 & =\frac{\sigma}{2\varepsilon_{0}}\left[\sqrt{z^{2}+a^{2}}-\sqrt{z^{2}+b^{2}}\right]_{\infty}^{h}\\
 & =\frac{\sigma}{2\varepsilon_{0}}\left[\frac{a^{2}-b^{2}}{\sqrt{z^{2}+a^{2}}+\sqrt{z^{2}+b^{2}}}\right]_{\infty}^{h}\\
 & =\frac{\sigma}{2\varepsilon_{0}}\left(\frac{a^{2}-b^{2}}{\sqrt{h^{2}+a^{2}}+\sqrt{h^{2}+b^{2}}}\right)
\end{align*}


\[
V_{O}=V_{P}|_{h=0}=\frac{\sigma}{2\varepsilon_{0}}\left(a-b\right)
\]


\begin{align*}
W_{P\rightarrow O} & =q\left(V_{O}-V_{P}\right)\\
 & =\frac{\sigma}{2\varepsilon_{0}}\left(a-b\right)\left(1-\frac{a+b}{\sqrt{h^{2}+a^{2}}+\sqrt{h^{2}+b^{2}}}\right)
\end{align*}



\paragraph{演習問題問1}

(1) 図電磁2-11

\[
\begin{cases}
z=r\cos\theta\\
\sqrt{x^{2}+y^{2}+z^{2}}=r
\end{cases}
\]


\begin{align*}
V & =\frac{aq}{2\pi\varepsilon_{0}}\frac{r\cos\theta}{r^{3}}\\
 & =\frac{aq}{2\pi\varepsilon_{0}}\frac{\cos\theta}{r^{2}}
\end{align*}


(2) 極座標では
\[
\nabla=\left(\pd{}r,\frac{1}{r}\pd{}{\theta},\frac{1}{r\sin\theta}\pd{}{\varphi}\right)
\]
より
\begin{align*}
\left(E_{r},E_{\theta},E_{\varphi}\right) & =-\nabla V\\
 & =\left(\frac{aq}{\pi\varepsilon_{0}}\frac{\cos\theta}{r^{3}},\frac{aq}{2\pi\varepsilon_{0}}\frac{\sin\theta}{r^{3}},0\right)
\end{align*}


(3) 極座標なので、
\begin{align*}
\div\boldsymbol{E} & =\frac{1}{r^{3}}\pd{\left(r^{2}E_{r}\right)}r+\frac{1}{r\sin\theta}\pd{\left(\sin\theta E_{\theta}\right)}{\theta}+\frac{1}{r\sin\theta}\pd{E_{\varphi}}{\varphi}\\
 & =\frac{1}{r^{2}}\left(-\frac{aq}{\pi\varepsilon_{0}}\frac{\cos\theta}{r^{2}}\right)+\frac{1}{r\sin\theta}\left(\frac{aq}{2\pi\varepsilon_{0}}\frac{2\sin\theta\cos\theta}{r^{3}}\right)\\
 & =0
\end{align*}



\paragraph{演習問題問2}

\[
\nabla^{2}V=-\frac{\rho}{\varepsilon_{0}}
\]


$\pdd Vy=\pdd Vz=0$より、
\[
\nabla^{2}V=\pdd Vx=-\frac{\rho}{\varepsilon_{0}}
\]


\[
\pd Vx=-\frac{\rho}{\varepsilon_{0}}x+C_{1}
\]


\[
V=-\frac{\rho}{2\varepsilon_{0}}x^{2}+C_{1}x+C_{2}
\]


$V\left(0\right)=0$より$C_{2}=0$

$V\left(d\right)=V_{0}$より
\[
-\frac{\rho}{2\varepsilon_{0}}d^{2}+C_{1}d=V_{0}
\]


\[
C_{1}=\frac{V_{0}}{d}+\frac{\rho d}{2\varepsilon_{0}}
\]


\[
\therefore V=-\frac{\rho}{2\varepsilon_{0}}x^{2}+\left(\frac{V_{0}}{d}+\frac{\rho d}{2\varepsilon_{0}}\right)x
\]


\begin{align*}
E_{x} & =-\left(\grad V\right)_{x}\\
 & =-\pd Vx\\
 & =\frac{\rho}{\varepsilon_{0}}x-\frac{V_{0}}{d}-\frac{\rho d}{2\varepsilon_{0}}
\end{align*}



\paragraph{演習問題問3}

$V\left(a\right)=0$

\[
\oint_{S}E_{n}\mathrm{d}S=\begin{cases}
\frac{1}{\varepsilon_{0}}\pi\left(a^{2}-b^{2}\right)h\rho & \left(r>a\right)\\
\frac{1}{\varepsilon_{0}}\pi\left(r^{2}-b^{2}\right)h\rho & \left(b<r<a\right)\\
0 & \left(r<b\right)
\end{cases}
\]


左辺は
\begin{align*}
\oint_{S}E_{n}\mathrm{d}S & =E\oint_{S}\mathrm{d}S\\
 & =2\pi rhE
\end{align*}


$\therefore$
\[
E=\begin{cases}
\frac{\left(a^{2}-b^{2}\right)\rho}{2\varepsilon_{0}r} & \left(r>a\right)\\
\frac{\left(r^{2}-b^{2}\right)\rho}{2\varepsilon_{0}r} & \left(b<r<a\right)\\
0 & \left(r<b\right)
\end{cases}
\]


$V$を求める。
\begin{enumerate}
\item $r>a$のとき


\begin{align*}
V & =\int_{a}^{r}E\left(r'\right)\mathrm{d}r'\\
 & =-\frac{\left(a^{2}-b^{2}\right)\rho}{2\varepsilon_{0}}\left[\ln r'\right]_{a}^{r}\\
 & =-\frac{\left(a^{2}-b^{2}\right)\rho}{2\varepsilon_{0}}\ln\frac{r}{a}
\end{align*}


\item $b<r<a$のとき


\begin{align*}
V & =-\int_{a}^{r}E\left(r'\right)\mathrm{d}r'\\
 & =\frac{\rho}{2\varepsilon_{0}}\left[\frac{r'^{2}}{2}-b^{2}\ln r'\right]_{r}^{a}\\
 & =\frac{\rho}{4\varepsilon_{0}}\left(a^{2}-r^{2}-2b^{2}\ln\frac{a}{r}\right)
\end{align*}


\item $r<b$のとき


\begin{align*}
V & =-\int_{a}^{r}E\left(r'\right)\mathrm{d}r'\\
 & =-\int_{a}^{b}E\left(r'\right)\mathrm{d}r'-\int_{b}^{r}E\left(r'\right)\mathrm{d}r'\\
 & =\frac{\rho}{4\varepsilon_{0}}\left(a^{2}-b^{2}-2b^{2}\ln\frac{a}{b}\right)
\end{align*}


\end{enumerate}

\section*{第3回}


\section{導体}

導体⇔(誘電体=絶縁体)

図電磁3-1

$K=0$:
\begin{itemize}
\item $E_{\text{in}}=0$→導体は等電位 ($\boldsymbol{E}=-\grad V$)
\item $\text{表面}\perp\boldsymbol{E}$

\begin{itemize}
\item 図電磁3-2
\end{itemize}
\end{itemize}

\subsection{導体表面の電荷密度}

図電磁3-3

\[
\oint E_{n}\mathrm{d}S=\frac{q}{\varepsilon_{0}}
\]


\[
ES=\frac{\sigma S}{\varepsilon_{0}}
\]


\[
\sigma=\varepsilon_{0}E
\]



\paragraph{例}

図電磁3-4

\begin{align*}
4\pi a^{2}E & =\frac{Q}{\varepsilon_{0}}\\
 & =\frac{1}{\varepsilon_{0}}4\pi a^{2}\sigma
\end{align*}


\[
E=\frac{\sigma}{\varepsilon_{0}}
\]


図電磁3-5

\[
E\cdot2S=\frac{\sigma S}{\varepsilon_{0}}
\]


\[
E=\frac{\sigma}{2\varepsilon_{0}}
\]


図電磁3-6


\subsection{大地面}

地球の電位は常に0である。

図電磁3-7

図電磁3-8


\paragraph{影像法}

図電磁3-9

例2.2

図電磁3-10

図電磁3-11

\begin{align*}
\sigma & =-\varepsilon_{0}E\\
 & =-\frac{hq}{2\pi\left(h^{2}+r^{2}\right)^{\frac{3}{2}}}
\end{align*}


図電磁3-12

\begin{align*}
\int\sigma\mathrm{d}S & =-\int_{0}^{\infty}\frac{hq}{2\pi\left(h^{2}+g^{2}\right)^{\frac{3}{2}}}\cdot2\pi r\mathrm{d}r\\
 & =-hq\int_{0}^{\infty}\frac{r}{\left(h^{2}+g^{2}\right)^{\frac{3}{2}}}\mathrm{d}r\\
 & =-hq\left[\frac{1}{\sqrt{h^{2}+r^{2}}}\right]_{0}^{\infty}=-q
\end{align*}



\subsection{同心球面}

図電磁3-13
\begin{enumerate}
\item $r<r_{1}$のとき


\[
\int E\mathrm{d}S=0
\]
\[
E=0
\]


\item $r_{1}<r<r_{2}$のとき


\[
\underbrace{\int E\mathrm{d}S}_{=4\pi r^{2}E}=\frac{Q_{1}}{\varepsilon_{0}}
\]
\[
\therefore E=\frac{Q}{4\pi\varepsilon_{0}r^{2}}
\]


\item $r>r_{2}$のとき


\[
\int E\mathrm{d}S=\frac{Q_{1}+Q_{2}}{\varepsilon_{0}}
\]
\[
E=\frac{Q_{1}+Q_{2}}{4\pi\varepsilon_{0}r^{2}}
\]


\end{enumerate}
$V$は?

\[
V=-\int E\cdot\mathrm{d}s
\]


\begin{align*}
V_{2} & =-\int_{\infty}^{r_{2}}E\mathrm{d}r\\
 & =-\int_{\infty}^{r_{2}}\frac{Q_{1}+Q_{2}}{4\pi\varepsilon_{0}r^{2}}\mathrm{d}r\\
 & =\frac{Q_{1}+Q_{2}}{4\pi\varepsilon_{0}r_{2}}
\end{align*}


\begin{align*}
V_{1} & =V_{2}-\int_{r_{2}}^{r_{1}}E\mathrm{d}r\\
 & =V_{2}-\int_{r_{2}}^{r_{1}}\frac{Q}{4\pi\varepsilon_{0}r^{2}}\mathrm{d}r\\
 & =\frac{1}{4\pi\varepsilon_{0}}\left(\frac{Q_{1}}{r_{1}}+\frac{Q_{2}}{r_{2}}\right)
\end{align*}


ここで外球を接地すると?

$V_{2}=0$なので$Q_{1}+Q_{2}=0$ $\therefore Q_{2}=-Q_{1}$

\[
V_{1}=\frac{Q_{1}}{4\pi\varepsilon_{0}}\left(\frac{1}{r_{1}}-\frac{1}{r_{2}}\right)
\]



\paragraph{中空導体}

図電磁3-14

静電シールド


\subsection{静電容量}

図電磁3-15

$\mathrm{C=\frac{Q}{V}}$: 静電容量{[}F{]}

非接地導体\texttimes 2のとき、

図電磁3-16

\begin{align*}
V & =Ed\\
 & =\frac{\sigma}{\varepsilon_{0}}d\\
 & =\frac{d}{\varepsilon_{0}S}Q
\end{align*}


図電磁3-17

\[
4\pi r^{2}E=\frac{Q}{\varepsilon_{0}}
\]


\[
E=\frac{Q}{4\pi\varepsilon_{0}r^{2}}
\]


\begin{align*}
V & =-\int_{b}^{a}\frac{Q}{4\pi\varepsilon_{0}r^{2}}\mathrm{d}r\\
 & =\frac{Q}{4\pi\varepsilon_{0}}\left(\frac{1}{a}-\frac{1}{b}\right)
\end{align*}


\begin{align*}
C & =\frac{Q}{V}\\
 & =\frac{4\pi\varepsilon_{0}}{\frac{1}{a}-\frac{1}{b}}
\end{align*}


図電磁3-18

$C=\frac{Q}{V}$、非接地導体\texttimes 1

\[
4\pi r^{2}E=\frac{Q}{\varepsilon_{0}}
\]


\[
E=\frac{Q}{4\pi\varepsilon_{0}r^{2}}
\]


\begin{align*}
V & =-\int_{\infty}^{a}\frac{Q}{4\pi\varepsilon_{0}r^{2}}\mathrm{d}r\\
 & =\frac{Q}{4\pi\varepsilon_{0}a}
\end{align*}


\begin{align*}
C & =\frac{Q}{V}\\
 & =4\pi\varepsilon_{0}a
\end{align*}



\subsection{電界のエネルギー}

図電磁3-19

\[
W=QV
\]


エネルギー

\begin{align*}
U & =W\\
 & =\frac{1}{2}CV^{2}\\
 & =\frac{1}{2}QV
\end{align*}


図電磁3-20

\[
W=\int\mathrm{d}W=\int_{0}^{Q}\frac{q}{c}\mathrm{d}q=\frac{Q^{2}}{2c}
\]



\paragraph{例題2-2}

図電磁3-21

くっつく前は
\[
C=4\pi\varepsilon_{0}r
\]
\[
U=2\cdot\frac{+Q^{2}}{2C}=\frac{Q^{2}}{4\pi\varepsilon_{0}r}
\]


くっついた後は
\[
U'=\frac{\left(2Q\right)^{2}}{2C'}=\frac{Q}{2\pi\varepsilon_{0}\sqrt[3]{2}r}
\]


\[
\frac{U'}{U}=\frac{4}{2\sqrt[3]{2}}=1.59
\]



\subsection{導体に働く力}

図電磁3-22

$\Delta x$の移動に要する仕事
\[
-F_{x}\Delta x
\]


$F_{x}$: $F$の$x$方向成分(外力)

$Q=\text{一定}$→エネルギー増分

\[
\Delta W_{e}=-F_{x}\Delta x
\]


\[
F_{x}=-\left(\frac{\Delta W_{e}}{\Delta x}\right)_{Q}\rightarrow-\left(\pd{W_{e}}x\right)_{Q}
\]


図電磁3-23

\begin{align*}
W_{e} & =\frac{Q^{2}}{2C}\\
 & =\frac{Q^{2}x}{2\varepsilon_{0}S}
\end{align*}


$C=\frac{\varepsilon_{0}S}{x},Q=CV$

\begin{align*}
F & =-\left(\pd{W_{e}}x\right)_{Q}\\
 & =\frac{-Q^{2}}{2\varepsilon_{0}S}\\
 & =-\frac{\varepsilon_{0}S}{2}\left(\frac{V}{x}\right)^{2}
\end{align*}


図電磁3-24

$V$一定: $x\rightarrow x+\Delta x\Rightarrow Q\rightarrow Q+\Delta Q$

電源が$V\Delta Q$の仕事をする。

\[
\begin{cases}
\Delta W_{e}=-F\Delta x+V\Delta Q\\
\Delta W_{e}=\frac{1}{2}V\Delta Q\left(\because W_{e}=\frac{1}{2}VQ\right)
\end{cases}
\]


\[
\therefore F\Delta x=\frac{1}{2}V\Delta Q=\Delta W_{e}
\]


\[
F=\left(\pd{W_{e}}x\right)_{V}
\]



\paragraph{前回課題(1)}

図電磁3-25

\[
\rho\left(r\right)=\frac{\rho_{0}}{r^{4}}
\]


\[
\frac{1}{r^{2}}\pd{}r\left(r^{2}\pd Vr\right)=-\frac{\rho_{0}}{\varepsilon_{0}r^{4}}
\]


\[
\pd{}r\left(r^{2}\pd Vr\right)=-\frac{\rho_{0}}{\varepsilon_{0}r^{2}}
\]


\[
r^{2}\pd Vr=\frac{\rho_{0}}{\varepsilon_{0}r}+C_{1}
\]


\[
\pd Vr=\frac{\rho_{0}}{\varepsilon_{0}r^{3}}+\frac{C_{1}}{r^{2}}
\]


\[
V=-\frac{\rho_{0}}{2\varepsilon_{0}r^{2}}-\frac{C_{1}}{r}+C_{2}
\]


$V\left(\infty\right)=0$より$C_{2}=0$

$V\left(a\right)=0$より$C_{1}=-\frac{\rho_{0}}{2\varepsilon_{0}a}$

\[
\therefore V=\frac{\rho_{0}}{2\varepsilon_{0}r}\left(\frac{1}{a}-\frac{1}{r}\right)
\]



\paragraph{前回課題(2)}

\[
\int E\mathrm{d}S=\frac{1}{\varepsilon_{0}}\left(q+\int_{a}^{r}\frac{\rho_{0}}{r^{4}}4\pi r^{3}\mathrm{d}r\right)
\]


\[
4\pi r^{2}E=\frac{1}{\varepsilon_{0}}\left(q+4\pi\rho_{0}\left(\frac{1}{a}-\frac{1}{r}\right)\right)
\]
\[
E=\frac{q}{4\pi\varepsilon_{0}r^{2}}+\frac{\rho_{0}}{\varepsilon_{0}r^{2}}\left(\frac{1}{a}-\frac{1}{r}\right)
\]


\begin{align*}
V_{a} & =-\int_{\infty}^{a}E\mathrm{d}r\\
 & =\left[\frac{\rho_{0}}{\varepsilon_{0}}\left(\frac{1}{ar}-\frac{1}{2r^{2}}\right)+\frac{q}{4\pi\varepsilon_{0}r}\right]_{\infty}^{a}\\
 & =\frac{\rho_{0}}{2\varepsilon_{0}a^{2}}+\frac{q}{4\pi\varepsilon_{0}a}
\end{align*}


$V_{a}=0$より、
\[
q=-\frac{2\pi\rho_{0}}{a}
\]



\paragraph{演習課題問1}

図電磁3-26

\begin{align*}
F & =\frac{q_{1}}{4\pi\varepsilon_{0}}\left(\frac{q_{2}}{a^{2}}+\frac{q_{1}}{\left(2a\right)^{2}}+\frac{q_{2}}{\left(3a\right)^{2}}\right)\\
 & =\frac{q_{1}}{4\pi\varepsilon_{0}a^{2}}\left(\frac{q_{1}}{4}+\frac{10q_{2}}{9}\right)
\end{align*}


\begin{align*}
V & =\frac{1}{4\pi\varepsilon_{0}}\left(\frac{q_{2}}{a}-\frac{q_{1}}{2a}-\frac{q_{2}}{3a}\right)\\
 & =\frac{1}{4\pi\varepsilon_{0}a}\left(-\frac{q_{1}}{2}+\frac{2q_{2}}{3}\right)\\
V' & =\frac{1}{4\pi\varepsilon_{0}}\left(\frac{q_{2}}{\sqrt{2}a}-\frac{q_{1}}{2a}-\frac{q_{2}}{\sqrt{10}a}\right)\\
 & =\frac{1}{4\pi\varepsilon_{0}a}\left(-\frac{q_{1}}{2}+\frac{\left(\sqrt{5}-1\right)q_{1}}{\sqrt{10}}\right)
\end{align*}


\begin{align*}
W & =\Delta U\\
 & =q_{1}\left(V'-V\right)\\
 & =\frac{q_{1}q_{2}}{4\pi\varepsilon_{0}a}\left(\frac{\sqrt{5}-1}{\sqrt{10}}-\frac{2}{3}\right)
\end{align*}



\paragraph{演習問題問2(1)}

図電磁3-27

\[
\int E\mathrm{d}S=\frac{q}{\varepsilon_{0}}
\]


\[
4\pi r^{2}E=\frac{q}{\varepsilon_{0}}
\]


\[
E=\frac{q}{4\pi\varepsilon_{0}r^{2}}
\]


\begin{align*}
V & =-\int_{b}^{a}E\mathrm{d}r\\
 & =\frac{q}{4\pi\varepsilon_{0}}\left(\frac{1}{a}-\frac{1}{b}\right)
\end{align*}


\[
C=\frac{q}{V}=\frac{4\pi\varepsilon_{0}}{\frac{1}{a}-\frac{1}{b}}=\frac{4\pi\varepsilon_{0}ab}{b-a}
\]



\paragraph{演習問題問2(2)}

非接地\texttimes 1

図電磁3-28

\[
\begin{cases}
E_{1}=\frac{-q'}{4\pi\varepsilon_{0}r^{2}}\\
E_{2}=\frac{q-q'}{4\pi\varepsilon_{0}r^{2}}
\end{cases}
\]


\begin{align*}
V_{b} & =-\int_{a}^{b}E_{1}\mathrm{d}r\\
 & =-\int_{a}^{b}\frac{-q'}{4\pi\varepsilon_{0}r^{2}}\mathrm{d}r\\
 & =\frac{q'}{4\pi\varepsilon_{0}}\left(\frac{1}{a}-\frac{1}{b}\right)
\end{align*}


\begin{align*}
V_{b} & =-\int_{\infty}^{b}E_{2}\mathrm{d}r\\
 & =-\int_{\infty}^{b}\frac{q-q'}{4\pi\varepsilon_{0}r^{2}}\mathrm{d}r\\
 & =\frac{q-q'}{4\pi\varepsilon_{0}b}
\end{align*}


上の2つの式は等しいので、
\[
\frac{q'}{q}=\frac{a}{b}
\]


$q'$を$V_{b}$の式に代入

\[
V_{b}=\frac{aq}{4\pi\varepsilon_{0}b}\left(\frac{1}{a}-\frac{1}{b}\right)
\]


\begin{align*}
C & =\frac{q}{V_{0}}\\
 & =\frac{4\pi\varepsilon_{0}b^{2}}{b-a}
\end{align*}



\paragraph{演習問題問2(3)}

\begin{align*}
f & =\frac{1}{4\pi b^{2}}F\\
 & =-\frac{1}{4\pi b^{2}}\left(\pd{W_{e}}b\right)_{q}\\
 & =-\frac{1}{4\pi b^{2}}\left(\pd{}b\frac{q^{2}}{2C}\right)_{q}\\
 & =-\frac{q^{2}}{8\pi b^{2}}\left(\pd{}b\frac{1}{4\pi\varepsilon_{0}}\left(\frac{1}{a}-\frac{1}{b}\right)\right)\\
 & =\frac{-q^{2}}{32\pi^{2}\varepsilon_{0}b^{2}}\frac{1}{b^{2}}\\
 & =\frac{-q^{2}}{32\pi^{2}\varepsilon_{0}b^{4}}
\end{align*}


別解

Maxwell応力
\[
\sigma=\frac{-q}{4\pi b^{2}}
\]


\begin{align*}
E & =\frac{\sigma}{\varepsilon_{0}}\\
 & =\frac{-q}{4\pi\varepsilon_{0}b^{2}}
\end{align*}


\begin{align*}
f & =\frac{1}{2}\varepsilon_{0}E^{2}\left(r=b\right)\\
 & =\frac{\varepsilon_{0}}{2}\frac{q^{2}}{16\pi^{2}\varepsilon_{0}^{2}b^{4}}
\end{align*}



\paragraph{演習問題問3}

図電磁3-30

球電荷の半径を$r\rightarrow r+\mathrm{d}r$とするのに必要な仕事$\mathrm{d}w$

\[
\mathrm{d}w=V_{1}\mathrm{d}q=\frac{q_{r}}{4\pi\varepsilon_{0}r}\mathrm{d}q
\]


\[
q_{r}=\frac{4}{3}\pi r^{3}\rho
\]


\[
\therefore q_{r}=\frac{r^{3}}{a^{3}Q}
\]


および

\begin{align*}
\mathrm{d}q & =4\pi r^{2}\mathrm{d}r\cdot\rho\\
 & =\frac{3Qr^{2}}{a^{3}}\mathrm{d}r
\end{align*}


より、
\[
\mathrm{d}w=\frac{3Q^{2}r^{4}}{4\pi\varepsilon_{0}a^{3}}\mathrm{d}r
\]


よって、$q:0\rightarrow Q$に必要な$W$は、
\begin{align*}
W & =\int\mathrm{d}w=\int_{0}^{a}\frac{3Q^{2}r^{4}}{4\pi\varepsilon_{0}a^{6}}\mathrm{d}r\\
 & =\frac{3Q^{2}}{20\pi\varepsilon_{0}a}
\end{align*}

\end{document}
