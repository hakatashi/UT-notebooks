%% LyX 2.2.2 created this file.  For more info, see http://www.lyx.org/.
%% Do not edit unless you really know what you are doing.
\documentclass[english]{article}
\usepackage[T1]{fontenc}
\usepackage[utf8]{inputenc}
\usepackage[a5paper]{geometry}
\geometry{verbose,tmargin=2cm,bmargin=2cm,lmargin=1cm,rmargin=1cm}
\setlength{\parskip}{\smallskipamount}
\setlength{\parindent}{0pt}
\usepackage{textcomp}
\usepackage{amsmath}
\usepackage{amssymb}

\makeatletter
%%%%%%%%%%%%%%%%%%%%%%%%%%%%%% User specified LaTeX commands.
\usepackage[dvipdfmx]{hyperref}
\usepackage[dvipdfmx]{pxjahyper}

\makeatother

\usepackage{babel}
\begin{document}

\title{2016-A 数理手法V}

\author{教員: 藤原 入力: 高橋光輝}

\maketitle
\global\long\def\pd#1#2{\frac{\partial#1}{\partial#2}}
\global\long\def\d#1#2{\frac{\mathrm{d}#1}{\mathrm{d}#2}}
\global\long\def\pdd#1#2{\frac{\partial^{2}#1}{\partial#2^{2}}}
\global\long\def\dd#1#2{\frac{\mathrm{d}^{2}#1}{\mathrm{d}#2^{2}}}
\global\long\def\e{\mathrm{e}}
\global\long\def\i{\mathrm{i}}
\global\long\def\j{\mathrm{j}}
\global\long\def\grad{\mathrm{grad}}
\global\long\def\rot{\mathrm{rot}}
\global\long\def\div{\mathrm{div}}
\global\long\def\diag{\mathrm{diag}}
\global\long\def\rank{\mathrm{rank}}
\global\long\def\cov{\mathrm{Cov}}


\section*{第1回}

\paragraph{講義の内容}

講義の内容: 確率と確率過程

比較的user-orientedな内容

休日授業日は休み

評価はレポート数回
\begin{enumerate}
\item 統計と確率
\item 確率過程

\begin{itemize}
\item 確率微分方程式
\end{itemize}
\end{enumerate}

\paragraph{参考書}
\begin{enumerate}
\item 『確率・統計1』縄田 (丸善) 東京大学工学教程シリーズ
\item 『確率と確率過程』伏見 (朝倉)
\item 『入門 確率過程』松原 (東京図書)

\begin{itemize}
\item 主に後半の内容
\end{itemize}
\item 『確率論の基礎』伊藤 (岩波)
\item 『確率論』伊藤 (岩波)
\end{enumerate}

\section{事象と確率}

試行: 標本点$\Omega$ 例) $\left\{ 1,2,3,4,5,6\right\} $

事象: 例) $A_{1}=\left\{ 1,3,5\right\} ,A_{2}=\left\{ 2,4,6\right\} ,A_{3}=\left\{ 3,4,5,6\right\} $

「$A_{i}$は$\Omega$の部分集合である」と(定義上)言える。

余事象$A_{i}^{c}$: $A_{i}^{c}=\left\{ \omega\in\Omega,\omega\notin A_{i}\right\} $

空集合$\phi$: $\phi=\Omega^{c}$

\paragraph{演算の定義}
\begin{enumerate}
\item 和

\[
A_{1}\cup A_{2}=\left\{ \omega\in\Omega,\omega\in A_{1}\text{OR}\omega\in A_{2}\right\} 
\]

\item 積

\[
A_{1}\cap A_{2}=\left\{ \omega\in\Omega,\omega\in A_{1}\text{AND}\omega\in A_{2}\right\} 
\]

\item $A_{1},A_{2}$が互いに排他的である時、
\[
A_{1}\cap A_{2}=\phi
\]
\end{enumerate}

\subsection{事象の公理}

$\mathcal{A}$: 事象の全体集合
\begin{enumerate}
\item $\Omega\in\mathcal{A}$
\item $A\in\mathcal{A}$ならば$A^{c}\in\mathcal{A}$
\item $A_{1},A_{2},A_{3}\cdots\in\mathcal{A}$ならば
\[
A_{1}\cup A_{2}\cup A_{3}\cdots=\bigcup_{i=1}^{\infty}A_{i}\in\mathcal{A}
\]
\end{enumerate}
1. から 3. を満足する集合をBorel集合という。

公理1.\textasciitilde{}3.を認めれば、
\begin{enumerate}
\item $\phi\in\mathcal{A}$ (1.と2.から)
\item $A_{1},A_{2},\cdots\in\mathcal{A}$ならば
\[
A_{1}\cap A_{2}\cap\cdots=\bigcap_{i=1}^{\infty}A_{i}\in\mathcal{A}
\]

($\left(\bigcap_{i=1}^{n}A_{i}\right)^{c}=\bigcup_{i=1}^{n}A_{i}^{c}$から)
\end{enumerate}

\subsection{確率の定義(Kolmogorovの定理)}

次の特徴を持つ数を定義する。
\begin{enumerate}
\item $0\leq P\left(A\right)\leq1$
\item $P\left(\Omega\right)=1$ ($\Omega$は$\mathcal{A}$の最大の部分空間)
\item $A_{1},A_{2},A_{3}\cdots$が排他的であれば、
\[
P\left(\bigcup_{i=1}^{\infty}A_{i}\right)=\sum_{i=1}^{\infty}P\left(A_{i}\right)
\]

(完全加法性)
\end{enumerate}
このような数を確率(測度)という。

これら$\left(\Omega,\mathcal{A},P\right)$の組み合わせで確率空間が構成される。

\subsection{確率の性質}
\begin{enumerate}
\item $P\left(\phi\right)=0$
\item $A_{1},A_{2}\cdots\in\mathcal{A}$、$A_{1},A_{2},\cdots,A_{n}$が排他的であれば
\[
P\left(\bigcup_{i=1}^{n}A_{i}\right)=\sum_{i=1}^{n}P\left(A_{i}\right)
\]
\item $P\left(A^{c}\right)=1-P\left(A\right)$
\item $P\left(\bigcup_{i}A_{i}\right)=1-P\left(\bigcap_{i}A_{i}^{c}\right)$
\item $B_{1},B_{2},\cdots$が排他的で$\bigcup_{i}B_{i}=\Omega$ならば、任意の$A$に対して
\[
P\left(A\right)=\sum_{i}P\left(A\cap B_{i}\right)
\]
\item 和の公式$P\left(A_{1}\cup A_{2}\right)=P\left(A_{1}\right)+P\left(A_{2}\right)-P\left(A_{1}\cap A_{2}\right)$
\item 和の公式2 
\begin{align*}
P\left(A_{1}\cup A_{2}\cup A_{3}\right)= & P\left(A_{1}\right)+P\left(A_{2}\right)+P\left(A_{3}\right)\\
 & -P\left(A_{1}\cap A_{2}\right)-P\left(A_{2}\cap A_{3}\right)-P\left(A_{3}\cap A_{1}\right)\\
 & +P\left(A_{1}\cap A_{2}\cap A_{3}\right)
\end{align*}
\item 和の公式3 
\begin{align*}
P\left(\bigcup_{i=1}^{n}A_{i}\right)= & P\left(A_{1}\right)+P\left(A_{2}\right)+\cdots+P\left(A_{n}\right)\\
 & -\sum_{i<j}P\left(A_{i}\cap A_{j}\right)\\
 & +\sum_{i<j<k}P\left(A_{i}\cap A_{j}\cap A_{k}\right)\\
 & \cdots\\
 & +\left(-1\right)^{m+1}P\left(\bigcap_{i=1}^{m}A_{i}\right)
\end{align*}

(証明は帰納法など)
\end{enumerate}

\section*{第2回}

\paragraph{講義全体の流れ}
\begin{enumerate}
\item 事象と確率

\begin{enumerate}
\item Borel集合
\item 確率の測度とKolomogorovの公理
\item 確率の性度
\item 確率の連続性
\item 条件確率
\item 独立性
\end{enumerate}
\item 確率変数

\begin{enumerate}
\item 確率変数と分布関数
\item 様々な分布関数
\end{enumerate}
\end{enumerate}
評価は数回のレポートで行う。

可能な限り情報基盤センターのPCを使って計算を行ってもらう。

\subsection{確率の連続性}

1. 無限個の系列$A_{1},A_{2},A_{3}\cdots$に包含関係$A_{1}\subset A_{2}\subset A_{3}\cdots$がある時、
\[
A\equiv\bigcup_{i=1}^{\infty}A_{i}
\]
とすると、
\[
P\left(A\right)=\lim_{i\rightarrow\infty}P\left(A_{i}\right)
\]

$B_{1}=A_{1},B_{2}=A_{2}\cap A_{1}^{c},B_{3}=A_{3}\cap A_{2}^{c}\cdots$、$B_{1},B_{2},B_{3}$は互いに素とする。

\begin{align*}
 & \bigcup_{i=1}^{n}B_{i}=A_{m}\\
\rightarrow\therefore & P\left(A_{m}\right)=P\left(\bigcup_{i=1}^{n}B_{i}\right)=\sum_{i=1}^{m}P\left(B_{i}\right)\\
\therefore & \lim_{n\rightarrow\infty}P\left(A_{n}\right)=\sum_{i=1}^{\infty}P\left(B_{i}\right)
\end{align*}

2. $A_{1},A_{2},\cdots,A_{n}\cdots$が$A_{1}\supset A_{2}\supset\cdots\supset A_{n}\supset\cdots$ならば
\[
P\left(A\right)=\lim_{n\rightarrow\infty}P\left(A_{n}\right)
\]

$A_{1}^{c}\subset A_{2}^{c}\subset A_{3}^{c}\cdots$

1. から
\[
P\left(A^{c}\right)=\lim_{i\rightarrow\infty}P\left(A_{i}^{c}\right)
\]

\[
\therefore P\left(A\right)=\lim_{i\rightarrow\infty}P\left(A_{i}\right)
\]

$B_{1},B_{2},\cdots$は互いに素だから($A_{n}\rightarrow A$)、
\[
P\left(A\right)=\sum_{i=1}^{n}P\left(B_{i}\right)
\]

\[
\therefore P\left(A\right)=\lim_{i\rightarrow\infty}P\left(A_{i}\right)
\]


\subsection{条件付き確率}

$B$: 最初に起きる事象

$A$: 次に起きる事象

赤玉が7個、白玉が3個入ってる箱から連続して2個玉を取り出す作業を考える。

赤→赤: $\frac{7}{10}\times\frac{6}{9}$

白→赤: $\frac{7}{10}\times\frac{3}{9}$

赤→白: $\frac{3}{10}\times\frac{7}{9}$

白→白: $\frac{3}{10}\times\frac{2}{9}$

\[
P\left(A|B\right)=\frac{P\left(A\cap B\right)}{P\left(B\right)}\:\left(P\left(B\right)\neq0\right)
\]

最小に$B$が起こった条件下で、次に$A$が起こる確率

\paragraph{条件付き確率の性質}
\begin{enumerate}
\item a
\item a
\item $B_{1},B_{2}\cdots$が互いに素で$\bigcup_{i}B_{i}=\Omega$であるなら、任意の$A$に対して、
\[
P\left(A\right)=\sum_{i}P\left(B_{i}\right)P\left(A|B_{i}\right)
\]
$P\left(A\right)>0$であるなら、
\[
P\left(B_{j}|A\right)=\frac{P\left(B_{j}\right)P\left(A|B_{j}\right)}{\sum_{i}P\left(B_{i}\right)P\left(A|B_{i}\right)}
\]

\begin{align*}
\because P\left(B_{j}|A\right) & =\frac{P\left(B_{j}\cap A\right)}{P\left(A\right)}=\frac{P\left(B_{j}\right)P\left(A|B_{j}\right)}{P\left(A\right)}\\
 & =\frac{P\left(B_{j}\right)P\left(A|B_{j}\right)}{\sum_{i}P\left(B_{i}\right)P\left(A|B_{i}\right)}
\end{align*}

\end{enumerate}

\subsection{独立性}

\[
P\left(A|B\right)\Leftrightarrow\text{「AとBは独立する」}\Leftrightarrow P\left(A\cap B\right)=P\left(A\right)\cdot P\left(B\right)\Leftrightarrow\frac{P\left(A\cap B\right)}{P\left(B\right)}=P\left(A\right)
\]

例として、

$\Omega=\left\{ 1,2,3,4\right\} $

$A=\left\{ 1,2\right\} $: 1または2が出る

$B=\left\{ 1,3\right\} $: 1または3が出る

$C=\left\{ 1,4\right\} $: 1または4が出る

\[
P\left(A\right)=\frac{1}{2},P\left(B\right)=\frac{1}{2},P\left(C\right)=\frac{1}{2}
\]

\[
A\cap B=\left\{ 1\right\} ,A\cap C=\left\{ 1\right\} ,B\cap C=\left\{ 1\right\} 
\]

\[
P\left(A\cap B\right)=\frac{1}{4},P\left(A\cap C\right)=\frac{1}{4},P\left(B\cap C\right)=\frac{1}{4}
\]

$P\left(A\cap B\right)=P\left(A\right)P\left(B\right)$: $A$と$B$は独立

\[
A\cap B\cap C=\left\{ 1\right\} 
\]

\[
P\left(C|A\cap B\right)=1\neq P\left(A\cap B\right)P\left(C\right)
\]
から、$A$と$B$、$B$と$C$、$C$と$A$が独立であっても、$P\left(C|A\cap B\right)=P\left(C\right)P\left(A\cap B\right)$とはならない。

\[
P\left(A_{i_{1}}\cap A_{i_{2}}\cap\cdots\right)=P\left(A_{i_{1}}\right)P\left(A_{i_{2}}\right)P\left(A_{i_{3}}\right)\cdots
\]
が成立するとき、$A_{1},A_{2},A_{3}\cdots$が独立であるという。

\paragraph{2点分布}

$f\left(0\right)=1-p\equiv q$

$f\left(1\right)=p$ (1が出る確率が$p$)

\[
p^{k}q^{1-k}
\]

→ベルヌーイ試行

\paragraph{2項分布}

ベルヌーイ試行を$n$回

1が$k$回出現

0が$n-k$回出現

\[
\left(\begin{array}{c}
n\\
k
\end{array}\right)p^{k}q^{n-k}
\]


\paragraph{Poisson分布}

\[
F\left(N\right)=\sum_{n<N}f\left(n\right)
\]

\[
\left(\text{二項分布}\right)\xrightarrow{n\rightarrow\infty}\text{ポアソン分布}
\]


\section*{第3回}

\paragraph{レポート}

ITC-LMSで提出する。

必ずPDFで提出のこと。

ファイル名に学籍番号と氏名を明記。

また、資料もITC-LMSで配布するので必ず確認のこと。

\section{確率変数}

\subsection{確率変数と分布関数}

\[
\left\{ \Omega,\mathcal{A},P\right\} 
\]


\paragraph{確率変数}

全ての実数$x$に対して
\[
\left\{ \omega;X\left(\omega\right)\leqq x\right\} \in\mathcal{A}
\]
を満足するとき、$X\left(\omega\right)$を確率変数という。

すなわち、$\omega\rightarrow X\left(\omega\right)$を対応させた上で、$X\left(\omega\right)\leqq x,\omega\in\mathcal{A}$ならば確率変数である。

\paragraph{分布関数}

\[
F_{x}\left(x\right)=P\left(\left\{ \omega;X\left(\omega\right)\leqq x\right\} \right)
\]


\paragraph{分布関数の性質}
\begin{enumerate}
\item $a\leq b\Rightarrow F\left(a\right)\leq F\left(b\right)$
\item 右連続 
\[
\lim_{x\rightarrow+a}F\left(x\right)=F\left(a\right)
\]
\item $\lim_{x\rightarrow-\infty}F\left(x\right)=0$
\item $\lim_{x\rightarrow+\infty}F\left(x\right)=1$
\end{enumerate}
ここで
\[
f\left(x\right)=F\left(x_{k}\right)-F\left(x_{k-1}\right)
\]
(ただし$x_{k}=X\left(\omega_{k}\right)$)は確率関数である。

\subsection{様々な分布}

\paragraph{ポアソン分布}

(プリントも参照)

\[
\Delta P_{0}\left(\Delta t\right)\sim-\lambda\Delta tP_{0}\left(t\right)
\]

これが0回起きる場合、
\[
\d{P_{0}\left(t\right)}t=-\lambda P_{0}\left(t\right)
\]
\[
P_{0}\left(t\right)=\e^{-\lambda t}
\]

$n$回起きる場合、
\[
P_{n}\left(t+\Delta t\right)=\left(1-\lambda\Delta t\right)P_{n}\left(t\right)+\lambda\Delta tP_{n-1}\left(t\right)
\]

これは$n$に対する差分方程式。

$\Delta t\rightarrow0$で
\[
\frac{P_{n}\left(t+\Delta t\right)-P_{n}\left(t\right)}{\Delta t}=-\lambda\left(P_{n}\left(t\right)-P_{n-1}\left(t\right)\right)
\]

\[
\d{}tP_{n}\left(t\right)=\lambda\left(P_{n-1}\left(t\right)-P_{n}\left(t\right)\right)
\]

これを初期条件$P_{0}\left(t\right)=\e^{-\lambda t}$で解くと、
\[
P_{n}\left(t\right)=\e^{-\lambda t}\frac{\left(\lambda t\right)^{n}}{n!}
\]
となる。

離散分布について
\[
F\left(X\left(\omega_{k}\right)\right)=\sum_{i=1}^{k}P\left(\omega_{i}\right)
\]
\[
f\left(X\left(\omega_{k}\right)\right)=F\left(X\left(\omega_{k}\right)\right)-F\left(X\left(\omega_{k-1}\right)\right)
\]
であったものを、連続分布について
\[
F\left(x\right)=\int_{-\infty}^{x}f\left(v\right)\mathrm{d}v
\]
\[
f\left(x\right)=\d{}xF\left(x\right)
\]
とする。

\paragraph{正規分布}

\[
f\left(x\right)=\frac{1}{\sqrt{2\pi}\sigma}\exp\left[-\frac{1}{2}\left(\frac{x-\mu}{\sigma}\right)^{2}\right]
\]

しばしば$\mathcal{N}\left(\mu,\sigma^{2}\right)$と書く。

\paragraph{コーシー分布}

\[
f\left(x\right)=\frac{1}{\pi}\frac{\alpha}{\left(x-\mu\right)^{2}+\alpha^{2}}
\]

$C\left(\mu,\alpha\right)$と書く。

\subsection{中心極限定理1 (ド・モアブル-ラプラスの定理)}

$X$が二項分布$B\left(n;p\right)$に従うとき、$n$が大きいならば$X$は$\mu=np,\sigma^{2}=np\left(1-p\right)$であるような正規分布$\mathcal{N}\left(np;np\left(1-p\right)\right)$に従う。

\paragraph{証明}

二項分布
\[
P\left(a\leq\frac{X_{n}-np}{\sqrt{npq}}\leq b\right)\approx^{n\rightarrow\infty}\int_{a}^{b}\frac{\e^{-\frac{1}{2}x^{2}}}{\sqrt{2\pi}}\mathrm{d}x
\]
を示せばよい。($q=1-p$)

∵
\[
P\left(x=k\right)=_{n}C_{k}p^{k}q^{n-k}
\]

すなわち$n\rightarrow\infty$で
\[
\sum_{np+a\sqrt{npq}\leqq k\leqq np+b\sqrt{npq}}{}_{n}C_{k}p^{k}q^{n-k}\rightarrow\frac{1}{2\pi}\int_{a}^{b}\e^{-\frac{1}{2}x^{2}}\mathrm{d}x
\]
を示す。
\begin{enumerate}
\item $n\rightarrow\infty$なら$k\rightarrow\infty$となるはず。
\item $n\gg1$で$n!\approx\sqrt{2\pi n}n^{n}\e^{-n}$ (スターリングの公式)

(ガンマ関数
\[
\Gamma\left(n\right)=\left(n+1\right)!
\]
の議論で後々出てくる。)
\end{enumerate}
これを用いて
\begin{align*}
_{n}C_{k}=\frac{n!}{k!\left(n-k\right)!} & \approx\frac{\sqrt{2\pi n}n^{n}\e^{-n}}{\sqrt{2\pi k}k^{k}\e^{-k}+\sqrt{2\pi\left(n-k\right)}\left(n-k\right)^{n-k}\e^{-\left(n-k\right)}}\\
 & =\frac{1}{\sqrt{2\pi n\cdot\frac{k}{n}\cdot\frac{n-k}{n}}}\left(\frac{n}{k}\right)^{k}\left(\frac{n}{n-k}\right)^{n-k}
\end{align*}
\[
_{n}C_{k}p^{k}q^{n-k}\simeq\frac{1}{\sqrt{2\pi n\cdot\frac{k}{n}\cdot\frac{n-k}{n}}}\left(\frac{n}{k}p\right)^{k}\left(\frac{n}{n-k}q\right)^{n-k}
\]

\[
F\equiv\left(\frac{n}{k}p\right)^{k}\left(\frac{n}{n-k}q\right)^{n-k}
\]
を評価する。

$\delta\equiv k-np$とすると、
\begin{align*}
\log F & =-\left(np+\delta\right)\log\left(1+\frac{\delta}{np}\right)-\left(nq-k\right)\log\left(1-\frac{\delta}{nq}\right)\\
 & \approx-\frac{\delta^{2}}{2n}\left(\frac{1}{p}+\frac{1}{q}\right)=-\frac{\delta^{2}}{2npq}
\end{align*}

$p+q=1$を使った。

\[
F\approx\exp\left[\frac{\left(k-np\right)^{2}}{2npq}\right]
\]

ところで
\[
np+a\sqrt{npq}\leqq k\leqq np+b\sqrt{npq}
\]
より
\[
\frac{k}{n}\rightarrow p,\frac{n-k}{n}\rightarrow q
\]

\begin{align*}
P\left(a\leq\frac{X-np}{\sqrt{npq}}\leq b\right) & =P\left(a\sqrt{npq}+np\leq X\leq b\sqrt{npq}+np\right)\\
 & =\sum_{a\sqrt{npq}+np\leq k\leq b\sqrt{npq}+np}\frac{1}{\sqrt{2\pi npq}}\exp\left[-\frac{\left(k-np\right)^{2}}{2npq}\right]\\
 & =\sum_{a\leq z_{k}\leq b}\frac{1}{\sqrt{2\pi}}\e^{-\frac{z^{2}}{2}}\frac{1}{\sqrt{npq}}\rightarrow\int_{a}^{b}\frac{1}{\sqrt{2\pi}}\e^{-\frac{z^{2}}{2}}\mathrm{d}z
\end{align*}

ただし$z_{k}=\frac{k-np}{\sqrt{npq}},\Delta z=\frac{1}{\sqrt{npq}}$である。

\section*{第4回}

\section{確率関数に関する特性値}

\subsection{平均値、期待値}

離散: $E\left[X\right]=\sum_{k}v_{k}f\left(v_{k}\right)$

連続: $E\left[X\right]=\int_{-\infty}^{\infty}xf\left(x\right)\mathrm{d}x$

\paragraph{例}

2点分布 (0: $1-p=q$, 1: $p$) 
\[
E\left[X\right]=0\cdot q+1\cdot p=p
\]

2項分布 ($B\left(n,p\right)$) 
\[
E\left[X\right]=\sum_{k}k\cdot\binom{n}{k}p^{k}\left(1-p\right)^{n-k}=np
\]

正規分布 
\[
E\left[X\right]=\int_{-\infty}^{\infty}x\frac{1}{\sqrt{2\pi}}\e^{-\frac{1}{2}\left(\frac{x-\mu}{\sigma}\right)^{2}}\mathrm{d}x
\]

\[
\int_{-\infty}^{\infty}\mathrm{d}x\cdot x\e^{-x^{2}}=\int_{0}^{-\infty}\mathrm{d}\left(x^{2}\right)\e^{-x^{2}}
\]

コーシー分布 
\begin{align*}
E\left[X\right] & =\int_{-\infty}^{\infty}\left|x\right|\frac{\mathrm{d}x}{\prod\left(1+x^{2}\right)}\rightarrow\infty\\
 & \sim\int_{0}^{\infty}\frac{\mathrm{d}x}{1+x}
\end{align*}


\paragraph{性質1}

$X$が非負の値のみを取る確率変数であるならば
\begin{align*}
E\left(X\right) & =\int_{0}^{\infty}\left(1-F\left(x\right)\right)\mathrm{d}x\\
 & =\int_{0}^{\infty}P\left(X>x\right)\mathrm{d}x
\end{align*}

$F\left(x\right)=1-P\left(X>x\right)$

\paragraph{性質2}

$X$が正負両方をとる確率変数である場合、
\[
E\left(X\right)=\int_{-\infty}^{\infty}P\left(X>x\right)\mathrm{d}x-\int_{-\infty}^{0}P\left(x\leq X\right)\mathrm{d}x
\]


\paragraph{平均値}

\[
E\left[X\right]=\int_{-\infty}^{\infty}x\mathrm{d}F\left(x\right)
\]

離散: $\mathrm{d}F\left(v_{k}\right)=F\left(v_{k}\right)-F\left(v_{k-1}\right)$

連続: $\mathrm{d}F\left(x\right)=f\left(x\right)\mathrm{d}x$

\[
E\left[X\right]\equiv\sum_{i=1}^{n-1}\tilde{x}_{k}\left[F\left(x_{k+1}\right)-F\left(x_{k}\right)\right];\tilde{x}_{k}\in\left[x_{k},x_{k+1}\right]
\]

\[
\int_{-\infty}^{\infty}f\left(x\right)\mathrm{d}x=1
\]

\[
\mu=\int_{-\infty}^{\infty}xf\left(x\right)\mathrm{d}x
\]

\[
\sigma^{2}=\int_{-\infty}^{\infty}\left(x-\mu\right)^{2}f\left(x\right)\mathrm{d}x
\]

$X$: 確率変数

$\mu=E\left[X\right]$

$\left(X-\mu\right)^{2}$も確率変数と考えることができる。

\begin{align*}
E\left[\left(X-\mu\right)^{2}\right] & =V_{av}\left[X\right]\text{or}V\left[X\right]\\
 & =\int_{-\infty}^{\infty}\left(x-\mu\right)^{2}f\left(x\right)\mathrm{d}x=\int_{-\infty}^{\infty}\left(x-\mu\right)^{2}\mathrm{d}F\left(x\right)
\end{align*}

$\sqrt{V\left[X\right]}$: 標準偏差

\paragraph{分散の性質}
\begin{enumerate}
\item $V_{av}\left[X\right]\geqq0$
\item $V_{av}\left[X\right]=E\left[X^{2}\right]-\mu^{2}$
\item $V_{av}\left[ax\right]=a^{2}V_{av}\left[X\right]$
\item $V_{av}\left[X_{1}+X_{2}\right]=V_{av}\left[X_{1}\right]+V_{av}\left[X_{2}\right]+2E\left[\left(X_{1}-\mu_{1}\right)\left(X_{2}-\mu_{2}\right)\right]$
\end{enumerate}

\subsection{???}

\subsection{モーメント母関数}

\begin{align*}
M_{X}\left(t\right) & =E\left[\e^{tX}\right]\\
 & =\begin{cases}
\sum\e^{tx}f\left(x\right)\\
\int_{-\infty}^{\infty}\e^{tx}f\left(x\right)\mathrm{d}x
\end{cases}
\end{align*}

\begin{align*}
M_{X}\left(t\right) & =E\left[1+tX+\frac{t^{2}}{2!}X^{2}+\cdots\right]\\
 & =1+\underbrace{tE\left[X\right]}_{\equiv\mu=\mu_{1}}+\underbrace{\frac{t^{2}}{2}E\left[X^{2}\right]}_{\equiv\mu_{2}\text{(2次モーメント)}}+\cdots
\end{align*}


\subsection{特性関数}

\[
\varphi\left(t\right)=E\left[\e^{\i tX}\right]=\int_{-\infty}^{\infty}\e^{\i tx}f\left(x\right)\mathrm{d}x
\]

\[
\left|\varphi\left(t\right)\right|\leqq\int_{-\infty}^{\infty}f\left(x\right)\mathrm{d}x=1
\]
(可積分)

\[
f\left(x\right)\xrightarrow[\text{フーリエ逆変換}(\text{反転公式})]{\text{フーリエ変換}(\text{特性関数})}\varphi\left(t\right)
\]


\section{多次元確率分布}

\subsection{複数の確率関数}

$\left(\Omega,\mathcal{A},P\right)$

$X_{1}\left(\omega\right),X_{2}\left(\omega\right),\cdots$

\[
\left\{ \omega:X_{1}\left(\omega\right)\leq x_{1},X_{2}\left(\omega\right)\leq x_{2},\cdots\right\} =\bigcap_{k=1}^{n}\left\{ \omega:X\left(\omega\right)\leq x_{k}\right\} 
\]
も$\mathcal{A}$に属する。

\[
F\left(x_{1},x_{2},\cdots\right)=P\left(\bigcap_{k=1}^{n}\left\{ \omega:X_{k}\left(\omega\right)\leq x_{n}\right\} \right)
\]
=同時分布関数

\paragraph{性質}

1. $a_{1}\leq b_{1},a_{2}\leq b_{2}$ならば、
\[
F\left(b_{1},b_{2}\right)-F\left(b_{1},a_{2}\right)-F\left(a_{1},b_{2}\right)+F\left(a_{1},a_{2}\right)\geq0
\]

=単調増加

2. 
\[
\lim_{x_{1}\rightarrow a1+,x_{2}\in\rightarrow}F\left(x_{1},x_{2}\right)=F\left(a_{1},a_{2}\right)
\]

3. 
\[
\lim_{x_{1}\rightarrow-\infty}F\left(x_{1},x_{2}\right)=\lim_{x_{2}\rightarrow-\infty}F\left(x_{1},x_{2}\right)=0
\]

\[
\lim_{x_{1},x_{2}\rightarrow+\infty}F\left(x_{1},x_{2}\right)=1
\]


\section*{第5回}

\subsection{?}

\subsection{共分散と相関}

\begin{align*}
E\left[X+Y\right] & =E\left[X\right]+E\left[Y\right]\\
V\left[X+Y\right] & =V\left[X\right]+V\left[Y\right]+2\cov\left(X,Y\right)
\end{align*}

\begin{align*}
V\left[X+Y\right] & =\int\left\{ \left(x-\mu_{X}\right)+\left(y-\mu_{Y}\right)\right\} ^{2}\mathrm{d}F\left(X,Y\right)\\
 & \int\left\{ \left(x-\mu_{X}\right)^{2}+\left(y-\mu_{Y}\right)^{2}+2\left(x-\mu_{X}\right)\left(y-\mu_{Y}\right)\right\} \mathrm{d}F\left(X,Y\right)
\end{align*}

\begin{align*}
\cov\left(X,Y\right) & =\int\left(x-\mu_{X}\right)\left(y-\mu_{Y}\right)\mathrm{d}F\left(X,Y\right)\\
 & =E\left[\left(X-\mu_{X}\right)\left(Y-\mu_{Y}\right)\right]
\end{align*}

共分散という。

\[
\rho_{XY}\equiv\frac{\cov\left(X,Y\right)}{\sqrt{V\left(X\right)V\left(Y\right)}}
\]

相関関数という。

\[
-1\leq\rho_{XY}\leq1
\]

$\rho_{XY}=0$: 無相関

\[
\cov\left(X,Y\right)=E\left(X,Y\right)-\mu_{X}\mu_{Y}
\]

$V$: 独立な確率変数

\[
F\left(x_{1},x_{2},\cdots\right)=F_{1}\left(x_{1}\right)\cdot F_{2}\left(x_{2}\right)\cdot F_{3}\left(x_{3}\right)\cdots
\]


\paragraph{性質}
\begin{enumerate}
\item 積の期待値 $E\left[XY\right]=E\left[X\right]E\left[Y\right]$
\item 相関 $\cov\left[X,Y\right]=0$

独立ならば$\cov=0$。ただし逆は成立しない。
\item モーメント母関数
\[
M_{X+Y}\left(t\right)=M_{X}\left(t\right)M_{Y}\left(t\right)
\]
\item 分散の加法性
\[
V\left[X\pm Y\right]=V\left[X\right]+V\left[Y\right]
\]
\item $X_{1},X_{2},\cdots,X_{n}$が独立で同一分布に従い、
\[
E\left[X_{1}\right]=E\left[X_{2}\right]=\cdots=\mu
\]
\[
V\left[X_{1}\right]=V\left[X_{2}\right]=\cdots=\sigma^{2}
\]
ならば、
\begin{align*}
 & E\left[X_{1}+X_{2}+\cdots+X_{n}\right]=n\mu\\
 & V\left[X_{1}+X_{2}+\cdots+X_{n}\right]=n\sigma^{2}
\end{align*}

$\overline{X}=\frac{1}{n}\left(X_{1}+X_{2}+\cdots+X_{n}\right)$とすれば、
\begin{align*}
E\left[\overline{X}\right] & =\mu\\
V\left[\overline{X}\right] & =\frac{\sigma^{2}}{n}
\end{align*}

\item $X_{1},X_{2},\cdots,X_{n}$が独立で、各々が$\mathcal{N}\left(\mu_{i},\sigma_{i}^{2}\right)\:\left(i=1,2\cdots n\right)$に従うならば、
\[
X=X_{1}+X_{2}+\cdots+X_{n}
\]
は
\[
\mathcal{N}\left(\mu_{1}+\mu_{2}+\cdots+\mu_{n},\sigma_{1}^{2}+\sigma_{2}^{2}+\cdots+\sigma_{n}^{2}\right)
\]
に従う。

もし
\[
\mu_{1}=\mu_{2}=\cdots=\mu_{n}=\mu
\]
\[
\sigma_{1}=\sigma_{2}=\cdots=\sigma_{n}=\sigma
\]
ならば、
\[
\overline{X}=\frac{1}{n}\left(X_{1}+X_{2}+\cdots+X_{n}\right)
\]
は$\mathcal{N}\left(\mu,\frac{\sigma^{2}}{n}\right)$に従う。
\end{enumerate}

\section{?}

\subsection{?}

\subsection{条件付き確率}

\[
P\left(A|B\right)
\]

$X$と$Y$とが独立でないとき、
\[
E_{Y}\left(E\left(X|Y\right)\right)=E\left(X\right)
\]

$X$と$Y$とが独立であるならば
\[
E\left(X|Y\right)=E\left(X\right)
\]

$X$と$Y$とが独立であるならば
\[
E\left[X_{1}+X_{2}|Y\right]=E\left[X_{1}\right]+E\left[X_{2}\right]
\]
\[
E\left[cX|Y\right]=cE\left[X|Y\right]
\]
\[
E\left[aX+b|Y\right]=aE\left[X\right]+b
\]
\[
E\left[c|Y\right]=c
\]


\section{中心極限定理}

\subsection{上極限集合と下極限集合}

\paragraph{例}

\[
\left\{ A_{n}\right\} =A,B,A,B,\cdots
\]

無限に多くの$A_{n}$は$A\cup B$に含まれる。有限個の場合を除いて$A\cap B$に含まれる。

\paragraph{例2}

$n$奇

\[
A_{n}=\left(-\frac{1}{n},1-\frac{1}{n}\right)
\]

$n$偶

\[
A_{n}=\left(\frac{1}{n},1+\frac{1}{n}\right)
\]

上極限: 無限に多くの列$A_{n}$に属する元を全て集めた集合

下極限: ほとんど全ての(有限個の$A_{m}$を覗いて)$A_{n}$に含まれる元全体からなる集合

上極限集合:

\[
\lim_{n}\sup A_{n}=\bigcap_{k=1}^{\infty}\bigcup_{n=k}^{\infty}A_{n}
\]

\[
\omega\in\lim_{n}\sup A_{n}\leftrightarrows\text{無限個の}n\text{について}\omega\in A_{n}
\]

i.o. (infinity often)

下極限集合:

\[
\lim_{n}\inf A_{n}=\bigcup_{k=1}^{\infty}\bigcap_{n=k}^{\infty}A_{n}
\]

(ある番号より先の全ての$A_{n}$で)

\paragraph{性質}
\begin{enumerate}
\item $\lim_{n}\inf A_{n}\subset\lim_{n}\sup A_{n}$
\item $\left(\lim\sup A_{n}\right)^{c}=\lim\inf A_{n}^{c}$
\end{enumerate}

\paragraph{1.の証明}

\[
x\in\bigcap_{k=1}^{\infty}\bigcup_{n=k}^{\infty}A_{n}\rightarrow x\in\bigcup_{k=1}^{\infty}\bigcap_{n=k}^{\infty}A_{n}
\]
を示す。

\[
x\in\bigcap_{k=1}^{\infty}\bigcup_{n=k}^{\infty}A_{n}\rightarrow x\in x\in\left(\bigcap_{n=1}^{\infty}A_{n}\right)\cup\left(\bigcap_{n=2}^{\infty}A_{n}\right)\cup\cdots
\]

$\bigcap_{n=k}^{\infty}A_{n}$は減少列。

十分大きな$N$に対して$x\in\bigcap_{n=N}^{\infty}A_{n}$を考えると、任意の$K$を考えて
\[
M=\max\left(K,N\right)
\]

\[
A_{M}\subset\bigcup_{k=K}^{\infty}A_{n}
\]

\[
\bigcap_{k=N}^{\infty}A_{n}\subset A_{M}
\]

\[
\therefore x\in\bigcup_{n=K}^{\infty}A_{n}
\]


\paragraph{2.の証明}

\[
\left(\lim\sup A_{n}\right)^{c}=\lim\inf A_{n}^{c}
\]

\begin{align*}
\because\left(\lim\sup A_{n}\right)^{c} & =\left(\bigcup_{k=1}^{\infty}\bigcap_{n=k}^{\infty}A_{n}\right)^{c}\\
 & =\bigcup_{k=1}^{\infty}\bigcap_{n=k}^{\infty}A_{n}^{c}=\lim\inf A_{n}^{c}
\end{align*}


\subsection{Borel-Canntellの補題}
\begin{enumerate}
\item 第1定理: $\sum_{n=1}^{\infty}P\left(A_{n}\right)<\infty$ならば、
\[
P\left(\lim\sup A_{n}\right)=0
\]
\[
P\left(\lim\sup A_{n}^{c}\right)=1
\]
\item 第2定理: $\sum_{n=1}^{\infty}P\left(A_{n}\right)=\infty$ならば、
\[
P\left(\lim\sup A_{n}\right)=1
\]
\[
P\left(\lim\sup A_{n}^{c}\right)=0
\]
\end{enumerate}

\section*{第6回}

\paragraph{証明(第1定理)}

すべての$m$に対して$\lim_{n}\sup A_{n}\subset\bigcup_{k>m}A_{k}$

\[
0\leqq P\left(\bigcup_{k\geq m}A_{k}\right)\leqq\sum_{k\geqq m}P\left(A_{m}\right)\sum_{k=1}^{\infty}P\left(A_{k}\right)-\sum_{k=1}^{m}P\left(A_{k}\right)
\]

ここで$m\rightarrow\infty$ とすると$\text{右辺}\rightarrow0$

\[
\therefore P\left(\lim\sup A_{n}\right)=0
\]
となり前半が成立。

\[
\left(\lim\sup A_{n}\right)^{c}=\lim\inf A_{n}^{c}
\]

$P\left(\left(\lim\sup A_{n}\right)^{c}\right)=1$であるから、
\[
P\left(\lim\inf A_{n}^{c}\right)=1
\]
となり後半成立。

\paragraph{証明(第2定理)}

$A_{1},A_{2},\cdots$は独立である

\[
P\left(\bigcap_{k=m}^{l}A_{k}^{c}\right)=\prod_{k=m}^{l}P\left(A_{k}^{c}\right)=\prod_{k=m}^{/}\left\{ 1-P\left(A_{k}\right)\right\} \leqq\e^{-\sum_{k=m}^{l}P\left(A_{k}\right)}\rightarrow0
\]

\[
\therefore P\left(\lim\inf A_{n}^{c}\right)=0
\]

\begin{align*}
\therefore P\left(\lim\sup A_{n}^{c}\right) & =1-P\left(\left(\lim\sup A_{n}\right)^{c}\right)\\
 & =1-P\left(\lim\inf A_{n}^{c}\right)=1
\end{align*}


\subsection{収束}

(1) 概収束

「確率変数$X_{1},X_{2},\cdots$が$X$に概収束する」とは、
\[
X_{1},X_{2},\cdots\xrightarrow[n\rightarrow\infty]{}X\left(\text{a.s}\right)
\]

(a.s): almost sevly

(a.e): almost everywhere

(w.): with probability
\begin{itemize}
\item $P\left(\lim\inf\left(\omega\in\Omega;\left|X\left(\omega\right)-X\right|<\varepsilon\right)\right)=1\text{ for all }\varepsilon>0$
\item ある$m$があって、ほとんどすべての$n\geq m$および$\varepsilon>0$に対して
\[
\left|X_{n}\left(\omega\right)-X\right|\leq\varepsilon
\]
\item $P\left(\omega:\lim_{n\rightarrow\infty}X_{n}\left(\omega\right)=X\right)=1$
\end{itemize}
(2) 確率収束

「$X_{1},X_{2},\cdots,X_{n},\cdots$が$X$に確率収束する」とは、
\begin{itemize}
\item 任意の小さな$\varepsilon>0$に対して、
\[
P\left(\left|X_{n}-X\right|<\varepsilon\right)\xrightarrow[n\rightarrow\infty]{}0
\]
\end{itemize}
(3)平均収束

「$X_{1},X_{2},\cdots,X_{n}$が$X$に$p$次平均収束する」とは、

\[
E\left[\left|X_{n}-X\right|^{p}\right]\xrightarrow{n\rightarrow\infty}0
\]

4) 法則収束 (弱収束)
\begin{itemize}
\item すべての連続点$x$で$F\left(x_{1}\right),F\left(x_{2}\right),\cdots$
\item $\lim_{n\rightarrow\infty}P\left(X_{n}\leq x\right)=P\left(X\leq x\right)$
\end{itemize}

\paragraph{レポート問題}

概収束と確率収束の例を1つずつ挙げよ。

$X_{n}$が$X$に概収束するなら、確率収束する。

\paragraph{証明}

概収束:
\[
\lim_{n\rightarrow\infty}X_{n}\left(\omega\right)=X\left(\omega\right)\text{a.s}
\]

\[
P\left(\lim\inf\left\{ \omega\in\Omega;\left|X_{n}\left(\omega\right)-X\right|<\varepsilon\right\} \right)=1\text{ for all }\varepsilon
\]

すなわち
\[
\bigcup_{m=1}^{\infty}\bigcap_{n=m}^{\infty}\left\{ \omega\in\Omega;\left|X_{n}\left(\omega\right)-X\right|<\varepsilon\right\} 
\]
の確率

これの余集合について
\[
P\left(\bigcap_{m=1}^{\infty}\bigcup_{n=m}^{\infty}\left\{ \omega\in\Omega;\left|X_{n}\left(\omega\right)-X\right|\geq\varepsilon\right\} \right)=0
\]

事象の列$\bigcup_{n=m}^{\infty}\left\{ \omega\in\Omega;\left|X_{n}\left(\omega\right)-X\right|\geq\varepsilon\right\} $は$n$に関して単調減少

ゆえに
\[
\lim_{n\rightarrow\infty}P\left(\left\{ \omega\in\Omega;\left|X_{n}\left(\omega\right)-X\right|\geqq\varepsilon\right\} \right)=0
\]

これは確率収束1の表現

\subsection{大数の法則(弱法則)}
\begin{itemize}
\item 概収束$\leftrightarrows$強法則
\item 確率収束$\leftrightarrows$弱法則
\item 平均収束$\leftrightarrows$確率積分の近似
\item 法則収束$\leftrightarrows$中心極限定理
\end{itemize}

\paragraph{弱法則}

$X_{1},X_{2},\cdots$が独立な確率変数で、それぞれの平均値と分散が$\mu,\sigma^{2}$であるならば、
\[
\bar{X}_{n}=\frac{X_{1}+X_{2}+\cdots+X_{n}}{n}
\]
は$\mu$に確率収束する。

∵Chebyshevの不等式

$\varepsilon>0,p>0$

\[
P\left(\left|X\right|\leq\varepsilon\right)\leqq\frac{1}{\varepsilon^{p}}E\left[\left|X\right|^{p}\right]
\]

\begin{align*}
P\left(\left|X\right|\geqq\varepsilon\right) & =\int_{-\infty}^{-\varepsilon}f_{x}\left(x\right)\mathrm{d}+\int_{\varepsilon}^{\infty}f_{x}\left(x\right)\mathrm{d}x\\
 & =\frac{1}{\varepsilon^{p}}\left[\int_{-\infty}^{-\varepsilon}\varepsilon^{p}f_{x}\left(x\right)\mathrm{d}x+\int_{\varepsilon}^{\infty}\varepsilon^{p}f_{x}\left(x\right)\mathrm{d}x\right]\\
 & \leqq\frac{1}{\varepsilon^{p}}\left[\int_{-\infty}^{-\varepsilon}\left|x\right|^{p}f_{x}\left(x\right)\mathrm{d}x+\int_{\varepsilon}^{\infty}x^{p}f_{x}\left(x\right)\mathrm{d}x\right]\\
 & \leqq\frac{1}{\varepsilon^{p}}\int_{-\infty}^{\infty}\left|x\right|^{p}f_{x}\left(x\right)\mathrm{d}x=\frac{1}{\varepsilon^{p}}E\left[\left|X\right|^{p}\right]
\end{align*}

$p=2$として、

\begin{align*}
P\left(\left|\bar{X}_{n}-\mu\right|\geqq\varepsilon\right) & \leqq\frac{1}{\varepsilon^{2}}E\left(\left|\bar{X}_{n}-\mu\right|\right)\\
 & =\frac{1}{\varepsilon^{2}}
\end{align*}

\[
\mathrm{Var}\left(\bar{X}_{n}\right)=\frac{1}{\varepsilon^{2}},\frac{n\sigma^{2}}{n^{2}}=\frac{\sigma^{2}}{n\varepsilon^{2}}
\]

$n\rightarrow\infty$とすれば、右辺$\rightarrow\infty$

証明終

\subsection{大数の強法則}

$X_{1},X_{2},\cdots$が独立、同じ分布に従い、$X\left[X_{k}\right]=\mu$が存在するならば
\[
\bar{X}=\frac{1}{n}\left(X_{1}+X_{2}+\cdots+X_{n}\right)
\]
は$\mu$に概収束する。

\paragraph{証明}

証明を簡単にするために、4次のモーメントが存在するとする。

\begin{align*}
E\left[\left\{ \sum_{k=1}^{n})X_{k}-\mu\right\} ^{4}\right] & =\sum_{k=1}^{n}E\left[\left(X_{k}-\mu\right)^{4}\right]+6\sum_{i<j}E\left[\left(X_{i}-\mu\right)^{2}\right]\times E\left[\left(X_{j}-\mu\right)^{2}\right]\\
 & \leqq_{n\rightarrow\infty}C\sigma^{4}n^{2}
\end{align*}

\end{document}
