%% LyX 2.2.2 created this file.  For more info, see http://www.lyx.org/.
%% Do not edit unless you really know what you are doing.
\documentclass[english]{article}
\usepackage[T1]{fontenc}
\usepackage[utf8]{inputenc}
\usepackage[a5paper]{geometry}
\geometry{verbose,tmargin=2cm,bmargin=2cm,lmargin=1cm,rmargin=1cm}
\setlength{\parskip}{\smallskipamount}
\setlength{\parindent}{0pt}
\usepackage{amsmath}

\makeatletter
%%%%%%%%%%%%%%%%%%%%%%%%%%%%%% User specified LaTeX commands.
\usepackage[dvipdfmx]{hyperref}
\usepackage[dvipdfmx]{pxjahyper}

\makeatother

\usepackage{babel}
\begin{document}

\title{数理手法V 第1回レポート}

\author{電気電子工学科2年 340728B 高橋光輝}

\maketitle
\global\long\def\pd#1#2{\frac{\partial#1}{\partial#2}}
\global\long\def\d#1#2{\frac{\mathrm{d}#1}{\mathrm{d}#2}}
\global\long\def\pdd#1#2{\frac{\partial^{2}#1}{\partial#2^{2}}}
\global\long\def\dd#1#2{\frac{\mathrm{d}^{2}#1}{\mathrm{d}#2^{2}}}
\global\long\def\ddd#1#2{\frac{\mathrm{d}^{3}#1}{\mathrm{d}#2^{3}}}
\global\long\def\e{\mathrm{e}}
\global\long\def\i{\mathrm{i}}
\global\long\def\j{\mathrm{j}}
\global\long\def\grad{\operatorname{grad}}
\global\long\def\rot{\operatorname{rot}}
\global\long\def\div{\operatorname{div}}
\global\long\def\diag{\operatorname{diag}}
\global\long\def\rank{\operatorname{rank}}
\global\long\def\prob{\operatorname{Prob}}
\global\long\def\cov{\operatorname{Cov}}
\global\long\def\when#1{\left.#1\right|}


\paragraph{証明}

数学的帰納法を用いる。
\begin{enumerate}
\item $n=1$のとき題意は
\[
P\left(A_{1}\right)=P\left(A_{1}\right)
\]
となり自明。
\item $n=k$のとき題意が満たされると仮定する。

すなわち、
\[
P\left(\bigcup_{i=1}^{k}A_{i}\right)=\sum_{m=1}^{k}\left(-1\right)^{m-1}\sum_{i_{1}<\cdots<i_{m}}P\left(\bigcap_{l=1}^{m}A_{i_{l}}\right)
\]
が成立。このとき、

\begin{align*}
 & P\left(\bigcup_{i=1}^{k+1}A_{i}\right)\\
= & P\left(\left(\bigcup_{i=1}^{k}A_{i}\right)\cup A_{k+1}\right)\\
= & P\left(\bigcup_{i=1}^{k}A_{i}\right)+P\left(A_{k+1}\right)-P\left(\left(\bigcup_{i=1}^{k}A_{i}\right)\cap A_{k+1}\right)\\
= & \sum_{m=1}^{k}\left(-1\right)^{m-1}\sum_{i_{1}<\cdots<i_{m}\in\left(i_{1}\cdots i_{k}\right)}P\left(\bigcap_{l=1}^{m}A_{i_{l}}\right)+P\left(A_{k+1}\right)\\
 & -\sum_{m=1}^{k}\left(-1\right)^{m-1}\sum_{i_{1}<\cdots<i_{m}\in\left(i_{1}\cdots i_{k}\right)}P\left(\left(\bigcap_{l=1}^{m}A_{i_{l}}\right)\cap A_{k+1}\right)\\
= & \sum_{l=1}^{k}P\left(A_{l}\right)+P\left(A_{k+1}\right)\\
 & +\sum_{m=2}^{k}\left(-1\right)^{m-1}\sum_{i_{1}<\cdots<i_{m}\in\left(i_{1}\cdots i_{k}\right)}P\left(\bigcap_{l=1}^{m}A_{i_{l}}\right)\\
 & -\sum_{m=1}^{k-1}\left(-1\right)^{m-1}\sum_{i_{1}<\cdots<i_{m}\in\left(i_{1}\cdots i_{k}\right)}P\left(\left(\bigcap_{l=1}^{m}A_{i_{l}}\right)\cap A_{k+1}\right)\\
 & -\left(-1\right)^{k-1}\sum_{i_{1}<\cdots<i_{k}\in\left(i_{1}\cdots i_{k}\right)}P\left(\left(\bigcap_{l=1}^{k}A_{i_{l}}\right)\cap A_{k+1}\right)\\
= & \sum_{l=1}^{k+1}P\left(A_{l}\right)\\
 & +\sum_{m=2}^{k}\left(-1\right)^{m-1}\\
 & \quad\left(\sum_{i_{1}<\cdots<i_{m}\in\left(i_{1}\cdots i_{k}\right)}P\left(\bigcap_{l=1}^{m}A_{i_{l}}\right)+\sum_{i_{1}<\cdots<i_{m-1}\in\left(i_{1}\cdots i_{k}\right)}P\left(\left(\bigcap_{l=1}^{m-1}A_{i_{l}}\right)\cap A_{k+1}\right)\right)\\
 & +\left(-1\right)^{k}\sum_{i_{1}<\cdots<i_{k+1}\in\left(i_{1}\cdots i_{k+1}\right)}P\left(\bigcap_{l=1}^{k+1}A_{i_{l}}\right)
\end{align*}

ここで
\[
\sum_{i_{1}<\cdots<i_{m}\in\left(i_{1}\cdots i_{k}\right)}P\left(\bigcap_{l=1}^{m}A_{i_{l}}\right)+\sum_{i_{1}<\cdots<i_{m-1}\in\left(i_{1}\cdots i_{k}\right)}P\left(\left(\bigcap_{l=1}^{m-1}A_{i_{l}}\right)\cap A_{k+1}\right)
\]
とは$k+1$個の整数から$k+1$を含まない$m$個を選ぶ$_{k}C_{m}$通りと、$k+1$を含む$m$個を選ぶ$_{k}C_{m-1}$通りの2つの組み合わせの和となっている。この数は公式より
\[
_{k}C_{m}+{}_{k}C_{m-1}={}_{k+1}C_{m}
\]
通りであり、$k+1$個の整数から任意の$m$個を選ぶ組み合わせを重複なく列挙している。即ち、
\begin{align*}
 & \sum_{i_{1}<\cdots<i_{m}\in\left(i_{1}\cdots i_{k}\right)}P\left(\bigcap_{l=1}^{m}A_{i_{l}}\right)+\sum_{i_{1}<\cdots<i_{m-1}\in\left(i_{1}\cdots i_{k}\right)}P\left(\left(\bigcap_{l=1}^{m-1}A_{i_{l}}\right)\cap A_{k+1}\right)\\
= & \sum_{i_{1}<\cdots<i_{m}\in\left(i_{1}\cdots i_{k+1}\right)}P\left(\bigcap_{l=1}^{m}A_{i_{l}}\right)
\end{align*}
である。

これを用いて前述の式は、
\begin{align*}
 & P\left(\bigcup_{i=1}^{k+1}A_{i}\right)\\
= & \sum_{l=1}^{k+1}P\left(A_{l}\right)\\
 & +\sum_{m=2}^{k}\left(-1\right)^{m-1}\sum_{i_{1}<\cdots<i_{m}\in\left(i_{1}\cdots i_{k+1}\right)}P\left(\bigcap_{l=1}^{m}A_{i_{l}}\right)\\
 & +\left(-1\right)^{k}\sum_{i_{1}<\cdots<i_{k+1}\in\left(i_{1}\cdots i_{k+1}\right)}P\left(\bigcap_{l=1}^{k+1}A_{i_{l}}\right)\\
= & \sum_{m=1}^{k+1}\left(-1\right)^{m-1}\sum_{i_{1}<\cdots<i_{m}}P\left(\bigcap_{l=1}^{m}A_{i_{l}}\right)
\end{align*}

よって$k+1$において題意が成立する。
\end{enumerate}
1. および 2. より、数学的帰納法から任意の$n$に対して題意が成立する。よって証明された。

終
\end{document}
