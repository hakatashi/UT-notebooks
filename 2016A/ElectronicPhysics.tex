%% LyX 2.2.2 created this file.  For more info, see http://www.lyx.org/.
%% Do not edit unless you really know what you are doing.
\documentclass[oneside,english]{book}
\usepackage[T1]{fontenc}
\usepackage[utf8]{inputenc}
\usepackage[a5paper]{geometry}
\geometry{verbose,tmargin=2cm,bmargin=2cm,lmargin=1cm,rmargin=1cm}
\setcounter{secnumdepth}{3}
\setcounter{tocdepth}{3}
\setlength{\parskip}{\smallskipamount}
\setlength{\parindent}{0pt}
\usepackage{textcomp}
\usepackage{amsmath}
\usepackage{amssymb}
\usepackage{graphicx}
\PassOptionsToPackage{normalem}{ulem}
\usepackage{ulem}

\makeatletter
%%%%%%%%%%%%%%%%%%%%%%%%%%%%%% User specified LaTeX commands.
\usepackage[dvipdfmx]{hyperref}
\usepackage[dvipdfmx]{pxjahyper}

\makeatother

\usepackage{babel}
\begin{document}

\title{2016-A 電子基礎物理}

\author{教員: 入力: 高橋光輝}

\maketitle
\global\long\def\pd#1#2{\frac{\partial#1}{\partial#2}}
\global\long\def\d#1#2{\frac{\mathrm{d}#1}{\mathrm{d}#2}}
\global\long\def\pdd#1#2{\frac{\partial^{2}#1}{\partial#2^{2}}}
\global\long\def\dd#1#2{\frac{\mathrm{d}^{2}#1}{\mathrm{d}#2^{2}}}
\global\long\def\ddd#1#2{\frac{\mathrm{d}^{3}#1}{\mathrm{d}#2^{3}}}
\global\long\def\e{\mathrm{e}}
\global\long\def\i{\mathrm{i}}
\global\long\def\j{\mathrm{j}}
\global\long\def\grad{\operatorname{grad}}
\global\long\def\rot{\operatorname{rot}}
\global\long\def\div{\operatorname{div}}
\global\long\def\diag{\operatorname{diag}}
\global\long\def\rank{\operatorname{rank}}
\global\long\def\prob{\operatorname{Prob}}
\global\long\def\cov{\operatorname{Cov}}
\global\long\def\when#1{\left.#1\right|}


\section*{第1回}

\paragraph{講義のウェブサイト}

http://www.ee.t.u-tokyo.ac.jp/\textasciitilde{}sugiyama/lecture/phys/

中間試験を受けないと単位は与えられない。

成績は毎回の演習問題を50\%加味した成績と試験のみの成績の高い方で評価する。

科目の総合成績は授業数を加重した2人の担当者の平均点で計算される。

レポート・演習問題は本人が提出する必要はない。

\paragraph{アインシュタインの光量子仮説}

\[
E=h\nu
\]
\[
c=\nu\lambda
\]
\[
E=\frac{hc}{\lambda}
\]
\[
E\left(\text{eV}\right)=\frac{1240}{\lambda}
\]

$h$: プランク定数

$\nu$:光の振動数

$c$: 光速

$\lambda$: 光の波長

\paragraph{c.f. 古典力学: 光は波である}

振幅$A=A_{0}\cos\left(kx-\omega t\right)$

波のエネルギー$I\propto A_{0}^{2}$

\section*{第2回}

\[
E=h\nu=h\frac{c}{\lambda}
\]

\[
\frac{hc}{\lambda}=E_{2}-E_{1}
\]

光子(光を構成する粒子)の
\begin{itemize}
\item 質量→ゼロ
\item 運動量$p$→$E=cp$を満たす

\[
\frac{hc}{\lambda}=cp\Rightarrow p=\frac{h}{\lambda}
\]

\end{itemize}
単位面積、単位時間あたり、運動量$p\left[\mathrm{kg\cdot m\cdot s^{-2}}\right]$の粒子が$N$個入射しているとする。

→運動量$Np$

c.f. 圧力の単位$\left[\mathrm{\frac{kg\cdot m\cdot s^{-2}}{m^{2}}}\right]=\left[\mathrm{kg\cdot m^{-1}\cdot s^{-2}}\right]$
→ 運動量の和 / 面積・時間

$1\mathrm{kWm^{-2}}$の強度の光が入射するとき、光子の運動量によって発生する圧力はどれくらいか?

光の波長620nm(=2eV)とする。

1光子の運動量は
\begin{equation}
\frac{\left(\mathrm{2eV}\right)\left(\mathrm{1.6\times10^{-19}JeV^{-1}}\right)}{\left(\mathrm{3\times10^{8}ms^{-1}}\right)}\doteqdot\mathrm{1\times10^{-27}kgms^{-1}}
\end{equation}

$\mathrm{1m^{2},1s}$あたりの光子数は
\begin{equation}
\frac{\mathrm{1\times10^{3}Js^{-1}m^{-2}}}{\left(\mathrm{2eV}\right)\left(\mathrm{1.6\times10^{-19}JeV^{-1}}\right)}\doteqdot\mathrm{3\times10^{21}m^{-2}s^{-1}}
\end{equation}

∴光子の圧力
\[
\left(1\right)\times\left(2\right)\doteqdot\mathrm{3\times10^{-6}kgm^{-1}s^{-2}\left(=Pa\right)}
\]

参考: 1気圧=$\mathrm{1\times10^{5}Pa}$

$\mathrm{1kWm^{-2}}$の620nmの光が与える圧力→大気圧の$3\times10^{-11}$倍

\paragraph{物質波(ド・ブロイ)}

質量$m$

運動量$p=mv=\frac{h}{\lambda}$

\paragraph{ボーアの電子模型}
\begin{center}
\includegraphics{images/ElectronicPhysics/2-1}
\par\end{center}

$r$, $v$, エネルギーが離散的な値を取る。$m=1,2,\cdots$

発光波長 $\frac{hc}{\lambda}=E_{m}-E_{n}\left(m>n\right)$
\begin{center}
\includegraphics{images/ElectronicPhysics/2-2}
\par\end{center}

電子の物質波の波長が$\mathrm{1\times10^{-10}m}$

このとき必要な電子の加速電位は?

電子の質量$m=\mathrm{9.1\times10^{-31}kg}$

$\lambda=\frac{h}{mv}$

加速された電子の運動エネルギー
\[
E=\frac{1}{2}mv^{2}=\mathrm{1eV}
\]
\[
\therefore v=\sqrt{\frac{\mathrm{2eV}}{m}}
\]

\[
v=\sqrt{\frac{\mathrm{2eV}}{m}}\Rightarrow\lambda=\frac{h}{\sqrt{2m\mathrm{eV}}}\Rightarrow2m\mathrm{eV}\lambda^{2}=\frac{h^{2}}{V}
\]

\begin{align*}
V & =\frac{h^{2}}{2m\mathrm{eV}\lambda^{2}}=\frac{\left(\mathrm{6.6\times10^{-34}Js}\right)\left(\mathrm{6.6\times10^{-34}Js}\right)}{\left(2\right)\left(\mathrm{1\times10^{-30}kg}\right)\left(\mathrm{1.6\times10^{-19}C}\right)\left(\mathrm{1\times10^{-20}m^{2}}\right)}\\
 & \sim\mathrm{130JC^{-1}\left(=V\right)}
\end{align*}


\paragraph{Schrödinger方程式}

$\Psi$: 波動関数 粒子の運動に関する情報をすべて含む複素関数。Schrödinger方程式を解いて得られる。

\[
\i\hbar\pd{\Psi}t=\left(-\frac{\hbar^{2}}{2m}\pdd{}x+V\right)\Psi
\]

$\hbar$: プランク定数 $\hbar=\frac{h}{2\pi}$

$V$: ポテンシャル

\paragraph{もとになる考え方}

$\Psi$は波の形をとる。

波を表す関数

振幅
\[
A=\mathrm{Re}\left[\underbrace{A_{c}\e^{\i\left(kx-\omega t\right)}}_{\text{波動関数の基本形?}}\right]
\]

$k$: 波数 $k=\frac{2\pi}{\lambda}$

$\omega$: 角振動数 $\omega=\frac{2\pi}{T}$

\[
E=h\nu=\hbar\omega\left(\text{光量子仮説}\right)\Leftrightarrow\i h\pd{\Psi}t=\hbar\omega\Psi
\]

\[
p=\frac{h}{\lambda}=\hbar k\left(\text{物質波も}\right)\Leftrightarrow-\i h\pd{\Psi}{\chi}=\hbar k\Psi
\]

演算子→物理量を導く数学操作と考えると

エネルギー→$\i\hbar\pd{}t\equiv\hat{E}$

運動量→$\i h\pd{}{\chi}\equiv\hat{p}$

エネルギーのもう一つの表現=運動エネルギー+ポテンシャル $\frac{p^{2}}{2m}+V$

エネルギーに対応する演算子
\[
\frac{\hat{p}^{2}}{2m}+V=-\frac{\hbar^{2}}{2m}\pdd{}{\chi}+V
\]

→エネルギーに対応する2通りの演算子を対応させて、
\[
\left(-\frac{\hbar^{2}}{2m}\pdd{}{\chi}+V\right)\Psi=\i\hbar\pd{}t\Psi
\]


\section*{第3回}

演算子: ある関数に作用して別の関数に変える。

\[
Af=g
\]

内積:
\[
\left(f,g\right)=\left\langle f|g\right\rangle =\int_{\text{all space}}f^{*}g\mathrm{d}\vec{r}
\]

※$f^{*}$: 複素共役 $\i\rightarrow-\i$

エルミート演算子 $A$:

\[
\left(Af,g\right)=\left(f,Ag\right)
\]


\paragraph{定理}

エルミート演算子の固有値は実数

\[
Af=af
\]


\paragraph{証明}

\begin{align*}
\left(f,Af\right) & =\left(f,af\right)=a\left|f\right|^{2}\\
\left(Af,f\right) & =\left(af,f\right)=a^{*}\left|f\right|^{2}\\
\left(f,Af\right) & =\left(Af,f\right)\rightarrow a\left|f\right|^{2}=a^{*}\left|f\right|^{2}
\end{align*}

$a^{*}=a$なので$a$は実数。ただし$\left|f\right|^{2}=\left(f,f\right)$

\paragraph{定理}

エルミート演算子の固有値$a_{n}$に対応する固有関数を$f_{n}$とすると、

\[
\left(f_{m},f_{n}\right)=\delta_{m,n}
\]

ただし$\delta=\begin{cases}
1 & \left(m=n\right)\\
0 & \left(m\neq n\right)
\end{cases}$(クロネッカーのデルタ)

\paragraph{証明}

\[
Af_{m}=a_{m}f_{m}
\]

\[
Af_{n}=a_{n}f_{n}\left(m\neq n\right)
\]

\begin{equation}
\left(f_{m},Af_{n}\right)=a_{n}\left(f_{m},f_{n}\right)
\end{equation}

\begin{equation}
\left(Af_{m},f_{n}\right)=a_{m}^{*}\left(f_{m},f_{n}\right)=a_{m}\left(f_{m},f_{n}\right)
\end{equation}

$A$はエルミート演算子なので、$\left(f_{m},Af_{n}\right)=\left(Af_{m},f_{n}\right)$

(3), (4)を辺々引いて、
\[
\left(a_{m}-a_{n}\right)\left(f_{m},f_{n}\right)=0
\]

$a_{m}\neq a_{n}$なので、$\left(f_{m},f_{n}\right)=0$

\paragraph{交換関係}

演算子$A,B$について、
\begin{itemize}
\item $AB-BA=0$→$A$と$B$は交換可能
\item $AB-BA\neq0$→$A$と$B$は交換不可能
\end{itemize}

\paragraph{定理}

エルミート演算子$A,B$が交換不可能な時、
\[
Af=af
\]
\[
Bf=bf
\]
を同時に満たす固有関数$f$は存在しない。

\paragraph{証明}

エルミート演算子$A,B$が交換不可能かつ、$Af=af,Bf=bf$を同時に満たす固有関数$f$は存在すると仮定する。

\[
Af=af
\]

\[
B\left(Af\right)=a\left(Bf\right)=abf
\]

\[
A\left(Bf\right)=b\left(Af\right)=baf
\]

$a,b$は実数なので$ab=ba$

\[
BAf=ABf
\]

\[
\left(BA-AB\right)f=0
\]

これは$A,B$が交換不可能($AB-BA\neq0$)なことに矛盾

\paragraph{Schrödinger方程式}

原子・電子スケールの減少を記述する方程式

\[
\i\hbar\pd{\Psi}t=H\Psi
\]

$\Psi$: 波動関数

$H$: ハミルトニアン

\paragraph{定常状態の解}

定常状態では、
\begin{align*}
\Psi & =\Psi_{0}\left(x,y,z\right)\exp\left(\i\omega t\right)\\
E & =\hbar\omega
\end{align*}
とおくと、
\[
H\Psi_{0}\left(x,y,z\right)=E\Psi_{0}\left(x,y,z\right)
\]

(固有値問題)

\[
E=\hbar\omega
\]

(固有値)は$H$で記述されるエネルギーに対応

(板書)

\[
\i\hbar\pd{\Psi}t=\hat{H}\Psi\:\hat{H}=-\frac{\hbar^{2}}{2m}\pdd{}x+V
\]

$\Psi=\Psi_{0}\left(x\right)\e^{-\i\omega t}$を代入

\[
\hbar\omega\Psi_{0}\left(x\right)\e^{-\i\omega t}=\left\{ \hat{H}\Psi_{0}\left(x\right)\right\} \e^{-\i\omega t}
\]

$\hbar\omega=\varepsilon$: エネルギー とすると、
\[
\hat{H}\Psi_{0}\left(x\right)=\varepsilon\Psi_{0}\left(x\right)
\]

これを定常状態のSchrödinger方程式という。この式は時間に無依存であり、解くと$\Psi_{0}\left(x\right)$が求まる。これに時間依存性を加えると、
\[
\Psi\left(x,t\right)=\Psi_{0}\left(x\right)\e^{-\i\omega t}
\]

ただし、$\varepsilon=\hbar\omega$。

\paragraph{量子力学の基本仮定}

(1)

系の状態は波動関数$\Psi$によって表される。

「粒子」が座標$\left(x,y,z\right)$と$\left(x+\mathrm{d}x,y+\mathrm{d}y,z+\mathrm{d}z\right)$の間に存在する確率
\[
P\left(x,y,z\right)\mathrm{d}x\mathrm{d}y\mathrm{d}z=\left|\Psi\left(x,y,z\right)\right|^{2}\mathrm{d}x\mathrm{d}y\mathrm{d}z=\Psi^{*}\left(x,y,z\right)\Psi\left(x,y,z\right)\mathrm{d}x\mathrm{d}y\mathrm{d}z
\]

ただし、
\[
\iiint_{-\infty}^{\infty}\left|\Psi\right|^{2}\mathrm{d}x\mathrm{d}y\mathrm{d}z=1
\]

(2)

波動関数$\Psi$で表される状態は、いくつか(あるいは無限個)の固有状態$\phi_{n}$の重ね合わせで表される

\[
\Psi\left(x,y,z,t\right)=\sum_{n}c_{n}\phi_{n}\left(x,y,z\right)\e^{-\i\omega_{n}t}
\]

$c_{n}$: 係数(時間に無依存の複素数)

$\phi_{n}$は、定常状態のSchrödinger方程式を満たす。

\[
H\phi_{n}\left(x,y,z\right)=\epsilon_{n}\phi_{n}\left(x,y,z\right)
\]

\[
\omega_{n}=\frac{\epsilon_{n}}{\hbar}
\]

異なるエネルギー固有値に属する波動関数は直交する。

\[
\iiint_{-\infty}^{\infty}\phi_{n}^{*}\phi_{m}\mathrm{d}x\mathrm{d}y\mathrm{d}z=\delta_{nm}
\]

(3)

古典力学における物理量$a$→演算子$A$
\begin{itemize}
\item エネルギー $E\rightarrow\i\hbar\pd{}t$
\item 運動量 $p_{x}\rightarrow-\i\hbar\pd{}x$
\item 位置座標 $x\rightarrow x$
\end{itemize}
例えば、定常状態のエネルギー
\[
\hat{E}\Psi\left(x,t\right)=\i\hbar\pd{}t\Psi=\hbar\omega\Psi=\varepsilon\Psi
\]

古典力学では、運動量$p=mv$、運動エネルギー$E=\frac{1}{2}mv^{2}=\frac{p}{2m}$である。

一方、ハミルトニアン
\[
\hat{H}=-\frac{\hbar^{2}}{2m}\pdd{}x+V=\frac{\hat{p}^{2}}{2m}+V
\]
という対応関係が見える。ただし、
\[
\hat{p}^{2}=\left(-\i\hbar\pd{}x\right)\left(-\i\hbar\pd{}x\right)=-\hbar^{2}\pdd{}x
\]
を用いた。

物理量は観測しないとわからない。

ある固有状態(波動関数$\phi_{n}$で表される)に対応する物理量
\[
A\phi_{n}=a_{n}\phi_{n}
\]

→波動関数$\phi_{n}$にエルミート演算子$A$を左右させて得られる固有値$a_{0}$

固有値が1つなら観測される物理量は必ず一定。

固有値が複数なら、観測した「瞬間」に系がとる固有状態に対応する固有値をとる。

どの固有値を系が取っているかは観測するまでわからない。

・物理量$a$を多数回観察した時の期待値
\[
\left\langle a\right\rangle =\iiint_{-\infty}^{\infty}\Psi^{*}A\Psi\mathrm{d}x\mathrm{d}y\mathrm{d}z=\sum_{n}\left|c_{n}\right|^{2}a_{n}
\]

・ある瞬間に固有状態$n$が観測される確率

\[
\left|c_{n}\right|^{2}
\]


\paragraph{ポテンシャル一定の空間 (1次元)}

$V=U_{0}$(一定値)とする。

Schrödinger方程式(定常)

\[
\hat{H}\varphi=\varepsilon\varphi
\]

\[
\left(-\frac{\hbar^{2}}{2m}\pdd{}x+U_{0}\right)\varphi=\varepsilon\varphi
\]

\[
-\frac{\hbar^{2}}{2m}\pdd{}x\varphi=\left(\varepsilon-U_{0}\right)\varphi
\]

$\varepsilon-U_{0}$とは$\varepsilon$を$U_{0}$を基準として表しただけ。$U_{0}=0$でも同じこと。

\begin{equation}
\pdd{}x\varphi=-\frac{2m\varepsilon}{\hbar^{2}}\varphi
\end{equation}

\[
\varphi=A\e^{\i kx}
\]
が一般解。

なぜなら、
\[
-k^{2}\varphi=-\frac{2m\varepsilon}{\hbar^{2}}\varphi
\]
\begin{equation}
k=\pm\frac{\sqrt{2m\varepsilon}}{\hbar}
\end{equation}
として、$\varphi=A\e^{\i kx}$は(5)を満たす。

$\int_{-\infty}^{\infty}\left|\varphi\right|^{2}\mathrm{d}x=1$となるように係数を求める。

周期的境界条件
\begin{center}
\includegraphics{images/ElectronicPhysics/3-1}
\par\end{center}

を適用する。

規格化して
\[
\int_{0}^{L}\left|A\e^{\i kx}\right|^{2}\mathrm{d}x=1
\]
\[
A^{2}L=1
\]
\[
\therefore A=\frac{1}{\sqrt{L}}
\]

境界条件
\[
\varphi\left(0\right)=\varphi\left(L\right)
\]
\[
\pd{\varphi}x\left(L\right)=\pd{\varphi}x\left(0\right)
\]
より
\[
\varphi=\frac{1}{\sqrt{L}}\e^{\i kx}
\]

波長$\lambda$について
\[
n\lambda=L
\]
\[
n\frac{2\pi}{k}=L
\]
\[
\therefore k=\frac{2\pi}{L}n
\]

まとめると、ポテンシャル一定の空間中で\uline{粒子1個}に対する波動関数
\[
\varphi_{n}=\frac{1}{\sqrt{L}}\e^{\i k_{n}x}
\]

ただし(6)より$k_{n}=\frac{2\pi}{L}n,\varepsilon_{n}=\frac{\hbar^{2}k_{n}^{2}}{2m}$、$n$は整数。

時間項まで含めると、
\[
\Psi_{n}=\varphi_{n}\e^{-\i\omega t}=\frac{1}{\sqrt{L}}\e^{\i\left(k_{n}x-\omega_{n}t\right)}
\]

ただし、
\[
\hbar\omega_{n}=\varepsilon_{n}=\frac{\hbar k_{n}^{2}}{2m}
\]

粒子の運動量
\[
\hat{p}\Psi_{n}=-\i\hbar\pd{}t\Psi_{n}=\hbar k_{n}\Psi_{n}
\]

より、$\hbar k_{n}$が状態$n$に対応する運動量。

時刻$t$、位置$x$に粒子を見出す確率は(粒子が状態$n$をとるとき)
\[
P\left(x,t\right)=\left|\Psi_{n}\right|^{2}=\frac{1}{L}
\]
となり、位置にも時間にも依存しない。すなわちどの位置にも粒子は等確率で存在することになってしまう。

$\Psi_{n}$単独では粒子の位置を議論できない。

たくさんの$\Psi_{n}$の足し合わせで
\[
\Psi=\sum_{n}C_{n}\Psi_{n}=\sum_{n}C_{n}\e^{\i\left(k_{n}x-\omega_{n}t\right)}
\]

$C_{n}$: 複素数(時間に無依存)
\begin{center}
\includegraphics{images/ElectronicPhysics/3-2}
\par\end{center}

\paragraph{波束 (wave packet)}

波束: 粒子の運動を表す波動関数の形

\[
\Psi=\sum_{n}C_{n}\e^{\i\left(k_{n}x-\omega_{n}t\right)}
\]

$\e^{\i\left(k_{n}x-\omega_{n}t\right)}$は正弦波の動きを表す。この式は時間とともにどう運動するか?

\[
\e^{\i\left(kx-\omega t\right)}=\e^{\i k\left(x-\frac{\omega}{k}t\right)}
\]

$\frac{\omega}{k}t$: $x$軸上の平行移動

単位時間の間に同意蒼天が$\frac{\omega}{x}$(位相速度)動く。
\begin{center}
\includegraphics{images/ElectronicPhysics/3-4}
\par\end{center}

波束の移動速度

\[
k_{n}=k_{0}+\Delta k,\omega_{n}=\omega_{0}+\Delta\omega=\omega_{0}+\left.\pd{\omega}k\right|_{k_{0}}\Delta k
\]

\begin{align*}
\Psi & =\sum_{\Delta k}C_{\Delta k}\e^{\i\left\{ \left(k_{0}+\Delta k\right)x-\left(\omega_{0}+\pd{\omega}k\Delta k\right)t\right\} }\\
 & =\sum_{\Delta k}C_{\Delta k}\e^{\i\left(k_{0}x-\omega_{0}t\right)}\e^{\i\Delta k\left(x-\pd{\omega}kt\right)}\\
 & =\underbrace{\e^{\i k_{0}\left(x-\frac{\omega_{0}}{k_{0}}t\right)}}_{\text{細かい振動}}\sum_{\Delta k}C_{\Delta k}\underbrace{\e^{\i\Delta k\left(x-\pd{\omega}kt\right)}}_{\text{緩やかな振動}}
\end{align*}

ただし、$\Delta k\ll k_{0}$。

\section*{第4回}

\paragraph{前半メモ}

\[
\Psi=\sum_{n}C_{n}\phi_{n}
\]

$\phi_{n}$: 固有状態

の重ねあわせ

$\hat{A}$: 演算子

\[
\hat{A}\phi_{n}=a_{n}\phi_{n}
\]

$a_{n}$: 固有状態の物理量

\textbf{1回の観測で見えるのは$a_{n}$のどれかである。}

期待値

\[
\left\langle a\right\rangle =\int_{\text{全空間}}\Psi^{*}\hat{A}\Psi\mathrm{d}x=\sum_{n}\left|C_{n}\right|^{2}a_{n}
\]

\[
P\left(x,t\right)=\Psi^{*}\Psi=\left(\sum_{n}C_{n}^{*}\phi_{n}^{*}\right)\left(\sum_{n}C_{n}\phi_{n}\right)
\]

ある位置$x$に粒子が存在する確率
\[
\int_{\text{全空間}}P\left(x,t\right)\mathrm{d}x=\sum_{n}\left|C_{n}\right|^{2}\equiv\text{全空間での粒子の存在確率}=1
\]
になるように$C_{n}$を決める。

ガウス関数
\[
f\left(x\right)=\left(\frac{1}{\pi\alpha^{2}}\right)^{\frac{1}{4}}\e^{-\frac{x^{2}}{2\alpha^{2}}}\e^{\i k_{0}x}
\]

($\int_{-\infty}^{\infty}\left|f\left(x\right)\right|^{2}\mathrm{d}x=1$)

フーリエ変換して
\[
F\left(k\right)=\left(\frac{\alpha^{2}}{\pi}\right)^{\frac{1}{4}}\e^{-\frac{\alpha^{2}}{2}\left(k-k_{0}\right)^{2}}
\]

すなわち、
\[
f\left(x\right)=\frac{1}{\sqrt{2\pi}}\int_{-\infty}^{\infty}F\left(k\right)\e^{\i kx}\mathrm{d}k
\]
(フーリエ逆変換)

$f\left(x\right)$: ポテンシャル一定の空間中の粒子の固有関数$\phi_{k}=\e^{\i\left(kx-\omega t\right)}$

→$f\left(x\right)$は$\phi_{k}$の重ね合わせでできた波動関数$\Psi\left(x,t\right)$の$t=0$に相当している。
i.e. $\Psi\left(x,0\right)=f\left(x\right)$
\begin{center}
\includegraphics{images/ElectronicPhysics/4-1}
\par\end{center}

\[
\left|f\left(x\right)\right|^{2}=\sqrt{\frac{1}{\pi\alpha^{2}}}\e^{-\frac{x^{2}}{\alpha^{2}}}
\]

$t>0$について、
\[
\Psi\left(x,t\right)=\frac{1}{\sqrt{2\pi}}\int_{-\infty}^{\infty}F\left(k\right)\e^{\i\left(kx-\omega t\right)}\mathrm{d}k=\left(\frac{\alpha^{2}}{4\pi^{3}}\right)^{\frac{1}{4}}\int_{-\infty}^{\infty}\e^{-\frac{\alpha^{2}}{2}\left(k-k_{0}\right)^{2}}\e^{\i\left(kx-\omega t\right)}\mathrm{d}k
\]

\[
\Psi\left(x,t\right)=\left(\frac{1}{\pi\alpha^{2}}\right)^{\frac{1}{4}}\frac{1}{\sqrt{1+\i\xi t}}\exp\left(\frac{-\frac{x^{2}}{2\alpha^{2}}+\i\left(kx-\omega_{0}t\right)}{1+\i\xi t}\right)
\]

ただし$\xi=\frac{\hbar}{m\alpha^{2}},\omega_{0}=\frac{\hbar k_{0}^{2}}{2m}$

粒子の存在確率分布
\begin{align*}
P\left(x,t\right) & =\left|\Psi\left(x,t\right)\right|^{2}=\Psi^{*}\left(x,t\right)\Psi\left(x,t\right)\\
 & =\sqrt{\frac{1}{\alpha^{2}\pi\left(1+\xi^{2}t^{2}\right)}}\exp\left(-\frac{\left(x-v_{g}t\right)^{2}}{\alpha^{2}\left(1+\xi^{2}t^{2}\right)}\right)
\end{align*}

ただし$v_{g}=\frac{\hbar k_{0}}{m}$

この中心:$x=v_{g}t$ c.f. $v_{g}=\pd{\omega}k=\frac{\hbar k}{m}$

即ち、波束の中心が動く速度$v_{g}$が群速度を求める式と矛盾しない。

この幅: $\frac{\alpha}{\sqrt{2}}\sqrt{1+\xi^{2}t^{2}}$

$\sqrt{1+\xi^{2}t^{2}}$が時間の経過による波束の増加を表している。$t$が大きいとき、幅は$\frac{\alpha}{\sqrt{2}}\xi t$。

波束の幅増加を表す係数
\[
\sim\xi t=\frac{\hbar}{m\alpha^{2}}t
\]

$\alpha$: $t=0$における波束幅
\begin{center}
\includegraphics{images/ElectronicPhysics/4-2}
\par\end{center}

\[
\hat{H}\hat{p}-\hat{p}\hat{H}=\i\hbar\pd Vx
\]

\begin{align*}
\hat{a} & =\hat{\dot{v}}=\pd{}t\left(\frac{\hat{p}}{m}\right)+\frac{\i}{\hbar}\left[\hat{H}\frac{\hat{p}}{m}-\frac{\hat{p}}{m}\hat{H}\right]\\
 & =\frac{1}{m}\frac{\i}{\hbar}\i\hbar\pd Vx=\frac{1}{m}\pd Vx
\end{align*}

\[
m\left\langle a\right\rangle =-\pd Vx=F\left(\text{外力}\right)
\]


\paragraph{ポテンシャル障壁への衝突}
\begin{center}
\includegraphics{images/ElectronicPhysics/4-3}
\par\end{center}

解くべきSchrödinger方程式は
\[
\left(-\frac{\hbar^{2}}{2m}\dd{}x+V\right)\varphi=\varepsilon\varphi
\]

\begin{itemize}
\item 領域I

\[
-\frac{\hbar^{2}}{2m}\dd{\varphi}x=\varepsilon\varphi
\]

\[
\varphi_{I}=A\e^{\i kx}+B\e^{-\i kx}
\]

$A,B$: 係数、$\frac{\hbar^{2}k^{2}}{2m}=\varepsilon$

$k>0$→$+x$方向に進む波

$k<0$→$-x$方向に進む波
\item 領域III

粒子は$+x$方向に進むのみ。

\[
\varphi_{III}=T\e^{\i kx}
\]

$T$: 係数、$\frac{\hbar^{2}k^{2}}{2m}=\varepsilon$
\item 領域II

粒子の運動エネルギー$\varepsilon$ < 障壁高さ$V_{0}$

\[
\frac{\hbar^{2}}{2m}\d{\varphi_{II}}x=\left(V_{0}-\varepsilon\right)\varphi_{II}
\]

\[
\varphi_{II}=C\e^{-\beta x}+D\e^{\beta x}
\]

$C,D$: 係数、$\beta$: 実数、$\frac{\hbar^{2}\beta^{2}}{2m}=V_{0}-\varepsilon$
\end{itemize}
全空間でSchrödinger方程式が成立するためには、全空間で$\Psi$が$x$について2階微分可能である必要がある。→$\Psi$が連続、$\pd{\Psi}x$が連続

$x=0$で、
\begin{align*}
\Psi_{I}\left(0\right) & =\Psi_{II}\left(0\right)\\
\pd{\Psi_{I}}x\left(0\right) & =\pd{\Psi_{II}}x\left(0\right)
\end{align*}

$x=b$で、
\begin{align*}
\Psi_{II}\left(b\right) & =\Psi_{III}\left(b\right)\\
\pd{\Psi_{II}}x\left(b\right) & =\pd{\Psi_{III}}x\left(b\right)
\end{align*}

上式より$A,B,C,D$についての制約式が2個、下式より$C,D,T$についての制約式が2個出てくる。

これに規格化条件を加えて$A,B,C,D,T$が求まる。

\section*{第5回}

\paragraph{前半メモ}
\begin{center}
\includegraphics{images/ElectronicPhysics/5-1}
\par\end{center}

領域3

\[
\varphi_{III}=T\e^{\i kx}
\]

\[
T=\frac{2\i\beta k}{F}\e^{-\i kb}
\]

\[
F=\left(k^{2}-\beta^{2}\right)\sinh\left(\beta b\right)+2\i\beta k\cosh\left(\beta b\right)
\]

波束の群速度
\[
v_{G}=\frac{1}{\hbar}\underbrace{\when{\pd Ek}_{k=k_{0}}}_{V\text{が一定の空間中}}
\]

\[
E=\frac{\hbar^{2}k^{2}}{2m}
\]

\[
v_{G}=\frac{\hbar k_{0}}{m}
\]

電子を例に取り、$m=9.1\times10^{-31}\mathrm{kg},v_{G}=1\times10^{6}\mathrm{m}$の時を考える。

\[
k_{0}=\frac{mv_{G}}{\hbar}=1\times10^{10}\mathrm{m^{-1}}
\]

\begin{center}
\includegraphics{images/ElectronicPhysics/5-2}
\par\end{center}

フーリエ級数より
\[
\left(\delta x\right)\left(\delta k\right)\sim1
\]

\[
\delta x=10\mathrm{nm}\rightarrow\delta k\sim10\mathrm{m^{-1}}
\]

\[
\left\langle E\right\rangle =\frac{\hbar^{2}k_{0}^{2}}{2m}\sim\frac{\left(10^{-34}\mathrm{Js}\right)^{2}\left(10^{10}\mathrm{m^{-1}}\right)^{2}}{\left(2\right)\left(10^{-30}\mathrm{kg}\right)}\sim0.5\times10^{-18}\mathrm{J}\longrightarrow3\mathrm{eV}
\]

\begin{center}
\includegraphics{images/ElectronicPhysics/5-3}
\par\end{center}

フラックス(J)=単位時間.単位時間あたりの量
\begin{center}
\includegraphics{images/ElectronicPhysics/5-4}
\par\end{center}

粒子数の収支(単位断面積あたり)
\[
J\left(x\right)\Delta t=\pd{\rho}t\Delta x\Delta t+J\left(x+\Delta x\right)\Delta t
\]

\[
\frac{J\left(x+\Delta x\right)-J\left(x\right)}{\Delta x}=-\pd{\rho}t
\]

\[
\pd Jx=-\pd{\rho}t
\]

連続の式である。

古典力学的に運動する粒子のフラックス$J$
\begin{center}
\includegraphics{images/ElectronicPhysics/5-5}
\par\end{center}

\[
J=\rho v
\]

量子力学的に運動する粒子のフラックス→確率密度のフラックス

\[
J_{p}=\Re\left[\Psi^{*}\hat{v_{x}}\Psi\right]
\]

(連続の式$\pd{J_{p}}x=-\pd{\left|\Psi\right|^{2}}t$を満たす。$\left|\Psi\right|^{2}$:
ある場所における1つの粒子を見出す確率の密度)

\[
\hat{v_{x}}=\frac{\hat{p_{x}}}{m}
\]

\begin{align*}
\Psi^{*}\hat{v_{x}}\Psi & =\Psi^{*}\frac{\hat{p_{x}}}{m}\Psi\\
 & =\Psi^{*}\left(\frac{-\i\hbar}{m}\right)\pd{\Psi}x\\
 & =-\frac{\i\hbar}{m}\Psi^{*}\pd{\Psi}x\\
\Re\left[\Psi^{*}\hat{v_{x}}\Psi\right] & =\frac{1}{2}\left(\Psi^{*}\hat{v}_{x}\Psi+\left(\Psi^{*}\hat{v_{x}}\Psi\right)^{*}\right)\\
 & =\frac{1}{2}\left(\left(\frac{-\i\hbar}{m}\right)\Psi^{*}\pd{\Psi}x+\left(\frac{\i\hbar}{m}\right)\Psi\pd{\Psi^{*}}x\right)\\
 & =\frac{-\i\hbar}{2m}\left(\Psi^{*}\pd{\Psi}x-\Psi\pd{\Psi^{*}}x\right)
\end{align*}

\begin{equation}
\pd{J_{p}}x=\frac{-\i\hbar}{2m}\left(\Psi^{*}\pdd{\Psi}x-\Psi\pdd{\Psi^{*}}x\right)
\end{equation}

Schrödinger方程式を用いて
\[
\i\hbar\pd{\Psi}t=-\frac{\hbar^{2}}{2m}\pdd{\Psi}x+V\Psi
\]

\[
\pdd{\Psi}x=-\frac{2\i m}{\hbar}\pd{\Psi}t+\frac{2m}{\hbar^{2}}V\Psi
\]

c.c.をとって、
\[
\pdd{\Psi^{*}}x=-\frac{2m}{\hbar}\pd{\Psi^{*}}t+\frac{2m}{\hbar^{2}}V\Psi^{*}
\]

(7)に代入して、
\begin{align*}
\pd{J_{p}}x & =\frac{-\i\hbar}{2m}\left[-\frac{2\i m}{\hbar}\Psi^{*}\pd{\Psi}t+\frac{2m}{\hbar^{2}}V\Psi^{*}\Psi+\frac{2m}{\hbar}\Psi\pd{\Psi^{*}}t-\frac{2m}{\hbar^{2}}V\Psi\Psi^{*}\right]\\
 & =-\left(\Psi^{*}\pd{\Psi}t+\Psi\pd{\Psi^{*}}t\right)\\
 & =-\pd{\Psi^{*}\Psi}t\\
 & =-\pd{\left|\Psi\right|^{2}}t
\end{align*}

\begin{center}
\includegraphics{images/ElectronicPhysics/5-6}
\par\end{center}

粒子が壁を透過する確率
\[
\eta=\frac{\int_{0}^{\eta}J_{p}\left(x_{\text{out}},t\right)\mathrm{d}t}{\int_{0}^{\eta}J_{p}\left(x_{in},t\right)\mathrm{d}t}
\]

固有関数$\varphi_{k}=A_{k}\e^{\i\left(kx-\omega t\right)}$に関して、
\begin{align*}
J_{p} & =\Re\left[A_{k}^{*}\e^{-\i\left(kx-\omega t\right)}\frac{-\i\hbar}{m}\pd{}x\left(A_{k}\e^{\i\left(kx-\omega t\right)}\right)\right]\\
 & =\Re\left[A_{k}^{*}A_{k}\frac{\hbar k}{m}\right]\\
 & =\underbrace{\left|A_{k}\right|^{2}}_{\text{粒子の存在確率密度}}\underbrace{\frac{\hbar k}{m}}_{\text{(粒子の)速度}}
\end{align*}

透過確率
\begin{align*}
\eta & =\left|T\right|^{2}\\
 & =\frac{4\beta^{2}k^{2}}{\left(k^{2}-\beta^{2}\right)^{2}\sinh^{2}\left(\beta b\right)+4\beta^{2}k^{2}\cosh^{2}\left(\beta b\right)}
\end{align*}

ただし、
\[
T=\frac{2\i\beta k\e^{-\i kb}}{\left(k^{2}-\beta^{2}\right)\sinh\left(\beta b\right)+2\i\beta k\cosh\left(\beta b\right)}
\]
を用いた。

\[
\cosh^{2}\left(x\right)-\sinh^{2}\left(x\right)=1
\]
を用いて、
\begin{align*}
\eta & =\frac{4\beta^{2}k^{2}}{4\beta^{2}k^{2}+\left(k^{2}+\beta^{2}\right)^{2}\sinh^{2}\left(\beta b\right)}\\
 & =\frac{1}{1+\frac{\left(k^{2}+\beta^{2}\right)^{2}}{4\beta^{2}k^{2}}\sinh^{2}\left(\beta b\right)}
\end{align*}

ここで領域I, IIIにおいて
\[
\frac{\hbar^{2}k^{2}}{2m}=E\rightarrow k^{2}=\frac{2mE}{k^{2}}
\]

領域IIにおいて
\[
\frac{\hbar^{2}\beta^{2}}{2m}=V_{0}-E\rightarrow\beta^{2}=\frac{2m}{k^{2}}\left(V_{0}-E\right)
\]

以上を代入して
\[
\eta=\frac{1}{1+\frac{V_{0}^{2}}{4E\left(V_{0}-E\right)}\sinh\left(\beta b\right)}
\]
が粒子が障壁を透過する確率となる。

\paragraph{例}

電子 $E=3\mathrm{eV},V_{0}=10\mathrm{eV}$

\begin{align*}
\beta^{2} & =\frac{\left(2\times10^{-30}\mathrm{kg}\right)}{\left(10^{-34}\mathrm{kgm^{2}s^{-1}}\right)}\left(7\times1.6\times10^{-19}\mathrm{kgm^{2}s^{-1}}\right)\\
 & =2\times10^{20}\mathrm{m^{-2}}\\
\beta & \sim10^{10}\mathrm{m^{-1}}
\end{align*}

領域II
\[
\varphi=C\e^{-\beta x}+D\e^{\beta x}
\]

\begin{center}
\includegraphics{images/ElectronicPhysics/5-8}
\par\end{center}

$\beta b\gg1$であることが多いので、
\[
\eta\sim\e^{-2\beta b}=\e^{-\frac{2b}{\frac{1}{\beta}}}
\]

\begin{center}
\includegraphics{images/ElectronicPhysics/5-9}
\par\end{center}

\begin{center}
\includegraphics{images/ElectronicPhysics/5-10}
\par\end{center}

\[
F=U-TS
\]

$F$: 自由エネルギー

$U$: 内部エネルギー

$TS$: エントロピー

$f\left(E\right)$: 電子が量子状態に見いだされる確率

\rule[0.5ex]{1\columnwidth}{1pt}

1. 1次元のSchrödinger方程式の場合

1つのエネルギー固有値⇔1つの波動関数(縮退がない)

\[
\left(-\frac{\hbar^{2}}{2m}\dd{}x+V\left(X\right)\right)\varphi=E\varphi
\]
を$\varphi_{1},\varphi_{2}$ともに満たすと仮定
\[
-\frac{\hbar^{2}}{2m}\frac{1}{\varphi_{1}}\dd{\varphi_{1}}x=-\frac{\hbar^{2}}{2m}\frac{1}{\varphi_{2}}\dd{\varphi_{2}}x=E-V\left(X\right)
\]

\[
\frac{1}{\varphi_{1}}\dd{\varphi_{1}}x-\varphi_{2}\dd{\varphi_{2}}x=0
\]

\begin{center}
\includegraphics{images/ElectronicPhysics/5-11}
\par\end{center}

定理より、ポテンシャル$V\left(X\right)$が$X=0$に対して対称(偶関数)
\[
V\left(-X\right)=V\left(X\right)
\]
ならば、波動関数は$x=0$に対して
\begin{itemize}
\item 偶関数 $\varphi\left(-x\right)=\varphi\left(x\right)$ もしくは
\item 奇関数 $\varphi\left(-x\right)=-\varphi\left(x\right)$
\end{itemize}
\[
\varphi_{2}\dd{\varphi_{1}}x-\varphi_{1}\dd{\varphi_{2}}x=0
\]

\[
\d{}x\left(\d{\varphi_{1}}x\varphi_{2}-\varphi_{1}\d{\varphi_{2}}x\right)=0
\]

\[
\d{\varphi_{1}}x\varphi_{2}-\varphi_{1}\d{\varphi_{2}}x=C
\]

$\int_{-\infty}^{\infty}\left|\varphi\right|^{2}\mathrm{d}x$が有限の値を取る→$x\rightarrow I\infty$のとき$\varphi\rightarrow0$

\[
\Rightarrow\d{\varphi_{1}}x\varphi_{2}-\varphi_{1}\d{\varphi_{2}}x=0=\Rightarrow\frac{1}{\varphi_{1}}\d{\varphi_{1}}x=\frac{1}{\varphi_{2}}\d{\varphi_{2}}x
\]

片々積分して、
\[
\ln\left(\varphi_{1}\right)=\ln\left(\varphi_{2}\right)+C'
\]

\[
\varphi_{2}=C''\varphi_{1}
\]
より、仮定に矛盾する。従って、1つの$E$には1つの独立な$\varphi$。

2.

\[
\left(-\frac{\hbar^{2}}{2m}\dd{}x+V\left(x\right)\right)\varphi\left(x\right)=E\varphi\left(x\right)
\]

$x\rightarrow-x$に置換

\[
\left(-\frac{\hbar^{2}}{2m}\dd{}x+V\left(-x\right)\right)\varphi\left(-x\right)=E\varphi\left(-x\right)
\]

$\varphi\left(x\right)$と$\varphi\left(-x\right)$は同一の$E$についてSchrödinger方程式の固有関数

→1. より$\varphi\left(x\right)=C\varphi\left(-x\right)$

\[
\varphi\left(-x\right)=C\varphi\left(-\left(-x\right)\right)=C\varphi\left(x\right)
\]
を用いて、
\[
\varphi\left(x\right)=C^{2}\varphi\left(x\right)
\]

\[
C^{2}=1,C\pm1
\]


\section*{第6回}

\paragraph{試験について}
\begin{itemize}
\item 9時開始
\item 持ち込み不可の第1部と持ち込み可の第2部の2部に分かれている。配点は1:1。
\end{itemize}

\paragraph{定理}

ポテンシャル$V\left(x\right)$が$x=0$について線対称
\[
V\left(-x\right)=V\left(x\right)
\]

⇒波動関数$\varphi\left(x\right)$は$x=0$について線対称/点対称
\begin{enumerate}
\item 1次元の問題では「1つのエネルギー$\varepsilon$→1つだけの独立な$\varphi\left(x\right)$」
\item 
\[
\left(\frac{\hat{p}^{2}}{2m}+V\left(x\right)\right)\varphi\left(x\right)=\varepsilon\varphi\left(x\right)
\]

$x\rightarrow-x$に置換して、
\[
\left(\frac{\hat{p}^{2}}{2m}+V\left(-x\right)\right)\varphi\left(-x\right)=\varepsilon\varphi\left(-x\right)
\]

$\varphi\left(x\right)$と$\varphi\left(-x\right)$は同じエネルギー$\varepsilon$に対応\\
⇒$\varphi\left(-x\right)=C\varphi\left(x\right)$ ($C$: 複素係数)

∵$\varphi\left(-x\right)$と$\varphi\left(x\right)$は従属

\[
\varphi\left(x\right)=\varphi\left(-\left(-x\right)\right)=C\varphi\left(-x\right)=C^{2}\varphi\left(x\right)
\]

$C^{2}=1\Rightarrow C=\pm1$したがって$\varphi\left(-x\right)=\pm\varphi\left(x\right)$
\end{enumerate}

\paragraph{無限大ポテンシャル障壁による閉じ込め}
\begin{center}
\includegraphics{images/ElectronicPhysics/6-1}
\par\end{center}
\begin{enumerate}
\item $V\left(x\right)=0$ for $-\frac{l}{2}\leq x\leq\frac{l}{2}$
\item $V\left(x\right)\rightarrow\infty$ for $x<-\frac{l}{2},\frac{l}{2}<x$
\end{enumerate}
2. について、
\[
\left(\frac{\hat{p}^{2}}{2m}+V\left(x\right)\right)\varphi\left(x\right)=\varepsilon\varphi\left(x\right)\Rightarrow\varphi\left(x\right)=0
\]

1.について、
\[
-\frac{\hbar^{2}}{2m}\dd{\varphi\left(x\right)}x=\varepsilon\varphi\left(x\right)
\]

一般解
\[
\varphi=A\e^{\i kx}+B\e^{-\i kx}
\]

境界条件

\[
\varphi\left(-\frac{l}{2}\right)=\varphi\left(\frac{l}{2}\right)=0
\]

\[
\frac{\hbar^{2}k^{2}}{2m}=\varepsilon
\]

$V\left(x\right)$が$x=0$に対称→$\varphi\left(x\right)$は$x=0$について、
\begin{itemize}
\item 線対称→$\cos\left(kx\right)$
\item 点対称→$\sin\left(kx\right)$
\end{itemize}
\begin{center}
\includegraphics{images/ElectronicPhysics/6-2}
\par\end{center}

$\varphi$の波長 $\lambda=2l,l,\frac{2l}{3},\frac{l}{2},\cdots,\frac{2l}{n}$
($n$: 自然数)

\[
k=\frac{2\pi}{\lambda}=\frac{\pi}{l},\frac{2\pi}{l},\frac{3\pi}{l},\cdots,\frac{\pi}{l}n
\]

\[
\varphi\left(x\right)=\begin{cases}
C\cos\left(kx\right) & n\text{が奇数}\\
C\sin\left(kx\right) & n\text{が偶数}
\end{cases}
\]

ただし$k=\frac{\pi}{l}n$

規格化条件

\[
1=\int_{-\frac{l}{2}}^{\frac{l}{2}}\mathrm{d}x\left|\varphi\left(x\right)\right|^{2}=\int_{-\frac{l}{2}}^{\frac{l}{2}}\mathrm{d}xC^{2}\cos^{2}\left(kx\right)=C^{2}\int_{-\frac{l}{2}}^{\frac{l}{2}}\mathrm{d}x\left(\frac{1+\cos\left(2kx\right)}{2}\right)=C^{2}\left[\frac{1}{2}+\frac{\sin\left(2kx\right)}{4}\right]_{-\frac{l}{2}}^{\frac{l}{2}}=\frac{C^{2}}{2}l
\]

\[
\therefore C=\pm\sqrt{\frac{2}{l}}
\]

\[
\varepsilon=\frac{\hbar^{2}k^{2}}{2m}=\frac{\pi^{2}\hbar^{2}}{2ml^{2}}n^{2}
\]

\[
n=1\rightarrow\varepsilon=\varepsilon_{0}=\frac{\pi^{2}\hbar^{2}}{2ml^{2}}>0
\]

ゼロ点エネルギーである。

c.f. 熱振動のエネルギー

温度$T$→$\left\langle \varepsilon\right\rangle \sim k_{B}T$ (粒子1個あたり)

これをエネルギーの等分配則という。

\[
T=300\mathrm{K}\Rightarrow\left\langle \varepsilon\right\rangle =k_{B}T=\frac{1}{40}\mathrm{eV}
\]

$\varepsilon_{0}=\frac{1}{40}\mathrm{eV}$になる$l$の値

\[
\frac{1}{40}\times\left(1.6\times10^{-19}J\right)=\frac{\left(10\right)\left(1\times10^{-19}\mathrm{J^{2}s^{2}}\right)}{\left(2\right)\left(1\times10^{-30}\mathrm{kg}\right)l^{2}}
\]

\[
l^{2}=\frac{5\times10^{-38}\mathrm{m^{2}s^{-2}Js^{2}}}{4\times10^{-21}\mathrm{J}}\doteqdot1\times10^{-17}\mathrm{m^{2}}
\]

\[
l\sim3\times10^{-9}\mathrm{m}
\]


\paragraph{3次元の閉じ込め空間 (ポテンシャル障壁無限大)}

$V=0$ for $-\frac{l}{2}\leq x,y,z\leq\frac{l}{2}$

その他の領域 $V\rightarrow\infty\longrightarrow\varphi\left(x,y,z\right)=0$

$V=0$の領域
\[
-\frac{\hbar^{2}}{2m}\left(\pdd{}x+\pdd{}y+\pdd{}z\right)\Psi\left(x,y,z\right)=E\Psi\left(x,y,z\right)
\]

$\Psi\left(x,y,z\right)=\varphi\left(x\right)\varphi\left(y\right)\varphi\left(z\right)$:
変数分離

\[
-\left(\pdd{\varphi\left(x\right)}x\varphi\left(y\right)\varphi\left(z\right)+\varphi\left(x\right)\pdd{\varphi\left(y\right)}y\varphi\left(z\right)+\varphi\left(x\right)\varphi\left(y\right)\pdd{\varphi\left(z\right)}z\right)=\frac{2mE}{\hbar^{2}}\varphi\left(x\right)\varphi\left(y\right)\varphi\left(z\right)
\]

両辺を$\varphi\left(x\right)\varphi\left(y\right)\varphi\left(z\right)$で割る。

\[
-\left(\frac{1}{\varphi\left(x\right)}\pdd{\varphi\left(x\right)}x+\frac{1}{\varphi\left(y\right)}\pdd{\varphi\left(y\right)}y+\frac{1}{\varphi\left(z\right)}\pdd{\varphi\left(z\right)}z\right)=\frac{2mE}{\hbar^{2}}
\]

右辺が定数なので、一般で成り立つために左辺各項も定数。

$E=\varepsilon_{x}+\varepsilon_{y}+\varepsilon_{z}$として、
\[
-\frac{1}{\varphi\left(x\right)}\pdd{\varphi\left(x\right)}x=\frac{2m\varepsilon_{x}}{\hbar^{2}}
\]

$y,z$に関しても同様。

\[
-\pdd{\varphi\left(x\right)}x=\frac{2m\varepsilon_{x}}{\hbar^{2}}\varphi\left(x\right)
\]

←1次元閉じ込めの場合と同じSchrödinger方程式

すなわち、
\begin{enumerate}
\item 1次元の問題を解く→$\varphi\left(x\right),\varphi\left(y\right),\varphi\left(z\right),\varepsilon_{x},\varepsilon_{y},\varepsilon_{z}$
\item $\Psi\left(x,y,z\right)=\varphi\left(x\right)\varphi\left(y\right)\varphi\left(z\right),E=\varepsilon_{x}+\varepsilon_{y}+\varepsilon_{z}$

\[
E=\frac{\hbar^{2}}{2m}\left(k_{x}^{2}+k_{y}^{2}+k_{z}^{2}\right)=\frac{\pi^{2}\hbar^{2}}{2ml^{2}}\left(n_{x}^{2}+n_{y}^{2}+n_{z}^{2}\right)
\]

\end{enumerate}
1つのエネルギー固有値に複数の量子状態が対応⇒「縮退(degeneration)」

\paragraph{有限高さのポテンシャル}
\begin{center}
\includegraphics{images/ElectronicPhysics/6-3}
\par\end{center}

$V\left(x\right)=0$ for $-\frac{l}{2}\leq x\leq\frac{l}{2}$

$V\left(x\right)=V_{0}$ for $x<-\frac{l}{2},\frac{l}{2}<x$

Schrödinger方程式

領域A: $-\frac{l}{2}\leq x\leq\frac{l}{2}$ :
\[
-\frac{\hbar^{2}}{2m}\dd{\varphi}x=\varepsilon\varphi
\]

領域B: $x<-\frac{l}{2},\frac{l}{2}<x$:
\[
-\frac{\hbar^{2}}{2m}\dd{\varphi}x=\left(\varepsilon-V_{0}\right)\varphi
\]

※$V_{0}>\varepsilon$とする。

$V\left(x\right)$は$x=0$について線対称⇒$\varphi\left(x\right)$は$x=0$について線対称(偶関数)/点対称(奇関数)
\begin{enumerate}
\item $\varphi\left(x\right)$が偶関数

領域Aでは
\[
\varphi\left(x\right)=A\cos\left(kx\right)
\]

ただし、
\begin{equation}
\frac{\hbar^{2}k^{2}}{2m}=\varepsilon
\end{equation}

領域Bでは、$\varphi\left(x\right)\propto\e^{\beta x}$ or $\e^{-\beta x}$
($\beta$: 実数)
\begin{center}
\includegraphics{images/ElectronicPhysics/6-4}
\par\end{center}

$x\rightarrow\pm\infty$で$\varphi\left(x\right)$が発散しない(規格化条件)

領域Bで、
\begin{align*}
\varphi\left(x\right) & =C\e^{-\beta x}\:\left(x>\frac{l}{2}\right)\\
\varphi\left(x\right) & =C\e^{\beta}\:\left(x<-\frac{l}{2}\right)
\end{align*}

ただし、
\begin{equation}
\frac{\hbar^{2}\beta^{2}}{2m}=V_{0}-\varepsilon
\end{equation}

接続条件 $x=\frac{l}{2}$
\begin{itemize}
\item $\varphi\left(\frac{l}{2}\right)$が連続
\begin{equation}
A\cos\left(\frac{kx}{2}\right)=C\e^{-\frac{\beta l}{2}}
\end{equation}
\item $\d{\varphi}x\left(\frac{l}{2}\right)$が連続
\begin{equation}
-kA\sin\left(\frac{kx}{2}\right)=-\beta C\e^{-\frac{\beta l}{2}}
\end{equation}
\end{itemize}
(8),(9), (10), (11) より、$k,\beta$の制約式

(8)と(9)より、
\[
k^{2}+\beta^{2}=\frac{2mV_{0}}{\hbar^{2}}
\]

(10), (11) を辺々割る。

\[
k\tan\left(\frac{kl}{2}\right)=\beta
\]

これを解くと、$k,\beta$の値が求まる。

変数変換 $\xi=\frac{kl}{2},\eta=\frac{\beta l}{2}$

\begin{align*}
\xi^{2}+\eta^{2} & =\frac{mV_{0}l^{2}}{2\hbar^{2}}\\
\eta & =\xi\tan\left(\xi\right)
\end{align*}

\begin{center}
\includegraphics{images/ElectronicPhysics/6-5}
\par\end{center}

$k>0,\beta>0$を考える。
\begin{center}
\includegraphics{images/ElectronicPhysics/6-6}
\par\end{center}
\item $\varphi\left(x\right)$が奇関数
\end{enumerate}
領域Aでは$\varphi=A\sin\left(kx\right)$

領域Bでは$\varphi=C\e^{-\beta x}\left(x>\frac{l}{2}\right),\varphi=-C\e^{\beta x}\left(x<\frac{l}{2}\right)$
\begin{center}
\includegraphics{images/ElectronicPhysics/6-7}
\par\end{center}

接続条件 $\left(x=\frac{l}{2}\right)$
\begin{itemize}
\item $\varphi\left(\frac{l}{2}\right)$が連続
\begin{equation}
A\sin\left(\frac{kl}{2}\right)=C\e^{-\frac{\beta l}{2}}
\end{equation}
\item $\d{\varphi}x\left(\frac{l}{2}\right)$が連続
\begin{equation}
kA\cos\left(\frac{kl}{2}\right)=-\beta C\e^{-\frac{\beta l}{2}}
\end{equation}
\end{itemize}
(12), (13) を辺々割る
\[
\beta=-\frac{k}{\tan\left(\frac{kl}{2}\right)}
\]

変数変換 ($\xi=\frac{kl}{2},\eta=\frac{\beta l}{2}$)

\[
\eta=-\frac{\xi}{\tan\left(\xi\right)}
\]

ここから$k,\beta$が求められる。

\paragraph{後半メモ}

$\xi=\frac{kl}{2},\eta=\frac{\beta l}{2}$

漸近線 $\xi=\frac{\pi}{2}n,k=\frac{\pi}{l}n$

無限大ポテンシャルに対する$k$の値である。

c.f. 無限大ポテンシャルによる閉じ込め

$\varphi$の波長 $\frac{2l}{n}$

$k=\frac{2\pi}{\lambda}=\frac{\pi}{l}n$
\begin{center}
\includegraphics{images/ElectronicPhysics/6-8}
\par\end{center}

$k>\frac{l}{2}$ $\varphi\left(x\right)=C\e^{-\beta x}$

$\frac{1}{\beta}$: 波動関数の侵入長

\[
\frac{\hbar^{2}\beta^{2}}{2m}=V_{0}-\varepsilon,\frac{1}{\beta}=\frac{\hbar}{\sqrt{2m\left(V_{0}\varepsilon\right)}}
\]

$V_{0}-\varepsilon$大→侵入長小

\section*{第7回}

\paragraph{Blochの定理}

周期$a$のポテンシャル→
\begin{equation}
\varphi\left(x\right)=u\left(x\right)\e^{\i kx}
\end{equation}

もう一つの表現($x\mapsto x+a$)
\begin{equation}
\varphi\left(x+a\right)=\e^{\i ka}\varphi\left(x\right)
\end{equation}

(15)→(14)の証明:

$u\left(x\right)=\varphi\left(x\right)\e^{-\i kx}$を考える。
\[
u\left(x+a\right)=\varphi\left(x+a\right)\e^{-\i k\left(x+a\right)}=\varphi\left(x\right)\e^{\i kx}=u\left(x\right)
\]

⋯$u\left(x\right)$は周期$a$の関数

\[
\varphi\left(x\right)=u\left(x\right)\e^{\i kx}
\]


\paragraph{Blochの定理の証明}

Schrödinger方程式
\begin{equation}
\left(\frac{\hat{p}^{2}}{2m}+V\left(x\right)\right)\varphi\left(x\right)=E\varphi\left(x\right)
\end{equation}

$x\mapsto x+a$
\begin{equation}
\left(\frac{\hat{p}^{2}}{2m}+V\left(x+a\right)\right)\varphi\left(x+a\right)=E\varphi\left(x+a\right)
\end{equation}

(16) と (17) は同じSchrödinger方程式である。(1次元問題)

⇒$\varphi\left(x\right),\varphi\left(x+a\right)$は従属

\[
\varphi\left(x+a\right)=C\varphi\left(x\right)
\]

($C$: 複素定数)

\[
\varphi\left(x+Na\right)=C\varphi\left(x+\left(N-a\right)a\right)=C^{2}\varphi\left(x+\left(N-2\right)a\right)=\cdots=C^{N}\varphi\left(x\right)
\]

同期境界条件を考える。
\[
\varphi\left(x+Na\right)=\varphi\left(x\right)\Rightarrow C^{N}=1
\]

\begin{align*}
C & =\e^{\i2\pi\frac{m}{N}}\\
 & =\e^{\i\frac{2\pi m}{Na}a}\\
 & =\e^{\i k_{m}a}
\end{align*}

ただし$k_{m}=\frac{2\pi m}{Na}\rightarrow\lambda_{m}=\frac{Na}{m}$
\begin{center}
\includegraphics{images/ElectronicPhysics/7-1}
\par\end{center}

\[
\varphi\left(x+a\right)=\e^{\i k_{m}a\varphi\left(x\right)}\Rightarrow\varphi\left(x\right)=u\left(x\right)\e^{\i kx}
\]

\[
\varphi\left(x\right)=\e^{\i kx}
\]

\[
\Psi\left(x,t\right)=\varphi\left(x\right)\e^{-\i\omega t}=\e^{\i\left(kx-\omega t\right)}=\e^{\i k\left(x-\frac{\omega}{k}t\right)}
\]

$\frac{\omega}{k}$: 波の進む速度
\begin{center}
\includegraphics{images/ElectronicPhysics/7-2}
\par\end{center}

意味のある$m$の範囲

\[
m=0,1,2,\cdots,N-1\left(m=N\text{は}m=0\text{と同}\right)
\]

→むしろ

\[
m=-\frac{N}{2},\cdots,0,1,\cdots,\frac{N}{2}-1
\]

⇒$k$が-も+もとれる。-方向方向両方に進む波を表せる。

$\left|k_{m}\right|$の最大値($\left|m\right|=\frac{N}{2}$のとき)

\[
k_{max}=\frac{\pi}{a},\lambda_{min}=2a
\]

\begin{center}
\includegraphics{images/ElectronicPhysics/7-3}
\par\end{center}

ポテンシャル一定の空間 $\varphi=A\e^{\i kx}$

\[
E_{f}=\frac{\hbar^{2}k^{2}}{2m}
\]

→$k$の値に制限なし。

周期的なポテンシャル $\varphi=u\left(x\right)\e^{\i kx}$

ポテンシャルの1周期分に注目 $-b<x\leq a-b$
\begin{itemize}
\item 領域I: $0\leq x\leq a-b,V\left(x\right)=0$

\[
-\frac{\hbar^{2}}{2m}\dd{\varphi_{I}}x=E\varphi_{I}
\]

$\frac{\hbar^{2}\alpha^{2}}{2m}=E$より、
\[
\varphi_{I}=A\e^{\i\alpha x}+B\e^{-\i\alpha x}
\]

\item 領域II: $-b<x<0$

\[
-\frac{\hbar^{2}}{2m}\dd{\varphi_{II}}x=\left(V_{0}-E\right)\varphi_{II}
\]

ただし$V_{0}>E$とする。

$\frac{\hbar^{2}\beta^{2}}{2m}=V_{0}-E$より、
\[
\varphi_{II}=C\e^{\beta x}+D\e^{-\beta x}
\]

\end{itemize}
接続条件

$x=0$
\begin{equation}
\varphi_{I}\left(x\right)=\varphi_{II}\left(x\right)\Rightarrow A+B=C+D
\end{equation}
\begin{equation}
\d{\varphi_{I}}x\left(x\right)=\d{\varphi_{II}}x\left(x\right)\Rightarrow\i\alpha\left(A-B\right)=\beta\left(C-D\right)
\end{equation}

$x=a-b$

\begin{equation}
\varphi_{I}\left(x\right)=\varphi_{III}\left(x\right)\Rightarrow A\e^{\i\alpha\left(a-b\right)}+B\e^{-\i\alpha\left(a-b\right)}=\e^{\i ka}\left(C\e^{\beta\left(a-b\right)}+D\e^{-\beta\left(a-b\right)}\right)
\end{equation}

$a-b<x\leq a$での$\varphi$
\[
\varphi_{III}\left(x\right)=\varphi_{II}\left(x\right)\e^{\i ka}
\]
(Blochの定理)

\begin{align}
\d{\varphi_{I}\left(x\right)}x=\d{\varphi_{III}\left(x\right)}x & =\i\alpha\left(A\e^{\i\alpha\left(a-b\right)}+B\e^{-\i\alpha\left(a-b\right)}\right)\\
 & =\e^{\i ka}\beta\left(C\e^{\beta\left(a-b\right)}-D\e^{-\beta\left(a-b\right)}\right)
\end{align}

(18), (19), (20), (22) より、
\[
\left(\begin{array}{cccc}
1 & 1 & -1 & -1\\
\i\alpha & -\i\alpha & -\beta & \beta\\
\e^{\i\alpha\left(a-b\right)} & \e^{-\i\alpha\left(a-b\right)} & -\e^{\i ka}\e^{-\beta b} & -\e^{\i ka}\e^{\beta b}\\
\i\alpha\e^{\i\alpha\left(a-b\right)} & -\i\alpha\e^{-\alpha\left(a-b\right)} & -\e^{\i ka}\beta\e^{-\beta b} & \e^{\i ka}\beta\e^{\beta b}
\end{array}\right)\left(\begin{array}{c}
A\\
B\\
C\\
D
\end{array}\right)=\left(\begin{array}{c}
0\\
0\\
0\\
0
\end{array}\right)
\]

$A,B,C,D$を求めたい⇒行列が逆行列を持たない⇒行列式=0

\[
\e^{ika}+\e^{-\i ka}=\frac{\beta^{2}-\alpha^{2}}{\alpha\beta}\sin\alpha\left(a-b\right)\sinh\left(\beta b\right)+2\cos\alpha\left(a-b\right)\cosh\left(\beta b\right)
\]

\begin{align*}
\cosh\left(x\right) & =\frac{\e^{x}-\e^{-x}}{2}\rightarrow1\left(x\rightarrow0\right)\\
\sinh\left(x\right) & =\frac{\e^{x}-\e^{-x}}{2}=\frac{\left(1+x\right)-\left(1-x\right)}{2}\rightarrow x\left(x\rightarrow0\right)
\end{align*}

\[
2\cos\left(ka\right)=\frac{\beta^{2}-\alpha^{2}}{\alpha\beta}\left(\sin\left(\alpha a\right)\right)\beta b+2\cos\left(\alpha a\right)
\]

\[
\beta^{2}-\alpha^{2}=\frac{2m\left(V_{0}-2E\right)}{\hbar^{2}}\doteqdot\frac{2mV_{0}}{\hbar^{2}}\:\because V_{0}\left(\rightarrow\infty\right)\gg E
\]

\[
\cos\left(ka\right)=\frac{mV_{0}b}{\hbar^{2}}a\frac{\sin\left(\alpha a\right)}{\alpha a}+\cos\left(\alpha a\right)
\]
($a$: ポテンシャルの周期、$k$: $\varphi$に含まれる $\alpha=\frac{\sqrt{2mE}}{\hbar}$
($E$: 粒子のエネルギー固有値))\\
←$A,B,C,D$が0にならないための制約式

⇒$\varphi$が存在するための制約式\\
←これを満たす$\left(k,E\right)$の組

⇒$A,B,C,D$が決まる($\varphi$が求められる)

\paragraph{後半メモ}

\[
\Psi=\sum_{k}C_{k}u_{k}\left(x\right)\e^{\i\left(kx-\omega t\right)}
\]
: 波束

$V\left(x\right)$が一定($u_{k\left(x\right)}$は$x$に依存しない)
\[
v_{g}=\frac{1}{\hbar}\when{\d Ek}_{k=k_{0}}
\]
: 群速度

$V\left(x\right)$が周期的→$u_{k}\left(x\right)$の$k$に関する変化が小さければ

\[
\Psi=u_{k_{0}}\left(x\right)\underbrace{\sum_{k}C_{k}\e^{\i\left(kx-\omega t\right)}}_{\text{ポテンシャル一定の場合の波束}}
\]

$\Rightarrow v_{g}=\frac{1}{\hbar}\when{\d Ek}_{k=k_{0}}$
\begin{center}
\includegraphics{images/ElectronicPhysics/7-4}
\par\end{center}

←光の電場(x方向) $E_{0}\cos\left(\omega t\right)$

空間的には電界一定
\begin{center}
\includegraphics{images/ElectronicPhysics/7-5}
\par\end{center}

⇔光のエネルギー 1eV $\lambda\sim1\mu m$

一様な電界の存在下(x方向に$E$)
\begin{center}
\includegraphics{images/ElectronicPhysics/7-6}
\par\end{center}

→ポテンシャル$eEx$

\[
E=E_{0}\cos\left(\omega t\right)
\]

ポテンシャル$-exE_{0}\cos\left(\omega t\right)$

$\varepsilon_{2}-\varepsilon_{1}=\hbar\omega$: 光を光子として暑かった時のエネルギー
\end{document}
