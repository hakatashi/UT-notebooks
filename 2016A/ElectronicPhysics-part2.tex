%% LyX 2.2.2 created this file.  For more info, see http://www.lyx.org/.
%% Do not edit unless you really know what you are doing.
\documentclass[oneside,english]{book}
\usepackage[T1]{fontenc}
\usepackage[utf8]{inputenc}
\usepackage[a5paper]{geometry}
\geometry{verbose,tmargin=2cm,bmargin=2cm,lmargin=1cm,rmargin=1cm}
\setcounter{secnumdepth}{3}
\setcounter{tocdepth}{3}
\setlength{\parskip}{\smallskipamount}
\setlength{\parindent}{0pt}
\usepackage{textcomp}
\usepackage{amsmath}
\usepackage{amssymb}

\makeatletter
%%%%%%%%%%%%%%%%%%%%%%%%%%%%%% User specified LaTeX commands.
\usepackage[dvipdfmx]{hyperref}
\usepackage[dvipdfmx]{pxjahyper}

\makeatother

\usepackage{babel}
\begin{document}

\title{2016-A 電子基礎物理 後半}

\author{教員: 山下真司 入力: 高橋光輝}

\maketitle
\global\long\def\pd#1#2{\frac{\partial#1}{\partial#2}}
\global\long\def\d#1#2{\frac{\mathrm{d}#1}{\mathrm{d}#2}}
\global\long\def\pdd#1#2{\frac{\partial^{2}#1}{\partial#2^{2}}}
\global\long\def\dd#1#2{\frac{\mathrm{d}^{2}#1}{\mathrm{d}#2^{2}}}
\global\long\def\ddd#1#2{\frac{\mathrm{d}^{3}#1}{\mathrm{d}#2^{3}}}
\global\long\def\e{\mathrm{e}}
\global\long\def\i{\mathrm{i}}
\global\long\def\j{\mathrm{j}}
\global\long\def\grad{\operatorname{grad}}
\global\long\def\rot{\operatorname{rot}}
\global\long\def\div{\operatorname{div}}
\global\long\def\diag{\operatorname{diag}}
\global\long\def\rank{\operatorname{rank}}
\global\long\def\prob{\operatorname{Prob}}
\global\long\def\cov{\operatorname{Cov}}
\global\long\def\when#1{\left.#1\right|}


\section*{第1回}
\begin{itemize}
\item 統計力学
\begin{itemize}
\item 古典統計
\item 量子統計
\end{itemize}
\end{itemize}
全微分
\[
\mathrm{d}E=\left(\pd ET\right)_{V}\mathrm{d}T+\left(\pd EV\right)_{T}\mathrm{d}V
\]

\begin{align*}
\Delta W & =F\cdot\Delta x\\
 & =P\cdot A\cdot\Delta x\\
 & =-P\Delta V
\end{align*}

比熱: 温度を上げるのに必要な熱量

→
\[
C_{v}=\left(\pd QT\right)_{V}=\left(\pd ET\right)_{V}
\]

(∵(0.3)式で$\mathrm{d}V=0,\mathrm{d}'Q=0$)

1mol

\[
E=\frac{3}{2}RT
\]

1mol当たり
\[
C_{p}=C_{v}+R
\]

マイアーの関係式

\paragraph{演習(1)}

\[
\mathrm{d}'Q=C_{v}\mathrm{d}T+\left[P+\left(\pd EV\right)_{T}\right]\mathrm{d}V
\]

$\mathrm{d}'Q=0,P=\frac{RT}{V},\pd EV=0$より、
\[
0=C_{v}\frac{\mathrm{d}T}{T}+\frac{R}{V}\cdot\mathrm{d}V
\]

積分して、
\[
\text{const}=C_{v}\log T+R\log V
\]

\[
\log T+\left(\gamma-1\right)\log V=\text{const}
\]

\[
\log T\cdot V^{\gamma-1}=\text{const}
\]

$T\propto PV$より、
\[
P\cdot V^{\gamma}=\text{const}
\]


\paragraph{演習(3)}

\[
F=E-TS
\]

\begin{align*}
\mathrm{d}F=\mathrm{d}E-\mathrm{d}\left(TS\right) & =\mathrm{d}E-T\mathrm{d}S-S\mathrm{d}T\\
 & =-S\mathrm{d}T-P\mathrm{d}V
\end{align*}

\begin{align*}
\mathrm{d}\left(\frac{F}{T}\right) & =\mathrm{d}\left(\frac{E}{T}\right)-\mathrm{d}S\\
 & =\frac{\mathrm{d}E}{T}-\frac{E}{T^{2}}\mathrm{d}T-\mathrm{d}S\\
 & =-\frac{E}{T^{2}}\mathrm{d}T-\frac{P}{T}\mathrm{d}V
\end{align*}

熱力学的エントロピー
\[
\mathrm{d}S=\frac{\mathrm{d}'Q}{T}
\]

統計力学的エントロピー
\[
S=k\log W_{N}\left(E\right)
\]


\paragraph{演習(6)}

\begin{align*}
\log G_{N}\left(E\right) & =\log\left(M+N-1\right)!-\log M!-\log\left(N-1\right)!\\
 & \sim\log\frac{\left(M+N-1\right)^{M+N-1}}{M^{M}\left(N-1\right)^{N-1}}\\
 & \sim\log\frac{\left(M+N\right)^{M+N}}{M^{M}N^{N}}\\
 & =\log\frac{\left(2N\right)^{2N}}{N^{2N}}\\
 & =\log2^{2N}
\end{align*}

\begin{align*}
G_{N}\left(E\right) & =2^{2N}\\
 & \sim10^{0.6N}
\end{align*}


\paragraph{演習(7)}

\begin{align*}
P_{1}\left(x\right) & =\frac{\left(M+N-x-2\right)!}{\left(M-x\right)!\left(N-2\right)!}\frac{M!\left(N-1\right)!}{\left(M+N-1\right)!}\\
 & =\frac{M\cdot\left(M-1\right)\cdots\left(M-x+1\right)\left(N-1\right)}{\left(M+N-1\right)\left(M+N-2\right)\cdots\left(M+N-x-1\right)}\\
 & \sim\frac{M^{x}N}{\left(M+N\right)^{x+1}}=\frac{m^{x}}{\left(1+m\right)^{x}}\cdot\frac{1}{1+m}
\end{align*}


\paragraph{演習(8)}

\begin{align*}
\sum_{x=0}^{\infty}p_{1}\left(x\right) & =\frac{1}{1+m}\sum_{x=0}^{\infty}\left(\frac{m}{1+m}\right)^{x}\\
 & =\frac{1}{1+m}\cdot\frac{1}{1-\frac{m}{1+m}}=1
\end{align*}

\begin{align*}
\sum_{x=0}^{\infty}xp_{1}\left(x\right) & =\frac{1}{1+m}\sum_{x=0}^{\infty}xa^{x}\\
 & =\frac{a}{1+m}\sum_{x=0}^{\infty}\d{}a\left(a^{x}\right)\\
 & =\frac{a}{1+m}\d{}a\left(\sum_{x=0}^{\infty}a^{x}\right)\\
 & =\frac{a}{1+m}\d{}a\left(\frac{1}{1-a}\right)=m
\end{align*}


\paragraph{演習(9)}

\[
P_{N_{1}}\left(M_{1}\right)=\frac{W_{N_{1}}\left(M_{1}\right)\cdot W_{N-N_{1}}\left(M-M_{1}\right)}{W_{N}\left(M\right)}
\]

\begin{align*}
\frac{W_{N-N_{1}}\left(M-M_{1}\right)}{W_{N}\left(M\right)} & =\frac{\left(M+N-M_{1}-N_{1}-1\right)!}{\left(M-M_{1}\right)!\left(N-N_{1}-1\right)!}\frac{M!\left(N-1\right)!}{\left(M+N-1\right)!}\\
 & =\frac{M\cdot\cdots\left(M-M_{1}-1\right)\left(N-1\right)\cdots\left(N-N_{1}\right)}{\left(M+N-1\right)\cdots\left(M+N-N_{1}-M_{1}\right)}\\
 & \sim\frac{M^{M_{1}}N^{N_{1}}}{\left(M+N\right)^{M_{1}+N_{1}}}\\
 & =\frac{1}{\left(1+m\right)^{N_{1}}}\left(\frac{m}{1+m}\right)^{M_{1}}
\end{align*}


\section*{第2回}

ミクロカノニカル分布/集団 N: 粒子数, M: 量子数 一定

カノニカル分布/集団 N: 一定, M: 一定でない

外界のエネルギー/熱のやり取りのある物理系

グランドカノニカル分布 N: 一定でない, M: 一定でない

パラメータ((1.23)→(1.25)) $\theta=kT$

\[
p\left(E_{1}\right)=\e^{-\frac{E_{i}}{\theta}}
\]

$\e^{-\frac{E_{i}}{\theta}}$: すべての状態の和→$Z\left(\theta\right)$ 分配関数・状態和

\paragraph{演習(1)}

\[
\pd{}{\theta}\log Z\left(\theta\right)=\frac{1}{Z\left(\theta\right)}\pd{Z\left(\theta\right)}{\theta}
\]

\[
\pd{}{\theta}Z\left(\theta\right)=\pd{}{\theta}\sum\e^{-\frac{E_{i}}{\theta}}=\frac{1}{\theta^{2}}\sum E_{i}\e^{-\frac{E_{i}}{\theta}}
\]

\[
\bar{E}=\frac{\sum E_{i}\e^{-\frac{E_{i}}{\theta}}}{Z\left(\theta\right)}=\theta^{2}\frac{1}{Z\left(\theta\right)}\pd{Z\left(\theta\right)}{\theta}
\]


\paragraph{演習(2)}

\begin{align*}
\sum_{n=1}^{\infty}\e^{-a^{2}n^{2}}\cdot\Delta n & =\sum_{n=1}^{\infty}\e^{-x^{2}}\frac{\Delta x}{a}\\
 & \rightarrow\frac{1}{a}\int_{0}^{\infty}\e^{-x^{2}}\mathrm{d}x=\frac{\sqrt{\pi}}{2a}
\end{align*}


\paragraph{演習(3)}

\begin{align*}
S & =xek\log W_{N}\left(M\right)\\
 & =k\log\left(M+N-1\right)!-k\log M!-k\log\left(N-1\right)!\\
 & \sim k\left(M+N\right)\log\left(M+N\right)-kM\log M-kN\log N
\end{align*}

$E=\varepsilon_{0}M$なので、
\begin{align*}
\pd SE & =\frac{1}{\varepsilon_{0}}\pd SM=\frac{k}{\varepsilon_{0}}+\frac{k}{\varepsilon_{0}}\log\left(M+N\right)-\frac{k}{\varepsilon_{0}}-\frac{k}{\varepsilon_{0}}\log M\\
 & =\frac{k}{\varepsilon_{0}}\log\left(\frac{M+N}{M}\right)=\frac{k}{\varepsilon_{0}}\log\left(1+\frac{1}{M}\right)
\end{align*}

これが$\frac{1}{T}$に等しいので、

\[
kT=\frac{\varepsilon_{0}}{\log\left(1+\frac{1}{M}\right)}\equiv\theta
\]


\paragraph{演習(4)}

\[
Z_{A}=\sum_{i}\e^{-\frac{E_{A_{i}}}{kT}},Z_{B}=\sum_{i}\e^{-\frac{E_{B_{i}}}{kT}},\cdots
\]

$A,B,C$は独立なので、
\begin{align*}
E_{\left(A+B+C\cdots\right)_{n}} & =E_{A_{i}}+E_{B_{j}}+E_{C_{k}}+\cdots\\
Z_{A+B+\cdots} & =\sum_{n}\e^{-\frac{E_{\left(A+B+\cdots\right)_{n}}}{kT}}\\
 & =\sum_{i}\sum_{j}\cdots\sum\e^{-\frac{E_{A_{i}}}{kT}}\cdot\e^{-\frac{E_{B_{j}}}{kT}}\cdot\cdots\\
 & =\left(\sum_{i}\e^{-\frac{E_{A_{i}}}{kT}}\right)\left(\sum\cdots\right)\cdots=Z_{A}Z_{B}\cdots
\end{align*}


\paragraph{演習(5)}

(5-1)

\[
Z_{1}=\e^{-\frac{0}{kT}}+\e^{-\frac{\varepsilon_{0}}{kT}}=1+\e^{-\frac{\varepsilon_{0}}{kT}}
\]

(5-2)

\begin{align*}
Z_{N} & =Z_{1}^{N}=\left(1+\e^{-\frac{\varepsilon_{0}}{kT}}\right)^{N}\\
 & =\e^{-\frac{0}{kT}}+N\e^{-\frac{\varepsilon_{0}}{kT}}+{}_{n}C_{2}\e^{-\frac{2\varepsilon_{0}}{kT}}+\cdots+\e^{-\frac{N\varepsilon_{0}}{kT}}
\end{align*}

(5-3)

\[
\bar{E}=kT^{2}\pd{}T\log Z_{N}=kT^{2}\frac{-\e^{\frac{\varepsilon_{0}}{kT}}}{1+\e^{\frac{\varepsilon_{0}}{kT}}}\frac{\varepsilon_{0}}{kT^{2}}=\frac{\varepsilon_{0}N}{\e^{\frac{\varepsilon_{0}}{kT}}+1}
\]


\paragraph{演習(6)}

$\lambda\rightarrow\text{大}$で$\frac{hc}{\lambda kT}\rightarrow0$、$x$が小さい時$\e^{x}\sim1+x$

(2.32)の分母$\sim\frac{hc}{\lambda kT}$

\[
\therefore\bar{E_{\lambda}}\mathrm{d}\lambda=\frac{8\pi hc}{\lambda^{5}}\cdot\frac{\mathrm{d}\lambda}{\frac{hc}{\lambda kT}}=\frac{8\pi kT}{\lambda^{4}}\mathrm{d}\lambda
\]


\paragraph{演習(7)}

$\nu=\frac{c}{\lambda}$より、$\mathrm{d}\nu=-\frac{c}{\lambda}\mathrm{d}\lambda$

\begin{align*}
\bar{E_{\nu}}\mathrm{d}\nu & =\frac{8\pi}{c^{3}}\cdot\frac{h\nu^{3}}{\e^{\frac{h\nu}{kT}}-1}\mathrm{d}\nu=-\frac{8\pi}{c^{3}}\cdot\frac{\frac{hc^{3}}{\lambda^{3}}}{\e^{\frac{hc}{\lambda kT}}-1}\cdot\frac{c}{\lambda}\mathrm{d}\lambda\\
 & =-\frac{8\pi hc}{\lambda^{5}}\cdot\frac{\mathrm{d}\lambda}{\e^{\frac{hc}{\lambda kT}}-1}=-\bar{E_{\lambda}}\mathrm{d}\lambda
\end{align*}


\paragraph{演習(8)}

$\lambda_{m}T=0.2918\mathrm{cmK}$

\begin{align*}
\lambda_{m} & =2.918\mathrm{\mu m}\quad\mathrm{\left(i\right)}\\
\lambda_{m} & =486.3\mathrm{nm}\quad\mathrm{\left(ii\right)}\\
\lambda_{m} & =29.18\mathrm{nm}\quad\mathrm{\left(iii\right)}
\end{align*}

$1\mathrm{eV}=1.60\times10^{-19}\mathrm{C}\times1\mathrm{J/C}=1.60\times10^{-19}\mathrm{J}$

\begin{align*}
h\nu & =\frac{hc}{\lambda}\\
 & =6.63\times10^{-32}\times3\times10^{8}/10^{-6}=1.989\times10^{-19}\mathrm{J}
\end{align*}

(i) 0.425eV

(ii) 2.55eV

(iii) 42.5eV

\section*{第3回}

\paragraph{演習(2)}

\begin{align*}
S & =\pd FT=k\pd{}T\left(T\log Z\right)=k\log Z+\frac{kT}{Z}\pd ZT\\
\frac{kT}{Z}\pd ZT & =\frac{kT}{Z}\sum_{i}\pd{}T\e^{-\frac{E_{i}}{kT}}=\frac{kT}{Z}\sum\frac{E_{i}}{kT^{2}}\e^{-\frac{E_{i}}{kT}}=\frac{1}{T}\sum E\frac{\e^{-\frac{E_{i}}{kt}}}{Z}\\
\therefore S & =k\log Z+\frac{1}{T}\sum E_{_{i}}p\left(E_{i}\right)
\end{align*}

逆算して、
\begin{align*}
-k\overline{\log p\left(E_{i}\right)} & =-k\sum p\left(E_{i}\right)\log p\left(E_{i}\right)\\
 & =\frac{1}{T}\sum E_{i}p\left(E_{i}\right)+k\log Z\sum p\left(E_{i}\right)\quad\left(\because p\left(E_{i}\right)=\frac{\e^{-\frac{E_{i}}{kt}}}{Z}\right)\\
 & =S
\end{align*}


\paragraph{演習(4)}

フェルミ

\[
\pd{}{\mu}\log\left(1+\e^{-\frac{\varepsilon_{i}-\mu}{kT}}\right)=\frac{\frac{1}{kT}\e^{-\frac{\varepsilon_{i}-\mu}{kT}}}{1+\e^{-\frac{\varepsilon_{i}-\mu}{kT}}}=\frac{1}{kT}\frac{1}{\e^{\frac{\varepsilon_{i}-\mu}{kT}}+1}=\overline{N_{1}}
\]

ボーズ

\[
\pd{}{\mu}\log\frac{1}{1-\e^{-\frac{\varepsilon_{i}-\mu}{kT}}}=-\frac{-\frac{1}{kT}\e^{-\frac{\varepsilon_{i}-\mu}{kT}}}{1-\e^{-\frac{\varepsilon_{i}-\mu}{kT}}}=\frac{1}{kT}\frac{1}{\e^{\frac{\varepsilon_{i}-\mu}{kT}}-1}=\overline{N_{2}}
\]


\paragraph{演習(1-1)}

$\frac{\lambda}{2}\cdot n=L$より$\nu=\frac{c}{2L}n$ 状態数$\frac{2L\Delta\nu}{c}$

\paragraph{演習(1-2)}

$\int_{0}^{v_{D}}\frac{2L}{c}\cdot\mathrm{d}\nu=N$より$v_{D}=\frac{cN}{2L}$

\paragraph{演習(1-3)}

\begin{align*}
\bar{E} & =\int_{0}^{v_{D}}\frac{h\nu}{\e^{\frac{h\nu}{kT}}-1}\cdot\frac{2L}{c}\mathrm{d}v=\int_{0}^{\frac{\Theta}{T}}\frac{xkT}{\e^{x}-1}\cdot\frac{2L}{c}\cdot\frac{kT}{h}\mathrm{d}x\\
 & =Nk\frac{T^{2}}{\Theta}\int_{0}^{\frac{\Theta}{T}}\frac{x}{\e^{x}-1}\mathrm{d}x
\end{align*}


\paragraph{演習(1-4)}

$T\rightarrow\infty$で$\e^{x}-1\rightarrow x$

$\bar{E}=NkT$

$\therefore C_{v}=Nk$

\paragraph{演習(1-5)}

$T\rightarrow0$で$\Theta/T\rightarrow\infty$

\[
\bar{E}=Nk\frac{T^{2}}{\Theta}\int_{0}^{\infty}\frac{x}{\e^{x}-1}\mathrm{d}x=\frac{\pi^{2}}{6}Nk\frac{T^{2}}{\Theta}
\]

\[
C_{v}=\frac{\pi^{2}}{3}Nk\frac{T}{\Theta}
\]


\paragraph{演習(6)}

$v_{D}=10^{13}\mathrm{Hz}$

\begin{align*}
\Theta & =\frac{hv_{D}}{k}=\frac{6.63\times10^{-34}\times10^{13}\mathrm{\left[J\right]}}{1.38\times10^{-23}\mathrm{\left[J/K\right]}}\\
 & =4.80\times10^{2}\mathrm{\left[K\right]}
\end{align*}


\paragraph{演習(3)}

状態量 示量性(F.S.V)↔示強性(P.T)

\begin{align*}
F & =-kT\log\frac{\left(2\pi mkT\right)^{\frac{3}{2}N}\cdot V^{N}}{h^{3N}\cdot N!}=-NkT\log\frac{\left(2\pi mxT\right)^{\frac{3}{2}}}{h^{3}}-NkT\left(\log\frac{V}{N}+1\right)\\
S & =-\left(\pd FT\right)=Nk\log\frac{\left(\qquad\right)^{\frac{3}{2}}}{h^{3}}+Nk\left(\log\frac{V}{N}+1\right)+NkT\d{}T\left(\frac{3}{2}\log T\right)\\
 & =Nk\log\frac{\left(\qquad\right)^{\frac{3}{2}}}{h^{3}}+Nk\left(\log\frac{V}{N}+\frac{5}{2}\right)
\end{align*}

\begin{itemize}
\item 等温: $T$が同じ
\item 等圧: $\frac{V}{N}$が同じ
\end{itemize}

\section*{第4回}

自主休講(歯医者)

\section*{第5回}

\paragraph{第4回演習 (4-6)}

\begin{align*}
\bar{\varepsilon} & =\frac{\int_{0}^{\infty}\varepsilon Z_{0}f\left(\varepsilon\right)\mathrm{d}\varepsilon}{\int_{0}^{\infty}Z_{0}f\left(\varepsilon\right)\mathrm{d}\varepsilon}\sim\frac{Z_{0}\int_{0}^{\infty}\varepsilon\mathrm{d}\varepsilon+Z_{0}\frac{\pi^{2}\left(kT\right)^{2}}{6}}{Z_{0}\mu}=\frac{\mu}{2}+\frac{\pi^{2}\left(kT\right)^{2}}{6\mu}\\
 & =\frac{n}{2Z_{0}}+\frac{\pi^{2}\left(kT^{2}\right)}{6n}Z_{0}\;\left(\because\mu=\frac{n}{Z_{0}}\right)
\end{align*}


\paragraph{(4-7)}

\[
f\left(\varepsilon\right)=\frac{\e^{-\frac{\varepsilon-\mu}{kT}}}{1+\e^{-\frac{\varepsilon-\mu}{kT}}}=-kT\frac{\left(\hfill\right)'}{\left(1+\e^{-\frac{\varepsilon-\mu}{kT}}\right)}
\]

\[
n=-Z_{0}kT\left[\log\left(\hfill\right)\right]_{0}^{\infty}=Z_{0}kT\log\left(1+\e^{\frac{\mu}{kT}}\right)
\]

\[
\mu=kT\log\left(\e^{-\frac{n}{Z_{0}kT}}-1\right)
\]


\paragraph{演習問題 (1-1)}

\[
n=\int_{0}^{\infty}Z_{0}f\left(\varepsilon\right)\mathrm{d}\varepsilon=Z_{0}\varepsilon_{a}
\]


\paragraph{(1-2)}

\[
\bar{\varepsilon_{c0}}=\frac{\int_{0}^{\infty}\varepsilon Z_{0}f\left(\varepsilon\right)\mathrm{d}\varepsilon}{\int_{0}^{\infty}Z_{0}f\left(\varepsilon\right)\mathrm{d}\varepsilon}=\frac{\int_{0}^{\varepsilon_{a}}\varepsilon\mathrm{d}\varepsilon}{\int_{0}^{\varepsilon_{a}}\mathrm{d}\varepsilon}=\frac{\varepsilon_{a}}{2}
\]


\paragraph{(1-3)}

\begin{align*}
n & =Z_{0}\int_{0}^{\varepsilon_{a}}f\left(\varepsilon\right)\mathrm{d}\varepsilon+Z_{0}\int_{\varepsilon_{b}}^{\infty}f\left(\varepsilon\right)\mathrm{d}\varepsilon\\
 & =-Z_{0}kT\left[\log\left(1+\e^{-\frac{\varepsilon-\mu}{kT}}\right)\right]_{0}^{\varepsilon_{0}}-Z_{0}kT\left[\log\left(1+\e^{-\frac{\varepsilon-\mu}{kT}}\right)\right]_{\varepsilon_{b}}^{\infty}\\
 & =Z_{0}kT\log\frac{\left(1+\e^{\frac{\mu}{kT}}\right)\left(1+\e^{-\frac{\varepsilon_{b}-\mu}{kT}}\right)}{\left(1+\e^{-\frac{\varepsilon_{a}-\mu}{kT}}\right)}=Z_{0}\varepsilon_{a}
\end{align*}

\[
\rightarrow\frac{\varepsilon_{a}+\varepsilon_{b}}{2}
\]


\paragraph{(1-4)}

\begin{align*}
n_{c}=Z_{0}\int_{\varepsilon_{b}}^{\infty}f\left(\varepsilon\right)\mathrm{d}\varepsilon & =Z_{0}kT\log\left(1+\e^{-\frac{\varepsilon_{b}-\varepsilon_{a}}{2}}\right)\\
 & \sim Z_{0}kT\e^{-\frac{\varepsilon_{a}-\varepsilon_{b}}{2kT}}
\end{align*}


\paragraph{(1-5)}

\begin{align*}
\bar{\varepsilon_{c}} & =\frac{Z_{0}}{n_{c}}\int\varepsilon f\left(\varepsilon\right)\mathrm{d}\varepsilon=\frac{\e^{\frac{\varepsilon_{b}-\varepsilon_{a}}{2kT}}}{kT}\int_{\varepsilon_{b}}^{\infty}\varepsilon\e^{-\frac{\varepsilon-\mu}{kT}}\mathrm{d}\varepsilon\\
 & =\varepsilon_{b}+kT
\end{align*}


\paragraph{(2)}

\[
n=\frac{2\times10^{19}}{6.7\times10^{-29}}=1.18\times10^{44}\mathrm{\text{個}/m^{2}}
\]

\begin{align*}
T_{F} & =\frac{h^{2}}{2mk}\left(\frac{2n}{8\pi}\right)^{\frac{2}{3}}\\
 & =\left(9.34\times10^{-18}\right)\left(1.41\times10^{43}\right)^{\frac{2}{3}}\\
 & =9.34\times\left(1.41\right)^{\frac{2}{3}}\times10^{10}\mathrm{K}
\end{align*}


\paragraph{(4)}

\[
n=6.12\times10^{23}\times\frac{0.917}{23.0}=2.4\times10^{22}\mathrm{/m^{2}}
\]

\[
\frac{3n}{8\pi}=2.86\times10^{27}\mathrm{/m^{2}}
\]

\[
T_{F}=3.53\times10^{4}\mathrm{K}
\]

(A.2)式で$Z\left(\varepsilon\right)=Z_{0}$とおくと、
\[
Z_{0}\int_{0}^{\infty}\frac{1}{\e^{\frac{\varepsilon-\mu}{kT}}-1}\mathrm{d}\varepsilon=N
\]

$f\left(\varepsilon\right)=kT\frac{\left(\hfill\right)'}{\left(1-\e^{-\frac{\varepsilon-\mu}{kT}}\right)}$より$kTZ_{0}\left[\log\left(\hfill\right)\right]_{0}^{\infty}=N$

\[
\therefore-kTZ_{0}\log\left(1-\e^{-\frac{\varepsilon-\mu}{kT}}\right)=N
\]

\[
\mu=kT\log\left(1-\e^{-\frac{N}{kTZ_{0}}}\right)\xrightarrow[T\rightarrow\infty]{}-kT\e^{-\frac{N}{kTZ_{0}}}
\]

\end{document}
