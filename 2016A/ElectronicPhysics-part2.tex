%% LyX 2.2.2 created this file.  For more info, see http://www.lyx.org/.
%% Do not edit unless you really know what you are doing.
\documentclass[oneside,english]{book}
\usepackage[T1]{fontenc}
\usepackage[utf8]{inputenc}
\usepackage[a5paper]{geometry}
\geometry{verbose,tmargin=2cm,bmargin=2cm,lmargin=1cm,rmargin=1cm}
\setcounter{secnumdepth}{3}
\setcounter{tocdepth}{3}
\setlength{\parskip}{\smallskipamount}
\setlength{\parindent}{0pt}
\usepackage{textcomp}
\usepackage{amsmath}
\usepackage{amssymb}

\makeatletter
%%%%%%%%%%%%%%%%%%%%%%%%%%%%%% User specified LaTeX commands.
\usepackage[dvipdfmx]{hyperref}
\usepackage[dvipdfmx]{pxjahyper}

\makeatother

\usepackage{babel}
\begin{document}

\title{2016-A 電子基礎物理 後半}

\author{教員: 山下真司 入力: 高橋光輝}

\maketitle
\global\long\def\pd#1#2{\frac{\partial#1}{\partial#2}}
\global\long\def\d#1#2{\frac{\mathrm{d}#1}{\mathrm{d}#2}}
\global\long\def\pdd#1#2{\frac{\partial^{2}#1}{\partial#2^{2}}}
\global\long\def\dd#1#2{\frac{\mathrm{d}^{2}#1}{\mathrm{d}#2^{2}}}
\global\long\def\ddd#1#2{\frac{\mathrm{d}^{3}#1}{\mathrm{d}#2^{3}}}
\global\long\def\e{\mathrm{e}}
\global\long\def\i{\mathrm{i}}
\global\long\def\j{\mathrm{j}}
\global\long\def\grad{\operatorname{grad}}
\global\long\def\rot{\operatorname{rot}}
\global\long\def\div{\operatorname{div}}
\global\long\def\diag{\operatorname{diag}}
\global\long\def\rank{\operatorname{rank}}
\global\long\def\prob{\operatorname{Prob}}
\global\long\def\cov{\operatorname{Cov}}
\global\long\def\when#1{\left.#1\right|}


\section*{第1回}
\begin{itemize}
\item 統計力学
\begin{itemize}
\item 古典統計
\item 量子統計
\end{itemize}
\end{itemize}
全微分
\[
\mathrm{d}E=\left(\pd ET\right)_{V}\mathrm{d}T+\left(\pd EV\right)_{T}\mathrm{d}V
\]

\begin{align*}
\Delta W & =F\cdot\Delta x\\
 & =P\cdot A\cdot\Delta x\\
 & =-P\Delta V
\end{align*}

比熱: 温度を上げるのに必要な熱量

→
\[
C_{v}=\left(\pd QT\right)_{V}=\left(\pd ET\right)_{V}
\]

(∵(0.3)式で$\mathrm{d}V=0,\mathrm{d}'Q=0$)

1mol

\[
E=\frac{3}{2}RT
\]

1mol当たり
\[
C_{p}=C_{v}+R
\]

マイアーの関係式

\paragraph{演習(1)}

\[
\mathrm{d}'Q=C_{v}\mathrm{d}T+\left[P+\left(\pd EV\right)_{T}\right]\mathrm{d}V
\]

$\mathrm{d}'Q=0,P=\frac{RT}{V},\pd EV=0$より、
\[
0=C_{v}\frac{\mathrm{d}T}{T}+\frac{R}{V}\cdot\mathrm{d}V
\]

積分して、
\[
\text{const}=C_{v}\log T+R\log V
\]

\[
\log T+\left(\gamma-1\right)\log V=\text{const}
\]

\[
\log T\cdot V^{\gamma-1}=\text{const}
\]

$T\propto PV$より、
\[
P\cdot V^{\gamma}=\text{const}
\]


\paragraph{演習(3)}

\[
F=E-TS
\]

\begin{align*}
\mathrm{d}F=\mathrm{d}E-\mathrm{d}\left(TS\right) & =\mathrm{d}E-T\mathrm{d}S-S\mathrm{d}T\\
 & =-S\mathrm{d}T-P\mathrm{d}V
\end{align*}

\begin{align*}
\mathrm{d}\left(\frac{F}{T}\right) & =\mathrm{d}\left(\frac{E}{T}\right)-\mathrm{d}S\\
 & =\frac{\mathrm{d}E}{T}-\frac{E}{T^{2}}\mathrm{d}T-\mathrm{d}S\\
 & =-\frac{E}{T^{2}}\mathrm{d}T-\frac{P}{T}\mathrm{d}V
\end{align*}

熱力学的エントロピー
\[
\mathrm{d}S=\frac{\mathrm{d}'Q}{T}
\]

統計力学的エントロピー
\[
S=k\log W_{N}\left(E\right)
\]


\paragraph{演習(6)}

\begin{align*}
\log G_{N}\left(E\right) & =\log\left(M+N-1\right)!-\log M!-\log\left(N-1\right)!\\
 & \sim\log\frac{\left(M+N-1\right)^{M+N-1}}{M^{M}\left(N-1\right)^{N-1}}\\
 & \sim\log\frac{\left(M+N\right)^{M+N}}{M^{M}N^{N}}\\
 & =\log\frac{\left(2N\right)^{2N}}{N^{2N}}\\
 & =\log2^{2N}
\end{align*}

\begin{align*}
G_{N}\left(E\right) & =2^{2N}\\
 & \sim10^{0.6N}
\end{align*}


\paragraph{演習(7)}

\begin{align*}
P_{1}\left(x\right) & =\frac{\left(M+N-x-2\right)!}{\left(M-x\right)!\left(N-2\right)!}\frac{M!\left(N-1\right)!}{\left(M+N-1\right)!}\\
 & =\frac{M\cdot\left(M-1\right)\cdots\left(M-x+1\right)\left(N-1\right)}{\left(M+N-1\right)\left(M+N-2\right)\cdots\left(M+N-x-1\right)}\\
 & \sim\frac{M^{x}N}{\left(M+N\right)^{x+1}}=\frac{m^{x}}{\left(1+m\right)^{x}}\cdot\frac{1}{1+m}
\end{align*}


\paragraph{演習(8)}

\begin{align*}
\sum_{x=0}^{\infty}p_{1}\left(x\right) & =\frac{1}{1+m}\sum_{x=0}^{\infty}\left(\frac{m}{1+m}\right)^{x}\\
 & =\frac{1}{1+m}\cdot\frac{1}{1-\frac{m}{1+m}}=1
\end{align*}

\begin{align*}
\sum_{x=0}^{\infty}xp_{1}\left(x\right) & =\frac{1}{1+m}\sum_{x=0}^{\infty}xa^{x}\\
 & =\frac{a}{1+m}\sum_{x=0}^{\infty}\d{}a\left(a^{x}\right)\\
 & =\frac{a}{1+m}\d{}a\left(\sum_{x=0}^{\infty}a^{x}\right)\\
 & =\frac{a}{1+m}\d{}a\left(\frac{1}{1-a}\right)=m
\end{align*}


\paragraph{演習(9)}

\[
P_{N_{1}}\left(M_{1}\right)=\frac{W_{N_{1}}\left(M_{1}\right)\cdot W_{N-N_{1}}\left(M-M_{1}\right)}{W_{N}\left(M\right)}
\]

\begin{align*}
\frac{W_{N-N_{1}}\left(M-M_{1}\right)}{W_{N}\left(M\right)} & =\frac{\left(M+N-M_{1}-N_{1}-1\right)!}{\left(M-M_{1}\right)!\left(N-N_{1}-1\right)!}\frac{M!\left(N-1\right)!}{\left(M+N-1\right)!}\\
 & =\frac{M\cdot\cdots\left(M-M_{1}-1\right)\left(N-1\right)\cdots\left(N-N_{1}\right)}{\left(M+N-1\right)\cdots\left(M+N-N_{1}-M_{1}\right)}\\
 & \sim\frac{M^{M_{1}}N^{N_{1}}}{\left(M+N\right)^{M_{1}+N_{1}}}\\
 & =\frac{1}{\left(1+m\right)^{N_{1}}}\left(\frac{m}{1+m}\right)^{M_{1}}
\end{align*}

\end{document}
