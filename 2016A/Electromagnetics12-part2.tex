%% LyX 2.2.2 created this file.  For more info, see http://www.lyx.org/.
%% Do not edit unless you really know what you are doing.
\documentclass[english]{article}
\usepackage[T1]{fontenc}
\usepackage[utf8]{inputenc}
\usepackage[a5paper]{geometry}
\geometry{verbose,tmargin=2cm,bmargin=2cm,lmargin=1cm,rmargin=1cm}
\setlength{\parskip}{\smallskipamount}
\setlength{\parindent}{0pt}
\usepackage{textcomp}
\usepackage{amsmath}
\usepackage{amssymb}
\usepackage{esint}
\PassOptionsToPackage{normalem}{ulem}
\usepackage{ulem}

\makeatletter
%%%%%%%%%%%%%%%%%%%%%%%%%%%%%% User specified LaTeX commands.
\usepackage[dvipdfmx]{hyperref}
\usepackage[dvipdfmx]{pxjahyper}

\makeatother

\usepackage{babel}
\begin{document}

\title{2016-A 電気磁気学 後半}

\author{教員: 種村 入力: 高橋光輝}

\maketitle
\global\long\def\pd#1#2{\frac{\partial#1}{\partial#2}}
\global\long\def\d#1#2{\frac{\mathrm{d}#1}{\mathrm{d}#2}}
\global\long\def\pdd#1#2{\frac{\partial^{2}#1}{\partial#2^{2}}}
\global\long\def\dd#1#2{\frac{\mathrm{d}^{2}#1}{\mathrm{d}#2^{2}}}
\global\long\def\e{\mathrm{e}}
\global\long\def\i{\mathrm{i}}
\global\long\def\j{\mathrm{j}}
\global\long\def\grad{\mathrm{grad}}
\global\long\def\rot{\mathrm{rot}}
\global\long\def\div{\mathrm{div}}


\section*{第1回}

\paragraph{成績}
\begin{itemize}
\item 出席は取らない
\item 必修
\end{itemize}

\section*{0 イントロ}

電磁気学を完全に記述する方程式 Maxwell's Equations

\begin{align*}
\nabla\cdot\boldsymbol{E} & =\frac{\rho}{\varepsilon_{0}}\:\left(\text{ガウス}\right)\\
\nabla\times\boldsymbol{E} & =-\pd{\boldsymbol{B}}t\:\left(\text{ファラデー}\right)\\
\nabla\cdot\boldsymbol{B} & =0\:\left(\text{磁荷の非存在}\right)\\
\nabla\times\boldsymbol{B} & =\mu_{0}\varepsilon_{0}\pd{\boldsymbol{E}}t+\mu_{0}\boldsymbol{i}\:\left(\text{アンペア}\right)
\end{align*}

静電磁気学$\left(\pd{}t=0\right)$では、

図電後1-1

つまり時間不変の上等では``電気''と``磁気''は別個の現象(のように見える)。

本当に面白いのは$\pd{\boldsymbol{E}}t,\pd{\boldsymbol{B}}t$の項。これにより、電磁誘導→電磁波(光)が記述される。

\paragraph{後半}
\begin{enumerate}
\item $\boldsymbol{B}$って何?
\item (※)の式 $\left(\pd{}t=0\right)$
\item 磁性体
\item $\pd{}t$の項
\end{enumerate}

\section{磁界}

\subsection{アンペール力}

\paragraph{クーロン力}

\uline{静止した}2つの電荷間に働く力

図電後1-2

\[
F=\frac{1}{4\pi\varepsilon_{0}}\frac{Q_{1}Q_{2}}{r^{2}}
\]

電荷が動いているときは、その\uline{速さと向き}に依存した力を発生する。→アンペール力(Ampere)

\paragraph{アンペール力 (アンペール 1820年)}

距離$R$離れた2本の平行な導線に電流$I_{1},I_{2}$が流れているとき、電荷が動いているとき、
\begin{equation}
F=\frac{\mu_{0}}{2\pi}\frac{I_{1}I_{2}}{R}
\end{equation}
の力が働く。

ただし、$I_{1},I_{2}$が同じ向きなら引力、逆向きなら斥力。

導線が直交してたら力は働かない。

\[
\mu_{0}=4\pi\times10^{-7}\left[\mathrm{N\cdot m/A^{2}}\right]
\]

透磁率という。

\paragraph{なぜきれいな値なのか?}

→実は(1)により電流の単位が定義されている。

1Aとは「真空中で1m離れた平衡な直線導体に流れた時に、単位長さあたり$2\times10^{-7}\mathrm{N}$のちからが働くような電流」と定義される(SI単位系)。

→その結果、1Cが定義されて、$\varepsilon_{0}$が求まる。

\paragraph{復習}

帯電下電線管に働くクーロン力は、単位長あたり、
\begin{equation}
F=\frac{1}{2\pi\varepsilon_{0}}\frac{\lambda_{1}\lambda_{2}}{R}
\end{equation}

∵ガウスの定理

図電後1-3

クーロン力(2)と同様に、直線導体間のアンペール力(1)は$R$に反比例する。

→クーロン力(遠隔作用)が電界(近接作用)を用いて言い換えられたように、アンペール力も同様の考え方で説明できるはず。

つまり、
\begin{enumerate}
\item 導線1の微小区間$\delta\boldsymbol{s}_{1}$を流れる電流$I_{1}$(電流要素$I\mathrm{d\boldsymbol{s}_{1}}$)が周囲に、なんらかの``場''(これを磁場または磁界と呼ぶ)を作る。
\item 導線2の微小区間$\mathrm{d}\boldsymbol{s}_{2}$を流れる電流$I_{2}$(電流要素$I_{2}\mathrm{d}\boldsymbol{s}_{2}$)がその場を感じて力を受ける。
\end{enumerate}
図電後1-4

「電流要素」→「電荷」に置き換えるとクーロン力

\subsection{ビオ・サバールの法則}

位置$\boldsymbol{r}'$にある電流要素$I'\mathrm{d}\boldsymbol{s}'$が、位置$\boldsymbol{r}$に
\begin{equation}
\mathrm{d}\boldsymbol{B}=\frac{\mu_{0}}{4\pi}\frac{I'\mathrm{d}\boldsymbol{s}\times\boldsymbol{u}}{\left|\boldsymbol{r}-\boldsymbol{r}'\right|^{2}}
\end{equation}
で表される微小な``場''を作る。

図電後1-5

\[
\boldsymbol{u}\equiv\frac{\boldsymbol{r}-\boldsymbol{r}'}{\left|\boldsymbol{r}-\boldsymbol{r}'\right|}
\]

$\boldsymbol{r}-\boldsymbol{r}'$方向の単位ベクトル

($I$は$\mathrm{d}\boldsymbol{s}$の向きを性とした電流)

電線全体が作る場は
\[
\boldsymbol{B}=\int\mathrm{d}\boldsymbol{B}
\]

\begin{equation}
\therefore\boldsymbol{B}=\frac{\mu_{0}}{4\pi}\oint\frac{I'\mathrm{d}s'\times\left(\boldsymbol{r}-\boldsymbol{r}'\right)}{\left|\boldsymbol{r}-\boldsymbol{r'}\right|^{3}}
\end{equation}

図電後1-6

そしてこの時、位置$\boldsymbol{r}$にある電流要素$I\mathrm{d}\boldsymbol{s}$は、
\begin{equation}
\boldsymbol{F}=I\mathrm{d}\boldsymbol{s}\times\boldsymbol{B}
\end{equation}
の力を受ける。

図電後1-7

すべての$I\mathrm{d}\boldsymbol{s}$において積分すれば導線全体が受ける力が求まる。

ここで``場''$\boldsymbol{B}$を磁束密度と呼び、単位T(テスラ)を用いる。(このあとで出てくる$\boldsymbol{H}$を磁界と呼ぶため)

直線電流にとうして(4)、(5)を用いて計算すると(1)が得られる。(演習問題)

\paragraph{ローレンツ力}

もともと電流は電荷の流れ。

\[
I\mathrm{d}\boldsymbol{s}-qN\boldsymbol{v}
\]

$q$: でん過疎量

$N$: $\mathrm{d}\boldsymbol{s}$無いの電荷数

$v$: 電荷の速度

を用いると、(5)は
\[
\boldsymbol{F}=qN\boldsymbol{v}\times\boldsymbol{B}
\]

∴電荷1個あたり$\boldsymbol{F}=q\boldsymbol{v}\times\boldsymbol{B}$のアンペール力を受けるとも言える。

クーロン力と合わせて一般に電荷が受ける力
\begin{equation}
\boldsymbol{F}=q\left(\boldsymbol{E}+\boldsymbol{v}\times\boldsymbol{B}\right)
\end{equation}
をローレンツ力と呼ぶ。

\paragraph{電磁場の相対性}

静止した電荷間の力がクーロン力(電界)\uline{動いている}電荷間の力がアンペール力(磁界)と区別したが、「動いているかどうか?」は相対的なもの。

例えば、電荷と一緒に動いている系絡みたら、(6)の$\boldsymbol{v}=0$になり、アンペール力は発生しない。おかしい?

答え: このとき、相対論的に長さが縮むことに起因して\uline{クーロン力}が余分に発生し、合計の$\boldsymbol{F}$がつじつまが合う。(配布資料)

クーロンの法則+相対論→ビオ・サバールの法則が導出できる。(太田浩一『電磁気学の基礎II』)

\paragraph{TAオフィスアワーのご案内}

数学演習と電磁気学でわからないことを聞きに行ける制度
\begin{itemize}
\item 場所: 駒場IIキャンパス 先端研3号館308号室
\item 日程: 11/24, 12/1, 12/8, 12/15, 12/22 (毎週木曜)
\item 時間: 18:30~19:30
\end{itemize}

\paragraph{演習}

1(a)

図電後1-8
\end{document}
