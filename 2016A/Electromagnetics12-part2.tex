%% LyX 2.2.2 created this file.  For more info, see http://www.lyx.org/.
%% Do not edit unless you really know what you are doing.
\documentclass[english]{article}
\usepackage[T1]{fontenc}
\usepackage[utf8]{inputenc}
\usepackage[a5paper]{geometry}
\geometry{verbose,tmargin=2cm,bmargin=2cm,lmargin=1cm,rmargin=1cm}
\setlength{\parskip}{\smallskipamount}
\setlength{\parindent}{0pt}
\usepackage{textcomp}
\usepackage{amsmath}
\usepackage{amssymb}
\usepackage{stmaryrd}
\usepackage{graphicx}
\usepackage{esint}
\PassOptionsToPackage{normalem}{ulem}
\usepackage{ulem}

\makeatletter
%%%%%%%%%%%%%%%%%%%%%%%%%%%%%% Textclass specific LaTeX commands.
\numberwithin{equation}{section}

%%%%%%%%%%%%%%%%%%%%%%%%%%%%%% User specified LaTeX commands.
\usepackage[dvipdfmx,hidelinks]{hyperref}
\usepackage[dvipdfmx]{pxjahyper}

\makeatother

\usepackage{babel}
\begin{document}

\title{2016-A 電気磁気学 後半}

\author{教員: 種村 入力: 高橋光輝}

\maketitle
\global\long\def\pd#1#2{\frac{\partial#1}{\partial#2}}
\global\long\def\d#1#2{\frac{\mathrm{d}#1}{\mathrm{d}#2}}
\global\long\def\pdd#1#2{\frac{\partial^{2}#1}{\partial#2^{2}}}
\global\long\def\dd#1#2{\frac{\mathrm{d}^{2}#1}{\mathrm{d}#2^{2}}}
\global\long\def\ddd#1#2{\frac{\mathrm{d}^{3}#1}{\mathrm{d}#2^{3}}}
\global\long\def\e{\mathrm{e}}
\global\long\def\i{\mathrm{i}}
\global\long\def\j{\mathrm{j}}
\global\long\def\grad{\operatorname{grad}}
\global\long\def\rot{\operatorname{rot}}
\global\long\def\div{\operatorname{div}}
\global\long\def\diag{\operatorname{diag}}
\global\long\def\rank{\operatorname{rank}}
\global\long\def\prob{\operatorname{Prob}}
\global\long\def\cov{\operatorname{Cov}}
\global\long\def\when#1{\left.#1\right|}
gi

\section*{第1回}

\paragraph{成績}
\begin{itemize}
\item 出席は取らない
\item 必修
\end{itemize}

\section*{0 イントロ}

電磁気学を完全に記述する方程式 Maxwell's Equations

\begin{align*}
\nabla\cdot\boldsymbol{E} & =\frac{\rho}{\varepsilon_{0}}\:\left(\text{ガウス}\right)\\
\nabla\times\boldsymbol{E} & =-\pd{\boldsymbol{B}}t\:\left(\text{ファラデー}\right)\\
\nabla\cdot\boldsymbol{B} & =0\:\left(\text{磁荷の非存在}\right)\\
\nabla\times\boldsymbol{B} & =\mu_{0}\varepsilon_{0}\pd{\boldsymbol{E}}t+\mu_{0}\boldsymbol{i}\:\left(\text{アンペア}\right)
\end{align*}

静電磁気学$\left(\pd{}t=0\right)$では、
\begin{center}
\includegraphics{images/Electromagnetics12-part2/1-1}
\par\end{center}

つまり時間不変の上等では``電気''と``磁気''は別個の現象(のように見える)。

本当に面白いのは$\pd{\boldsymbol{E}}t,\pd{\boldsymbol{B}}t$の項。これにより、電磁誘導→電磁波(光)が記述される。

\paragraph{後半}
\begin{enumerate}
\item $\boldsymbol{B}$って何?
\item (※)の式 $\left(\pd{}t=0\right)$
\item 磁性体
\item $\pd{}t$の項
\end{enumerate}

\section{磁界}

\subsection{アンペール力}

\paragraph{クーロン力}

\uline{静止した}2つの電荷間に働く力
\begin{center}
\includegraphics{images/Electromagnetics12-part2/1-2}
\par\end{center}

\[
F=\frac{1}{4\pi\varepsilon_{0}}\frac{Q_{1}Q_{2}}{r^{2}}
\]

電荷が動いているときは、その\uline{速さと向き}に依存した力を発生する。→アンペール力(Ampere)

\paragraph{アンペール力 (アンペール 1820年)}

距離$R$離れた2本の平行な導線に電流$I_{1},I_{2}$が流れているとき、電荷が動いているとき、
\begin{equation}
F=\frac{\mu_{0}}{2\pi}\frac{I_{1}I_{2}}{R}\label{1.1}
\end{equation}
の力が働く。

ただし、$I_{1},I_{2}$が同じ向きなら引力、逆向きなら斥力。

導線が直交してたら力は働かない。

\[
\mu_{0}=4\pi\times10^{-7}\left[\mathrm{N\cdot m/A^{2}}\right]
\]

透磁率という。

\paragraph{なぜきれいな値なのか?}

→実は\ref{1.1}により電流の単位が定義されている。

1Aとは「真空中で1m離れた平衡な直線導体に流れた時に、単位長さあたり$2\times10^{-7}\mathrm{N}$のちからが働くような電流」と定義される(SI単位系)。

→その結果、1Cが定義されて、$\varepsilon_{0}$が求まる。

\paragraph{復習}

帯電下電線管に働くクーロン力は、単位長あたり、
\begin{equation}
F=\frac{1}{2\pi\varepsilon_{0}}\frac{\lambda_{1}\lambda_{2}}{R}\label{1.2}
\end{equation}

∵ガウスの定理
\begin{center}
\includegraphics{images/Electromagnetics12-part2/1-3}
\par\end{center}

クーロン力\ref{1.2}と同様に、直線導体間のアンペール力\ref{1.1}は$R$に反比例する。

→クーロン力(遠隔作用)が電界(近接作用)を用いて言い換えられたように、アンペール力も同様の考え方で説明できるはず。

つまり、
\begin{enumerate}
\item 導線1の微小区間$\delta\boldsymbol{s}_{1}$を流れる電流$I_{1}$(電流要素$I\mathrm{d\boldsymbol{s}_{1}}$)が周囲に、なんらかの``場''(これを磁場または磁界と呼ぶ)を作る。
\item 導線2の微小区間$\mathrm{d}\boldsymbol{s}_{2}$を流れる電流$I_{2}$(電流要素$I_{2}\mathrm{d}\boldsymbol{s}_{2}$)がその場を感じて力を受ける。
\end{enumerate}
\begin{center}
\includegraphics{images/Electromagnetics12-part2/1-4}
\par\end{center}

「電流要素」→「電荷」に置き換えるとクーロン力

\subsection{ビオ・サバールの法則}

位置$\boldsymbol{r}'$にある電流要素$I'\mathrm{d}\boldsymbol{s}'$が、位置$\boldsymbol{r}$に
\begin{equation}
\mathrm{d}\boldsymbol{B}=\frac{\mu_{0}}{4\pi}\frac{I'\mathrm{d}\boldsymbol{s}\times\boldsymbol{u}}{\left|\boldsymbol{r}-\boldsymbol{r}'\right|^{2}}\label{1.3}
\end{equation}
で表される微小な``場''を作る。
\begin{center}
\includegraphics{images/Electromagnetics12-part2/1-5}
\par\end{center}

\[
\boldsymbol{u}\equiv\frac{\boldsymbol{r}-\boldsymbol{r}'}{\left|\boldsymbol{r}-\boldsymbol{r}'\right|}
\]

$\boldsymbol{r}-\boldsymbol{r}'$方向の単位ベクトル

($I$は$\mathrm{d}\boldsymbol{s}$の向きを性とした電流)

電線全体が作る場は
\[
\boldsymbol{B}=\int\mathrm{d}\boldsymbol{B}
\]

\begin{equation}
\therefore\boldsymbol{B}=\frac{\mu_{0}}{4\pi}\oint\frac{I'\mathrm{d}s'\times\left(\boldsymbol{r}-\boldsymbol{r}'\right)}{\left|\boldsymbol{r}-\boldsymbol{r'}\right|^{3}}\label{1.4}
\end{equation}

\begin{center}
\includegraphics{images/Electromagnetics12-part2/1-6}
\par\end{center}

そしてこの時、位置$\boldsymbol{r}$にある電流要素$I\mathrm{d}\boldsymbol{s}$は、
\begin{equation}
\boldsymbol{F}=I\mathrm{d}\boldsymbol{s}\times\boldsymbol{B}\label{1.5}
\end{equation}
の力を受ける。
\begin{center}
\includegraphics{images/Electromagnetics12-part2/1-7}
\par\end{center}

すべての$I\mathrm{d}\boldsymbol{s}$において積分すれば導線全体が受ける力が求まる。

ここで``場''$\boldsymbol{B}$を磁束密度と呼び、単位T(テスラ)を用いる。(このあとで出てくる$\boldsymbol{H}$を磁界と呼ぶため)

直線電流に対して\ref{1.4}、\ref{1.5}を用いて計算すると\ref{1.1}が得られる。(演習問題)

\paragraph{ローレンツ力}

もともと電流は電荷の流れ。

\[
I\mathrm{d}\boldsymbol{s}-qN\boldsymbol{v}
\]

$q$: 電荷疎量

$N$: $\mathrm{d}\boldsymbol{s}$無いの電荷数

$v$: 電荷の速度

を用いると、(5)は
\[
\boldsymbol{F}=qN\boldsymbol{v}\times\boldsymbol{B}
\]

∴電荷1個あたり$\boldsymbol{F}=q\boldsymbol{v}\times\boldsymbol{B}$のアンペール力を受けるとも言える。

クーロン力と合わせて一般に電荷が受ける力
\begin{equation}
\boldsymbol{F}=q\left(\boldsymbol{E}+\boldsymbol{v}\times\boldsymbol{B}\right)\label{1.6}
\end{equation}
をローレンツ力と呼ぶ。

\paragraph{電磁場の相対性}

静止した電荷間の力がクーロン力(電界)\uline{動いている}電荷間の力がアンペール力(磁界)と区別したが、「動いているかどうか?」は相対的なもの。

例えば、電荷と一緒に動いている系絡みたら、\ref{1.6}の$\boldsymbol{v}=0$になり、アンペール力は発生しない。おかしい?

答え: このとき、相対論的に長さが縮むことに起因して\uline{クーロン力}が余分に発生し、合計の$\boldsymbol{F}$がつじつまが合う。(配布資料)

クーロンの法則+相対論→ビオ・サバールの法則が導出できる。(太田浩一『電磁気学の基礎II』)

\paragraph{TAオフィスアワーのご案内}

数学演習と電磁気学でわからないことを聞きに行ける制度
\begin{itemize}
\item 場所: 駒場IIキャンパス 先端研3号館308号室
\item 日程: 11/24, 12/1, 12/8, 12/15, 12/22 (毎週木曜)
\item 時間: 18:30~19:30
\end{itemize}

\paragraph{演習問題1(a)}
\begin{center}
\includegraphics{images/Electromagnetics12-part2/1-8}
\par\end{center}

微小区間$\left[x+x+\mathrm{d}x\right]$の電流要素$I\mathrm{d}x$が点$P$に作る磁界密度を考える。

\[
\left|\boldsymbol{r}-\boldsymbol{r}'\right|\equiv r=\frac{d}{\sin\theta}
\]

$x=-\frac{d}{\tan\theta}$より$\d x{\theta}=\frac{d}{\sin^{2}\theta}$

\[
\mathrm{}\boldsymbol{s}\times\left(\boldsymbol{r}-\boldsymbol{r}'\right)=\left(r\mathrm{d}x\sin\theta\right)
\]

∴(4)より、
\begin{align*}
B & =\frac{\mu_{0}}{4\pi}\int_{x_{1}}^{x_{2}}\frac{I\sin x}{r^{2}}\mathrm{d}x\\
 & =\frac{\mu_{0}I}{4\pi}\int_{\alpha}^{\beta}\frac{\sin^{3}x}{d^{2}}\frac{d}{\sin^{2}\theta}\mathrm{d}\theta\\
 & =\frac{\mu_{0}I}{4\pi d}\int_{\alpha}^{\beta}\sin\theta\mathrm{d}\theta\\
 & =\frac{\mu_{0}I}{4\pi d}\left(\cos\alpha-\cos\beta\right)
\end{align*}

向きは+z方向。

\paragraph{(b)}

$\alpha=0,\beta=\pi$

$I=I_{1},d=R$を代入して、
\[
B=\frac{\mu_{0}I_{1}}{2\pi R}
\]

(5)より単位長さあたり($\left|\mathrm{d}\boldsymbol{s}\right|=1$)

\[
F=I_{2}B=\frac{\mu_{0}I_{1}I_{2}}{2\pi R}
\]


\paragraph{(c)}
\begin{center}
\includegraphics{images/Electromagnetics12-part2/1-9}
\par\end{center}

まず、線分$QR$を流れる電流g点$P$に作る磁束密度成分$\boldsymbol{B}_{1}$を考える。

$\Delta PQR$に(a)の結果を適用
\begin{center}
\includegraphics{images/Electromagnetics12-part2/1-10}
\par\end{center}

\[
\cos\alpha=-\cos\beta=\frac{\overline{O'Q}}{\overline{PR}}=\frac{a}{\sqrt{a^{2}+b^{2}+x^{2}}}
\]

\[
d=\overline{PO'}=\sqrt{b^{2}+x^{2}}
\]

(a)の結果より、
\[
B_{1}=\frac{\mu_{0}I}{2\pi\sqrt{b^{2}+x^{2}}}\frac{a}{\sqrt{a^{2}+b^{2}+x^{2}}}
\]

$ST$が作る磁場を加えると、残るのは$OP$方向のみなので、その成分だけを考えると、
\[
B_{1\bot}=B_{1}\frac{b}{\sqrt{b^{2}+x^{2}}}=\frac{\mu_{0}Iab}{2\pi\left(b^{2}+x^{2}\right)\sqrt{a^{2}+b^{2}+x^{2}}}
\]

同様に、$RS$が作る磁場は、
\[
B_{2\bot}=\frac{\mu_{0}Iab}{2\pi\left(a^{2}+x^{2}\right)\sqrt{a^{2}+b^{2}+x^{2}}}
\]

∴合計
\[
B=2B_{1\bot}+2B_{2\bot}=\frac{\mu_{0}Iab}{\pi\sqrt{a^{2}+b^{2}+x^{2}}}\left(\frac{1}{a^{2}+x^{2}}+\frac{1}{b^{2}+x^{2}}\right)
\]


\paragraph{演習問題2}
\begin{center}
\includegraphics{images/Electromagnetics12-part2/1-11}
\par\end{center}

まず下の円環が作る磁場を考える。

周方向に角度$\theta$をとる。

微小電流要素$I\mathrm{d}s\left(=Ia\mathrm{d}\theta\right)$が点$P$につくる$\mathrm{d}\boldsymbol{B}$は、
\[
\mathrm{d}B=\frac{\mu_{0}}{4\pi}\frac{Ia\mathrm{d}\theta}{a^{2}+h^{2}}
\]

ただし$h\equiv d+z$
\begin{center}
\includegraphics{images/Electromagnetics12-part2/1-12}
\par\end{center}

円環一周がつくる磁場は$z$成分以外は打ち消し合うので$z$成分のみを考える。

\[
\mathrm{d}B_{z}=\mathrm{d}B\cdot\frac{a}{\sqrt{a^{2}+h^{2}}}=\frac{\mu_{0}}{4\pi}\frac{Ia^{2}}{\left(a^{2}+h^{2}\right)}\mathrm{d}\theta
\]

∴円環全体で
\[
B_{z}=\int_{0}^{2\pi}\frac{\mu_{0}}{4\pi}\frac{Ia^{2}}{\left(a^{2}+b^{2}\right)^{\frac{3}{2}}}\mathrm{d}\theta=\frac{\mu_{0}Ia^{2}}{2\left(a^{2}+b^{2}\right)^{\frac{3}{2}}}
\]

2つの合計は、
\[
B=-\frac{\mu_{0}Ia^{2}}{2}\left[\frac{1}{\left\{ a^{2}+\left(d+z\right)^{2}\right\} ^{\frac{3}{2}}}+\frac{1}{\left\{ a^{2}+\left(d-z\right)^{2}\right\} ^{\frac{3}{2}}}\right]
\]

\begin{center}
\includegraphics{images/Electromagnetics12-part2/1-13}
\par\end{center}

\section*{第2回}

\subsection{ベクトルポテンシャル}
\begin{center}
\includegraphics{images/Electromagnetics12-part2/2-1}
\par\end{center}

点電荷→電荷密度と一般化したのと同様に、まず電流→電流密度で考え直す。
\begin{center}
\includegraphics{images/Electromagnetics12-part2/2-2}
\par\end{center}

$\boldsymbol{r}'$における電流要素は、
\[
I'\mathrm{d}s'=\boldsymbol{i}\left(\boldsymbol{r}'\right)\mathrm{d}A'\mathrm{d}s'=\boldsymbol{i}\left(\boldsymbol{r}'\right)\mathrm{d}v'
\]

$\mathrm{d}v'\equiv\mathrm{d}A'\mathrm{d}s'$: 体積要素

∴ビオ・サバール

\ref{1.4}→
\begin{equation}
\boldsymbol{B}\left(\boldsymbol{r}\right)=\frac{\mu_{0}}{4\pi}\int_{v}\frac{\boldsymbol{i}\left(\boldsymbol{r}'\right)\times\left(\boldsymbol{r}-\boldsymbol{r}'\right)}{\left|\boldsymbol{r}-\boldsymbol{r}'\right|^{3}}\mathrm{d}v'\label{1.7}
\end{equation}

ここで、
\begin{equation}
\nabla\left(\frac{1}{\left|\boldsymbol{r}-\boldsymbol{r}'\right|}\right)=-\frac{\boldsymbol{r}-\boldsymbol{r}'}{\left|\boldsymbol{r}-\boldsymbol{r}'\right|^{3}}\label{1.8}
\end{equation}

ただし、$\nabla$は$\boldsymbol{r}$に対する演算子
\[
\nabla=\left(\begin{array}{c}
\pd{}x\\
\pd{}y\\
\pd{}z
\end{array}\right)
\]

\ref{1.8}の証明は演習でやる。

\ref{1.7}→
\begin{align*}
\boldsymbol{B}\left(\boldsymbol{r}\right) & =\frac{\mu_{0}}{4\pi}\int\boldsymbol{i}\left(\boldsymbol{r}\right)\times\left\{ -\nabla\left(\frac{1}{\left|\boldsymbol{r}-\boldsymbol{r}'\right|}\right)\right\} \mathrm{d}v'\\
 & =\frac{\mu_{0}}{4\pi}\int_{V}\left\{ \nabla\left(\frac{1}{\left|\boldsymbol{r}-\boldsymbol{r}'\right|}\right)\right\} \times\boldsymbol{i}\left(\boldsymbol{r}'\right)\mathrm{d}v'
\end{align*}

ベクトル公式 ($f$: 任意のスカラー関数)
\[
\nabla\times\left(f\boldsymbol{i}\right)=\left(\nabla f\right)\times\boldsymbol{i}+f\left(\nabla\times\boldsymbol{i}\right)
\]
より
\[
\left(\nabla f\right)\times\boldsymbol{i}=\nabla\times\left(f\boldsymbol{i}\right)-f\left(\nabla\times\boldsymbol{i}\right)
\]
を用いると、
\begin{equation}
\boldsymbol{B}\left(\boldsymbol{r}\right)=\frac{\mu_{0}}{4\pi}\int_{V}\mathrm{d}v'\left[\nabla\times\frac{\boldsymbol{i}\left(\boldsymbol{r}'\right)}{\left|\boldsymbol{r}-\boldsymbol{r}'\right|}-\frac{1}{\left|\boldsymbol{r}-\boldsymbol{r}'\right|}\left\{ \nabla\times\boldsymbol{i}\left(\boldsymbol{r}'\right)\right\} \right]\label{1.9}
\end{equation}

ここで、$\nabla$は$\boldsymbol{r}$に対する微分なので、
\[
\nabla\times\boldsymbol{i}\left(\boldsymbol{r}'\right)=0
\]

また、$\nabla$は$\int\mathrm{d}v'$の外に出せる。

\ref{1.7}→
\begin{equation}
\boldsymbol{B}\left(\boldsymbol{r}\right)=\nabla\times\frac{\mu_{0}}{4\pi}\int_{V}\frac{\boldsymbol{i}\left(\boldsymbol{r}'\right)}{\left|\boldsymbol{r}-\boldsymbol{r}'\right|}\mathrm{d}v'\label{1.10}
\end{equation}

そこで、ベクトル
\begin{equation}
\boldsymbol{A}\left(\boldsymbol{r}\right)\equiv\frac{\mu_{0}}{4\pi}\int_{V}\frac{\boldsymbol{i}\left(\boldsymbol{r}'\right)}{\left|\boldsymbol{r}-\boldsymbol{r}'\right|}\mathrm{d}v'\label{1.11}
\end{equation}
を定義すると、
\begin{equation}
\boldsymbol{B}\left(\boldsymbol{r}\right)=\nabla\times\boldsymbol{A}\left(\boldsymbol{r}\right)\label{1.12}
\end{equation}
と表される。

\paragraph{クーロン力の場合}

\begin{align}
V\left(\boldsymbol{r}\right) & =\frac{1}{4\pi\varepsilon_{0}}\int\frac{\rho\left(\boldsymbol{r}'\right)}{\left|\boldsymbol{r}-\boldsymbol{r}'\right|}\mathrm{d}\boldsymbol{r}'\label{1.13}\\
\boldsymbol{E} & =-\nabla V\label{1.14}
\end{align}

\begin{itemize}
\item \ref{1.13}↔\ref{1.11}
\item \ref{1.14}↔\ref{1.12}
\end{itemize}
に対応している。

そこで、\ref{1.11}の$\boldsymbol{A}$をベクトルポテンシャルと呼ぶ。

(一部紛失)

\begin{equation}
?
\end{equation}

\begin{equation}
\boldsymbol{i}\left(\boldsymbol{r}\right)=\frac{1}{4\pi\varepsilon_{0}}\int_{V}\frac{\rho\left(\boldsymbol{r}'\right)}{\left|\boldsymbol{r}-\boldsymbol{r}'\right|}\mathrm{d}v'=-\frac{\rho\left(\boldsymbol{r}'\right)}{\varepsilon_{0}}\label{1.16}
\end{equation}
 

\ref{1.16}は、任意の電荷分布$\rho\left(\boldsymbol{r}\right)$について成立する。つまり、任意のスカラー関数$\rho\left(\boldsymbol{r}\right)$について成り立つ恒等式である。

∴(1.15)の両辺$\nabla^{2}$して、右辺に\ref{1.16}を適用すると、
\[
\nabla^{2}A_{x}\left(\boldsymbol{r}\right)=-\mu_{0}i_{x}\left(\boldsymbol{r}\right)
\]

\begin{equation}
\nabla^{2}\boldsymbol{A}=-\mu_{0}\boldsymbol{A}\label{1.17}
\end{equation}

つまり、$\boldsymbol{i}$と$\boldsymbol{A}$のあいだ市、ポアソンの式と同じ形の式が成り立つ。(例えば、$i_{x}$の分布から、これを$\rho$だと思って$V$を求めれば、直ちに$A_{x}$が求まる)

\[
\nabla^{2}\boldsymbol{A}=\left(\begin{array}{c}
\pdd{A_{x}}x+\pdd{A_{x}}y+\pdd{A_{x}}z\\
\pdd{A_{y}}x+\cdots\\
\pdd{A_{z}}x+\cdots
\end{array}\right)
\]

\ref{1.11}の両辺$\div$を取ると、
\begin{equation}
\nabla\cdot\boldsymbol{A}=0\label{1.18}
\end{equation}

(クーロンげージ)

導出は、教p.109-110(注)

\paragraph{$\boldsymbol{A}$のゲージ}

電位$V$に定数$V_{0}$(任意のエネルギー基準)を加えても$\boldsymbol{E}$が変わらないのと同様に、$\boldsymbol{A}$の決め方も任意性がある。

\ref{1.11}で定義した$\boldsymbol{A}$に、定ベクトル$\boldsymbol{A}_{0}$を加えても、\ref{1.12}で求まる$\boldsymbol{B}$は変わらない。

それ以上に、任意のスカラー関数$\phi\left(\boldsymbol{r}\right)$を考え、その$\nabla\phi$を$\boldsymbol{A}$に加え、
\begin{equation}
\boldsymbol{A}'\equiv\boldsymbol{A}+\nabla\phi\label{1.19}
\end{equation}
としてみても、
\[
\nabla\times\boldsymbol{A}'=\nabla\times\boldsymbol{A}+\nabla\times\left(\nabla\phi\right)
\]

(ベクトル公式より$\nabla\times\left(\nabla\phi\right)=0$)なので、$\boldsymbol{B}$は変わらない。

そこで、より広義に、\ref{1.19}で変換(ゲージ変換という)できる$\boldsymbol{A}'$をすべてベクトルポテンシャルと呼ぶ。

ただし、このとき、一般に$\nabla\cdot\boldsymbol{A}'\neq0$となる。

$\boldsymbol{A}$の選び方を「ゲージ」と言い、特に、\ref{1.18}を満たすように$\boldsymbol{A}$を選ぶこと(例えば\ref{1.11})を「クーロンゲージ」という。

\subsection{アンペアの法則}
\begin{itemize}
\item 電場については$\nabla\times\boldsymbol{E}=0$ 磁界はどうか?
\end{itemize}
\ref{1.12}より
\begin{align*}
\nabla\times\boldsymbol{B} & =\nabla\times\nabla\times\boldsymbol{A}\\
 & =\nabla\left(\nabla\cdot\boldsymbol{A}\right)-\nabla^{2}\boldsymbol{A}
\end{align*}

\ref{1.18}より$\nabla\cdot\boldsymbol{A}=0$

また\ref{1.17}より $\nabla^{2}\boldsymbol{A}=-\mu_{0}\boldsymbol{i}$

\paragraph{アンペアの法則(微分系)}

\begin{equation}
\nabla\times\boldsymbol{B}=\mu_{0}\boldsymbol{i}\label{1.20}
\end{equation}

つまり、電流$\boldsymbol{i}$があると$\boldsymbol{B}$の``渦''が発生

\paragraph{積分系}

任意の閉曲面$S$について、\ref{1.20}の両辺を面積分する。
\begin{center}
\includegraphics{images/Electromagnetics12-part2/2-3}
\par\end{center}

\[
\int_{S}\nabla\times\boldsymbol{B}\mathrm{d}S=\int_{S}\mu_{0}\boldsymbol{i}\mathrm{d}S
\]

ストークスの定理より、
\[
\text{左辺}=\oint_{C}\boldsymbol{B}\cdot\mathrm{d}\boldsymbol{s}
\]

一方、
\[
\text{右辺}=\mu_{0}I
\]

$I$: $S$を貫く全電流

∴アンペアの法則 (積分系)

\begin{equation}
\oint_{C}\boldsymbol{B}\cdot\mathrm{d}\boldsymbol{s}=\mu_{0}I\label{1.21}
\end{equation}

\begin{center}
\includegraphics{images/Electromagnetics12-part2/2-4}
\par\end{center}

$C$: 任意の閉曲線

$I$: $C$を鎖交する全電流。$C$に沿って右ねじを回すときネジが進む向きを正とする。

\paragraph{例}
\begin{center}
\includegraphics{images/Electromagnetics12-part2/2-5}
\par\end{center}

\begin{align*}
I & =I_{1}-I_{2}+I_{3}-I_{3}\\
 & =I_{1}-I_{2}
\end{align*}


\paragraph{例題}

ソレノイドの中の磁界
\begin{center}
\includegraphics{images/Electromagnetics12-part2/2-6}
\par\end{center}

うえんような閉経路$C$を考え、\ref{1.21}を適用

\[
\text{左辺}=\int_{\text{左}}\boldsymbol{B}\cdot\mathrm{d}\boldsymbol{s}+\int_{\text{下}}\boldsymbol{B}\cdot\mathrm{d}\boldsymbol{s}+\int_{\text{上}}\boldsymbol{B}\cdot\mathrm{d}\boldsymbol{s}+\int_{\text{右}}\boldsymbol{B}\cdot\mathrm{d}\boldsymbol{s}
\]

$\int_{\text{下}}\boldsymbol{B}\cdot\mathrm{d}\boldsymbol{s}+\int_{\text{上}}\boldsymbol{B}\cdot\mathrm{d}\boldsymbol{s}$:
ソレノイドが無限であれば、対称性より打ち消し合う。 

\paragraph{先週の課題1.}
\begin{center}
\includegraphics{images/Electromagnetics12-part2/2-7}
\par\end{center}

当日問題2より、円電流$I$が高さ$h$に作る$B$は
\[
B=\frac{\mu_{0}Ia^{2}}{2\left(a^{2}+h^{2}\right)^{\frac{3}{2}}}
\]

\begin{center}
\includegraphics{images/Electromagnetics12-part2/2-8}
\par\end{center}

位置$\left[x,x+\mathrm{d}x\right]$の円電流が点$P$につくる$\mathrm{d}B$を求める。

\[
r\equiv\sqrt{a^{2}+h^{2}}=\frac{a}{\sin\theta}
\]

\[
\therefore\mathrm{d}B=\frac{\mu_{0}a^{2}\left(In\mathrm{d}x\right)}{2\left(\frac{a}{\sin\theta}\right)^{3}}=\frac{\mu_{0}In\sin^{3}\theta}{2a}\mathrm{d}x
\]

\[
B=\int_{x_{1}}^{x_{2}}\frac{\mu_{0}In\sin^{3}\theta}{2a}\mathrm{d}x
\]

ここで$h=x_{p}-x=\frac{a}{\tan\theta}$

\[
x=x_{p}-\frac{a}{\tan\theta}
\]

\[
\d x{\theta}=-a\frac{\left(-\sin^{2}\theta-\cos^{2}\theta\right)}{\sin^{2}\theta}=\frac{a}{\sin^{2}\theta}
\]

$x\rightarrow\theta$に置換

\begin{align*}
\therefore B & =\int_{\theta_{1}}^{\pi-\theta_{2}}\frac{\mu_{0}In\sin^{3}\theta}{2a}\frac{a}{\sin^{2}\theta}\mathrm{d}\theta\\
 & =\frac{\mu_{0}nI}{2}\left(\cos\theta_{1}+\cos\theta_{2}\right)
\end{align*}

無限の場合は$\theta_{1}=\theta_{2}=0$ $B=\mu_{0}nI$となり、講義(1.26)と一致。

\paragraph{当日問題1(a)①}

\ref{1.11}から計算

$x,y$方向は電流=0なので$A_{x}=A_{y}=0$
\begin{center}
\includegraphics{images/Electromagnetics12-part2/2-9}
\par\end{center}

\begin{align*}
A_{z} & =\frac{\mu_{0}}{4\pi}\int_{-L}^{L}\frac{I}{r}\mathrm{d}z\\
 & =\frac{\mu_{0}}{4\pi}\int_{-L}^{L}\frac{I}{\sqrt{z^{2}+a^{2}}}\mathrm{d}z\\
 & =\frac{\mu_{0}I}{4\pi}\left[\log\left(z+\sqrt{z^{2}+a^{2}}\right)\right]_{-L}^{L}\\
 & =\frac{\mu_{0}I}{4\pi}\log\frac{L+\sqrt{L^{2}+a^{2}}}{-L+\sqrt{L^{2}+a^{2}}}\\
 & =\frac{\mu_{0}I}{4\pi}\log\frac{\left\{ L+\sqrt{L^{2}+a^{2}}\right\} ^{2}}{a^{2}}\\
 & =\frac{\mu_{0}I}{2\pi}\log\frac{L+\sqrt{L^{2}+a^{2}}}{a}\\
 & =\frac{\mu_{0}I}{2\pi}\left[\log\frac{1+\sqrt{1+\left(\frac{a}{L}\right)^{2}}}{a}-\log L\right]
\end{align*}

$\log L$の項は定数なので無視する。($A$の原点を$\frac{\mu_{0}I}{2\pi}\log L$だけずらすと考える)

$L\rightarrow\infty$の極限を考えると、
\begin{align*}
A_{z} & =\frac{\mu_{0}I}{2\pi}\log\frac{2}{a}=\frac{\mu_{0}I}{2\pi}\left[\log2-\log a\right]\\
 & =-\frac{\mu_{0}I}{2\pi}\log a+\left(\nabla\phi\right)
\end{align*}


\paragraph{当日問題1(a)②}

教p.111 例題6.5

$z$軸上に線密度$I$の一様な電荷がある時の$V$を計算する。

ガウスの法則より$z$軸から距離$r$の位置の電界は$E\left(r\right)=\frac{I}{2\pi\varepsilon_{0}r}$

\begin{align*}
V\left(a\right) & =-\int_{r_{0}}^{a}E\left(r\right)\mathrm{d}r\\
 & =-\frac{I}{2\pi\varepsilon_{0}}\left(\log-\log r_{0}\right)
\end{align*}

$\log r_{0}$は無視。$\varepsilon_{0}\rightarrow\frac{1}{\mu_{0}}$とすると、
\[
A_{z}=-\frac{\mu I}{2\pi}\log a
\]


\paragraph{当日問題1(b)}

\begin{align*}
A_{z} & =-\frac{\mu_{0}I}{2\pi}\log\sqrt{x^{2}+y^{2}}\\
 & =-\frac{\mu_{0}I}{4\pi}\log\left(x^{2}+y^{2}\right)
\end{align*}

$\therefore\boldsymbol{B}=\nabla\times\boldsymbol{A}$より、
\[
B_{x}=\pd{A_{z}}y=-\frac{\mu_{0}I}{2\pi}\frac{y}{x^{2}+y^{2}}
\]

\[
B_{y}=-\pd{A_{z}}x=\frac{\mu_{0}I}{2\pi}\frac{x}{x^{2}+y^{2}}
\]

\[
B_{z}=0
\]

\begin{center}
\includegraphics{images/Electromagnetics12-part2/2-10}
\par\end{center}

先週の演習問題1(b)$B=\frac{\mu_{0}I}{2\pi a}$に一致

\paragraph{当日問題1(c)}

\begin{align*}
\pd{B_{x}}x & =\frac{\mu_{0}I}{\pi}\frac{xy}{\left(x^{2}+y^{2}\right)^{2}}\\
\pd{B_{y}}y & =-\frac{\mu_{0}I}{\pi}\frac{xy}{\left(x^{2}+y^{2}\right)^{2}}\\
\pd{B_{z}}z & =0
\end{align*}

\[
\therefore\nabla\cdot\boldsymbol{B}=0
\]


\paragraph{当日問題2}
\begin{center}
\includegraphics{images/Electromagnetics12-part2/2-11}
\par\end{center}

円柱の軸を中心に半径$r$の円経路$C$を考える。

(i) $r<a$のとき
\begin{center}
\includegraphics{images/Electromagnetics12-part2/2-12}
\par\end{center}

$C$を貫く電流は
\[
I=\pi r^{2}J
\]

一方
\[
\oint\boldsymbol{B}\cdot\mathrm{d}\boldsymbol{s}=2\pi rB
\]

対称性より$C$上で$B$一定

∴アンペアの法則

\begin{align*}
2\pi rB & =\pi\mu_{0}r^{2}J\\
B & =\frac{\mu_{0}rJ}{2}
\end{align*}

(ii) $r>a$のとき
\begin{center}
\includegraphics{images/Electromagnetics12-part2/2-13}
\par\end{center}

\[
I=\pi a^{2}J
\]

\begin{align*}
\therefore2\pi rB & =\pi\mu_{0}a^{2}J\\
B & =\frac{\mu_{0}a^{2}J}{2r}
\end{align*}

\begin{center}
\includegraphics{images/Electromagnetics12-part2/2-14}
\par\end{center}

\paragraph{当日問題3}

\[
\left|\boldsymbol{r}-\boldsymbol{r}'\right|=\sqrt{\left(x-x'\right)^{2}+\left(y-y'\right)^{2}+\left(z-z'\right)^{2}}
\]

\begin{align*}
\pd{}x\left(\frac{1}{\left|\boldsymbol{r}-\boldsymbol{r}'\right|}\right) & =-\frac{1}{\left|\boldsymbol{r}-\boldsymbol{r}'\right|^{2}}\pd{}x\left(\left|\boldsymbol{r}-\boldsymbol{r}'\right|\right)\\
 & =-\frac{1}{\left|\boldsymbol{r}-\boldsymbol{r}'\right|^{2}}\frac{2\left(x-x'\right)}{2\sqrt{\left(x-x'\right)^{2}+\left(y-y'\right)^{2}+\left(z-z'\right)^{2}}}=-\frac{\left(x-x'\right)}{\left|\boldsymbol{r}-\boldsymbol{r}'\right|^{3}}
\end{align*}

$y,z$成分も同様

\[
\therefore\nabla\left(\frac{1}{\left|\boldsymbol{r}-\boldsymbol{r}'\right|}\right)=-\frac{\boldsymbol{r}-\boldsymbol{r}'}{\left|\boldsymbol{r}-\boldsymbol{r}'\right|^{3}}
\]


\section*{第3回}

(ぽつぽつ寝落ちして抜けてるかも)

\subsection{磁荷の非存在}

\paragraph{ガウスの法則}

\[
\nabla\cdot\boldsymbol{E}=\frac{\rho}{\varepsilon_{0}}
\]

つまり電荷$\rho$により$\boldsymbol{E}$が発生。

$\boldsymbol{B}$についても同様にdivを考える。

\ref{1.12}より、
\[
\nabla\cdot\boldsymbol{B}=\nabla\cdot\left(\nabla\times\boldsymbol{A}\right)=0
\]

\begin{equation}
?
\end{equation}

\begin{equation}
\nabla\cdot\boldsymbol{B}=0\label{1.23}
\end{equation}

つもり``磁荷''は存在しない。

ガウスの定理より\ref{1.23}の積分形は、
\begin{equation}
\int_{S}\boldsymbol{B}\cdot\boldsymbol{n}\mathrm{d}S=0\label{1.24}
\end{equation}

$S$: 任意の閉曲面

図電後3-1

一般に、ある面$S$を通過する$\boldsymbol{B}$の合計
\begin{equation}
\Phi\equiv\int_{S}\boldsymbol{B}\cdot\boldsymbol{n}\mathrm{d}S\label{1.25}
\end{equation}
を磁束と呼び、単位をWb(ウェーバー)で表す。つまり$\mathrm{T=\frac{Wb}{m^{2}}}$。

図電後3-2

\ref{1.25}に\ref{1.12}を代入し、ストークスの定理を用いると、
\begin{align*}
\int\boldsymbol{B}\cdot\boldsymbol{n}\mathrm{d}S & =\int_{S}\left(\nabla\times\boldsymbol{A}\right)\cdot\boldsymbol{n}\mathrm{d}S\\
 & =\oint_{C}\boldsymbol{A}\cdot\mathrm{d}\boldsymbol{s}
\end{align*}

$C$: $S$の境界

\begin{equation}
\therefore\ref{1.25}\Leftrightarrow\Phi=\int_{C}\boldsymbol{A}\cdot\mathrm{d}\boldsymbol{s}\label{1.26}
\end{equation}
とも書ける。

\textbf{境界上の$\boldsymbol{A}$を一階積分すると、その中の磁束が求まる。}

$S$が閉曲面のとき、$C$はないので$\Phi=0$⇔\ref{1.24}

\paragraph{まとめ}

今、

図電後3-3

の順に導出したが、逆に①→③→②→①の順でも導出できる(教科書の順)。

つまり①、②、③は等価(教p.117 図6.21参照)。

\section{磁性体}

\subsection{磁化と磁性体}

\paragraph{エールステーエッドの実験 (1820年)}

電流が流れる導線に方位磁石を近づけると回転する。

図電後3-4

→どうやら電流と磁石に関連がある。

\paragraph{磁石の正体}

内部に無数の電流ループが存在し、これらが磁界を生み出している。

図電後3-5

「微小な電流ループ」の大きさを表すために、磁気モーメント
\begin{equation}
\boldsymbol{m}\equiv IA\boldsymbol{n}\label{2.1}
\end{equation}
を定義する。

図電後3-6

$I$: 電流

$A$: ループの面積

$\boldsymbol{n}$: 電流に対して右ねじが進む向きの単位ベクトル

\paragraph{$\boldsymbol{m}$の起源}
\begin{enumerate}
\item 電子の自転(スピン)。多くの場合、これが主要因。

図電後3-7
\item 原子核を回る電子の軌道運動

図電後3-8
\end{enumerate}
単位体積あたり$N$個の磁気モーメント$\boldsymbol{m}_{i}\left(i=1,2,\cdots,N\right)$があるとき、単位体積あたりの巨視的な磁気モーメントは、
\begin{equation}
\boldsymbol{M}=\sum_{i=1}^{N}\boldsymbol{m}_{1}=N\left\langle \boldsymbol{m}_{i}\right\rangle \label{2.2}
\end{equation}

これを磁化ベクトルと呼ぶ。

一般に物質中の$\boldsymbol{m}_{i}$はランダムな方向を向くので、$\boldsymbol{M}=0$。しかし、外部から磁界をかけると$\boldsymbol{m}_{i}$がせいろ津市、全体として$\boldsymbol{M}\neq0$になる物質がある。このような物質を磁性体と呼び、$\boldsymbol{m}_{i}$がある方向に整列することを「磁化する(magnetize)」と言う。

自選の起源は量子力学的現象。電磁気学では、$\boldsymbol{B}$と$\boldsymbol{M}$の関係が与えられたものとして、議論を進める。

(2) 強磁性体 (ferromagnetic)

常磁性体の中でも、特に$\boldsymbol{M}$が強いもの。(鉄、ニッケルなど)

外部磁界を取り去ったあとでも$\boldsymbol{M}$が保たれる。これが磁石。

(3) 反磁性体 (diamagnetic)

$\boldsymbol{B}$に対して\uline{反対の向き}に$\boldsymbol{m}$がそろおうとする材料。(ビスマスなど)つまり、$\boldsymbol{B}$を打ち消そうとする。

\subsection{磁性体中のアンペアの法則}

図電後3-9

今、磁性体の中を通る経路$C$について\ref{1.21}を考える。

$C$に鎖交する電流$I$として、通常の電流$I_{F}$(自由電流と呼ぶ)肉を得て、磁性体内の微小電流ループの総和(磁化電流)$I_{M}$も考える必要がある。

つまり\ref{1.21}は
\begin{equation}
\oint_{C}\boldsymbol{B}\cdot\mathrm{d}\boldsymbol{s}=\mu_{0}I=\mu_{0}\left(I_{E}+I_{M}\right)\label{2.3}
\end{equation}

と分割して書ける。

1つの$\boldsymbol{m}$が$C$と鎖交する確率は
\[
\frac{A\cos\theta}{S}=\frac{A}{S}\left(\boldsymbol{n}\cdot\frac{\mathrm{d}\boldsymbol{s}}{\left|\mathrm{d}\boldsymbol{s}\right|}\right)
\]

∴この領域で$C$に鎖交する電流の期待値は、
\[
\mathrm{d}I_{M}=\left\langle \frac{IA}{S}\left(\boldsymbol{n}\cdot\frac{\mathrm{d}\boldsymbol{s}}{\left|\mathrm{d}\boldsymbol{s}\right|}\right)\right\rangle NS
\]

全$I_{M}$はこれを$C$全体で積分したもの
\begin{equation}
\therefore I_{M}=\oint_{C}\boldsymbol{M}\cdot\mathrm{d}\boldsymbol{s}\label{2.4}
\end{equation}

ストークスの定理より\ref{2.4}の微分系は
\begin{equation}
\boldsymbol{i}_{M}=\nabla\times\boldsymbol{M}\label{2.5}
\end{equation}

$\boldsymbol{i}_{M}$: 磁化電流密度

\ref{2.4}を\ref{2.3}に代入して、
\begin{equation}
\oint_{C}\left(\frac{1}{\mu_{0}}\boldsymbol{B}-\boldsymbol{M}\right)\cdot\mathrm{d}\boldsymbol{s}=I_{F}\label{2.6}
\end{equation}

そして、新しい場
\begin{equation}
\boldsymbol{H}\equiv\frac{1}{\mu}\boldsymbol{B}-\boldsymbol{M}\label{2.7}
\end{equation}
を定義すると、\ref{2.6}は、
\begin{equation}
\oint_{C}\boldsymbol{H}\cdot\mathrm{d}\boldsymbol{s}=I_{F}\label{2.8}
\end{equation}
→微分系
\begin{equation}
\boldsymbol{i}_{F}=\nabla\times\boldsymbol{H}\label{2.9}
\end{equation}

$\boldsymbol{i}_{F}$自由電流密度

$\boldsymbol{H}$は全磁束密度$\boldsymbol{B}$のうち、z祐電流$I_{F}$と直接関連付けられる成分を表すため、「磁界」と呼ばれる。

一方、\ref{1.23}、\ref{1.24}
\[
\int\boldsymbol{B}\cdot\boldsymbol{n}\mathrm{d}S=0
\]
は、電流と関係ないので、磁性体の中でもそのまま成立する。

\subsection{境界条件}

2種種類の磁性体の界面における条件

\paragraph{$\boldsymbol{H}$の条件}

図電後3-10

上図の$C$で\ref{2.8}を考えると、
\[
H_{1\sslash}=H_{2\sslash}
\]

界面で$I_{F}=0$のとき\uline{$\boldsymbol{H}$の接線成分}は界面で連続。

\paragraph{$\boldsymbol{B}$の条件}

図電後3-11

上図の$S$で\ref{1.24}を考える。

\[
B_{1\bot}=B_{2\bot}
\]

\uline{$\boldsymbol{B}$の垂直成分}が連続

\subsection{磁化曲線}

磁性体に外部磁界$\boldsymbol{H}$をかけた時の磁化の様子を示すのにM-H特性が用いられる。

(1)常磁性体

図電後3-12

(2)反磁性体

図電後3-13

(3)強磁性体

図電後3-14

ヒステリシス(行きと帰りで異なる曲線になること)を示す。

\subsection{透磁率}

一般にM-H特性は複雑。しかし、常磁性体、反磁性体で$H$が十分小さい領域(線形領域)では、
\begin{equation}
\boldsymbol{M}=\chi\boldsymbol{H}\label{2.10}
\end{equation}
と近似できる。

・強磁性体でも、ヒステリシスが小さい場合や、最小に磁化する領域で\ref{2.10}を用いる。

・磁石($H=0$で$M\neq0$)は、は使えない。$M$=一定 とするか、M-H特性を厳密に考える。

$\chi$を磁化率と呼ぶ。(常磁性体: $\chi>0$、反磁性体: $-1<\chi<0$)

\ref{2.10}を\ref{2.7}に代入し、
\[
\boldsymbol{B}=\mu_{0}\left(\boldsymbol{H}+\boldsymbol{M}\right)=\left(1+\chi\right)\mu_{0}\boldsymbol{H}
\]

\begin{equation}
\therefore\boldsymbol{B}=\mu\boldsymbol{H}\label{2.11}
\end{equation}

ただし
\begin{equation}
\mu\equiv\left(1+\chi\right)\mu_{0}\label{2.12}
\end{equation}

$\mu$を透磁率と呼ぶ。

\paragraph{先週の宿題1}

(a)

\[
\boldsymbol{B}=\nabla\times\boldsymbol{A}=\left(\begin{array}{c}
0\\
0\\
B_{0}
\end{array}\right)
\]

(b)

図電後3-15

$A$は電流を流すべき方向と同じ方向になる。

\paragraph{先週の宿題2}

図電後3-16

$z$字句を中心都市、半径$r$の炎上でアンペアの法則(積分系)を適用する。

(i) $r<a$

$C_{1}$に鎖交する電流はない。

\[
\oint\boldsymbol{B}\cdot\mathrm{d}\boldsymbol{s}=2\pi rB=0
\]

\[
B=0
\]

(ii) $b<r<a$

\[
2\pi rB=-\mu I
\]

\[
\therefore B=-\frac{\mu_{0}I}{2\pi r}
\]

(iii) $a<r$

\[
2\pi rB=\mu\left(-I+I\right)=0
\]

\[
B=0
\]

図電後3-17

\paragraph{前回の宿題3}

(a)

\[
\nabla\times\left(f\boldsymbol{u}\right)=\nabla f\times\boldsymbol{u}+f\left(\nabla\times\boldsymbol{u}\right)
\]

$x$成分は
\[
\pd{\left(fu_{z}\right)}y-\pd{\left(fu_{y}\right)}z=\left(\pd fyu_{z}-\pd fzu_{y}\right)+f\left(\pd{u_{z}}y-\pd{u_{y}}z\right)
\]

$y,z$成分も同様

(b)

\[
\nabla\times\left(\nabla f\right)=0
\]

$x$成分は
\[
\pd{}y\left(\pd fz\right)-\pd{}z\left(\pd fy\right)=0
\]

$y,z$も同様

(c)

\[
\nabla\cdot\left(\nabla\times\boldsymbol{u}\right)=0
\]

\[
\text{左辺}=\pd{}x\left(\pd{u_{z}}y-\pd{u_{y}}z\right)+\pd{}y\left(\pd{u_{x}}z-\pd{u_{z}}x\right)+\pd{}z\left(\pd{u_{y}}x-\pd{u_{x}}y\right)=0
\]


\paragraph{当日問題1}

点電荷は時間$\frac{2\pi}{\omega}$の間にいっしゅうするので、電流$I=\frac{q\omega}{2\pi}$

\[
\therefore\left|\boldsymbol{m}\right|=IS=\frac{q\omega}{2\pi}\cdot a^{2}\pi=\frac{qL}{2m}\quad\left(\because L=ma^{2}\omega\right)
\]

$L$→量子化することで$m$が説明される。

\paragraph{当日問題2}

教科書p.152参照

図電後3-18
\begin{itemize}
\item $M$は磁石の中だけ。外では$M=0$
\item $H$は、磁石の外では$H=\frac{1}{\mu_{0}}B$
\item $I_{F}=0$なので
\[
\int H\cdot\mathrm{d}s=0
\]
\item $\nabla\cdot\boldsymbol{B}=0$なので$B$は連続

図電後3-19

図電後3-20

一般に磁石の内部で$H$と$B$は逆ムキになる。このような$H$を減磁力や反磁界と呼ぶ。
\end{itemize}

\paragraph{当日問題3}

(a)

図電後3-21

\[
H_{\text{外}}=H_{\text{中}}
\]
\[
H_{\text{外}}=\frac{B_{\text{外}}}{\mu_{0}},H_{\text{中}}=\frac{B_{\text{中}}}{\mu_{0}}
\]

磁気回路(教.131)
\begin{itemize}
\item $B$→電流$I$
\item $H$→電界$E$
\item $\frac{1}{\mu}$→抵抗
\end{itemize}
とみなして、電気回路の問題と同様に解くことができる。

(b)

図電後3-22

考え方
\begin{enumerate}
\item $\nabla\cdot B=0$より経路上で(断面積が一定)なら$B$は一定
\item 磁性体(磁石ではない)の中で$H=\frac{1}{\mu}B$
\item アンペアの法則 $\oint H\cdot\mathrm{d}\boldsymbol{s}=I_{F}$
\end{enumerate}
\[
H_{m}=\frac{B}{\mu},H_{g}=\frac{B}{\mu_{0}}
\]

\[
\oint H\cdot\mathrm{d}\boldsymbol{s}=\frac{B}{\mu}\left(l-g\right)+\frac{B}{\mu_{0}}g=NI
\]

\[
B=\frac{NI}{\frac{l}{\mu}+\frac{g}{\mu_{0}}}
\]

(鉄心中もギャップも同じ)

(c)

\begin{align*}
H_{m} & =\frac{B}{\mu}=\frac{NI}{l+\frac{\mu}{\mu_{0}}g}\quad\left(\text{鉄心}\right)\\
H_{g} & =\frac{B}{\mu_{0}}=\frac{NI}{\frac{\mu_{0}}{\mu}l+g}\quad\left(\text{ギャップ}\right)
\end{align*}

$g\ll l,\mu\gg\mu_{0}$ならギャップの中心のほうが強い。

\section*{第4回}

\paragraph{試験}

2/2 10-12

\section*{第5回}

\section{電磁誘導の法則}

\subsection{ファラデーの法則 (1831年)}

電流により磁界が発生 (エルスレッド 1820)

→逆に磁界が

(一部紛失)

\subsection{???}

$+y$方向に動かすことを考える。導体中の自由電荷$q\left(>0\right)$は、

(一部紛失)

つまり、一般に磁束密度$\boldsymbol{B}$の磁界の中を速度$\boldsymbol{v}$で移動すると、単位長さあたり、
\begin{equation}
\varepsilon=\boldsymbol{v}\times\boldsymbol{B}\label{3.1}
\end{equation}
の起電力が発生する。

\paragraph{(復習)}

起電力は、非保存的な力(ここではローレンツ力)によって、単位電荷に与えられる仕事。(教科書 第5章 p.87)

例: 電池

つまり、

図電後10-1

と同じ。

今、$z$方向の一様な磁界$B_{z}$の中で、回路のの一部である棒ABを速度$v_{y}$で動かす。

図電後10-2

Ab間には、\ref{3.1}より、A→Bの向きに
\begin{equation}
\varepsilon=lv_{y}B_{z}\label{3.2}
\end{equation}
の起電力が発生し、電流が流れる。

その結果、正電荷がAB間を$+x$方向に流れるため、今度は棒ABに$-y$方向のローレンツ力が加わる。

つまり導体棒の運動を阻止しようとする方向に起電力は発生する。

\subsection{ファラデーの法則 (磁束の変化)}

上の例を別の見方で見る。

図電後10-3

棒ABが位置$y$から$\Delta t$病後リハ$v_{y}\Delta t$移動するとする。

閉回路ABCDを貫く磁束$\Phi$は、
\[
\Phi=l_{y}B_{y}
\]

時間$\Delta t$の間に
\[
\Delta\Phi=lB_{z}v_{y}\Delta t
\]
だけ増える。

つまり\ref{3.2}の速度起電力の大きさは、
\[
\left|\d{\Phi}t\right|=\left|lB_{y}v_{y}\right|
\]
に一致する。

$\Phi$を増やす方向に電流を流す起電力の向きを性と定めると、\ref{3.2}の$\varepsilon$は、
\begin{equation}
\varepsilon=-\d{\Phi}t\label{3.3}
\end{equation}
(ファラデーの法則)と表すことができる。

つまり、\uline{$\Phi$の変化を抑えようとする方向}に$\varepsilon$は発生する。(レンツの法則)

図電後10-4

次に、$B$が空間的に一様でなく、$y$の関数のとき、閉回路全体が速度$v_{y}$で動いている場合を考える。

回路一周の起電力を図中の$\varepsilon$飲む気($+z$方向の磁束を強める向き)を性として求めてみる。

辺ABと辺CDに働く速度起電力の和として考えると、
\[
\varepsilon=lv_{y}B_{DC}-lv_{y}B_{AB}
\]

$B_{DC}$: DCにおける$B$

$B_{AB}$: ABにおける$B$

一方、ファラデーの法則で考えると、
\[
\d{\Phi}t=lv_{y}B_{AB}-lv_{y}B_{CD}
\]

$lv_{y}B_{AB}$: 増えたぶん

$lv_{y}B_{CD}$: 減ったぶん

$\varepsilon=-\d{\Phi}t$となり、\ref{3.3}が満たされる。

任意の形の閉回路、および、$\boldsymbol{B}$と$\boldsymbol{v}$の向きについて(回路の変形があるときも)

$\boldsymbol{B}$が時間的に一定のときは、\ref{3.1}と\ref{3.3}が一致することが示すことができる。

図電後10-5

(参考 教科書p.150 図8.15~)

$\left(\pd{\Phi}t\right)$($B$内での面$S$の移動)の項。

その結果、

(一部紛失)

つまり、
\begin{itemize}
\item $\varepsilon$は$\boldsymbol{v}\times\boldsymbol{B}$もしくは、$\pd{\boldsymbol{B}}t$により生じる。
\item 本質的なのは\ref{3.3}えあり、これはどの完成形デマ変わらない。
\item どの慣性経過によ宛て、$\boldsymbol{v}\times\boldsymbol{B}$と$\pd{\boldsymbol{B}}t$の内訳が変わる。
\end{itemize}

\subsection{ファラデーの法則 (微分系)}

速度起電力$\varepsilon=\boldsymbol{v}\times\boldsymbol{B}$は、ローレンツ力$\boldsymbol{F}=q\left(v\times\boldsymbol{B}\right)$から導出できた。

では、
\begin{equation}
\varepsilon=\int_{S}\left(-\pd{\boldsymbol{B}}t\right)\cdot\boldsymbol{n}\mathrm{d}\mathrm{S}\label{3.4}
\end{equation}
の成分は何に対応するのか?

→答え、ローレンツ力は、\ref{1.6}より、
\[
F=q\left(\boldsymbol{E}+\boldsymbol{v}\times\boldsymbol{B}\right)
\]

$\boldsymbol{v}\times\boldsymbol{B}$: 速度起電力を生む。

$\boldsymbol{E}$: クーロン力(保存力)に加えて$\pd{\boldsymbol{B}}t$に対応する成分(非保存力)があるはず。

\[
\boldsymbol{E}_{\text{クーロン}}=-\nabla V,\nabla\times\boldsymbol{E}_{\text{クーロン}}=0
\]

$\boldsymbol{E}$(非保存力成分)と$\pd{\boldsymbol{B}}t$の関係を求める。

簡単のため$v=0$($\pd{\boldsymbol{B}}t$のみ)の場合を考える。(一般的な$v\neq0$の場合については教p.151)

$v=0$なので$\boldsymbol{F}=q\boldsymbol{E}$

回路一周分の起電力は
\begin{equation}
\varepsilon=\underbrace{\oint_{C}\frac{\boldsymbol{F}}{q}\cdot\mathrm{d}\boldsymbol{s}}_{\text{単位電荷が受ける仕事}}=\oint_{C}\boldsymbol{E}\cdot\mathrm{d}\boldsymbol{s}=\int_{S}\left(\nabla\times\boldsymbol{E}\right)\cdot\boldsymbol{n}\mathrm{d}S\label{3.5}
\end{equation}

\ref{3.4}より\ref{3.5}の左辺は、
\[
\varepsilon=\int_{S}\left(-\pd{\boldsymbol{B}}t\right)\cdot\boldsymbol{n}\mathrm{d}S=\int_{S}\left(\nabla\times\boldsymbol{E}\right)\cdot\boldsymbol{n}\mathrm{d}S
\]

常識が任意の$S$について成り立つには、
\begin{equation}
\nabla\times\boldsymbol{E}=-\pd{\boldsymbol{B}}t\label{3.6}
\end{equation}
(ファラデーの法則(微分系))

\paragraph{まとめ}

積分系 $\varepsilon=-\d{\Phi}t$

⇔微分系 $\boldsymbol{F}=q\left(\boldsymbol{E}+\boldsymbol{v}\times\boldsymbol{B}\right)$
かつ $\nabla\times\boldsymbol{E}=-\d{\boldsymbol{B}}t$

\paragraph{\ref{3.3}の$\Phi$について注意}

$\Phi$は$\d{\boldsymbol{B}}t$と$\boldsymbol{v}\times\boldsymbol{B}$の両方の成分を表していないといけない。

$\Phi$は任意の閉経路で定義されているものでは\uline{ない}。

図電後10-6

経路$C$を貫く$\Phi$は変化しない。しかし、円盤沿いに沿ったループ$C'$を貫く$\Phi$は変化するので、$\varepsilon$が発生する。

導体ループが移動(あるいは変形)しているときは、\uline{ループを作る物質と一緒に動く}(変形する)用に閉経路を定義しないといけない。

\subsection{インダクタンス}

\paragraph{自己インダクタンス}

図電後10-7
\begin{enumerate}
\item コイルに流す電流$I$を増やす
\item $I$によって作られる磁界$B$が増大
\item コイルを貫く磁束$\Phi$が増大
\item ファラデーの法則より、
\[
\varepsilon=-\d{\Phi}t
\]
($I$の増加を抑える向き)の起電力が発生する。
\end{enumerate}
この減少を自己誘導と呼ぶ。

この時、$B\propto I,\Phi\propto B$なので、
\begin{equation}
\Phi=LI\label{3.7}
\end{equation}
と表せられる。

比例係数$L$を自己インダクタンスと予備、単位H(ヘンリー)を用いる。$L$はコイルの形状などにより決まり、常に正である。

\ref{3.7}、\ref{3.3}より、コイルの両端には、
\begin{equation}
V=L\d It\label{3.8}
\end{equation}
の電位が現れる。

図電後10-8

$V$は正電荷が集まる側を性と定義されており、$\varepsilon$の向きとは逆である。

\paragraph{相互インダクタンス}

図電後10-9

コイルが2つ近くにあるとき、それぞれに流れる電流を$I_{1},I_{2}$、鎖交磁束を$\Phi_{1},\Phi_{2}$とすると、
\begin{equation}
\begin{cases}
\Phi_{1}=L_{11}I_{1}+L_{12}I_{2}\\
\Phi_{2}=L_{21}I_{1}+L_{22}I_{2}
\end{cases}\label{3.10}
\end{equation}
と線形和で表される。

$L_{11},L_{22}$はコイル1,2の自己インダクタンス

相互性より、
\[
L_{12}=L_{21}=M
\]
が必ず成立する。(今週の宿題)

$M$を相互インダクタンスと呼ぶ。$M$は一般に正負どちらにもなる。

$I_{1}$が作る全磁束$\Phi_{11}=L_{11}I_{1}$のうち、\uline{一部}がコイル2と鎖交する。

∴この磁束$\Phi_{21}\left(=MI_{1}\right)$は
\[
\Phi_{21}=k_{21}\Phi_{11}=k_{21}L_{11}I_{1}\quad\left(\left|k_{21}\right|<1\right)
\]

同様に
\[
\Phi_{12}\equiv MI_{2}=k_{12}L_{22}=k_{12}L_{22}I_{2}
\]

\[
\therefore\Phi_{12}\Phi_{21}=M^{2}I_{1}I_{2}=k_{12}k_{21}L_{11}L_{22}I_{1}I_{2}
\]

\begin{equation}
\frac{M}{\sqrt{L_{12}L_{21}}}=\sqrt{k_{12}k_{21}}\equiv k\quad\left(\left|k\right|<1\right)\label{3.12}
\end{equation}

$k$を結合係数という。$\left|k\right|=1$のとき、磁束の漏れがない。

\paragraph{先週からの宿題1(a)}

図電後1010

境界面で$H$の接線成分が連続

\[
H_{m}=H
\]

\[
B=\frac{H}{\mu_{0}},B_{m}=\frac{H_{m}}{\mu}=\frac{H}{\mu}
\]

$M=\chi H_{m}$なので、$H=\frac{1}{\chi}M$

\paragraph{宿題1(b)}

図電後10-11

境界面で$B$のほうせんせいぶんが連続

\[
H_{m}=\frac{B}{\mu_{0}\left(1+\chi\right)},H=\frac{B}{\mu_{0}}
\]

\[
M=\chi H_{m}=\frac{\chi B}{\mu_{0}\left(1+\chi\right)}=\frac{\chi}{1+\chi}H
\]


\paragraph{宿題2}

図電後10-12

断面積一定で、$\nabla\cdot\boldsymbol{B}=0$より、$C$上で$B$一定。

$C$上で$H$に対するアンペアの法則を考える。

磁石内 $H_{PM}$

磁性体内 $H_{M}=\frac{B}{\mu}$

ギャップ内 $H_{g}=\frac{B}{\mu_{0}}$

\[
\oint_{C}\boldsymbol{H}\cdot\mathrm{d}\boldsymbol{s}=H_{PM}L+\frac{B}{\mu}l+\frac{B}{\mu_{0}}2s=0\quad\left(\because\text{自由電流がないので}\right)
\]


\paragraph{(a)}

$B_{PM}=B$なので、
\[
H_{PM}L+\frac{B_{PM}}{\mu}l+\frac{B_{M}}{\mu_{0}}2S=0
\]

(b)

\[
M_{PM}=\frac{\mu_{0}L}{2\delta}H_{PM}
\]
⋯1

\paragraph{(c)}

M-H特性は
\[
B_{PM}=B_{r}\left(1+\frac{H_{LM}}{H_{C}}\right)
\]
⋯2

2つの線の交点が会となる。1,2より$H_{PM}$を消去して、
\[
B=B_{PM}=\frac{1}{1+\frac{28B_{r}}{\mu_{0}H_{C}L}}B_{r}
\]

$H_{PM}<0$なので、$B_{PM}$および$M$と同じ逆向きになる。

\paragraph{1(a)}

図電後10-13

コイルの左右両辺は速度$v=\frac{b\omega}{2}$で回転している。

左側の変に発生する起電力は、
\[
\varepsilon_{L}=-a\left|\boldsymbol{v}\times\boldsymbol{B}\right|=-\frac{ab\omega}{2}\cos\omega t
\]

ただし、回転を加速させる$\varepsilon$の向きを正とした。

同様に右側も$\varepsilon_{R}=\varepsilon_{L}$の起電力が発生。

∴コイル一周分の起電力は
\[
\varepsilon=\varepsilon_{L}+\varepsilon_{R}=-abB\omega\cos\omega t
\]


\paragraph{(b)}

コイルを貫く磁束は、
\[
\Phi=abB\sin\omega t
\]

\[
\varepsilon=-\d{\Phi}t=-abB\omega\cos\omega t
\]


\paragraph{2}

図電後10-14

対称性より、発生する電界$E$は、中心軸に対して軸対象となる。

半径$r$の演習$C$に沿ってファラデーの法則を適用する。

\[
\nabla\times\boldsymbol{E}=-\d{\boldsymbol{B}}t
\]
より、
\[
\oint\boldsymbol{E}\cdot\mathrm{d}\boldsymbol{s}=-\d{}t\int_{S}\boldsymbol{B}\cdot\boldsymbol{n}\mathrm{d}S
\]

\begin{align*}
\text{左辺} & =2\pi rE\\
\text{右辺} & =-\d{\Phi}t
\end{align*}

\[
\therefore E\left(r\right)=-\frac{1}{2\pi r}\d{\Phi}t
\]


\paragraph{3}

図電後10-15

\paragraph{(a)}

アンペアより、コイル1への電流$I_{1}$が鉄心内に作る磁界$H_{1}$は、
\[
H_{1}l=N_{1}I
\]

\[
B_{1}=\mu H_{1}=\frac{\mu N_{1}I_{1}}{l}
\]

これがコイル1(自分)を貫く磁束は、
\[
\Phi_{N}=N_{1}SB_{1}=\frac{\mu N_{1}^{2}S}{l}I_{1}
\]

\[
\therefore L_{1}=\frac{\mu N_{1}^{2}S}{l}
\]

同様に、
\[
L_{2}=\frac{\mu N_{2}^{2}S}{l}
\]


\paragraph{(b)}

$B_{1}=\frac{\mu_{1}N_{1}I_{1}}{l}$がコイル2を貫く磁束は、
\[
\Phi_{21}=N_{2}SB_{1}=\frac{\mu N_{1}N_{2}S}{l}I_{1}
\]

\[
\therefore M=\frac{\mu N_{1}N_{2}S}{l}
\]

\[
k=\frac{M}{\sqrt{L_{1}L_{2}}}=1
\]
(磁束の漏れがない)

\paragraph{(c)}

\begin{align*}
V_{1}\left(t\right) & =-L_{1}\d{I_{1}\left(t\right)}{}\\
V_{2}\left(t\right) & =-M\d{I_{1}\left(t\right)}t=\frac{M}{L_{1}}V_{1}\left(t\right)=\frac{N_{2}}{N_{1}}V_{1}\left(t\right)
\end{align*}

∴
\[
\text{電圧比}=\frac{N_{2}}{N_{1}}
\]


\section*{第5回}

\subsection{コイルに蓄えられるエネルギー}

コンデンサに電荷を帯電すると静電エネルギー$\frac{1}{2}CV^{2}$が蓄えられる。

→同様に、コイルに電流を供給するには、電磁誘導による起電力$V=\d Et$に打ち勝って仕事をする必要がある。

→\uline{磁気エネルギー}が貯まるはず。

図電後5-1

コイルを流れる電流を$i$

そのとき、コイルを鎖交する磁束を$\Phi$とする。微小時間$\mathrm{d}x$の間に、$i$を$\mathrm{d}i$だけ増やした時、$\Phi$が$\mathrm{d}\Phi$だけ増えたとする。

この間、\ref{3.3}より$V=\d{\Phi}t$の電圧が端子間に発生するので、$\mathrm{d}t$の間にした仕事量は、
\begin{equation}
\mathrm{d}W_{m}=iV\mathrm{d}t=i\mathrm{d}\Phi\label{3.13}
\end{equation}

($m$→magnetic)

今、コイルが1つで、\ref{3.7}のように、$\Phi=Li$と表されるとき、\ref{3.13}は、
\[
\mathrm{d}W_{m}=Li\mathrm{d}i
\]

図電後5-2

∴$i$を0から$I$まで増大させるには、
\[
W_{m}=\int\mathrm{d}W_{m}=L\int_{0}^{I}i\mathrm{d}i=\frac{1}{2}LI^{2}
\]
の仕事が必要(上図の三角の面積)

つまりコイル蓄えられたエネルギーは、
\begin{equation}
W_{m}=\frac{1}{2}LI^{2}\label{3.14}
\end{equation}

図電後5-3

コイルが2個ある時、\ref{3.13}

\begin{equation}
\mathrm{d}W_{j}=i_{j}\mathrm{d}\Phi_{j}\label{3.15}
\end{equation}
が\uline{コイルごとに}成り立つ。

コイル1,2に$i_{1},i_{2}$が流れている状態で、$i_{1}$を$\mathrm{d}i_{1}$だけ変化させると、
\[
\begin{cases}
\mathrm{d}\Phi_{1}=L_{1}\mathrm{d}i_{1}\\
\mathrm{d}\Phi_{2}=M\mathrm{d}i_{1}
\end{cases}
\]
変わるので、\ref{3.15}の合計は、
\begin{align*}
\mathrm{d}W_{1} & =\mathrm{d}w_{1}+\mathrm{d}w_{2}\\
 & =L_{1}i_{1}\mathrm{d}i_{1}+Mi_{1}\mathrm{d}i_{1}
\end{align*}

同様に、$i_{2}$を$\mathrm{d}i_{2}$だけ変化させたときの仕事量は、
\[
\mathrm{d}W_{2}=Mi_{1}\mathrm{d}i_{2}+L_{2}i_{2}\mathrm{d}i_{2}
\]

∴$\mathrm{d}t$の間に$\left[\mathrm{d}i_{1},\mathrm{d}_{2}\right]$だけ変化させたときの合計の仕事率は、
\begin{align*}
\d Wt & =\d{W_{1}}t+\d{W_{2}}t\\
 & =L_{1}i_{1}\d{i_{1}}t+L_{2}i_{2}\d{i_{2}}t+M\left(i_{1}\d{i_{2}}t+\d{i_{1}}ti_{2}\right)\\
 & =L_{1}i_{1}\d{i_{1}}t+L_{2}i_{2}\d{i_{2}}t+M\d{\left(i_{1}i_{2}\right)}t
\end{align*}

∴$i_{1},i_{2}$を$I_{1},I_{2}$まで増加させるのに要する仕事量は、
\begin{align*}
W_{m} & =\int\d Wt\mathrm{d}t=\int_{0}^{I_{1}}L_{1}i_{1}\mathrm{d}i_{1}+\int_{0}^{I_{2}}L_{2}i_{2}\mathrm{d}i_{2}+\int_{0}^{I_{1}I_{2}}M\mathrm{d}\left(i_{1}i_{2}\right)\\
 & =\frac{1}{2}L_{1}I_{1}^{2}+\frac{1}{2}L_{2}I_{2}^{2}+MI_{1}I_{2}\\
 & =\frac{1}{2}\sum_{i=1}^{2}\sum_{j=1}^{2}M_{ij}I_{i}I_{j}
\end{align*}

ただし、$L_{1}=M_{11},L_{2}=M_{22},M=M_{12}=M_{21}$

一般に、$n$個のコイル(ループ)があるとき、合計の磁気エネルギーは
\begin{equation}
W_{m}=\frac{1}{2}\sum_{i=1}^{n}\sum_{j=1}^{n}M_{ij}I_{i}I_{j}\label{3.15-1}
\end{equation}

ただし、$M_{ij}$は、
\begin{itemize}
\item $i=j$のとき、自己インダクタンス$L_{i}$ 
\item $i\neq j$のとき、相互インダクタンス
\end{itemize}
このとき、各ループの鎖交磁束は、
\begin{equation}
\Phi_{i}=\sum_{j=1}^{N}M_{ij}I_{j}\label{3.16}
\end{equation}
なので、\ref{3.15-1}は、
\begin{equation}
W_{m}=\frac{1}{2}\sum_{i=1}^{n}I_{i}\Phi_{i}\label{3.17}
\end{equation}
とも書ける。

\subsection{磁界のエネルギー密度}

\ref{3.17}は電流が流れる``コイル''それぞれにエネルギーが溜まるという解釈

→電流により周囲に発生した``磁界''にエネルギーが蓄えられていると考えることができる。(参: 静電エネルギー$\frac{1}{2}\boldsymbol{E}\cdot\boldsymbol{D}$と同じ考え方)

図電後5-4

自由電流ループ$i_{F}$があるとき、全空間を$\boldsymbol{B}$に沿ってパイプ感情の領域に分割し、角パイプを長さ方向に長さ$\Delta l_{i}\left(i=1,2,3\cdots\right)$に分割する。

今、微小時間$\mathrm{d}t$の間に、$i_{p}$を$\mathrm{d}i_{p}$だけ増やし、微小断面$\Delta S$
を貫く磁束を$\mathrm{d}\phi$だけ増加させる。この時した仕事は\ref{3.13}より、
\begin{equation}
\mathrm{d}W_{m}=i_{F}\mathrm{d}\phi=i_{p}\mathrm{d}B\Delta S\label{3.18}
\end{equation}

($\mathrm{d}\phi=\mathrm{d}B\Delta S$)

アンペアの法則\ref{2.8}より、
\[
i_{F}=\sum_{i=1}^{\text{一周}}\boldsymbol{H}\cdot\Delta\boldsymbol{l}_{i}
\]

∴\ref{3.18}は、
\[
\mathrm{d}W_{m}=\sum_{i}\boldsymbol{H}\cdot\Delta\boldsymbol{l}_{i}\mathrm{d}B\Delta S=\sum_{i}\boldsymbol{H}\cdot\mathrm{d}\boldsymbol{B}\Delta v_{i}
\]

(∵$\mathrm{d}\boldsymbol{B}\sslash\Delta\boldsymbol{l}_{i},\Delta v_{i}\equiv\left|\Delta\boldsymbol{l}_{i}\right|\Delta S$)

∴磁束密度を$0\rightarrow\boldsymbol{B}$にするには、
\begin{equation}
W_{m}=\int_{\text{全空間}}\left(\int_{0}^{B}\boldsymbol{H}\cdot\mathrm{d}\boldsymbol{B}\right)\mathrm{d}v\label{3.19}
\end{equation}
の仕事が必要。

つまり、各店において単位体積あたり
\begin{equation}
w_{m}=\int_{0}^{\boldsymbol{B}}\boldsymbol{H}\cdot\boldsymbol{B}\label{3.20}
\end{equation}
のエネルギーが蓄えられていると解釈できる。

線形な磁性体では、$\boldsymbol{H}=\frac{1}{\mu}\boldsymbol{B}$(比例)なので、
\[
w_{m}=\frac{1}{\mu}\int_{0}^{\boldsymbol{B}}\boldsymbol{B}\cdot\mathrm{d}\boldsymbol{B}
\]

\begin{equation}
w_{m}=\frac{1}{2\mu}B^{2}=\frac{1}{2}\mu H^{2}=\frac{1}{2}\boldsymbol{H}\cdot\boldsymbol{B}\label{3.21}
\end{equation}

特に真空中では
\begin{equation}
w_{m}=\frac{1}{2\mu_{0}}B^{2}\label{3.22}
\end{equation}

強磁性体を磁化する際に必要な仕事は、B-H特性曲線で$\int\boldsymbol{H}\cdot\mathrm{d}\boldsymbol{B}$を計算することで求まる。

図電後5-5

上図のように、ヒステリシスがあるときは、$\int\boldsymbol{H}\cdot\mathrm{d}\boldsymbol{B}$を1サイクル積分してもゼロにならない。

つまり、ヒストリシス内の面積分のエネルギーが失われて\uline{発熱}する。(ヒステリシス損失)

∵そもそもヒステリシスが散財するのは、磁区間の摩擦→発熱

\subsection{磁性体に働く力}

誘電体と同様に、仮想変位の考え方で力を求めることができる。

図電後5-6

$x$郷校の力は
\begin{equation}
F_{x}=-\left(\pd{W_{m}}x\right)_{\Phi\text{一定}}\label{3.23}
\end{equation}

$F_{x}$は、$x$を増やす向きが正。

($\Phi$が一定であれば、起電力$V=\d{\Phi}t=0$なので、外部から仕事をする必要がない)

\paragraph{電流$I$=一定だとどうなるか?}

$I$一定で動かすと、一般に$\Phi$が変化し、$V=\d{\Phi}t$が発生するので、$I$を一定に保つために、\ref{3.13}より、$\mathrm{d}W=I\mathrm{d}\Phi$の仕事を外部回路がする必要がある。

$F_{x}$(+x方向)に対抗して、$-F_{x}$の力を借りながら$+\Delta x$動かす時、
\[
-F_{x}\Delta x+I\Delta\Phi=\Delta W_{m}
\]

\[
F_{x}=-\when{\pd{\left(W_{m}-I\Phi\right)}x}_{I=\text{一定}}
\]

\ref{3.17}より$W_{m}=\frac{1}{2}I\Phi$より$I\Phi=2W_{m}$なので、
\begin{equation}
F_{x}=-\left(\pd{W_{m}}x\right)_{I=\text{一定}}\label{3.23'}
\end{equation}


\subsection{インダクタンスの計算}

有限の太さの電線等の自己インダクタンスを計算するには、断面内の電流分布を知っておく必要がある。

通常は次のどちらかを仮定する。
\begin{enumerate}
\item 断面内で一様の密度で流れる。
\item 導体の\uline{表面のみ}一様に流れる。(表皮効果)
\end{enumerate}
直流や低周波では1.になるが、高周波では2.の仮定が妥当になる。

\paragraph{表皮効果 (skin effect)}

図電後5-7

導体内の電流密度$J_{e}\left(t\right)$が増加

→周囲の$B\left(t\right)$も増加

→$B$の増加を妨げる向きに起電力が生じ、誘導電流$J_{i}\left(t\right)$が流れる。

→導体の中央では$J$が弱められ、外側に押しやられる。

この起電力$\d Bt$は周波数が高いほど大きいので、高周波で顕著。

電流が流れる厚み(表皮厚)は、
\begin{equation}
\delta=\sqrt{\frac{2}{\omega\sigma\mu}}\label{3.24}
\end{equation}
となる。(導出は教p.158)

$\omega$: 角周波数

$\sigma$: 導電率

\paragraph{電線インダクタンス}

図電後5-8

半径$a\left(>0\right)$の無限に長い電線を考える。

電流を流すには無限遠にループをなしており、このループに対して自己インダクタンスが定義される。

電流が導体内を一様に流れているとき、(1.の場合)は、導体外($r>a$)の$B$に加えて、\uline{導体内($r\leqq a$)にも$B$が発生し、``一部''の電流と鎖交する}ので、これも自己インダクタンス成分になる。

このように、導体内の磁束によるインダクタンス成分を``内部インダクタンス''と呼ぶ。(演習1)

電流が表面を流れるときは、導体内で$B=0$なので、内部インダクタンスもゼロ。

\paragraph{同軸ケーブル (来週の宿題)}

図電後5-9

\paragraph{先週の宿題}

解1

図電後5-10

速度起電力で考える。

中心から$r$の点では、$v=r\omega$で動いているので、\ref{3.1}より径方向に単位長さあたり
\[
\varepsilon\left(r\right)=v\times B=r\omega B
\]
の起電力を受ける。

\[
V=\int_{0}^{a}\varepsilon\left(r\right)\mathrm{d}r=\frac{1}{2}a^{2}\omega B
\]

解2

$\d{\Phi}t$で考える。

図電後5-11

経路を鎖交する磁束は、$\Delta\phi=\frac{1}{2}a^{2}\omega B\Delta t$だけ増える。

\[
\therefore V=\frac{\Delta\Phi}{\Delta t}=\frac{1}{2}a^{2}\omega B
\]


\paragraph{先週の宿題2}

図電後5-12

(a) 断面積が一定なので経路内で$B$は一定($\nabla\cdot B=0$)

$H$は、

\[
\begin{cases}
H_{m}=\frac{B}{\mu} & \left(\text{磁性体内}\right)\\
H_{g}=\frac{M}{\mu_{0}} & \left(\text{ギャップ}\right)
\end{cases}
\]

∴$\Phi_{11}=L_{1}I$より、

\[
L_{1}=\frac{N_{1}^{2}S}{\frac{2\pi R-S}{\mu}+\frac{\delta}{\mu_{0}}}
\]

(b) $B$により、コイル2を鎖交する磁束は$\Phi_{12}=N_{2}BS$

\[
\therefore M=\frac{\Phi_{12}}{I}=\frac{N_{1}N_{2}S}{\frac{2\pi R-S}{\mu}+\frac{\delta}{\mu_{0}}}
\]

アンペアの法則より
\[
H_{m}\left(2\pi R-\delta\right)+H_{g}\delta=N_{1}I
\]

\[
B=\frac{N_{1}I}{\frac{2\pi R-S}{\mu}+\frac{\delta}{\mu_{0}}}
\]

これによりコイル1を鎖交する磁束は
\[
\Phi_{11}=N_{1}BS=\frac{N_{1}^{2}SI}{\frac{2\pi R-S}{\mu}+\frac{\delta}{\mu_{0}}}
\]


\paragraph{先週の宿題3}

図電後5-13

$C_{2}$に電流$I_{2}$が流れているとき、これが、$C_{1}$上につくるベクトルポテンシャル$\boldsymbol{A}_{2}$は、\ref{1.11}より、
\[
\boldsymbol{A}=\frac{\mu_{0}I_{2}}{4\pi}\oint_{C_{2}}\frac{\mathrm{d}\boldsymbol{s}_{2}}{r}
\]

一方、$C_{1}$を貫く磁束は、\ref{1.11}より、
\[
\Phi_{11}=\oint_{C_{1}}A\cdot\mathrm{d}\boldsymbol{s}_{1}=\frac{\mu_{0}I_{2}}{4\pi}\oint_{C_{1}}\oint_{C_{2}}\frac{\mathrm{d}\boldsymbol{s}_{1}\cdot\mathrm{d}\boldsymbol{s}_{2}}{r}
\]

\[
\therefore M=\frac{\Phi_{12}}{I_{2}}=\frac{\mu_{0}}{4\pi}\oint_{C_{1}}\oint_{C_{2}}\frac{\mathrm{d}\boldsymbol{s}_{1}\cdot\mathrm{d}\boldsymbol{s}_{2}}{r}
\]

$C_{1}$と$C_{2}$に対して対称なので、$M_{12}=M_{21}$

\paragraph{1}

図電後5-14

(a)

(i) $r<a$のとき

アンペアの法則
\[
H\cdot2\pi r=\frac{r^{2}}{a^{2}}I
\]

\[
\therefore H=\frac{rI}{2\pi a^{2}},B=\mu H=\frac{\mu rI}{2\pi a^{2}}
\]

(ii) $r>a$のとき

\[
H\cdot2\pi r=I
\]

\[
H=\frac{I}{2\pi r},B=\frac{\mu_{0}I}{2\pi r}
\]

(b)

$\mathrm{d}r$を横切る磁束は
\[
\mathrm{d}\phi=B\mathrm{d}r=\frac{\mu rI}{2\pi a^{2}}\mathrm{d}r
\]

$\mathrm{d}\phi$が鎖交するのは、$I$ではなく$\frac{r^{2}}{a^{2}}I$

∴等価的に巻き数$N=\frac{r^{2}}{a^{2}}\left(<1\right)$

\[
\therefore\mathrm{d}\Phi=N\mathrm{d}\phi=\frac{\mu r^{3}I}{2\pi a^{4}}\mathrm{d}r
\]

(c) 全鎖交磁束は$\Phi=\int\mathrm{d}\Phi=\int_{0}^{a}\frac{\mu r^{3}I}{2\pi a^{4}}\mathrm{d}r=\frac{\mu}{8\pi}I$

(d) $\Phi=l_{i}I$より、
\[
l_{i}=\frac{\mu}{8\pi}
\]

($a$によらない)

(e) 導体内部の磁気エネルギーは、
\[
W_{m}=\int_{0}^{a}\frac{1}{2}HB\cdot2\pi r\mathrm{d}r=\frac{\mu I^{2}}{4\pi a^{4}}\int_{0}^{a}r^{3}\mathrm{d}r=\frac{\mu I^{2}}{16\pi}
\]

($\frac{1}{2}l_{i}I^{2}$と一致する)

\paragraph{2}

(a) 断面積$S$が一定なので$B$が一定。

鉄芯内で$H_{m}=\frac{B}{\mu}\simeq0$ ($\mu\rightarrow\infty$)

ギャップでは$H_{g}=\frac{B}{\mu_{0}}$

アンペアより$2gH_{g}=NI$

\[
\therefore H_{g}=\frac{NI}{2g}
\]

\[
B=\frac{\mu_{0}NI}{2g}
\]

(b) 鎖交磁束は、
\[
\Phi=NSB=\frac{\mu_{0}N^{2}SI}{2g}
\]

\[
\therefore L=\frac{\Phi}{I}=\frac{\mu_{0}N^{2}S}{2g}
\]

(c)
\[
W=\frac{1}{2}LI^{2}=\frac{\mu_{0}N^{2}SI^{2}}{4g}
\]

(d) $I=\text{一定}$の式を使うと、
\[
F=+\left(\pd Wg\right)_{I=\text{一定}}=-\frac{\mu_{0}N^{2}SI^{2}}{4g^{2}}
\]

$g$を増やす向きが正なので、引き合う向きに力が働く。

別解、$\Phi$一定で解く

\[
I=\frac{\Phi}{L}=\frac{2g}{\mu_{0}N^{2}S}\Phi
\]
を(c)の結果に代入して
\[
W=\frac{g\Phi^{2}}{\mu_{0}N^{2}S}
\]

\[
F=-\left(\pd Wg\right)_{\Phi\text{一定}}=-\frac{\Phi^{2}}{\mu_{0}N^{2}S}=-\frac{\mu_{0}N^{2}SI^{2}}{4g^{2}}
\]

(e)

図電後5-15

単位面積あたり$f=\frac{1}{2}\mu_{0}H^{2}$

\[
F=-f2S=\mu_{0}H_{g}^{2}S=\frac{\mu_{0}N^{2}SI^{2}}{4g^{2}}
\]

\end{document}
