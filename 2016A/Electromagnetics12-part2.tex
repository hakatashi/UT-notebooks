%% LyX 2.2.2 created this file.  For more info, see http://www.lyx.org/.
%% Do not edit unless you really know what you are doing.
\documentclass[english]{article}
\usepackage[T1]{fontenc}
\usepackage[utf8]{inputenc}
\usepackage[a5paper]{geometry}
\geometry{verbose,tmargin=2cm,bmargin=2cm,lmargin=1cm,rmargin=1cm}
\setlength{\parskip}{\smallskipamount}
\setlength{\parindent}{0pt}
\usepackage{textcomp}
\usepackage{amsmath}
\usepackage{amssymb}
\usepackage{esint}
\PassOptionsToPackage{normalem}{ulem}
\usepackage{ulem}

\makeatletter
%%%%%%%%%%%%%%%%%%%%%%%%%%%%%% User specified LaTeX commands.
\usepackage[dvipdfmx]{hyperref}
\usepackage[dvipdfmx]{pxjahyper}

\makeatother

\usepackage{babel}
\begin{document}

\title{2016-A 電気磁気学 後半}

\author{教員: 種村 入力: 高橋光輝}

\maketitle
\global\long\def\pd#1#2{\frac{\partial#1}{\partial#2}}
\global\long\def\d#1#2{\frac{\mathrm{d}#1}{\mathrm{d}#2}}
\global\long\def\pdd#1#2{\frac{\partial^{2}#1}{\partial#2^{2}}}
\global\long\def\dd#1#2{\frac{\mathrm{d}^{2}#1}{\mathrm{d}#2^{2}}}
\global\long\def\ddd#1#2{\frac{\mathrm{d}^{3}#1}{\mathrm{d}#2^{3}}}
\global\long\def\e{\mathrm{e}}
\global\long\def\i{\mathrm{i}}
\global\long\def\j{\mathrm{j}}
\global\long\def\grad{\operatorname{grad}}
\global\long\def\rot{\operatorname{rot}}
\global\long\def\div{\operatorname{div}}
\global\long\def\diag{\operatorname{diag}}
\global\long\def\rank{\operatorname{rank}}
\global\long\def\prob{\operatorname{Prob}}
\global\long\def\cov{\operatorname{Cov}}
\global\long\def\when#1{\left.#1\right|}


\section*{第1回}

\paragraph{成績}
\begin{itemize}
\item 出席は取らない
\item 必修
\end{itemize}

\section*{0 イントロ}

電磁気学を完全に記述する方程式 Maxwell's Equations

\begin{align*}
\nabla\cdot\boldsymbol{E} & =\frac{\rho}{\varepsilon_{0}}\:\left(\text{ガウス}\right)\\
\nabla\times\boldsymbol{E} & =-\pd{\boldsymbol{B}}t\:\left(\text{ファラデー}\right)\\
\nabla\cdot\boldsymbol{B} & =0\:\left(\text{磁荷の非存在}\right)\\
\nabla\times\boldsymbol{B} & =\mu_{0}\varepsilon_{0}\pd{\boldsymbol{E}}t+\mu_{0}\boldsymbol{i}\:\left(\text{アンペア}\right)
\end{align*}

静電磁気学$\left(\pd{}t=0\right)$では、

図電後1-1

つまり時間不変の上等では``電気''と``磁気''は別個の現象(のように見える)。

本当に面白いのは$\pd{\boldsymbol{E}}t,\pd{\boldsymbol{B}}t$の項。これにより、電磁誘導→電磁波(光)が記述される。

\paragraph{後半}
\begin{enumerate}
\item $\boldsymbol{B}$って何?
\item (※)の式 $\left(\pd{}t=0\right)$
\item 磁性体
\item $\pd{}t$の項
\end{enumerate}

\section{磁界}

\subsection{アンペール力}

\paragraph{クーロン力}

\uline{静止した}2つの電荷間に働く力

図電後1-2

\[
F=\frac{1}{4\pi\varepsilon_{0}}\frac{Q_{1}Q_{2}}{r^{2}}
\]

電荷が動いているときは、その\uline{速さと向き}に依存した力を発生する。→アンペール力(Ampere)

\paragraph{アンペール力 (アンペール 1820年)}

距離$R$離れた2本の平行な導線に電流$I_{1},I_{2}$が流れているとき、電荷が動いているとき、
\begin{equation}
F=\frac{\mu_{0}}{2\pi}\frac{I_{1}I_{2}}{R}
\end{equation}
の力が働く。

ただし、$I_{1},I_{2}$が同じ向きなら引力、逆向きなら斥力。

導線が直交してたら力は働かない。

\[
\mu_{0}=4\pi\times10^{-7}\left[\mathrm{N\cdot m/A^{2}}\right]
\]

透磁率という。

\paragraph{なぜきれいな値なのか?}

→実は(1)により電流の単位が定義されている。

1Aとは「真空中で1m離れた平衡な直線導体に流れた時に、単位長さあたり$2\times10^{-7}\mathrm{N}$のちからが働くような電流」と定義される(SI単位系)。

→その結果、1Cが定義されて、$\varepsilon_{0}$が求まる。

\paragraph{復習}

帯電下電線管に働くクーロン力は、単位長あたり、
\begin{equation}
F=\frac{1}{2\pi\varepsilon_{0}}\frac{\lambda_{1}\lambda_{2}}{R}
\end{equation}

∵ガウスの定理

図電後1-3

クーロン力(2)と同様に、直線導体間のアンペール力(1)は$R$に反比例する。

→クーロン力(遠隔作用)が電界(近接作用)を用いて言い換えられたように、アンペール力も同様の考え方で説明できるはず。

つまり、
\begin{enumerate}
\item 導線1の微小区間$\delta\boldsymbol{s}_{1}$を流れる電流$I_{1}$(電流要素$I\mathrm{d\boldsymbol{s}_{1}}$)が周囲に、なんらかの``場''(これを磁場または磁界と呼ぶ)を作る。
\item 導線2の微小区間$\mathrm{d}\boldsymbol{s}_{2}$を流れる電流$I_{2}$(電流要素$I_{2}\mathrm{d}\boldsymbol{s}_{2}$)がその場を感じて力を受ける。
\end{enumerate}
図電後1-4

「電流要素」→「電荷」に置き換えるとクーロン力

\subsection{ビオ・サバールの法則}

位置$\boldsymbol{r}'$にある電流要素$I'\mathrm{d}\boldsymbol{s}'$が、位置$\boldsymbol{r}$に
\begin{equation}
\mathrm{d}\boldsymbol{B}=\frac{\mu_{0}}{4\pi}\frac{I'\mathrm{d}\boldsymbol{s}\times\boldsymbol{u}}{\left|\boldsymbol{r}-\boldsymbol{r}'\right|^{2}}
\end{equation}
で表される微小な``場''を作る。

図電後1-5

\[
\boldsymbol{u}\equiv\frac{\boldsymbol{r}-\boldsymbol{r}'}{\left|\boldsymbol{r}-\boldsymbol{r}'\right|}
\]

$\boldsymbol{r}-\boldsymbol{r}'$方向の単位ベクトル

($I$は$\mathrm{d}\boldsymbol{s}$の向きを性とした電流)

電線全体が作る場は
\[
\boldsymbol{B}=\int\mathrm{d}\boldsymbol{B}
\]

\begin{equation}
\therefore\boldsymbol{B}=\frac{\mu_{0}}{4\pi}\oint\frac{I'\mathrm{d}s'\times\left(\boldsymbol{r}-\boldsymbol{r}'\right)}{\left|\boldsymbol{r}-\boldsymbol{r'}\right|^{3}}
\end{equation}

図電後1-6

そしてこの時、位置$\boldsymbol{r}$にある電流要素$I\mathrm{d}\boldsymbol{s}$は、
\begin{equation}
\boldsymbol{F}=I\mathrm{d}\boldsymbol{s}\times\boldsymbol{B}
\end{equation}
の力を受ける。

図電後1-7

すべての$I\mathrm{d}\boldsymbol{s}$において積分すれば導線全体が受ける力が求まる。

ここで``場''$\boldsymbol{B}$を磁束密度と呼び、単位T(テスラ)を用いる。(このあとで出てくる$\boldsymbol{H}$を磁界と呼ぶため)

直線電流にとうして(4)、(5)を用いて計算すると(1)が得られる。(演習問題)

\paragraph{ローレンツ力}

もともと電流は電荷の流れ。

\[
I\mathrm{d}\boldsymbol{s}-qN\boldsymbol{v}
\]

$q$: でん過疎量

$N$: $\mathrm{d}\boldsymbol{s}$無いの電荷数

$v$: 電荷の速度

を用いると、(5)は
\[
\boldsymbol{F}=qN\boldsymbol{v}\times\boldsymbol{B}
\]

∴電荷1個あたり$\boldsymbol{F}=q\boldsymbol{v}\times\boldsymbol{B}$のアンペール力を受けるとも言える。

クーロン力と合わせて一般に電荷が受ける力
\begin{equation}
\boldsymbol{F}=q\left(\boldsymbol{E}+\boldsymbol{v}\times\boldsymbol{B}\right)
\end{equation}
をローレンツ力と呼ぶ。

\paragraph{電磁場の相対性}

静止した電荷間の力がクーロン力(電界)\uline{動いている}電荷間の力がアンペール力(磁界)と区別したが、「動いているかどうか?」は相対的なもの。

例えば、電荷と一緒に動いている系絡みたら、(6)の$\boldsymbol{v}=0$になり、アンペール力は発生しない。おかしい?

答え: このとき、相対論的に長さが縮むことに起因して\uline{クーロン力}が余分に発生し、合計の$\boldsymbol{F}$がつじつまが合う。(配布資料)

クーロンの法則+相対論→ビオ・サバールの法則が導出できる。(太田浩一『電磁気学の基礎II』)

\paragraph{TAオフィスアワーのご案内}

数学演習と電磁気学でわからないことを聞きに行ける制度
\begin{itemize}
\item 場所: 駒場IIキャンパス 先端研3号館308号室
\item 日程: 11/24, 12/1, 12/8, 12/15, 12/22 (毎週木曜)
\item 時間: 18:30~19:30
\end{itemize}

\paragraph{演習問題1(a)}

図電後1-8

微小区間$\left[x+x+\mathrm{d}x\right]$の電流要素$I\mathrm{d}x$が点$P$に作る磁界密度を考える。

\[
\left|\boldsymbol{r}-\boldsymbol{r}'\right|\equiv r=\frac{d}{\sin\theta}
\]

$x=-\frac{d}{\tan\theta}$より$\d x{\theta}=\frac{d}{\sin^{2}\theta}$

\[
\mathrm{}\boldsymbol{s}\times\left(\boldsymbol{r}-\boldsymbol{r}'\right)=\left(r\mathrm{d}x\sin\theta\right)
\]

∴(4)より、
\begin{align*}
B & =\frac{\mu_{0}}{4\pi}\int_{x_{1}}^{x_{2}}\frac{I\sin x}{r^{2}}\mathrm{d}x\\
 & =\frac{\mu_{0}I}{4\pi}\int_{\alpha}^{\beta}\frac{\sin^{3}x}{d^{2}}\frac{d}{\sin^{2}\theta}\mathrm{d}\theta\\
 & =\frac{\mu_{0}I}{4\pi d}\int_{\alpha}^{\beta}\sin\theta\mathrm{d}\theta\\
 & =\frac{\mu_{0}I}{4\pi d}\left(\cos\alpha-\cos\beta\right)
\end{align*}

向きは+z方向。

\paragraph{(b)}

$\alpha=0,\beta=\pi$

$I=I_{1},d=R$を代入して、
\[
B=\frac{\mu_{0}I_{1}}{2\pi R}
\]

(5)より単位長さあたり($\left|\mathrm{d}\boldsymbol{s}\right|=1$)

\[
F=I_{2}B=\frac{\mu_{0}I_{1}I_{2}}{2\pi R}
\]


\paragraph{(c)}

図電後1-9

まず、線分$QR$を流れる電流g点$P$に作る磁束密度成分$\boldsymbol{B}_{1}$を考える。

$\Delta PQR$に(a)の結果を適用

図電後1-10

\[
\cos\alpha=-\cos\beta=\frac{\overline{O'Q}}{\overline{PR}}=\frac{a}{\sqrt{a^{2}+b^{2}+x^{2}}}
\]

\[
d=\overline{PO'}=\sqrt{b^{2}+x^{2}}
\]

(a)の結果より、
\[
B_{1}=\frac{\mu_{0}I}{2\pi\sqrt{b^{2}+x^{2}}}\frac{a}{\sqrt{a^{2}+b^{2}+x^{2}}}
\]

$ST$が作る磁場を加えると、残るのは$OP$方向のみなので、その成分だけを考えると、
\[
B_{1\bot}=B_{1}\frac{b}{\sqrt{b^{2}+x^{2}}}=\frac{\mu_{0}Iab}{2\pi\left(b^{2}+x^{2}\right)\sqrt{a^{2}+b^{2}+x^{2}}}
\]

同様に、$RS$が作る磁場は、
\[
B_{2\bot}=\frac{\mu_{0}Iab}{2\pi\left(a^{2}+x^{2}\right)\sqrt{a^{2}+b^{2}+x^{2}}}
\]

∴合計
\[
B=2B_{1\bot}+2B_{2\bot}=\frac{\mu_{0}Iab}{\pi\sqrt{a^{2}+b^{2}+x^{2}}}\left(\frac{1}{a^{2}+x^{2}}+\frac{1}{b^{2}+x^{2}}\right)
\]


\paragraph{演習問題2}

図電後1-11

まず下の円環が作る磁場を考える。

周方向に角度$\theta$をとる。

微小電流要素$I\mathrm{d}s\left(=Ia\mathrm{d}\theta\right)$が点$P$につくる$\mathrm{d}\boldsymbol{B}$は、
\[
\mathrm{d}B=\frac{\mu_{0}}{4\pi}\frac{Ia\mathrm{d}\theta}{a^{2}+h^{2}}
\]

ただし$h\equiv d+z$

図電後1-12

円環一周がつくる磁場は$z$成分以外は打ち消し合うので$z$成分のみを考える。

\[
\mathrm{d}B_{z}=\mathrm{d}B\cdot\frac{a}{\sqrt{a^{2}+h^{2}}}=\frac{\mu_{0}}{4\pi}\frac{Ia^{2}}{\left(a^{2}+h^{2}\right)}\mathrm{d}\theta
\]

∴円環全体で
\[
B_{z}=\int_{0}^{2\pi}\frac{\mu_{0}}{4\pi}\frac{Ia^{2}}{\left(a^{2}+b^{2}\right)^{\frac{3}{2}}}\mathrm{d}\theta=\frac{\mu_{0}Ia^{2}}{2\left(a^{2}+b^{2}\right)^{\frac{3}{2}}}
\]

2つの合計は、
\[
B=-\frac{\mu_{0}Ia^{2}}{2}\left[\frac{1}{\left\{ a^{2}+\left(d+z\right)^{2}\right\} ^{\frac{3}{2}}}+\frac{1}{\left\{ a^{2}+\left(d-z\right)^{2}\right\} ^{\frac{3}{2}}}\right]
\]

図電後1-13

\section*{第2回}

\subsection{ベクトルポテンシャル}

図電後2-1

点電荷→電荷密度と一般化したのと同様に、まず電流→電流密度で考え直す。

図電後2-2

$\boldsymbol{r}'$における電流要素は、
\[
I'\mathrm{d}s'=\boldsymbol{i}\left(\boldsymbol{r}'\right)\mathrm{d}A'\mathrm{d}s'=\boldsymbol{i}\left(\boldsymbol{r}'\right)\mathrm{d}v'
\]

$\mathrm{d}v'\equiv\mathrm{d}A'\mathrm{d}s'$: 体積要素

∴ビオ・サバール

(1.4) →
\begin{equation}
\boldsymbol{B}\left(\boldsymbol{r}\right)=\frac{\mu_{0}}{4\pi}\int_{v}\frac{\boldsymbol{i}\left(\boldsymbol{r}'\right)\times\left(\boldsymbol{r}-\boldsymbol{r}'\right)}{\left|\boldsymbol{r}-\boldsymbol{r}'\right|^{3}}\mathrm{d}v'
\end{equation}

ここで、
\begin{equation}
\nabla\left(\frac{1}{\left|\boldsymbol{r}-\boldsymbol{r}'\right|}\right)=-\frac{\boldsymbol{r}-\boldsymbol{r}'}{\left|\boldsymbol{r}-\boldsymbol{r}'\right|^{3}}
\end{equation}

ただし、$\nabla$は$\boldsymbol{r}$に対する演算子
\[
\nabla=\left(\begin{array}{c}
\pd{}x\\
\pd{}y\\
\pd{}z
\end{array}\right)
\]

(1.8)の証明は演習でやる。

(1.7)→
\begin{align*}
\boldsymbol{B}\left(\boldsymbol{r}\right) & =\frac{\mu_{0}}{4\pi}\int\boldsymbol{i}\left(\boldsymbol{r}\right)\times\left\{ -\nabla\left(\frac{1}{\left|\boldsymbol{r}-\boldsymbol{r}'\right|}\right)\right\} \mathrm{d}v'\\
 & =\frac{\mu_{0}}{4\pi}\int_{V}\left\{ \nabla\left(\frac{1}{\left|\boldsymbol{r}-\boldsymbol{r}'\right|}\right)\right\} \times\boldsymbol{i}\left(\boldsymbol{r}'\right)\mathrm{d}v'
\end{align*}

ベクトル公式 ($f$: 任意のスカラー関数)
\[
\nabla\times\left(f\boldsymbol{i}\right)=\left(\nabla f\right)\times\boldsymbol{i}+f\left(\nabla\times\boldsymbol{i}\right)
\]
より
\[
\left(\nabla f\right)\times\boldsymbol{i}=\nabla\times\left(f\boldsymbol{i}\right)-f\left(\nabla\times\boldsymbol{i}\right)
\]
を用いると、
\begin{equation}
\boldsymbol{B}\left(\boldsymbol{r}\right)=\frac{\mu_{0}}{4\pi}\int_{V}\mathrm{d}v'\left[\nabla\times\frac{\boldsymbol{i}\left(\boldsymbol{r}'\right)}{\left|\boldsymbol{r}-\boldsymbol{r}'\right|}-\frac{1}{\left|\boldsymbol{r}-\boldsymbol{r}'\right|}\left\{ \nabla\times\boldsymbol{i}\left(\boldsymbol{r}'\right)\right\} \right]
\end{equation}

ここで、$\nabla$は$\boldsymbol{r}$に対する微分なので、
\[
\nabla\times\boldsymbol{i}\left(\boldsymbol{r}'\right)=0
\]

また、$\nabla$は$\int\mathrm{d}v'$の外に出せる。

(1.7)→
\begin{equation}
\boldsymbol{B}\left(\boldsymbol{r}\right)=\nabla\times\frac{\mu_{0}}{4\pi}\int_{V}\frac{\boldsymbol{i}\left(\boldsymbol{r}'\right)}{\left|\boldsymbol{r}-\boldsymbol{r}'\right|}\mathrm{d}v'
\end{equation}

そこで、ベクトル
\begin{equation}
\boldsymbol{A}\left(\boldsymbol{r}\right)\equiv\frac{\mu_{0}}{4\pi}\int_{V}\frac{\boldsymbol{i}\left(\boldsymbol{r}'\right)}{\left|\boldsymbol{r}-\boldsymbol{r}'\right|}\mathrm{d}v'
\end{equation}
を定義すると、
\begin{equation}
\boldsymbol{B}\left(\boldsymbol{r}\right)=\nabla\times\boldsymbol{A}\left(\boldsymbol{r}\right)
\end{equation}
と表される。

\paragraph{クーロン力の場合}

\begin{align}
V\left(\boldsymbol{r}\right) & =\frac{1}{4\pi\varepsilon_{0}}\int\frac{\rho\left(\boldsymbol{r}'\right)}{\left|\boldsymbol{r}-\boldsymbol{r}'\right|}\mathrm{d}\boldsymbol{r}'\\
\boldsymbol{E} & =-\nabla V
\end{align}

\begin{itemize}
\item (1.13)↔(1.11)
\item (1.14)↔(1.12)
\end{itemize}
に対応している。

そこで、(1.11)の$\boldsymbol{A}$をベクトルポテンシャルと呼ぶ。

(一部紛失)

\begin{equation}
?
\end{equation}

\begin{equation}
\boldsymbol{i}\left(\boldsymbol{r}\right)=\frac{1}{4\pi\varepsilon_{0}}\int_{V}\frac{\rho\left(\boldsymbol{r}'\right)}{\left|\boldsymbol{r}-\boldsymbol{r}'\right|}\mathrm{d}v'=-\frac{\rho\left(\boldsymbol{r}'\right)}{\varepsilon_{0}}
\end{equation}
 

(1.16')は、任意の電荷分布$\rho\left(\boldsymbol{r}\right)$について成立する。つまり、任意のスカラー関数$\rho\left(\boldsymbol{r}\right)$について成り立つ恒等式である。

∴(1.15)の両辺$\nabla^{2}$して、右辺に(1.16')を適用すると、
\[
\nabla^{2}A_{x}\left(\boldsymbol{r}\right)=-\mu_{0}i_{x}\left(\boldsymbol{r}\right)
\]

\begin{equation}
\nabla^{2}\boldsymbol{A}=-\mu_{0}\boldsymbol{A}
\end{equation}

つまり、$\boldsymbol{i}$と$\boldsymbol{A}$のあいだ市、ポアソンの式と同じ形の式が成り立つ。(例えば、$i_{x}$の分布から、これを$\rho$だと思って$V$を求めれば、直ちに$A_{x}$が求まる)

\[
\nabla^{2}\boldsymbol{A}=\left(\begin{array}{c}
\pdd{A_{x}}x+\pdd{A_{x}}y+\pdd{A_{x}}z\\
\pdd{A_{y}}x+\cdots\\
\pdd{A_{z}}x+\cdots
\end{array}\right)
\]

(1.11)の両辺$\div$を取ると、
\begin{equation}
\nabla\cdot\boldsymbol{A}=0
\end{equation}

(クーロンげージ)

導出は、教p.109-110(注)

\paragraph{$\boldsymbol{A}$のゲージ}

電位$V$に定数$V_{0}$(任意のエネルギー基準)を加えても$\boldsymbol{E}$が変わらないのと同様に、$\boldsymbol{A}$の決め方も任意性がある。

(1.11)で定義した$\boldsymbol{A}$に、定ベクトル$\boldsymbol{A}_{0}$を加えても、(1.12)で求まる$\boldsymbol{B}$は変わらない。

それ以上に、任意のスカラー関数$\phi\left(\boldsymbol{r}\right)$を考え、その$\nabla\phi$を$\boldsymbol{A}$に加え、
\begin{equation}
\boldsymbol{A}'\equiv\boldsymbol{A}+\nabla\phi
\end{equation}
としてみても、
\[
\nabla\times\boldsymbol{A}'=\nabla\times\boldsymbol{A}+\nabla\times\left(\nabla\phi\right)
\]

(ベクトル公式より$\nabla\times\left(\nabla\phi\right)=0$)なので、$\boldsymbol{B}$は変わらない。

そこで、より広義に、(1.19)で変換(ゲージ変換という)できる$\boldsymbol{A}'$をすべてベクトルポテンシャルと呼ぶ。

ただし、このとき、一般に$\nabla\cdot\boldsymbol{A}'\neq0$となる。

$\boldsymbol{A}$の選び方を「ゲージ」と言い、特に、(1.18)を満たすように$\boldsymbol{A}$を選ぶこと(例えば(1.11))を「クーロンゲージ」という。

\subsection{アンペアの法則}
\begin{itemize}
\item 電場については$\nabla\times\boldsymbol{E}=0$ 磁界はどうか?
\end{itemize}
(1.12)より
\begin{align*}
\nabla\times\boldsymbol{B} & =\nabla\times\nabla\times\boldsymbol{A}\\
 & =\nabla\left(\nabla\cdot\boldsymbol{A}\right)-\nabla^{2}\boldsymbol{A}
\end{align*}

(1.18)より$\nabla\cdot\boldsymbol{A}=0$

また(1.17)より $\nabla^{2}\boldsymbol{A}=-\mu_{0}\boldsymbol{i}$

\paragraph{アンペアの法則(微分系)}

\begin{equation}
\nabla\times\boldsymbol{B}=\mu_{0}\boldsymbol{i}
\end{equation}

つまり、電流$\boldsymbol{i}$があると$\boldsymbol{B}$の``渦''が発生

\paragraph{積分系}

任意の閉曲面$S$について、(1.20)の両辺を面積分する。

図電後2-3

\[
\int_{S}\nabla\times\boldsymbol{B}\mathrm{d}S=\int_{S}\mu_{0}\boldsymbol{i}\mathrm{d}S
\]

ストークスの定理より、
\[
\text{左辺}=\oint_{C}\boldsymbol{B}\cdot\mathrm{d}\boldsymbol{s}
\]

一方、
\[
\text{右辺}=\mu_{0}I
\]

$I$: $S$を貫く全電流

∴アンペアの法則 (積分系)

\begin{equation}
\oint_{C}\boldsymbol{B}\cdot\mathrm{d}\boldsymbol{s}=\mu_{0}I
\end{equation}

図電後2-4

$C$: 任意の閉曲線

$I$: $C$を鎖交する全電流。$C$に沿って右ねじを回すときネジが進む向きを正とする。

\paragraph{例}

図電後2-5

\begin{align*}
I & =I_{1}-I_{2}+I_{3}-I_{3}\\
 & =I_{1}-I_{2}
\end{align*}


\paragraph{例題}

ソレノイドの中の磁界

図電後2-6

うえんような閉経路$C$を考え、(1.21)を適用

\[
\text{左辺}=\int_{\text{左}}\boldsymbol{B}\cdot\mathrm{d}\boldsymbol{s}+\int_{\text{下}}\boldsymbol{B}\cdot\mathrm{d}\boldsymbol{s}+\int_{\text{上}}\boldsymbol{B}\cdot\mathrm{d}\boldsymbol{s}+\int_{\text{右}}\boldsymbol{B}\cdot\mathrm{d}\boldsymbol{s}
\]

$\int_{\text{下}}\boldsymbol{B}\cdot\mathrm{d}\boldsymbol{s}+\int_{\text{上}}\boldsymbol{B}\cdot\mathrm{d}\boldsymbol{s}$:
ソレノイドが無限であれば、対称性より打ち消し合う。 

\paragraph{先週の課題1.}

図電後2-7

当日問題2より、円電流$I$が高さ$h$に作る$B$は
\[
B=\frac{\mu_{0}Ia^{2}}{2\left(a^{2}+h^{2}\right)^{\frac{3}{2}}}
\]

図電後2-8

位置$\left[x,x+\mathrm{d}x\right]$の円電流が点$P$につくる$\mathrm{d}B$を求める。

\[
r\equiv\sqrt{a^{2}+h^{2}}=\frac{a}{\sin\theta}
\]

\[
\therefore\mathrm{d}B=\frac{\mu_{0}a^{2}\left(In\mathrm{d}x\right)}{2\left(\frac{a}{\sin\theta}\right)^{3}}=\frac{\mu_{0}In\sin^{3}\theta}{2a}\mathrm{d}x
\]

\[
B=\int_{x_{1}}^{x_{2}}\frac{\mu_{0}In\sin^{3}\theta}{2a}\mathrm{d}x
\]

ここで$h=x_{p}-x=\frac{a}{\tan\theta}$

\[
x=x_{p}-\frac{a}{\tan\theta}
\]

\[
\d x{\theta}=-a\frac{\left(-\sin^{2}\theta-\cos^{2}\theta\right)}{\sin^{2}\theta}=\frac{a}{\sin^{2}\theta}
\]

$x\rightarrow\theta$に置換

\begin{align*}
\therefore B & =\int_{\theta_{1}}^{\pi-\theta_{2}}\frac{\mu_{0}In\sin^{3}\theta}{2a}\frac{a}{\sin^{2}\theta}\mathrm{d}\theta\\
 & =\frac{\mu_{0}nI}{2}\left(\cos\theta_{1}+\cos\theta_{2}\right)
\end{align*}

無限の場合は$\theta_{1}=\theta_{2}=0$ $B=\mu_{0}nI$となり、講義(1.26)と一致。

\paragraph{当日問題1(a)①}

(1.11)から計算

$x,y$方向は電流=0なので$A_{x}=A_{y}=0$

図電後2-9

\begin{align*}
A_{z} & =\frac{\mu_{0}}{4\pi}\int_{-L}^{L}\frac{I}{r}\mathrm{d}z\\
 & =\frac{\mu_{0}}{4\pi}\int_{-L}^{L}\frac{I}{\sqrt{z^{2}+a^{2}}}\mathrm{d}z\\
 & =\frac{\mu_{0}I}{4\pi}\left[\log\left(z+\sqrt{z^{2}+a^{2}}\right)\right]_{-L}^{L}\\
 & =\frac{\mu_{0}I}{4\pi}\log\frac{L+\sqrt{L^{2}+a^{2}}}{-L+\sqrt{L^{2}+a^{2}}}\\
 & =\frac{\mu_{0}I}{4\pi}\log\frac{\left\{ L+\sqrt{L^{2}+a^{2}}\right\} ^{2}}{a^{2}}\\
 & =\frac{\mu_{0}I}{2\pi}\log\frac{L+\sqrt{L^{2}+a^{2}}}{a}\\
 & =\frac{\mu_{0}I}{2\pi}\left[\log\frac{1+\sqrt{1+\left(\frac{a}{L}\right)^{2}}}{a}-\log L\right]
\end{align*}

$\log L$の項は定数なので無視する。($A$の原点を$\frac{\mu_{0}I}{2\pi}\log L$だけずらすと考える)

$L\rightarrow\infty$の極限を考えると、
\begin{align*}
A_{z} & =\frac{\mu_{0}I}{2\pi}\log\frac{2}{a}=\frac{\mu_{0}I}{2\pi}\left[\log2-\log a\right]\\
 & =-\frac{\mu_{0}I}{2\pi}\log a+\left(\nabla\phi\right)
\end{align*}


\paragraph{当日問題1(a)②}

教p.111 例題6.5

$z$軸上に線密度$I$の一様な電荷がある時の$V$を計算する。

ガウスの法則より$z$軸から距離$r$の位置の電界は$E\left(r\right)=\frac{I}{2\pi\varepsilon_{0}r}$

\begin{align*}
V\left(a\right) & =-\int_{r_{0}}^{a}E\left(r\right)\mathrm{d}r\\
 & =-\frac{I}{2\pi\varepsilon_{0}}\left(\log-\log r_{0}\right)
\end{align*}

$\log r_{0}$は無視。$\varepsilon_{0}\rightarrow\frac{1}{\mu_{0}}$とすると、
\[
A_{z}=-\frac{\mu I}{2\pi}\log a
\]


\paragraph{当日問題1(b)}

\begin{align*}
A_{z} & =-\frac{\mu_{0}I}{2\pi}\log\sqrt{x^{2}+y^{2}}\\
 & =-\frac{\mu_{0}I}{4\pi}\log\left(x^{2}+y^{2}\right)
\end{align*}

$\therefore\boldsymbol{B}=\nabla\times\boldsymbol{A}$より、
\[
B_{x}=\pd{A_{z}}y=-\frac{\mu_{0}I}{2\pi}\frac{y}{x^{2}+y^{2}}
\]

\[
B_{y}=-\pd{A_{z}}x=\frac{\mu_{0}I}{2\pi}\frac{x}{x^{2}+y^{2}}
\]

\[
B_{z}=0
\]

図電後2-10

先週の演習問題1(b)$B=\frac{\mu_{0}I}{2\pi a}$に一致

\paragraph{当日問題1(c)}

\begin{align*}
\pd{B_{x}}x & =\frac{\mu_{0}I}{\pi}\frac{xy}{\left(x^{2}+y^{2}\right)^{2}}\\
\pd{B_{y}}y & =-\frac{\mu_{0}I}{\pi}\frac{xy}{\left(x^{2}+y^{2}\right)^{2}}\\
\pd{B_{z}}z & =0
\end{align*}

\[
\therefore\nabla\cdot\boldsymbol{B}=0
\]


\paragraph{当日問題2}

図電後2-11

円柱の軸を中心に半径$r$の円経路$C$を考える。

(i) $r<a$のとき

図電後2-12

$C$を貫く電流は
\[
I=\pi r^{2}J
\]

一方
\[
\oint\boldsymbol{B}\cdot\mathrm{d}\boldsymbol{s}=2\pi rB
\]

対称性より$C$上で$B$一定

∴アンペアの法則

\begin{align*}
2\pi rB & =\pi\mu_{0}r^{2}J\\
B & =\frac{\mu_{0}rJ}{2}
\end{align*}

(ii) $r>a$のとき

図電後2-13

\[
I=\pi a^{2}J
\]

\begin{align*}
\therefore2\pi rB & =\pi\mu_{0}a^{2}J\\
B & =\frac{\mu_{0}a^{2}J}{2r}
\end{align*}

図電後2-14

\paragraph{当日問題3}

\[
\left|\boldsymbol{r}-\boldsymbol{r}'\right|=\sqrt{\left(x-x'\right)^{2}+\left(y-y'\right)^{2}+\left(z-z'\right)^{2}}
\]

\begin{align*}
\pd{}x\left(\frac{1}{\left|\boldsymbol{r}-\boldsymbol{r}'\right|}\right) & =-\frac{1}{\left|\boldsymbol{r}-\boldsymbol{r}'\right|^{2}}\pd{}x\left(\left|\boldsymbol{r}-\boldsymbol{r}'\right|\right)\\
 & =-\frac{1}{\left|\boldsymbol{r}-\boldsymbol{r}'\right|^{2}}\frac{2\left(x-x'\right)}{2\sqrt{\left(x-x'\right)^{2}+\left(y-y'\right)^{2}+\left(z-z'\right)^{2}}}=-\frac{\left(x-x'\right)}{\left|\boldsymbol{r}-\boldsymbol{r}'\right|^{3}}
\end{align*}

$y,z$成分も同様

\[
\therefore\nabla\left(\frac{1}{\left|\boldsymbol{r}-\boldsymbol{r}'\right|}\right)=-\frac{\boldsymbol{r}-\boldsymbol{r}'}{\left|\boldsymbol{r}-\boldsymbol{r}'\right|^{3}}
\]

\end{document}
