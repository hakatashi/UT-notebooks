%% LyX 2.2.2 created this file.  For more info, see http://www.lyx.org/.
%% Do not edit unless you really know what you are doing.
\documentclass[oneside,english]{book}
\usepackage[T1]{fontenc}
\usepackage[utf8]{inputenc}
\usepackage[a5paper]{geometry}
\geometry{verbose,tmargin=2cm,bmargin=2cm,lmargin=1cm,rmargin=1cm}
\setcounter{secnumdepth}{3}
\setcounter{tocdepth}{3}
\setlength{\parskip}{\smallskipamount}
\setlength{\parindent}{0pt}
\usepackage{textcomp}
\usepackage{amsmath}
\usepackage{amssymb}

\makeatletter
%%%%%%%%%%%%%%%%%%%%%%%%%%%%%% User specified LaTeX commands.
\usepackage[dvipdfmx]{hyperref}
\usepackage[dvipdfmx]{pxjahyper}

\makeatother

\usepackage{babel}
\begin{document}

\title{2016-A 数理手法I}

\author{教員: 入力: 高橋光輝}

\maketitle
\global\long\def\pd#1#2{\frac{\partial#1}{\partial#2}}
\global\long\def\d#1#2{\frac{\mathrm{d}#1}{\mathrm{d}#2}}
\global\long\def\pdd#1#2{\frac{\partial^{2}#1}{\partial#2^{2}}}
\global\long\def\dd#1#2{\frac{\mathrm{d}^{2}#1}{\mathrm{d}#2^{2}}}


\section*{第1回}

\paragraph{講義の進め方}

講義と演習で構成される。
\begin{itemize}
\item 講義4回
\item 演習3回
\item 講義?回
\item 演習2回
\end{itemize}

\paragraph{数値解析の目的}
\begin{itemize}
\item データの整理

\begin{itemize}
\item 確率的手法

\begin{itemize}
\item 母集団全てに対して調査を行うのではなく、サンプリングされた対象から結論を導き出す。
\end{itemize}
\end{itemize}
\end{itemize}

\paragraph{確率}

「物事の起こりやすさ」。物事=事象。

起こりうる事象全体 標本空間 $\Omega$

$\Omega$に含まれる点 標本点 $\omega$

例: $\omega=\left\{ 1\right\} \left\{ 2\right\} \cdots\left\{ 6\right\} $

事象は標本点の集合で、$\Omega$の部分集合。$A,B,C$などと表す。$A\subset\Omega$ (可則)

起こらないこと 空事象 $\phi$ ⇔ 数字の0

\paragraph{ベン図}

図MM-1

図MM-2

\begin{align*}
A & =\left\{ 1,2,3,5\right\} \\
B & =\left\{ 1,2,3\right\} 
\end{align*}

$B\subset A$ $B$が$A$の部分集合

図MM-3

\[
B=\left\{ 4,5,6\right\} 
\]

共通部分あり

図MM-4

\[
B=\left\{ 4,6\right\} 
\]

共通部分なし: 排反事象

図MM-5

\begin{align*}
A & =\left\{ 1,2,3,4\right\} \\
B & =\left\{ 5,6\right\} 
\end{align*}

$B$が「$A$以外」となっている: 余事象 $A^{c}$

\begin{align*}
\phi^{c} & =\Omega\\
\Omega^{c} & =\phi
\end{align*}


\paragraph{和事象・積事象}

和事象 $A\cup B$ $A$または$B$の(少なくとも一方)が起こる

図MM-6

積事象 $A\cap B$ $A$かつ$B$ $A$と$B$が同時に起こる

図MM-7

排反事象の定義 $A\cap B=\phi$

\paragraph{分配法則}

\[
\left(a+b\right)\times c=a\times c+b\times c
\]
\[
\left(A\cup B\right)\cap C=\left(A\cap C\right)\cup\left(A\cap B\right)
\]
\[
\left(A\cap B\right)\cup C=\left(A\cup C\right)\cap\left(A\cup B\right)
\]


\paragraph{ドモルガンの法則}

\begin{align*}
\left(A\cup B\right)^{c} & =A^{c}\cap B^{c}\\
\left(A\cap B\right)^{c} & =A^{c}\cup B^{c}
\end{align*}


\paragraph{確率}

$A$の起こる確率:$P\left(A\right)$

\paragraph{コロモゴロフによる3つの公理}
\begin{enumerate}
\item $0\leq P\left(A\right)\leq1$
\item $P\left(\Omega\right)=1\Leftrightarrow P\left(\phi\right)=0$
\item 違いに排反である可算個の事象$A_{1},A_{2},\cdots,A_{k}$に対して、
\[
P\left(A_{1}\cup A_{2}\cup\cdots\cup A_{k}\right)=\sum_{i=1}^{k}P\left(A_{i}\right)
\]

可算個: 有限もしくは自然数と同一の密度

有理数は自然数と同じ密度で存在する
\end{enumerate}

\paragraph{確率の意味}

大数の法則

例) コインを投げて表が出る確率が$\frac{1}{2}$であるとは、全体の試行回数を$n$、表の回数を$\gamma$としたときに、$n\rightarrow\infty$のとき$\frac{\gamma}{n}\rightarrow\frac{1}{2}$となることである。

\paragraph{加法定理}

\[
P\left(A\cup B\right)=P\left(A\right)+P\left(B\right)-P\left(A\cap B\right)
\]

\[
P\left(A\cup B\cup C\right)=P\left(A\right)+P\left(B\right)+P\left(C\right)-P\left(A\cap B\right)-P\left(B\cap C\right)-P\left(C\cap A\right)+P\left(A\cap B\cap C\right)
\]


\paragraph{条件付き確率}

白い1の玉2個と、白い2の玉1個と、赤い1の玉1個と、赤い2の玉2個が入った箱から1つを選ぶ事を考える。

\[
P\left(\text{“1”}\right)=\frac{1}{6}+\frac{1}{6}+\frac{1}{6}=\frac{3}{6}=\frac{1}{2}
\]
\[
P\left(\text{“2”}\right)=\frac{1}{2}
\]

ここで選んだ玉が白であるという情報を持っていると、
\[
P\left(A|B\right)=P\left(\text{“1”}|\text{“白”}\right)=\frac{2}{3}
\]
となる。このような確率を($B$を条件とする$A$の)条件付き確率という。

\[
P\left(A|B\right)=\frac{P\left(A\cap B\right)}{P\left(B\right)}
\]

乗法定理

\[
P\left(A\cap B\right)=P\left(A|B\right)P\left(B\right)=P\left(B|A\right)P\left(A\right)
\]


\paragraph{独立}

\[
P\left(A|B\right)=P\left(B\right)
\]
となるとき、$A$と$B$が独立であるという。

乗法定理に代入すると、
\[
P\left(A\cap B\right)=P\left(A|B\right)P\left(B\right)=P\left(A\right)P\left(B\right)
\]

この式は$P\left(B\right)=0$の場合にも拡張可能なので、独立の定義にはこちらを用いる。

\section*{第2回}

\paragraph{順列数}

\[
_{n}P_{r}=\frac{n!}{\left(n-r\right)!}
\]


\paragraph{確率}

\[
\frac{\text{目的の順列数}}{\text{全体の順列数}}
\]

同等に起こりやすい場合

\paragraph{組み合わせ数}

\[
_{n}C_{r}=\frac{_{n}P_{r}}{r!}=\frac{n!}{\left(n-r\right)!r!}n
\]


\paragraph{ベイズの定理}

原因⇔結果

\[
P\left(B|A\right)-\frac{P\left(A\cap B\right)}{P\left(A\right)}-\frac{P\left(A|B\right)P\left(B\right)}{P\left(A\right)}
\]

$B$は$B_{1},B_{2},\cdots B_{k}$

\[
B_{i}\cap B_{j}=\phi\:\left(i\neq j\right)
\]
\[
\bigcup_{i=1}^{k}B_{i}=B_{1}\cup B_{2}\cup\cdots\cup B_{k}=B=\Omega
\]


\paragraph{分配法則}

\begin{align*}
A\cap\Omega & =A\cap\left(B_{1}\cup B_{2}\cup\cdots\cup B_{k}\right)\\
 & =\left(A\cap B_{1}\right)\cup\left(A\cap B_{2}\right)\cdots\cup\left(A\cap B_{k}\right)\\
 & =\bigcup_{i=1}^{k}\left(A\cap B_{i}\right)
\end{align*}

排反事象である。

\begin{align*}
P\left(A\right) & =P\left(\bigcup_{i=1}^{k}A\cap B_{i}\right)\\
 & =\sum_{i=1}^{k}P\left(A\cap B_{i}\right)
\end{align*}
\[
P\left(B|A\right)=\frac{P\left(A|B\right)P\left(B\right)}{\sum_{i=1}^{k}P\left(A|B_{i}\right)P\left(B_{i}\right)}
\]

$P\left(B_{i}\right)$: 事前確率

$P\left(B_{i}|A\right)$: 事後確率

\paragraph{確率変数}

取りうる値 + その確率

確率変数: $X$

取りうる値:$x$

\[
P\left(X=x\right)=f\left(x\right)
\]


\paragraph{離散型の確率変数}

取りうる値がとびとび(可算個)

\paragraph{コイン投げ}

表(H)が出たら1点、裏(T)が出たら0点とし、2回投げる(独立)。→ベルヌーイ試行

合計点: $X$

\[
X=\begin{cases}
2 & \left(H,H\right)\\
1 & \left(H,T\right),\left(T,H\right)\\
0 & \left(T,T\right)
\end{cases}
\]

$P\left(X=2\right)=\frac{1}{2}\times\frac{1}{2}=\frac{1}{4}$

$P\left(X=1\right)=\frac{1}{4}+\frac{1}{4}=\frac{1}{2}$

$P\left(X=0\right)=\frac{1}{4}$

表が出る確率を$p$、裡が出る確率を$q=1-p$とすると、
\begin{align*}
P\left(X=i\right) & =_{n}C_{x}p^{x}\left(1-p\right)^{n-x}\\
 & =f\left(x\right)
\end{align*}

→二項分布

\paragraph{分布の中心は?: 期待値(平均)}

\[
\mathrm{ex}=E\left(X\right)=\sum_{x}xf\left(x\right)
\]

中央値 50\%点

\[
P\left(X\leq x\right)\geq\frac{1}{2}
\]
の最小の値。

\[
P\left(X\leq x\right)=\sum_{u\leq x}f\left(u\right)=F\left(x\right)
\]

(累積)分布関数

\paragraph{モード}

$f\left(x\right)$の最大の値を与える

標準偏差$\sigma=\sqrt{\text{分散}}$

\[
E\left(X^{2}\right)=\sum_{x}x^{2}f\left(x\right)
\]
\[
E\left(g\left(x\right)\right)=\sum_{x}g\left(x\right)f\left(x\right)
\]
\[
E\left(X+Y\right)=E\left(X\right)+E\left(Y\right)
\]

$n\rightarrow\infty\:\because p\rightarrow0$

$n\cdot p=\lambda$: ポアソンの小数の法則

\[
f\left(x\right)=\frac{\mathrm{e}^{x}\lambda^{x}}{x!}\:x=0,1,2\cdots
\]

$p\rightarrow0\:n\rightarrow\infty$

$\mu\rightarrow\lambda$

$\sigma^{2}=\lambda$

$\mu=\sigma^{2}$

\paragraph{連続型の確率変数}
\end{document}
