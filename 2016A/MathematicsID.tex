%% LyX 2.1.3 created this file.  For more info, see http://www.lyx.org/.
%% Do not edit unless you really know what you are doing.
\documentclass[oneside,english]{book}
\usepackage[T1]{fontenc}
\usepackage[utf8]{inputenc}
\usepackage[a5paper]{geometry}
\geometry{verbose,tmargin=2cm,bmargin=2cm,lmargin=1cm,rmargin=1cm}
\setcounter{secnumdepth}{3}
\setcounter{tocdepth}{3}
\setlength{\parskip}{\smallskipamount}
\setlength{\parindent}{0pt}
\usepackage{textcomp}
\usepackage{amsmath}
\usepackage{esint}

\makeatletter
%%%%%%%%%%%%%%%%%%%%%%%%%%%%%% User specified LaTeX commands.
\usepackage[dvipdfmx]{hyperref}
\usepackage[dvipdfmx]{pxjahyper}

\makeatother

\usepackage{babel}
\begin{document}

\title{2016-A 数学ID}


\author{教員: 入力: 高橋光輝}

\maketitle
\global\long\def\pd#1#2{\frac{\partial#1}{\partial#2}}
\global\long\def\d#1#2{\frac{\mathrm{d}#1}{\mathrm{d}#2}}
\global\long\def\pdd#1#2{\frac{\partial^{2}#1}{\partial#2^{2}}}
\global\long\def\dd#1#2{\frac{\mathrm{d}^{2}#1}{\mathrm{d}#2^{2}}}
\global\long\def\e{\mathrm{e}}
\global\long\def\i{\mathrm{i}}
\global\long\def\j{\mathrm{j}}
\global\long\def\grad{\mathrm{grad}}
\global\long\def\rot{\mathrm{rot}}
\global\long\def\div{\mathrm{div}}
\global\long\def\diag{\mathrm{diag}}



\section*{第1回}


\paragraph{この講義の内容}
\begin{enumerate}
\item 常微分方程式

\begin{itemize}
\item 一変数の微分に関する方程式
\end{itemize}
\item ベクトル解析(3回程度)

\begin{itemize}
\item ベクトルの内積・外積、微分
\end{itemize}
\item 変分法(3回程度)

\begin{itemize}
\item オイラー・ラグランジュ方程式
\end{itemize}
\end{enumerate}

\paragraph{副読本}

ジョージ・アルフケン

ハンス・ウェーバー


\paragraph{授業の流れ}
\begin{enumerate}
\item 常微分方程式

\begin{enumerate}
\item 変数分離系
\item 同次系
\item 定数添加法
\item 完全微分方程式
\item 線形微分方程式
\item 連立微分方程式
\end{enumerate}
\item ベクトル解析

\begin{enumerate}
\item 内積・外積
\item gradient $\nabla\phi$
\item divergence $\nabla\cdot A$
\item rotation $\nabla\times A$
\item Greenの定理
\item Stokesの定理
\item Gauussの定理
\item 直行曲線座標
\item 円柱座標
\item 3次元極座標
\end{enumerate}
\item 変分法

\begin{enumerate}
\item Eular-lagrange方程式
\item 解析力学
\item Lagrangian
\item Hamiltonian
\item Lagrangeの未定乗数法
\end{enumerate}
\end{enumerate}

\paragraph{成績評価}

期末試験+何回かのレポート


\chapter{第一章 常微分方程式}


\paragraph{変数分離系}

関数$u\left(x\right)$を考える。

\[
\d ux=g\left(x\right)h\left(u\right)
\]
と表せるとき、変数分離系という。

\begin{align*}
\frac{\mathrm{d}u}{h\left(u\right)} & =g\left(x\right)\mathrm{d}x\\
\int^{u}\frac{\mathrm{d}u}{h\left(u\right)} & =\int^{x}g\left(x\right)\mathrm{d}x
\end{align*}



\paragraph{放射性物質の自然崩壊}

$u\left(t\right)$: 放射性物質の量

$\gamma$: 消滅率

\begin{align*}
\mathrm{d}u & =\lim_{\delta\rightarrow0}\left[u\left(t+\delta t\right)-u\left(t\right)\right]\\
 & =-\gamma u\left(t\right)\mathrm{d}t
\end{align*}
\[
\d ut=-\gamma u\Rightarrow u\left(t\right)=u_{0}\mathrm{e}^{-\gamma t}
\]


半減期$\tau$は
\[
u\left(t\right)=u_{0}\mathrm{e}^{-\gamma t}=\frac{1}{2}u_{0}
\]
\[
-\gamma t=-\ln2
\]
\[
\tau-\frac{\ln2}{\gamma}
\]



\paragraph{Malthus模型}

バクテリアの増殖を考える。

$\alpha$: 増殖率

\begin{align*}
\mathrm{d}u=\alpha u\left(t\right)\mathrm{d}t & \Rightarrow\d ut=\alpha u\\
 & \Rightarrow u\left(t\right)=u_{0}\mathrm{e}^{\alpha t}
\end{align*}



\paragraph{logistic方程式}

増殖率$\alpha$を$\alpha-\beta u$としたもの。

\[
\d ut=\left(\alpha-\beta u\right)u=\alpha u-\beta u^{2}
\]


変数分離を用いて解く。

\[
\int_{u_{0}}^{u}\frac{\mathrm{d}u'}{u'\left(\alpha-\beta u'\right)}=\int_{t_{0}}^{t}\mathrm{d}t'=t-t_{0}
\]


部分分数分解を用いて、
\[
\frac{1}{u\left(\alpha-\beta u\right)}=\frac{A}{u}+\frac{B}{\alpha-\beta u}=\frac{A\alpha-\left(B-\beta A\right)u}{u\left(\alpha-\beta u\right)}
\]


$A=\frac{1}{\alpha},B=\frac{\beta}{\alpha}$より、
\begin{align*}
\int_{u_{0}}^{u}\frac{\mathrm{d}u'}{u'\left(\alpha-\beta u'\right)} & =\frac{1}{\alpha}\int_{u_{0}}^{u}\left(\frac{1}{u'}+\frac{\beta}{\alpha-\beta u'}\right)\mathrm{d}u'\\
 & =\frac{1}{\alpha}\left[\log\frac{u}{u_{0}}+\log\frac{\alpha-\beta u_{0}}{\alpha-\beta u}\right]\\
 & =\frac{1}{\alpha}\log\frac{u}{u_{0}}\frac{\alpha-\beta u_{0}}{\alpha-\beta u}\\
\mathrm{e}^{\alpha\left(t-t_{0}\right)} & =\frac{u}{u_{0}}\frac{\alpha-\beta u_{0}}{\alpha-\beta u}
\end{align*}
\[
\left(\alpha-\beta u_{0}+\beta u_{0}\mathrm{e}^{u\left(t-t_{0}\right)}\right)u\left(t\right)=\alpha u_{0}\mathrm{e}^{u\left(t-t_{0}\right)}
\]
\[
u\left(t\right)=\frac{\alpha u_{0}\mathrm{e}^{u\left(t-t_{0}\right)}}{\alpha-\beta u_{0}+\beta u_{0}\mathrm{e}^{u\left(t-t_{0}\right)}}
\]



\paragraph{Lotka-Volterra方程式}

被食者と捕食者がいる生態系を考える。

被食者: $x\left(t\right)$

捕食者: $y\left(t\right)$

\begin{align*}
\d xt & =x\left(\alpha-\beta y\right)\\
\d yt & =-y\left(\gamma-\delta x\right)
\end{align*}


$\alpha,\beta,\gamma,\delta$は正の実数

このような式を非線形連立方程式という。→原則、非線形は解けない

平衡点を考える。

\[
\d xt=\d yt=0
\]
\[
\begin{cases}
x\left(\alpha-\beta y\right)=0\\
-y\left(\gamma-\delta x\right)=0
\end{cases}\Rightarrow\begin{cases}
\left(\begin{array}{c}
x=0\\
y=0
\end{array}\right) & \text{絶滅状態}\\
\left(\begin{array}{c}
x=\frac{\gamma}{\delta}\\
y=\frac{\alpha}{\beta}
\end{array}\right) & \text{平衡状態}
\end{cases}
\]


$X=x-\frac{\gamma}{\delta},Y=y-\frac{\alpha}{\beta}$($\left|X\right|\ll1,\left|Y\right|\ll1$)とし、
\begin{align*}
\d Xt & =\left(X+\frac{\gamma}{\delta}\right)\left(\alpha-\beta\left(Y+\frac{\alpha}{\beta}\right)\right)\simeq-\frac{\beta\gamma}{\delta}Y\\
\d Yt & =-\left(Y+\frac{\alpha}{\beta}\right)\left(\gamma-\delta\left(X+\frac{\gamma}{\delta}\right)\right)\simeq\frac{\alpha\delta}{\gamma}X
\end{align*}
\[
X'=-\frac{\alpha\gamma}{\delta}Y'=-\alpha\gamma X
\]
\begin{align*}
X & =c\cos\left(\sqrt{\alpha\gamma}t+\theta_{0}\right)\\
Y & =-\frac{\delta}{\beta\gamma}X'=\sqrt{\frac{\alpha}{\gamma}}\frac{\delta}{\beta}c\sin\left(\sqrt{\alpha\gamma}t+\theta_{0}\right)
\end{align*}


このように一般に解けない方程式も、平衡点の周りの近似解は求めることができる。


\paragraph{Liourille方程式}

\begin{align*}
 & y''+P\left(x\right)y'+Q\left(y\right)\left(y'\right)^{2}=0\\
\Rightarrow & \frac{y''}{y'}+P\left(x\right)+Q\left(y\right)y'=0
\end{align*}


\begin{align*}
 & \d{}x\left[\log y'+\int^{x}P\left(x'\right)\mathrm{d}x'+\int^{y}Q\left(y\right)\mathrm{d}y\right]=0\\
\Rightarrow & \log y'+\int^{x}P\left(x\right)\mathrm{d}x+\int^{y}Q\left(y\right)\mathrm{d}y=C\\
\Rightarrow & \d yx=\exp\left[-\int^{x}P\left(x\right)\mathrm{d}x\right]\exp\left[-\int^{y}Q\left(y\right)\mathrm{d}y\right]\\
\Rightarrow & \int^{y}\exp\left[\int^{y}Q\left(y\right)\mathrm{d}y\right]\mathrm{d}y=C_{1}\int^{x}\exp\left[-\int^{x}P\left(x\right)\mathrm{d}x\right]\mathrm{d}x+C_{2}
\end{align*}


例1

\[
y''+2xy'+2y\left(y'\right)^{2}=0
\]


$P\left(x\right)=2x,Q\left(y\right)=2y$とし、
\[
\int^{y}\mathrm{e}^{y}\mathrm{d}y=C_{1}\int^{x}\mathrm{e}^{-x^{2}}\mathrm{d}x+C_{2}
\]


例2

\[
\d yx=\sin x\tan y\Rightarrow\sin\mathrm{d}x=\frac{\cos y}{\sin y}\mathrm{d}y=\frac{\mathrm{d}\sin y}{\sin y}=\mathrm{d}\log\left|\sin y\right|
\]


\[
\cos x=-\log\left|\sin y\right|+C\Rightarrow\sin y=C\mathrm{e}^{-\cos x}\Rightarrow y=\arcsin\left[C\mathrm{e}^{-\cos x}\right]
\]



\section*{第2回}


\paragraph{同次型微分方程式}

\[
\d yx=f\left(\frac{y}{x}\right)
\]


変換$x\mapsto\lambda x,y\mapsto\lambda y$に対して不変

$u=\frac{y}{x}$とおく。

\[
\d yx=y'=\left(xu\right)'=xu'+u
\]
\[
xu'+u=f\left(u\right)\Rightarrow\d ux=\frac{f\left(u\right)-u}{x}
\]


\[
\int^{u}\frac{\mathrm{d}u}{f\left(u\right)-u}=\int^{u}\frac{\mathrm{d}x}{x}=\log\left|x\right|+c
\]


\[
x\e^{c}=\exp\left[\int^{u}\frac{\mathrm{d}u}{f\left(u\right)-u}\right]\equiv G\left(u\right)
\]


\[
G\left(\frac{y}{x}\right)=Cx\:\left(C=\e^{c}\right)
\]



\paragraph{例}

\[
\d yx=\e^{\frac{y}{x}}+\frac{y}{x}
\]


$y=xu$とおく。

\[
xu'+u=\e^{u}+u\Rightarrow\e^{-u}\mathrm{d}u=\frac{\mathrm{d}x}{x}
\]


\[
-\e^{-u}=\log\left|x\right|-C
\]


\[
\e^{-\frac{y}{x}}=-\log\left|x\right|+C
\]
\[
-\frac{y}{x}=\log\left|-\log\left|x+C\right|\right|
\]


\[
y=-x\log\left|-\log\left|x\right|+C\right|
\]



\paragraph{例}

\[
y'=\frac{x-2y+3}{2x+y-4}
\]


連立方程式

\[
\begin{cases}
\alpha-2\beta+3=0\\
2\alpha+\beta-4=0
\end{cases}\rightarrow\alpha=1,\beta=2
\]


$x=\xi+1,y=\eta+2$とおく。

\[
\d yx=\d{\eta}{\xi},\frac{x-2y+3}{2x+y-4}=\frac{\xi-2\eta}{2\xi+\eta}=\frac{1-2\frac{\eta}{\xi}}{2+\frac{\eta}{\xi}}
\]


$\eta=\xi z$とおく。

\[
\xi\d z{\xi}+z=\frac{1-2z}{2+z}\Rightarrow\frac{z+4}{z^{2}+4z-1}\mathrm{d}z=-\frac{2}{\xi}\mathrm{d}\xi
\]


\[
\log\left|z^{2}+4z-1\right|=-2\log\left|\xi\right|+c
\]
\[
\Rightarrow z^{2}+4z-1=\e^{c}\xi^{-2}
\]


\[
\left(y-2\right)^{2}+4\left(y-2\right)\left(x-1\right)-\left(x-1\right)^{2}=\e^{c}
\]



\paragraph{例}

$P\left(x,y\right)$と$Q\left(x,y\right)$が$x,y$について$k$次。

\[
\d yx=\frac{P\left(x,y\right)}{Q\left(x,y\right)}
\]


$k=1$のとき
\[
\d yx=\frac{p_{0}x+p_{1}y}{q_{0}x+q_{1}y}
\]


$k=2$のとき
\[
\d yx=\frac{p_{0}x^{2}+p_{1}xy+p_{2}y^{2}}{q_{0}x^{2}+q_{1}xy+q_{2}y^{2}}
\]


$u=\frac{y}{x}$とおくと、
\[
u+xu'=\frac{p_{0}+p_{1}u}{q_{0}+q_{1}u}
\]


\[
\Rightarrow x\d ux=\frac{p_{0}+p_{1}u}{q_{0}+q_{1}u}-u=\frac{p_{0}+\left(p_{1}-q_{0}\right)u-q_{1}u^{2}}{q_{2}+q_{1}u}
\]


\[
\log\left|x\right|=\int^{u}\frac{q_{2}+q_{1}u}{p_{0}+\left(p_{1}-q_{0}\right)u-q_{1}u^{2}}\mathrm{d}u\equiv G\left(u\right)
\]


\[
x=\exp\left[G\left(\frac{y}{x}\right)\right]
\]


$k=2$の場合も同様。以下一般で成り立つ。


\paragraph{例}

\[
\d yx=f\left(\frac{ax+by+p}{cx+dy+q}\right)
\]


\[
\begin{cases}
ax+by+p=0\\
cx+dy+q=0
\end{cases}
\]

\begin{enumerate}
\item $ad-bc\neq0$のとき $\left(x,y\right)=\left(\alpha,\beta\right)$


$x=\xi+\alpha,y=\eta+\beta$とおく。


\[
\d{\eta}{\xi}+f\left(\frac{a\xi+b\eta}{c\xi+d\eta}\right)=f\left(\frac{a+b\frac{\eta}{\xi}}{c+d\frac{\eta}{\xi}}\right)\equiv g\left(\frac{\eta}{\xi}\right)
\]


\item $ad-bc=0$で$c$と$d$の少なくとも一方が0でないとき


$\frac{ax+by}{cx+dy}=k\left(\text{定数}\right),z=cx+dy+q$とおく。


\begin{align*}
ax+by+p & =k\left(cx+dy\right)+p\\
 & =kz+p-kq
\end{align*}


\item $c=d=0$のとき


$z=-\frac{ax+by+p}{q}$とおく。


\[
\d zx=\frac{a+by'}{q}=\frac{a+bf\left(z\right)}{q}
\]



\[
\int\frac{\mathrm{d}z}{a+bf\left(z\right)}=\frac{x}{q}+C
\]


\end{enumerate}

\paragraph{例}

同次形でないもの

\[
\d yx=\frac{xy}{x^{2}+y}
\]


$\left(x\mapsto\lambda x,y\mapsto\lambda^{2}y\right)$に対して不変。

$u=\frac{y}{x^{2}}$とおく。
\[
\d yx=x^{2}u'+2xu
\]


\[
\frac{xy}{x^{2}+y}=\frac{x^{2}u}{x^{3}+x^{2}u}=\frac{xu}{1+u}
\]


\[
\d ux=\frac{u+2u^{2}}{x\left(1+u\right)}
\]


\[
\frac{1+u}{u\left(1+2u\right)}\mathrm{d}u=-\frac{\mathrm{d}x}{x}
\]


部分分数分解を行う。

\begin{align*}
\frac{1+u}{u\left(1+2u\right)}\mathrm{d}u & =\frac{1+u}{u}\mathrm{d}u-\frac{2\left(1+u\right)}{1+2u}\mathrm{d}u\\
 & =\frac{\mathrm{d}u}{u}+\frac{1+2u-2\left(1+u\right)}{1+2u}\mathrm{d}u\\
 & =\frac{1}{2}\mathrm{d}\left[\log u^{2}-\log\left(1+2u\right)\right]
\end{align*}


\begin{align*}
 & \log\left|\frac{u^{2}}{1+2u}\right|=-2\log\left|x\right|+c\\
\Rightarrow & \frac{x^{2}u^{2}}{1+2u}=c'\Rightarrow\frac{y^{2}}{x^{2}+2y}=c'
\end{align*}



\paragraph{例}

\[
\d yz=\frac{x^{2}y}{x^{2}+y}
\]


$\left(x\mapsto\lambda x,y\mapsto\lambda^{2}x\right)$に対して不変。

$u=\frac{y}{x^{2}}$とおく。

\[
y'=x^{3}u'+3x^{2}u
\]


\[
x^{3}u'+3x^{2}u=\frac{x^{5}u}{x^{3}+x^{3}u}=\frac{x^{2}u}{1+u}
\]


\[
\d ux=-\frac{u\left(2+3u\right)}{x\left(1+u\right)}
\]


\[
\frac{1+u}{u\left(2+3u\right)}\mathrm{d}u=-\frac{\mathrm{d}x}{x}
\]


\[
\frac{1+u}{u\left(2+3u\right)}=\frac{1}{2}\left(\frac{1+u}{u}-\frac{3+3u}{2+3u}\right)=\frac{1}{2}\left(\frac{1}{u}+1-\frac{3+3u}{2+3u}\right)
\]


\begin{align*}
\log\left|x\right|+c & =\frac{1}{2}\int^{u}\left(1u-\frac{1}{2+3u}\right)\mathrm{d}u\\
 & =\frac{1}{2}\left[\log u-\frac{1}{3}\log\left(2+3u\right)\right]\\
 & =-\frac{1}{6}\log\frac{u^{3}}{2+3u}\\
\Rightarrow\frac{x^{6}u^{3}}{2+3u} & =6C\\
\Rightarrow\frac{y^{3}}{2x^{2}+3y} & =6C
\end{align*}



\paragraph{定数変化法}

\[
\d ux+P\left(x\right)u=Q\left(x\right)
\]


$Q\left(x\right)=0$のとき、
\[
u\left(x\right)=C\exp\left[-\int P\left(x\right)\mathrm{d}x\right]
\]


定数$C$を$x$の関数と見て計算してみる。

\[
C\mapsto C\left(x\right)\Rightarrow u\left(x\right)=C\left(x\right)\exp\left[-\int P\left(x\right)\mathrm{d}x\right]
\]


\[
\d ux=\d Cx\exp\left[-\int P\left(x\right)\mathrm{d}x\right]P\left(x\right)u
\]


\[
C'\e^{-\int^{x}P\left(x\right)\mathrm{d}x}-Pu+Pu^{2}Q
\]


\[
\mathrm{d}C=Q\left(x\right)\e^{\int^{x}P\left(x\right)\mathrm{d}x}\mathrm{d}x
\]


\[
C\left(x\right)=\int^{x}Q\left(x\right)\e^{\int^{x}P\left(x\right)\mathrm{d}x}\mathrm{d}x+c
\]


\[
u\left(x\right)=\e^{\int^{x}P\left(x\right)\mathrm{d}x}\left[\int^{x}Q\left(x\right)\e^{\int^{x}P\left(x\right)\mathrm{d}x}\mathrm{d}x+c\right]
\]



\paragraph{例}

\[
xy^{2}-2y=x^{3}\cos x
\]


$xy'-2y=0$を解く。$y=Cx^{2}$

$y=C\left(x\right)x^{2}$

\[
C'x^{3}=x^{3}\cos x\Rightarrow C\left(x\right)=\sin x+c
\]


\[
y=x^{2}\left(\sin x+c\right)
\]



\paragraph{来週やること}

Bernoulli型の微分方程式

\[
y'+P\left(x\right)y=Q\left(x\right)y^{m}
\]


Riccati

\[
y'=P\left(x\right)y^{2}+Q\left(x\right)y+R\left(x\right)
\]


d'Alembert

\[
y=xf\left(y^{2}\right)+g\left(y'\right)
\]


Clairaut

\[
y=xp+g\left(y'\right)
\]



\section*{第3回}


\paragraph{レポート課題}

\[
\frac{\mathrm{d}y}{\mathrm{d}x}=\frac{3x-2y+1}{2x+3y-2}
\]
を解け。ただし$y\left(x\right)$の形まで解くこと。


\paragraph{Bernoulli型微分方程式}

\[
y'+P\left(x\right)y=Q\left(x\right)y^{m}
\]


$z=y^{k}$とおくと$z'=ky^{k-1}y'$

両辺に$ky^{k-1}$を掛けて、
\[
z'+kP\left(x\right)z=kQ\left(x\right)y^{x-1+m}
\]


$k=1-m$とおく。

\[
z'+\left(1-m\right)P\left(x\right)z=\left(1-m\right)Q\left(x\right)
\]


\[
z=\e^{-\left(1-n\right)\int P\left(x\right)\mathrm{d}x}\left[\left(1-m\right)\int Q\left(x\right)\e^{\left(1-m\right)\int P\left(x\right)\mathrm{d}x}\mathrm{d}x+C\right]
\]


\[
y=\e^{-\int P\left(x\right)\mathrm{d}x}\left[\left(1-m\right)\int Q\left(x\right)\e^{\left(1-m\right)\int P\left(x\right)\mathrm{d}x}\mathrm{d}x+C\right]^{\frac{1}{1-m}}
\]



\paragraph{例}

\[
y'+xy=xy^{3}
\]


$m=3$なので$z=y^{-2}$とおく。

\[
z'=-2y^{-3}y'
\]


両辺に$-2y^{-3}$をかけて、
\[
z'-2xz=-2x
\]


\begin{align*}
z & =\e^{x^{2}}\left(-2\int x\e^{-x^{2}}\mathrm{d}x+C\right)\\
 & =1+C\e^{x^{2}}
\end{align*}


\[
y=\frac{1}{\sqrt{1+C\e^{x^{2}}}}
\]



\paragraph{Riccati型微分方程式}

\[
y'=P\left(x\right)y^{2}+Q\left(x\right)y+R\left(x\right)
\]


$y=f\left(x\right)$を一つの特殊解とする。

\[
z=y-f\left(x\right)
\]


\begin{align*}
z' & =y'-f'\left(x\right)\\
 & =P\left(x\right)\left(z+f\left(x\right)\right)^{2}+Q\left(x\right)\left(z+f\left(x\right)\right)+R\left(x\right)-f'\left(x\right)\\
 & =P\left(x\right)z^{2}+\left[2P\left(x\right)f\left(x\right)+Q\left(x\right)\right]z+\left[P\left(x\right)f^{2}\left(x\right)+Q\left(x\right)f\left(x\right)+R\left(x\right)-f'\left(x\right)\right]
\end{align*}


\[
z'-\left[2P\left(x\right)f'\left(x\right)+Q\left(x\right)\right]z=P\left(x\right)z^{2}
\]


これはBernoulliの$m=2$の場合である。

両辺に$-2y^{-3}$をかけて、
\[
z'-2xz=-2x
\]


\begin{align*}
z & =\e^{x^{2}}\left(-2\int x\e^{-x^{2}}\mathrm{d}x+C\right)\\
 & =1+C\e^{x^{2}}
\end{align*}


\[
z=\exp\left[\int\left(2P\left(x\right)f\left(x\right)+Q\left(x\right)\right)\mathrm{d}x\right]-\int P\left(x\right)\exp\left[\int\left(2P\left(x\right)f\left(x\right)+Q\left(x\right)\right)\mathrm{d}x\right]\mathrm{d}x+C
\]



\paragraph{例}

\[
y'=y^{2}+\left(2-x\right)y-2x+1\;\left(y=x\right)
\]


$z=y-x$とおくと、
\[
z'+1=\left(z+x\right)^{2}+\left(2-x\right)\left(z+x\right)-2x+1
\]


$z'-\left(x+2\right)z=z^{2},z=\frac{1}{u}$とおく。

\[
u'+\left(x+2\right)u=-1\Rightarrow u=\e^{-\frac{x^{2}}{2}-2x}\left(-\int\e^{\frac{x^{2}}{2}+2x}\mathrm{d}x+C\right)
\]


\[
y=\frac{\e^{\frac{x^{2}}{2}}+2x}{-\int\e^{\frac{x^{2}}{2}}\mathrm{d}x+C}+x
\]



\paragraph{例}

\[
y'-\left(x-1\right)y^{2}+\left(2x-1\right)y=x\;\left(y=1\right)
\]


$z=y-1$とおく。

\[
z'-\left(x-1\right)\left(z+1\right)^{2}+\left(2x-1\right)\left(;+1\right)-x=0
\]


$z'+z=\left(x-1\right)z^{2},u=\frac{1}{z}$とおく。

\[
u'-u=1-x
\]


\begin{align*}
u\left(x\right) & =\e^{\int^{x}\mathrm{d}x}\left[\int^{x}\left(1-x\right)\e^{-\int^{x}\mathrm{d}x}\mathrm{d}x+C\right]\\
 & =\e^{x}\left(x\e^{-x}+C\right)=x+C\e^{x}
\end{align*}


\[
y=1+\frac{1}{x+C\e^{x}}
\]



\paragraph{d'Alembert型微分方程式}

\[
y=xf\left(y'\right)+g\left(y'\right)
\]


$P=y'$と置くと$y=xf\left(P\right)+g\left(P\right)$

両辺を$x$で微分して、
\[
P=f\left(P\right)+\left[xf'\left(P\right)+g'\left(P\right)\right]\d Px
\]


ここで$f\left(P\right)\neq P$とすると、
\[
\d xP=\frac{xf'\left(P\right)+g'\left(P\right)}{P-f\left(P\right)}=-\frac{f'\left(P\right)}{f\left(P\right)-P}x+\frac{g'\left(P\right)}{f\left(P\right)-P}
\]


\[
x=\left[-\int\frac{g'\left(P\right)}{P-f\left(P\right)}\e^{\int\frac{f'\left(P\right)}{P-f\left(P\right)}\mathrm{d}P}\mathrm{d}P+C\right]\exp\left[-\int\frac{f'\left(P\right)}{P-f\left(P\right)}\mathrm{d}P\right]
\]



\paragraph{例}

\[
y=xy'\left(y'+1\right)+\frac{1}{y'}
\]


$P=y'$とおく。($x$で微分)
\[
y=xP\left(P+1\right)+\frac{1}{P}
\]


\[
P=P\left(P+1\right)+\frac{1}{P}
\]


\[
P=P\left(P+1\right)+\left[x\left(2P+1\right)-\frac{1}{P^{2}}\right]\d Px
\]


\[
\d xP+\frac{2P+1}{P^{2}}x=\frac{1}{P^{4}}
\]


\begin{align*}
x & =\left[\int\frac{1}{P^{4}}\e^{\int\frac{2P+1}{P^{2}}\mathrm{d}P}\mathrm{d}P+C\right]\e^{-\int\frac{2P+1}{P^{2}}\mathrm{d}P}\\
 & =\frac{1}{P^{2}}\left(1+C\e^{\frac{1}{P}}\right)
\end{align*}



\paragraph{Clairaut型微分方程式}

$f\left(P\right)=P$のとき$P=y'$とおいて$x$で微分

\[
\left[x+g'\left(P\right)\right]P'=0
\]
$x+g'\left(P\right)=0$か$P'=0$

$P'=0$のとき$P=C\Rightarrow y=Cx+g\left(C\right)$

$x+g'\left(P\right)=0$のときは$y=xP+g\left(P\right)$と連立して$P$を消去して特異解$y\left(x\right)$を得る。


\paragraph{例}

\[
y=xy'-\left(y'\right)^{2}
\]


\[
y=xP-P^{2}\Rightarrow\left(x-2P\right)P'=0
\]


$x-2P=0$か$P'=0$

$P'=0$のとき$y=Cx-C^{2}$

$x-2P=0$のとき$y=xP-P^{2}$と連立して$y=\frac{x^{2}}{4}$


\paragraph{全微分方程式て完全微分方程式}

\[
\d yx=-\frac{P\left(x,y\right)}{Q\left(x,y\right)}\Rightarrow P\left(x,y\right)\mathrm{d}x+Q\left(x,y\right)\mathrm{d}y=0
\]


ある関数$\Phi\left(x,y\right)$が存在して$\pd{\Phi}x=P,\pd{\Phi}y=Q$のとき完全微分方程式
\[
\mathrm{d}\Phi\left(x,y\right)=P\left(x,y\right)\mathrm{d}x+Q\left(x,y\right)\mathrm{d}y
\]


\[
\pd Py=\pd Qx
\]


\[
\int_{x_{0}}^{x}P\left(x,y\right)\mathrm{d}x+\int_{y_{0}}^{y}Q\left(x_{0},y\right)\mathrm{d}y+C
\]


積分因子$\mu\left(x,y\right)$をかけると完全微分方程式になる全微分方程式

条件は$\partial_{y}\left(\mu P\right)=\partial_{x}\left(\mu Q\right)$

もしも$\mu\left(x\right)$

\[
\mu\partial_{y}P=\partial_{x}\mu Q+\mu\partial_{x}Q
\]


\[
\frac{\partial_{y}P-\partial_{x}Q}{Q}=\frac{\partial_{x}\mu}{\mu}=g\left(x\right)\Rightarrow\mu\left(x\right)=\e^{\int Q\left(x\right)\mathrm{d}x}
\]



\paragraph{例}

\[
\left(2x+2y\right)\mathrm{d}x+\left(2x+\mathrm{e}^{y}\right)\mathrm{d}y=0
\]


\[
\partial_{y}\left(2x+2y\right)=\partial_{x}\left(2x+\e^{y}\right)
\]


\begin{align*}
u & =\int_{0}^{x}\left(2x+2y\right)\mathrm{d}x+\left.\int_{0}^{y}\left(2x+\e^{y}\right)\right|_{x=0}\mathrm{d}y\\
 & =x^{2}+2yx+\e^{y}-1=C
\end{align*}


\[
\mathrm{d}x^{2}+2\mathrm{d}\left(xy+\mathrm{d}\e^{y}\right)=\mathrm{d}\left(x^{2}+2xy+\e^{y}\right)=0
\]


\[
x^{2}+2xy+\e^{y}=C'
\]



\paragraph{例}

\[
\mathrm{d}x+2xy\mathrm{d}y=0
\]


\[
\partial_{y}P-\partial_{x}Q=-\partial_{x}\left(2xy\right)=-2y\neq0
\]


\[
\frac{\partial_{y}P-\partial_{x}Q}{Q}=\frac{-2y}{2xy}=-\frac{1}{x}\Rightarrow\mu\left(x\right)=\e^{-\int\frac{\mathrm{d}x}{x}}=\e^{\ln x}=\frac{1}{x}
\]


\[
\frac{1}{x}\mathrm{d}x+2y\mathrm{d}y=\mathrm{d}\left(\ln x+y^{2}\right)=0
\]


\[
\ln x+y^{2}=C\Rightarrow y=\pm\sqrt{C-\ln x}
\]



\paragraph{例}

\[
\e^{y}\mathrm{d}x+x\e^{y}\mathrm{d}y+2z\mathrm{d}z=0
\]


\[
\mathrm{d}\left(x\e^{y}+z^{2}\right)=0\Rightarrow x\e^{y}+z^{2}=C
\]



\paragraph{例}

\begin{align*}
 & \left(x+\frac{y^{2}}{x}\right)\mathrm{d}x+2y\ln x\mathrm{d}y+u^{2}\mathrm{d}u=0\\
\Rightarrow & \frac{1}{2}\mathrm{d}x^{2}+y^{2}\mathrm{d}\ln x+\ln x\mathrm{d}y'+\frac{u^{2}}{3}\mathrm{d}u^{3}\\
\Rightarrow & \frac{1}{2}\mathrm{d}x^{2}+\mathrm{d}\left(y^{2}\ln x\right)+\frac{1}{3}u^{2}\mathrm{d}u^{2}=0\\
 & \frac{1}{2}x^{2}+y^{2}\ln x+\frac{1}{3}y^{2}=C
\end{align*}

\end{document}
