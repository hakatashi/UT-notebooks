%% LyX 2.2.2 created this file.  For more info, see http://www.lyx.org/.
%% Do not edit unless you really know what you are doing.
\documentclass[oneside,english]{book}
\usepackage[T1]{fontenc}
\usepackage[utf8]{inputenc}
\usepackage[a5paper]{geometry}
\geometry{verbose,tmargin=2cm,bmargin=2cm,lmargin=1cm,rmargin=1cm}
\setcounter{secnumdepth}{3}
\setcounter{tocdepth}{3}
\setlength{\parskip}{\smallskipamount}
\setlength{\parindent}{0pt}
\usepackage{textcomp}
\usepackage{amsmath}
\usepackage{amssymb}
\usepackage{graphicx}
\usepackage{esint}

\makeatletter
%%%%%%%%%%%%%%%%%%%%%%%%%%%%%% User specified LaTeX commands.
\usepackage[dvipdfmx]{hyperref}
\usepackage[dvipdfmx]{pxjahyper}

\makeatother

\usepackage{babel}
\begin{document}

\title{2016-A 数学ID}

\author{教員: 入力: 高橋光輝}

\maketitle
\global\long\def\pd#1#2{\frac{\partial#1}{\partial#2}}
\global\long\def\d#1#2{\frac{\mathrm{d}#1}{\mathrm{d}#2}}
\global\long\def\pdd#1#2{\frac{\partial^{2}#1}{\partial#2^{2}}}
\global\long\def\dd#1#2{\frac{\mathrm{d}^{2}#1}{\mathrm{d}#2^{2}}}
\global\long\def\ddd#1#2{\frac{\mathrm{d}^{3}#1}{\mathrm{d}#2^{3}}}
\global\long\def\e{\mathrm{e}}
\global\long\def\i{\mathrm{i}}
\global\long\def\j{\mathrm{j}}
\global\long\def\grad{\operatorname{grad}}
\global\long\def\rot{\operatorname{rot}}
\global\long\def\div{\operatorname{div}}
\global\long\def\diag{\operatorname{diag}}
\global\long\def\rank{\operatorname{rank}}
\global\long\def\prob{\operatorname{Prob}}
\global\long\def\cov{\operatorname{Cov}}
\global\long\def\when#1{\left.#1\right|}


\section*{第1回}

\paragraph{この講義の内容}
\begin{enumerate}
\item 常微分方程式

\begin{itemize}
\item 一変数の微分に関する方程式
\end{itemize}
\item ベクトル解析(3回程度)

\begin{itemize}
\item ベクトルの内積・外積、微分
\end{itemize}
\item 変分法(3回程度)

\begin{itemize}
\item オイラー・ラグランジュ方程式
\end{itemize}
\end{enumerate}

\paragraph{副読本}

ジョージ・アルフケン

ハンス・ウェーバー

\paragraph{授業の流れ}
\begin{enumerate}
\item 常微分方程式

\begin{enumerate}
\item 変数分離系
\item 同次系
\item 定数添加法
\item 完全微分方程式
\item 線形微分方程式
\item 連立微分方程式
\end{enumerate}
\item ベクトル解析

\begin{enumerate}
\item 内積・外積
\item gradient $\nabla\phi$
\item divergence $\nabla\cdot A$
\item rotation $\nabla\times A$
\item Greenの定理
\item Stokesの定理
\item Gauussの定理
\item 直行曲線座標
\item 円柱座標
\item 3次元極座標
\end{enumerate}
\item 変分法

\begin{enumerate}
\item Eular-lagrange方程式
\item 解析力学
\item Lagrangian
\item Hamiltonian
\item Lagrangeの未定乗数法
\end{enumerate}
\end{enumerate}

\paragraph{成績評価}

期末試験+何回かのレポート

\chapter{第一章 常微分方程式}

\paragraph{変数分離系}

関数$u\left(x\right)$を考える。

\[
\d ux=g\left(x\right)h\left(u\right)
\]
と表せるとき、変数分離系という。

\begin{align*}
\frac{\mathrm{d}u}{h\left(u\right)} & =g\left(x\right)\mathrm{d}x\\
\int^{u}\frac{\mathrm{d}u}{h\left(u\right)} & =\int^{x}g\left(x\right)\mathrm{d}x
\end{align*}


\paragraph{放射性物質の自然崩壊}

$u\left(t\right)$: 放射性物質の量

$\gamma$: 消滅率

\begin{align*}
\mathrm{d}u & =\lim_{\delta\rightarrow0}\left[u\left(t+\delta t\right)-u\left(t\right)\right]\\
 & =-\gamma u\left(t\right)\mathrm{d}t
\end{align*}
\[
\d ut=-\gamma u\Rightarrow u\left(t\right)=u_{0}\mathrm{e}^{-\gamma t}
\]

半減期$\tau$は
\[
u\left(t\right)=u_{0}\mathrm{e}^{-\gamma t}=\frac{1}{2}u_{0}
\]
\[
-\gamma t=-\ln2
\]
\[
\tau-\frac{\ln2}{\gamma}
\]


\paragraph{Malthus模型}

バクテリアの増殖を考える。

$\alpha$: 増殖率

\begin{align*}
\mathrm{d}u=\alpha u\left(t\right)\mathrm{d}t & \Rightarrow\d ut=\alpha u\\
 & \Rightarrow u\left(t\right)=u_{0}\mathrm{e}^{\alpha t}
\end{align*}


\paragraph{logistic方程式}

増殖率$\alpha$を$\alpha-\beta u$としたもの。

\[
\d ut=\left(\alpha-\beta u\right)u=\alpha u-\beta u^{2}
\]

変数分離を用いて解く。

\[
\int_{u_{0}}^{u}\frac{\mathrm{d}u'}{u'\left(\alpha-\beta u'\right)}=\int_{t_{0}}^{t}\mathrm{d}t'=t-t_{0}
\]

部分分数分解を用いて、
\[
\frac{1}{u\left(\alpha-\beta u\right)}=\frac{A}{u}+\frac{B}{\alpha-\beta u}=\frac{A\alpha-\left(B-\beta A\right)u}{u\left(\alpha-\beta u\right)}
\]

$A=\frac{1}{\alpha},B=\frac{\beta}{\alpha}$より、
\begin{align*}
\int_{u_{0}}^{u}\frac{\mathrm{d}u'}{u'\left(\alpha-\beta u'\right)} & =\frac{1}{\alpha}\int_{u_{0}}^{u}\left(\frac{1}{u'}+\frac{\beta}{\alpha-\beta u'}\right)\mathrm{d}u'\\
 & =\frac{1}{\alpha}\left[\log\frac{u}{u_{0}}+\log\frac{\alpha-\beta u_{0}}{\alpha-\beta u}\right]\\
 & =\frac{1}{\alpha}\log\frac{u}{u_{0}}\frac{\alpha-\beta u_{0}}{\alpha-\beta u}\\
\mathrm{e}^{\alpha\left(t-t_{0}\right)} & =\frac{u}{u_{0}}\frac{\alpha-\beta u_{0}}{\alpha-\beta u}
\end{align*}
\[
\left(\alpha-\beta u_{0}+\beta u_{0}\mathrm{e}^{u\left(t-t_{0}\right)}\right)u\left(t\right)=\alpha u_{0}\mathrm{e}^{u\left(t-t_{0}\right)}
\]
\[
u\left(t\right)=\frac{\alpha u_{0}\mathrm{e}^{u\left(t-t_{0}\right)}}{\alpha-\beta u_{0}+\beta u_{0}\mathrm{e}^{u\left(t-t_{0}\right)}}
\]


\paragraph{Lotka-Volterra方程式}

被食者と捕食者がいる生態系を考える。

被食者: $x\left(t\right)$

捕食者: $y\left(t\right)$

\begin{align*}
\d xt & =x\left(\alpha-\beta y\right)\\
\d yt & =-y\left(\gamma-\delta x\right)
\end{align*}

$\alpha,\beta,\gamma,\delta$は正の実数

このような式を非線形連立方程式という。→原則、非線形は解けない

平衡点を考える。

\[
\d xt=\d yt=0
\]
\[
\begin{cases}
x\left(\alpha-\beta y\right)=0\\
-y\left(\gamma-\delta x\right)=0
\end{cases}\Rightarrow\begin{cases}
\left(\begin{array}{c}
x=0\\
y=0
\end{array}\right) & \text{絶滅状態}\\
\left(\begin{array}{c}
x=\frac{\gamma}{\delta}\\
y=\frac{\alpha}{\beta}
\end{array}\right) & \text{平衡状態}
\end{cases}
\]

$X=x-\frac{\gamma}{\delta},Y=y-\frac{\alpha}{\beta}$($\left|X\right|\ll1,\left|Y\right|\ll1$)とし、
\begin{align*}
\d Xt & =\left(X+\frac{\gamma}{\delta}\right)\left(\alpha-\beta\left(Y+\frac{\alpha}{\beta}\right)\right)\simeq-\frac{\beta\gamma}{\delta}Y\\
\d Yt & =-\left(Y+\frac{\alpha}{\beta}\right)\left(\gamma-\delta\left(X+\frac{\gamma}{\delta}\right)\right)\simeq\frac{\alpha\delta}{\gamma}X
\end{align*}
\[
X'=-\frac{\alpha\gamma}{\delta}Y'=-\alpha\gamma X
\]
\begin{align*}
X & =c\cos\left(\sqrt{\alpha\gamma}t+\theta_{0}\right)\\
Y & =-\frac{\delta}{\beta\gamma}X'=\sqrt{\frac{\alpha}{\gamma}}\frac{\delta}{\beta}c\sin\left(\sqrt{\alpha\gamma}t+\theta_{0}\right)
\end{align*}

このように一般に解けない方程式も、平衡点の周りの近似解は求めることができる。

\paragraph{Liourille方程式}

\begin{align*}
 & y''+P\left(x\right)y'+Q\left(y\right)\left(y'\right)^{2}=0\\
\Rightarrow & \frac{y''}{y'}+P\left(x\right)+Q\left(y\right)y'=0
\end{align*}

\begin{align*}
 & \d{}x\left[\log y'+\int^{x}P\left(x'\right)\mathrm{d}x'+\int^{y}Q\left(y\right)\mathrm{d}y\right]=0\\
\Rightarrow & \log y'+\int^{x}P\left(x\right)\mathrm{d}x+\int^{y}Q\left(y\right)\mathrm{d}y=C\\
\Rightarrow & \d yx=\exp\left[-\int^{x}P\left(x\right)\mathrm{d}x\right]\exp\left[-\int^{y}Q\left(y\right)\mathrm{d}y\right]\\
\Rightarrow & \int^{y}\exp\left[\int^{y}Q\left(y\right)\mathrm{d}y\right]\mathrm{d}y=C_{1}\int^{x}\exp\left[-\int^{x}P\left(x\right)\mathrm{d}x\right]\mathrm{d}x+C_{2}
\end{align*}

例1

\[
y''+2xy'+2y\left(y'\right)^{2}=0
\]

$P\left(x\right)=2x,Q\left(y\right)=2y$とし、
\[
\int^{y}\mathrm{e}^{y}\mathrm{d}y=C_{1}\int^{x}\mathrm{e}^{-x^{2}}\mathrm{d}x+C_{2}
\]

例2

\[
\d yx=\sin x\tan y\Rightarrow\sin\mathrm{d}x=\frac{\cos y}{\sin y}\mathrm{d}y=\frac{\mathrm{d}\sin y}{\sin y}=\mathrm{d}\log\left|\sin y\right|
\]

\[
\cos x=-\log\left|\sin y\right|+C\Rightarrow\sin y=C\mathrm{e}^{-\cos x}\Rightarrow y=\arcsin\left[C\mathrm{e}^{-\cos x}\right]
\]


\section*{第2回}

\paragraph{同次型微分方程式}

\[
\d yx=f\left(\frac{y}{x}\right)
\]

変換$x\mapsto\lambda x,y\mapsto\lambda y$に対して不変

$u=\frac{y}{x}$とおく。

\[
\d yx=y'=\left(xu\right)'=xu'+u
\]
\[
xu'+u=f\left(u\right)\Rightarrow\d ux=\frac{f\left(u\right)-u}{x}
\]

\[
\int^{u}\frac{\mathrm{d}u}{f\left(u\right)-u}=\int^{u}\frac{\mathrm{d}x}{x}=\log\left|x\right|+c
\]

\[
x\e^{c}=\exp\left[\int^{u}\frac{\mathrm{d}u}{f\left(u\right)-u}\right]\equiv G\left(u\right)
\]

\[
G\left(\frac{y}{x}\right)=Cx\:\left(C=\e^{c}\right)
\]


\paragraph{例}

\[
\d yx=\e^{\frac{y}{x}}+\frac{y}{x}
\]

$y=xu$とおく。

\[
xu'+u=\e^{u}+u\Rightarrow\e^{-u}\mathrm{d}u=\frac{\mathrm{d}x}{x}
\]

\[
-\e^{-u}=\log\left|x\right|-C
\]

\[
\e^{-\frac{y}{x}}=-\log\left|x\right|+C
\]
\[
-\frac{y}{x}=\log\left|-\log\left|x+C\right|\right|
\]

\[
y=-x\log\left|-\log\left|x\right|+C\right|
\]


\paragraph{例}

\[
y'=\frac{x-2y+3}{2x+y-4}
\]

連立方程式

\[
\begin{cases}
\alpha-2\beta+3=0\\
2\alpha+\beta-4=0
\end{cases}\rightarrow\alpha=1,\beta=2
\]

$x=\xi+1,y=\eta+2$とおく。

\[
\d yx=\d{\eta}{\xi},\frac{x-2y+3}{2x+y-4}=\frac{\xi-2\eta}{2\xi+\eta}=\frac{1-2\frac{\eta}{\xi}}{2+\frac{\eta}{\xi}}
\]

$\eta=\xi z$とおく。

\[
\xi\d z{\xi}+z=\frac{1-2z}{2+z}\Rightarrow\frac{z+4}{z^{2}+4z-1}\mathrm{d}z=-\frac{2}{\xi}\mathrm{d}\xi
\]

\[
\log\left|z^{2}+4z-1\right|=-2\log\left|\xi\right|+c
\]
\[
\Rightarrow z^{2}+4z-1=\e^{c}\xi^{-2}
\]

\[
\left(y-2\right)^{2}+4\left(y-2\right)\left(x-1\right)-\left(x-1\right)^{2}=\e^{c}
\]


\paragraph{例}

$P\left(x,y\right)$と$Q\left(x,y\right)$が$x,y$について$k$次。

\[
\d yx=\frac{P\left(x,y\right)}{Q\left(x,y\right)}
\]

$k=1$のとき
\[
\d yx=\frac{p_{0}x+p_{1}y}{q_{0}x+q_{1}y}
\]

$k=2$のとき
\[
\d yx=\frac{p_{0}x^{2}+p_{1}xy+p_{2}y^{2}}{q_{0}x^{2}+q_{1}xy+q_{2}y^{2}}
\]

$u=\frac{y}{x}$とおくと、
\[
u+xu'=\frac{p_{0}+p_{1}u}{q_{0}+q_{1}u}
\]

\[
\Rightarrow x\d ux=\frac{p_{0}+p_{1}u}{q_{0}+q_{1}u}-u=\frac{p_{0}+\left(p_{1}-q_{0}\right)u-q_{1}u^{2}}{q_{2}+q_{1}u}
\]

\[
\log\left|x\right|=\int^{u}\frac{q_{2}+q_{1}u}{p_{0}+\left(p_{1}-q_{0}\right)u-q_{1}u^{2}}\mathrm{d}u\equiv G\left(u\right)
\]

\[
x=\exp\left[G\left(\frac{y}{x}\right)\right]
\]

$k=2$の場合も同様。以下一般で成り立つ。

\paragraph{例}

\[
\d yx=f\left(\frac{ax+by+p}{cx+dy+q}\right)
\]

\[
\begin{cases}
ax+by+p=0\\
cx+dy+q=0
\end{cases}
\]

\begin{enumerate}
\item $ad-bc\neq0$のとき $\left(x,y\right)=\left(\alpha,\beta\right)$

$x=\xi+\alpha,y=\eta+\beta$とおく。

\[
\d{\eta}{\xi}+f\left(\frac{a\xi+b\eta}{c\xi+d\eta}\right)=f\left(\frac{a+b\frac{\eta}{\xi}}{c+d\frac{\eta}{\xi}}\right)\equiv g\left(\frac{\eta}{\xi}\right)
\]

\item $ad-bc=0$で$c$と$d$の少なくとも一方が0でないとき

$\frac{ax+by}{cx+dy}=k\left(\text{定数}\right),z=cx+dy+q$とおく。

\begin{align*}
ax+by+p & =k\left(cx+dy\right)+p\\
 & =kz+p-kq
\end{align*}

\item $c=d=0$のとき

$z=-\frac{ax+by+p}{q}$とおく。

\[
\d zx=\frac{a+by'}{q}=\frac{a+bf\left(z\right)}{q}
\]

\[
\int\frac{\mathrm{d}z}{a+bf\left(z\right)}=\frac{x}{q}+C
\]

\end{enumerate}

\paragraph{例}

同次形でないもの

\[
\d yx=\frac{xy}{x^{2}+y}
\]

$\left(x\mapsto\lambda x,y\mapsto\lambda^{2}y\right)$に対して不変。

$u=\frac{y}{x^{2}}$とおく。
\[
\d yx=x^{2}u'+2xu
\]

\[
\frac{xy}{x^{2}+y}=\frac{x^{2}u}{x^{3}+x^{2}u}=\frac{xu}{1+u}
\]

\[
\d ux=\frac{u+2u^{2}}{x\left(1+u\right)}
\]

\[
\frac{1+u}{u\left(1+2u\right)}\mathrm{d}u=-\frac{\mathrm{d}x}{x}
\]

部分分数分解を行う。

\begin{align*}
\frac{1+u}{u\left(1+2u\right)}\mathrm{d}u & =\frac{1+u}{u}\mathrm{d}u-\frac{2\left(1+u\right)}{1+2u}\mathrm{d}u\\
 & =\frac{\mathrm{d}u}{u}+\frac{1+2u-2\left(1+u\right)}{1+2u}\mathrm{d}u\\
 & =\frac{1}{2}\mathrm{d}\left[\log u^{2}-\log\left(1+2u\right)\right]
\end{align*}

\begin{align*}
 & \log\left|\frac{u^{2}}{1+2u}\right|=-2\log\left|x\right|+c\\
\Rightarrow & \frac{x^{2}u^{2}}{1+2u}=c'\Rightarrow\frac{y^{2}}{x^{2}+2y}=c'
\end{align*}


\paragraph{例}

\[
\d yz=\frac{x^{2}y}{x^{2}+y}
\]

$\left(x\mapsto\lambda x,y\mapsto\lambda^{2}x\right)$に対して不変。

$u=\frac{y}{x^{2}}$とおく。

\[
y'=x^{3}u'+3x^{2}u
\]

\[
x^{3}u'+3x^{2}u=\frac{x^{5}u}{x^{3}+x^{3}u}=\frac{x^{2}u}{1+u}
\]

\[
\d ux=-\frac{u\left(2+3u\right)}{x\left(1+u\right)}
\]

\[
\frac{1+u}{u\left(2+3u\right)}\mathrm{d}u=-\frac{\mathrm{d}x}{x}
\]

\[
\frac{1+u}{u\left(2+3u\right)}=\frac{1}{2}\left(\frac{1+u}{u}-\frac{3+3u}{2+3u}\right)=\frac{1}{2}\left(\frac{1}{u}+1-\frac{3+3u}{2+3u}\right)
\]

\begin{align*}
\log\left|x\right|+c & =\frac{1}{2}\int^{u}\left(1u-\frac{1}{2+3u}\right)\mathrm{d}u\\
 & =\frac{1}{2}\left[\log u-\frac{1}{3}\log\left(2+3u\right)\right]\\
 & =-\frac{1}{6}\log\frac{u^{3}}{2+3u}\\
\Rightarrow\frac{x^{6}u^{3}}{2+3u} & =6C\\
\Rightarrow\frac{y^{3}}{2x^{2}+3y} & =6C
\end{align*}


\paragraph{定数変化法}

\[
\d ux+P\left(x\right)u=Q\left(x\right)
\]

$Q\left(x\right)=0$のとき、
\[
u\left(x\right)=C\exp\left[-\int P\left(x\right)\mathrm{d}x\right]
\]

定数$C$を$x$の関数と見て計算してみる。

\[
C\mapsto C\left(x\right)\Rightarrow u\left(x\right)=C\left(x\right)\exp\left[-\int P\left(x\right)\mathrm{d}x\right]
\]

\[
\d ux=\d Cx\exp\left[-\int P\left(x\right)\mathrm{d}x\right]P\left(x\right)u
\]

\[
C'\e^{-\int^{x}P\left(x\right)\mathrm{d}x}-Pu+Pu^{2}Q
\]

\[
\mathrm{d}C=Q\left(x\right)\e^{\int^{x}P\left(x\right)\mathrm{d}x}\mathrm{d}x
\]

\[
C\left(x\right)=\int^{x}Q\left(x\right)\e^{\int^{x}P\left(x\right)\mathrm{d}x}\mathrm{d}x+c
\]

\[
u\left(x\right)=\e^{\int^{x}P\left(x\right)\mathrm{d}x}\left[\int^{x}Q\left(x\right)\e^{\int^{x}P\left(x\right)\mathrm{d}x}\mathrm{d}x+c\right]
\]


\paragraph{例}

\[
xy^{2}-2y=x^{3}\cos x
\]

$xy'-2y=0$を解く。$y=Cx^{2}$

$y=C\left(x\right)x^{2}$

\[
C'x^{3}=x^{3}\cos x\Rightarrow C\left(x\right)=\sin x+c
\]

\[
y=x^{2}\left(\sin x+c\right)
\]


\paragraph{来週やること}

Bernoulli型の微分方程式

\[
y'+P\left(x\right)y=Q\left(x\right)y^{m}
\]

Riccati

\[
y'=P\left(x\right)y^{2}+Q\left(x\right)y+R\left(x\right)
\]

d'Alembert

\[
y=xf\left(y^{2}\right)+g\left(y'\right)
\]

Clairaut

\[
y=xp+g\left(y'\right)
\]


\section*{第3回}

\paragraph{レポート課題}

\[
\frac{\mathrm{d}y}{\mathrm{d}x}=\frac{3x-2y+1}{2x+3y-2}
\]
を解け。ただし$y\left(x\right)$の形まで解くこと。

\paragraph{Bernoulli型微分方程式}

\[
y'+P\left(x\right)y=Q\left(x\right)y^{m}
\]

$z=y^{k}$とおくと$z'=ky^{k-1}y'$

両辺に$ky^{k-1}$を掛けて、
\[
z'+kP\left(x\right)z=kQ\left(x\right)y^{x-1+m}
\]

$k=1-m$とおく。

\[
z'+\left(1-m\right)P\left(x\right)z=\left(1-m\right)Q\left(x\right)
\]

\[
z=\e^{-\left(1-n\right)\int P\left(x\right)\mathrm{d}x}\left[\left(1-m\right)\int Q\left(x\right)\e^{\left(1-m\right)\int P\left(x\right)\mathrm{d}x}\mathrm{d}x+C\right]
\]

\[
y=\e^{-\int P\left(x\right)\mathrm{d}x}\left[\left(1-m\right)\int Q\left(x\right)\e^{\left(1-m\right)\int P\left(x\right)\mathrm{d}x}\mathrm{d}x+C\right]^{\frac{1}{1-m}}
\]


\paragraph{例}

\[
y'+xy=xy^{3}
\]

$m=3$なので$z=y^{-2}$とおく。

\[
z'=-2y^{-3}y'
\]

両辺に$-2y^{-3}$をかけて、
\[
z'-2xz=-2x
\]

\begin{align*}
z & =\e^{x^{2}}\left(-2\int x\e^{-x^{2}}\mathrm{d}x+C\right)\\
 & =1+C\e^{x^{2}}
\end{align*}

\[
y=\frac{1}{\sqrt{1+C\e^{x^{2}}}}
\]


\paragraph{Riccati型微分方程式}

\[
y'=P\left(x\right)y^{2}+Q\left(x\right)y+R\left(x\right)
\]

$y=f\left(x\right)$を一つの特殊解とする。

\[
z=y-f\left(x\right)
\]

\begin{align*}
z' & =y'-f'\left(x\right)\\
 & =P\left(x\right)\left(z+f\left(x\right)\right)^{2}+Q\left(x\right)\left(z+f\left(x\right)\right)+R\left(x\right)-f'\left(x\right)\\
 & =P\left(x\right)z^{2}+\left[2P\left(x\right)f\left(x\right)+Q\left(x\right)\right]z+\left[P\left(x\right)f^{2}\left(x\right)+Q\left(x\right)f\left(x\right)+R\left(x\right)-f'\left(x\right)\right]
\end{align*}

\[
z'-\left[2P\left(x\right)f'\left(x\right)+Q\left(x\right)\right]z=P\left(x\right)z^{2}
\]

これはBernoulliの$m=2$の場合である。

両辺に$-2y^{-3}$をかけて、
\[
z'-2xz=-2x
\]

\begin{align*}
z & =\e^{x^{2}}\left(-2\int x\e^{-x^{2}}\mathrm{d}x+C\right)\\
 & =1+C\e^{x^{2}}
\end{align*}

\[
z=\exp\left[\int\left(2P\left(x\right)f\left(x\right)+Q\left(x\right)\right)\mathrm{d}x\right]-\int P\left(x\right)\exp\left[\int\left(2P\left(x\right)f\left(x\right)+Q\left(x\right)\right)\mathrm{d}x\right]\mathrm{d}x+C
\]


\paragraph{例}

\[
y'=y^{2}+\left(2-x\right)y-2x+1\;\left(y=x\right)
\]

$z=y-x$とおくと、
\[
z'+1=\left(z+x\right)^{2}+\left(2-x\right)\left(z+x\right)-2x+1
\]

$z'-\left(x+2\right)z=z^{2},z=\frac{1}{u}$とおく。

\[
u'+\left(x+2\right)u=-1\Rightarrow u=\e^{-\frac{x^{2}}{2}-2x}\left(-\int\e^{\frac{x^{2}}{2}+2x}\mathrm{d}x+C\right)
\]

\[
y=\frac{\e^{\frac{x^{2}}{2}}+2x}{-\int\e^{\frac{x^{2}}{2}}\mathrm{d}x+C}+x
\]


\paragraph{例}

\[
y'-\left(x-1\right)y^{2}+\left(2x-1\right)y=x\;\left(y=1\right)
\]

$z=y-1$とおく。

\[
z'-\left(x-1\right)\left(z+1\right)^{2}+\left(2x-1\right)\left(;+1\right)-x=0
\]

$z'+z=\left(x-1\right)z^{2},u=\frac{1}{z}$とおく。

\[
u'-u=1-x
\]

\begin{align*}
u\left(x\right) & =\e^{\int^{x}\mathrm{d}x}\left[\int^{x}\left(1-x\right)\e^{-\int^{x}\mathrm{d}x}\mathrm{d}x+C\right]\\
 & =\e^{x}\left(x\e^{-x}+C\right)=x+C\e^{x}
\end{align*}

\[
y=1+\frac{1}{x+C\e^{x}}
\]


\paragraph{d'Alembert型微分方程式}

\[
y=xf\left(y'\right)+g\left(y'\right)
\]

$P=y'$と置くと$y=xf\left(P\right)+g\left(P\right)$

両辺を$x$で微分して、
\[
P=f\left(P\right)+\left[xf'\left(P\right)+g'\left(P\right)\right]\d Px
\]

ここで$f\left(P\right)\neq P$とすると、
\[
\d xP=\frac{xf'\left(P\right)+g'\left(P\right)}{P-f\left(P\right)}=-\frac{f'\left(P\right)}{f\left(P\right)-P}x+\frac{g'\left(P\right)}{f\left(P\right)-P}
\]

\[
x=\left[-\int\frac{g'\left(P\right)}{P-f\left(P\right)}\e^{\int\frac{f'\left(P\right)}{P-f\left(P\right)}\mathrm{d}P}\mathrm{d}P+C\right]\exp\left[-\int\frac{f'\left(P\right)}{P-f\left(P\right)}\mathrm{d}P\right]
\]


\paragraph{例}

\[
y=xy'\left(y'+1\right)+\frac{1}{y'}
\]

$P=y'$とおく。($x$で微分)
\[
y=xP\left(P+1\right)+\frac{1}{P}
\]

\[
P=P\left(P+1\right)+\frac{1}{P}
\]

\[
P=P\left(P+1\right)+\left[x\left(2P+1\right)-\frac{1}{P^{2}}\right]\d Px
\]

\[
\d xP+\frac{2P+1}{P^{2}}x=\frac{1}{P^{4}}
\]

\begin{align*}
x & =\left[\int\frac{1}{P^{4}}\e^{\int\frac{2P+1}{P^{2}}\mathrm{d}P}\mathrm{d}P+C\right]\e^{-\int\frac{2P+1}{P^{2}}\mathrm{d}P}\\
 & =\frac{1}{P^{2}}\left(1+C\e^{\frac{1}{P}}\right)
\end{align*}


\paragraph{Clairaut型微分方程式}

$f\left(P\right)=P$のとき$P=y'$とおいて$x$で微分

\[
\left[x+g'\left(P\right)\right]P'=0
\]
$x+g'\left(P\right)=0$か$P'=0$

$P'=0$のとき$P=C\Rightarrow y=Cx+g\left(C\right)$

$x+g'\left(P\right)=0$のときは$y=xP+g\left(P\right)$と連立して$P$を消去して特異解$y\left(x\right)$を得る。

\paragraph{例}

\[
y=xy'-\left(y'\right)^{2}
\]

\[
y=xP-P^{2}\Rightarrow\left(x-2P\right)P'=0
\]

$x-2P=0$か$P'=0$

$P'=0$のとき$y=Cx-C^{2}$

$x-2P=0$のとき$y=xP-P^{2}$と連立して$y=\frac{x^{2}}{4}$

\paragraph{全微分方程式て完全微分方程式}

\[
\d yx=-\frac{P\left(x,y\right)}{Q\left(x,y\right)}\Rightarrow P\left(x,y\right)\mathrm{d}x+Q\left(x,y\right)\mathrm{d}y=0
\]

ある関数$\Phi\left(x,y\right)$が存在して$\pd{\Phi}x=P,\pd{\Phi}y=Q$のとき完全微分方程式
\[
\mathrm{d}\Phi\left(x,y\right)=P\left(x,y\right)\mathrm{d}x+Q\left(x,y\right)\mathrm{d}y
\]

\[
\pd Py=\pd Qx
\]

\[
\int_{x_{0}}^{x}P\left(x,y\right)\mathrm{d}x+\int_{y_{0}}^{y}Q\left(x_{0},y\right)\mathrm{d}y+C
\]

積分因子$\mu\left(x,y\right)$をかけると完全微分方程式になる全微分方程式

条件は$\partial_{y}\left(\mu P\right)=\partial_{x}\left(\mu Q\right)$

もしも$\mu\left(x\right)$

\[
\mu\partial_{y}P=\partial_{x}\mu Q+\mu\partial_{x}Q
\]

\[
\frac{\partial_{y}P-\partial_{x}Q}{Q}=\frac{\partial_{x}\mu}{\mu}=g\left(x\right)\Rightarrow\mu\left(x\right)=\e^{\int Q\left(x\right)\mathrm{d}x}
\]


\paragraph{例}

\[
\left(2x+2y\right)\mathrm{d}x+\left(2x+\mathrm{e}^{y}\right)\mathrm{d}y=0
\]

\[
\partial_{y}\left(2x+2y\right)=\partial_{x}\left(2x+\e^{y}\right)
\]

\begin{align*}
u & =\int_{0}^{x}\left(2x+2y\right)\mathrm{d}x+\left.\int_{0}^{y}\left(2x+\e^{y}\right)\right|_{x=0}\mathrm{d}y\\
 & =x^{2}+2yx+\e^{y}-1=C
\end{align*}

\[
\mathrm{d}x^{2}+2\mathrm{d}\left(xy+\mathrm{d}\e^{y}\right)=\mathrm{d}\left(x^{2}+2xy+\e^{y}\right)=0
\]

\[
x^{2}+2xy+\e^{y}=C'
\]


\paragraph{例}

\[
\mathrm{d}x+2xy\mathrm{d}y=0
\]

\[
\partial_{y}P-\partial_{x}Q=-\partial_{x}\left(2xy\right)=-2y\neq0
\]

\[
\frac{\partial_{y}P-\partial_{x}Q}{Q}=\frac{-2y}{2xy}=-\frac{1}{x}\Rightarrow\mu\left(x\right)=\e^{-\int\frac{\mathrm{d}x}{x}}=\e^{\ln x}=\frac{1}{x}
\]

\[
\frac{1}{x}\mathrm{d}x+2y\mathrm{d}y=\mathrm{d}\left(\ln x+y^{2}\right)=0
\]

\[
\ln x+y^{2}=C\Rightarrow y=\pm\sqrt{C-\ln x}
\]


\paragraph{例}

\[
\e^{y}\mathrm{d}x+x\e^{y}\mathrm{d}y+2z\mathrm{d}z=0
\]

\[
\mathrm{d}\left(x\e^{y}+z^{2}\right)=0\Rightarrow x\e^{y}+z^{2}=C
\]


\paragraph{例}

\begin{align*}
 & \left(x+\frac{y^{2}}{x}\right)\mathrm{d}x+2y\ln x\mathrm{d}y+u^{2}\mathrm{d}u=0\\
\Rightarrow & \frac{1}{2}\mathrm{d}x^{2}+y^{2}\mathrm{d}\ln x+\ln x\mathrm{d}y'+\frac{u^{2}}{3}\mathrm{d}u^{3}\\
\Rightarrow & \frac{1}{2}\mathrm{d}x^{2}+\mathrm{d}\left(y^{2}\ln x\right)+\frac{1}{3}u^{2}\mathrm{d}u^{2}=0\\
 & \frac{1}{2}x^{2}+y^{2}\ln x+\frac{1}{3}y^{2}=C
\end{align*}


\section*{第4回}

\paragraph{減衰振動と定数係数2階斉次線形常微分方程式}

\paragraph{抵抗のあるバネ振動}

\[
m\ddot{x}=-kx-\mu\dot{x}
\]
\[
\Rightarrow\ddot{x}+2\gamma x+w^{2}\dot{x}=0\left(\gamma=\frac{k}{2m},v=\sqrt{\frac{\mu}{m}}\right)
\]


\paragraph{LCR回路}

Kirchhoffの法則より、
\[
V_{R}+V_{L}+V_{C}=V
\]

\[
V_{R}=IR
\]

\[
V_{C}=\frac{q}{C}
\]

$q$: 電荷

\[
I=\d qt
\]

\[
V_{L}=L\d It
\]

\[
V=IR+\frac{q}{C}+L\d It
\]

時間で微分して
\[
L\dd It+R\d It+\frac{I}{C}=0
\]

\[
\ddot{I}+2\gamma\dot{I}+w^{2}I=0\:\left(\gamma=\frac{R}{2L},w=\frac{1}{\sqrt{LC}}\right)
\]

\[
\ddot{u}+2\gamma\dot{u}+w^{2}u=0
\]

$u=C\e^{\lambda t}$を代入

特性方程式 $\lambda^{2}+\gamma\lambda+w^{2}=0$

\[
\lambda=-\gamma\pm\sqrt{\gamma^{2}-w^{2}}
\]

$w$: 固有振動数

初期条件 $u=u_{0},\dot{u}=0$

判別式 $D=\gamma^{2}-w^{2}$

\[
V_{L}=L\d It
\]

\[
V=IR+\frac{q}{C}+L\d It
\]

\begin{enumerate}
\item $\gamma>w$の時 ($D>0$)

解は2つの実数$\lambda_{\pm}=-\gamma\pm\nu$ ($\nu=\sqrt{\gamma^{2}+w^{2}}$)

一般解は
\[
u=\e^{-\gamma t}\left[A\cosh\nu t+B\sinh\nu t\right]
\]

\[
\dot{u}=-\gamma\e^{-\gamma t}\left[A\cosh\nu t+B\sinh\nu t\right]+\e^{-\gamma t}\left[A\nu\sinh\nu t+B\nu\cosh\nu t\right]
\]

$u=u_{0}\Rightarrow A=u_{0}$

$\dot{u}=0\Rightarrow-\gamma A+B\nu=0\Rightarrow B=\frac{\gamma A}{\nu}=\frac{\gamma u_{0}}{\nu}$

\[
u=u_{0}\e^{-\gamma t}\left[\cosh\nu t+\frac{\gamma}{\nu}\sinh\nu t\right]
\]

過減衰解 $\gamma>\nu=\sqrt{\gamma^{2}-w^{2}}$
\item $\gamma<w$の時 ($D<0$)

解は2つの複素数 $\lambda_{\pm}=-\gamma\pm\i\nu$ ($\nu=\sqrt{w^{2}-\gamma^{2}}$)

一般解は
\[
u=\e^{-\gamma t}\left[A\cos\nu t+B\sin\nu t\right]
\]

\[
\dot{u}=-\gamma\e^{-\gamma t}\left[A\cos\nu t+B\sin\nu t\right]+\e^{-\gamma t}\left[A\nu\sin\nu t+B\nu\cos\nu t\right]
\]

$u=u_{0}\Rightarrow A=u_{0}$

$\dot{u}=0\Rightarrow-\gamma A+B\nu\Rightarrow B=\frac{\gamma u_{0}}{\nu}$

\[
u=u_{0}\e^{-\gamma t}\left[\cos\nu t+\frac{\gamma}{\nu}\sin\nu t\right]
\]

減衰振動解
\item $\gamma=w$の時 ($D=0$)

解は重根 ($\lambda=-\gamma$)

$u=A\e^{-u_{0}t}$

定数変化法 $u=A\left(t\right)\e^{-wt}$

\[
\dot{u}=\dot{A}\e^{-wt}-wA\e^{-wt}
\]

\[
\ddot{u}=\ddot{A}\e^{-wt}-2w\dot{A}\e^{-u_{0}t}+w^{2}A\e^{-wt}
\]

\[
\ddot{u}+w^{2}u+2\gamma\dot{u}=\left[\ddot{A}-2w\dot{A}+w^{2}A+w^{2}A+2w\dot{A}-2w^{2}A\right]\e^{-wt}=\ddot{A}\e^{-wt}=0
\]

よって 
\[
A\left(t\right)=C_{1}t+C_{2}
\]

臨界減衰運動解 $u=u_{0}\e^{-\gamma t}\left(\gamma t+1\right)$
\end{enumerate}

\paragraph{強制振動と定数係数2階非斉次線形常微分方程式}

振動する外力$F\left(t\right)=F_{0}\cos\Omega t$を加える。

\begin{align*}
\ddot{u}+2\gamma\dot{u}+w^{2}u & =F_{0}\cos\Omega t\\
 & =\Re\left[F_{0}\e^{\i\Omega t}\right]
\end{align*}

$u=\Re\left[A\e^{\i\Omega t}\right]$を代入。

\[
\left(-\Omega^{2}+\i2\gamma\Omega+w^{2}\right)A=F_{0}
\]

\begin{align*}
A & =\frac{F_{0}}{Z}\left(Z=w^{2}-\Omega^{2}+\i2\gamma\Omega\right)\\
 & \equiv\left|Z\right|\e^{\i\alpha}
\end{align*}

複素インピーダンス

\[
u=\frac{F_{0}}{\left|Z\right|}\cos\left(\Omega t-\alpha\right)
\]

$\alpha$: 位相の遅れ

\begin{align*}
\left|Z\right| & =\left|\left(w^{2}-\Omega^{2}\right)+\i2\gamma\Omega\right|\\
 & =\sqrt{\left(w^{2}-\Omega^{2}\right)^{2}+4\gamma^{2}\Omega^{2}}
\end{align*}

\[
\tan\alpha=\frac{2\gamma\Omega}{w^{2}-\Omega^{2}}
\]

振幅最大は$\left|Z\right|$が最大のとき

\[
\left|Z\right|^{2}=\left[\Omega^{2}-\left(w^{2}-2\gamma^{2}\right)\right]^{2}+w^{4}-\left(2\gamma^{2}-w^{2}\right)^{2}
\]

\[
\Omega=\sqrt{w^{2}-2\gamma^{2}}
\]

共鳴または共振と呼ぶ。

抵抗が零のとき($\gamma=0$)、$\Omega=w$

\[
\left|Z\right|=2\gamma\Omega\rightarrow0
\]

実数変化法 $u=Ct\sin\Omega t$

\[
\dot{u}=C\sin\Omega t+\Omega Ct\cos\Omega t
\]

\[
\ddot{u}=2\Omega C\cos\Omega t-\Omega^{2}Ct\sin\Omega t
\]

\[
\dot{u}+w^{2}u=2\Omega C\cos\Omega t=F_{0}\cos\Omega t
\]

\[
C=\frac{F_{0}}{2\Omega}
\]

\[
\therefore u=\frac{F_{0}}{2\Omega}t\sin\Omega t
\]


\paragraph{定数係数n階斉次線形常微分方程式}

\[
\sum_{j=1}^{n}a_{j}\frac{\mathrm{d}^{3}}{\mathrm{d}x^{3}}u+a_{0}u=0
\]

$u=C\e^{\lambda x}$を代入

特性方程式 $\sum_{j=!}^{n}a_{j}\lambda^{3}+a_{0}=0$

→$n$個の解$\lambda_{k}\left(k=1,\cdots,n\right)$
\begin{enumerate}
\item 解$\lambda_{k}$がすべて異なるとき $u=\sum_{k=1}^{n}C_{k}\e^{\lambda_{k}x}$
\item 解$\lambda$が$n$重根のとき $u=\sum_{m=0}^{n-1}C_{m}x^{m}\e^{\lambda x}$
\item 異なる解$\lambda_{k}$が$n_{k}$重根のとき
\[
u=\sum_{k}\sum_{m=0}^{n_{k}-1}C_{m}^{k}x^{m}\e^{\lambda_{k}x}
\]
\end{enumerate}
\[
\sum_{j=1}^{n}a_{j}\lambda^{3}+a_{0}=a_{0}\prod_{k=1}^{n}\left(\lambda-\lambda_{k}\right)=0
\]
\[
\sum_{j=1}^{n}\frac{\mathrm{d}^{3}}{\mathrm{d}x^{3}}u+a_{0}u=a_{0}\prod_{k=1}^{n}\left(D-\lambda_{k}\right)u=0
\]

\[
D=\d{}x
\]

$\e^{\lambda x}$が$n$重根のとき

\[
\left(D-\lambda\right)^{n}\left[\e^{\lambda x}f\left(x\right)\right]=0
\]

\[
\left(D-\lambda\right)\left[\e^{\lambda x}f\left(x\right)\right]=\e^{\lambda x}Df\left(x\right)
\]

\[
\left(D-\lambda\right)^{2}\left[\e^{\lambda x}f\left(x\right)\right]=\e^{\lambda x}D^{2}f\left(x\right)
\]

\[
\left(D-\lambda\right)^{n}\left[f\left(x\right)\e^{\lambda x}\right]=\e^{\lambda x}D^{n}f\left(x\right)
\]

\[
f^{\left(n\right)}\left(x\right)=0\Rightarrow f\left(x\right)=1,x,x^{2},\cdots,x^{n-1}
\]


\paragraph{例}

\[
u''-u'-6u=0
\]

\[
\lambda^{2}-\lambda-6=\left(\lambda-3\right)\left(\lambda+2\right)=0
\]

\[
u=C_{1}\e^{3x}+C_{2}\e^{-2x}
\]

\begin{align*}
u''+9u=0 & \Rightarrow\lambda^{2}+9=\left(\lambda+3\i\right)\left(\lambda-3\i\right)=0\\
 & \Rightarrow u=C_{1}\cos3x+C_{2}\sin3x
\end{align*}

\begin{align*}
u''-6u'+9u=0 & \Rightarrow\lambda^{2}-6\lambda+9=\left(\lambda-3\right)^{2}=0\\
 & \Rightarrow u=C_{1}\e^{3x}+C_{2}x\e^{3x}
\end{align*}

\begin{align*}
 & u'''-5u''+8u'-4u=0\\
\Rightarrow & \lambda^{3}-5\lambda^{2}+8\lambda-4=\left(\lambda-2\right)^{2}\left(\lambda-1\right)=0
\end{align*}

\[
u=C_{1}\e^{x}+C_{2}\e^{2x}+C_{3}x\e^{2x}
\]

\[
u'''+u''-u'-u=0\Rightarrow u=C_{1}\e^{x}+\left(C_{2}+C_{3}x\right)\e^{4x}
\]

\[
u'''+3u''+3u'+u=0\Rightarrow u\left(C_{1}+C_{2}x+C_{3}x^{2}\right)\e^{4x}
\]

\[
u'''-8u''+16=0\Rightarrow u=\left(C_{1}+C_{2}x\right)\e^{2x}+\left(C_{3}+C_{4}x\right)\e^{-2x}
\]


\section*{第5回}

\paragraph{定数係数n階非斉次線形常微分方程式}

\[
\sum_{j=1}^{n}a_{j}\ddd{}xu+a_{0}u_{2}=f\left(x\right)
\]

\[
u\left(x\right)=\underbrace{u_{0}\left(x\right)}_{\text{特殊解}}+\underbrace{\sum_{R}\sum_{m=0}^{n_{k}-1}C_{m}^{k}x^{m}\e^{\lambda_{k}x}}_{\text{斉次方程式の一般解}}
\]

特殊解の求め方

$f\left(x\right)=\e^{\alpha x}$のとき、
\begin{enumerate}
\item $\alpha$が特性方程式の根でないとき、$u=C\e^{\alpha x}$を代入して$C$を決定
\item $\alpha$が特性方程式の単根なら、$u=Cx\e^{\alpha x}$を代入して$C$を決定
\item $\alpha$が特性方程式の$m$乗根なら、$u=Cx^{m}\e^{\alpha x}$を代入
\end{enumerate}
$f\left(x\right)$が$m$次の多項式のとき、
\[
u=C_{1}x^{m}+C_{2}x^{m-1}+\cdots+C_{m}
\]
を代入。

\paragraph{例}

\[
u''+3u'+2u=\e^{3x}
\]

\[
\lambda^{2}+3\lambda+2=\left(\lambda+1\right)\left(\lambda+2\right)=0\Rightarrow\lambda=-1,-2
\]

\[
u=C\e^{3x}\Rightarrow9C\e^{3x}+9C\e^{3x}+2C\e^{3x}=\e^{3x}
\]

\[
C=\frac{1}{20}\Rightarrow u\left(x\right)=\frac{1}{20}\e^{3x}+C_{1}\e^{-x}+C_{2}\e^{-2x}
\]


\paragraph{例}

\[
u''+2u'+u=\e^{-x}
\]

\[
\lambda^{2}+2\lambda+1=\left(\lambda+1\right)^{2}=0\Rightarrow\lambda=-1
\]

$u=Cx^{2}\e^{-x}$を代入

\[
C\left(2-2x\right)\e^{-x}-C\left(2x-x^{2}\right)\e^{-x}+2C\left(2x-x^{2}\right)\e^{-x}+Cx^{2}\e^{-x}=\e^{-x}
\]

\[
C=\frac{1}{2}\Rightarrow u\left(x\right)=\frac{1}{2}x^{2}\e^{-x}+C_{1}\e^{-x}+C_{2}\e^{-x}
\]


\paragraph{例}

\[
u''+3u'+2u=x
\]
\[
\lambda^{2}+3\lambda+2=\left(\lambda+1\right)\left(\lambda+2\right)=0
\]

$u=ax+b$を代入

\[
3a+2ax+2b=x\Rightarrow a=\frac{1}{2},b=-\frac{3a}{2}=-\frac{3}{4}
\]

\[
u\left(x\right)=\frac{1}{2}x-\frac{3}{4}+C_{1}\e^{-x}+C_{2}\e^{-2x}
\]


\paragraph{例}

\[
u''-u'-2u=x^{2}
\]

\[
\left(\lambda+1\right)\left(\lambda-2\right)
\]
$u=ax^{2}+bx+c$を代入

\[
2a-2ax-b-2ax^{2}-2ax-2c=x^{2}
\]

\[
a=-\frac{1}{2},b=-a=\frac{1}{2},c=a-\frac{b}{2}=-\frac{3}{4}
\]

\[
u\left(x\right)=\frac{x^{2}}{2}+\frac{x}{2}-\frac{3}{4}+C_{1}\e^{-x}+C_{2}\e^{2x}
\]


\paragraph{例}

\[
u''+2\eta u'+\eta^{2}u=x
\]

\[
\left(\lambda+\eta\right)^{2}=0
\]

$u\left(x\right)=a+b$を代入

\[
2\eta a+\eta^{2}ax+\eta^{2}b=x
\]

\[
a=\frac{1}{\eta^{2}}b=-\frac{2a}{\eta}=-\frac{2}{\eta^{3}}
\]

\[
u\left(x\right)=\frac{x}{\eta^{2}}-\frac{2}{\eta^{3}}+C_{1}\e^{-\eta x}+C_{2}x\e^{-\eta x}
\]


\paragraph{n階微分方程式と一階n次元連立微分方程式}

\[
\sum_{j=1}^{n}a_{j}\ddd{}xu+a_{0}u=0
\]

$a_{n}=1$をおいて良い。

\[
u^{\left(n\right)}+a_{n-1}u^{\left(n-1\right)}+\cdots+a_{2}u''+a_{1}u'+a_{0}u=0
\]

\begin{align*}
u_{0}=u\\
u_{1}=u'\\
u_{2}=u''\\
\vdots\\
u_{n-2}=u^{\left(n-2\right)}\\
u_{n-1}=u^{\left(n-1\right)}
\end{align*}

$n$変数

\begin{align*}
u_{1}=u_{0}'\\
u_{2}=u_{0}'\\
\vdots\\
u_{n=1}=u_{n-2}'
\end{align*}

\[
u_{n-1}'+a_{n-1}u_{n-1}+a_{n-2}u_{n-2}+\cdots+a_{1}u_{1}+a_{0}u_{0}=0
\]

\[
\boldsymbol{u}=\left(\begin{array}{c}
u_{n-1}\\
u_{n-2}\\
\vdots\\
u_{1}\\
u_{0}
\end{array}\right),A=\left(\begin{array}{cccccc}
-a_{n-1} & -a_{n-2} & \cdots & -a_{2} & -a_{1} & -a_{0}\\
1 & 0 &  &  &  & 0\\
0 & 1 & 0 &  &  & 0\\
 & 0 & \ddots & 0 &  & \vdots\\
 &  & 0 & 1 & 0 & 0\\
0 &  & \cdots & 0 & 1 & 0
\end{array}\right)
\]

\[
\boldsymbol{x}'=A\boldsymbol{x}
\]


\paragraph{参考: Jordan標準形}

\paragraph{二次元連立微分方程式}

\[
\dot{\boldsymbol{u}}=A\boldsymbol{u}
\]

\[
A=\left(\begin{array}{cc}
a_{11} & a_{12}\\
a_{21} & a_{22}
\end{array}\right)
\]

\[
\det\left(A-\lambda I\right)=\lambda^{2}-\left(a_{11}+a_{22}\right)\lambda+a_{11}a_{22}-a_{12}a_{21}
\]

判別式
\[
D=\left(a_{11}+a_{22}\right)^{2}-4\left(a_{11}a_{22}-a_{12}a_{21}\right)
\]

\begin{enumerate}
\item $D>0$のとき相異なる実数$a<b$を持つ

\begin{align*}
A\boldsymbol{p} & =a\boldsymbol{p}\\
A\boldsymbol{q} & =b\boldsymbol{q}
\end{align*}

\[
p=\binom{p_{1}}{p_{2}},q=\binom{q}{q_{2}}
\]

\[
P=\left(\boldsymbol{p},\boldsymbol{q}\right)=\left(\begin{array}{cc}
p_{1} & q_{1}\\
p_{2} & q_{2}
\end{array}\right)
\]

\[
A\left(\boldsymbol{p},\boldsymbol{q}\right)=\left(\boldsymbol{p},\boldsymbol{q}\right)\left(\begin{array}{cc}
a & 0\\
0 & 0
\end{array}\right)
\]

\[
AP=P\left(\begin{array}{cc}
a & 0\\
0 & b
\end{array}\right)
\]

\[
P^{-1}AP=\left(\begin{array}{cc}
a & 0\\
0 & b
\end{array}\right)
\]

対角化。

$\boldsymbol{u}=\binom{u}{v},\boldsymbol{x}=P\boldsymbol{u}$

\begin{align*}
\dot{\boldsymbol{u}} & =P^{-1}\dot{\boldsymbol{x}}=P^{-1}Ax=P^{-1}APu\\
 & =\left(\begin{array}{cc}
a & 0\\
0 & b
\end{array}\right)\boldsymbol{u}
\end{align*}

\begin{align*}
 & \begin{cases}
\dot{u}=au\\
\dot{v}=bv
\end{cases}\\
\Rightarrow & \begin{cases}
u\left(t\right)=\e^{at}u_{0}\\
v\left(t\right)=\e^{bt}v_{0}
\end{cases}
\end{align*}

\item $D=0$のとき1つの重根を持つ。
\begin{enumerate}
\item $\mathrm{rank}\left(aI-A\right)=0$のとき

\[
A=\left(\begin{array}{cc}
a & 0\\
0 & a
\end{array}\right)\Rightarrow u\left(t\right)=\e^{at}u_{0},v\left(t\right)=\e^{at}v_{0}
\]

\item $\mathrm{rank}\left(aI-A\right)=1$のとき

$A\boldsymbol{p}=a\boldsymbol{p},D=\left(\lambda-a\right)^{2}$なので$\boldsymbol{p}$と一次独立な$\boldsymbol{q}$があって、$\left(A-aI\right)\boldsymbol{q}=\boldsymbol{p}$を満たす。

\[
\left(A-aI\right)^{2}\boldsymbol{q}=\left(A-aI\right)\boldsymbol{p}=0
\]

\begin{align*}
A\boldsymbol{p} & =a\boldsymbol{p}\\
A\boldsymbol{q} & =a\boldsymbol{q}+\boldsymbol{p}
\end{align*}

\[
A\left(\boldsymbol{p}\cdot\boldsymbol{q}\right)=\left(\boldsymbol{p}\cdot\boldsymbol{q}\right)\left(\begin{array}{cc}
a & 1\\
0 & a
\end{array}\right)
\]

\[
AP=P\left(\begin{array}{cc}
a & 1\\
0 & b
\end{array}\right)\Rightarrow P^{-1}AP=\left(\begin{array}{cc}
a & 1\\
0 & b
\end{array}\right)
\]

\[
\dot{u}=P^{-1}\dot{x}=P^{-1}Ax=P^{-1}AP\boldsymbol{u}=\left(\begin{array}{cc}
a & 1\\
0 & b
\end{array}\right)\boldsymbol{u}
\]

\[
u\left(t\right)=\e^{at}\left(u_{0}+tv_{0}\right),v\left(t\right)=\e^{bt}v_{0}
\]

\end{enumerate}
\item $D<0$複素共役な解$a\pm\i b$をもつ。

\[
A\boldsymbol{z}=\left(a+\i b\right)\boldsymbol{z}\:\left(\boldsymbol{z}=\boldsymbol{p}+\i\boldsymbol{q}\right)
\]

\[
A\boldsymbol{p}+A\boldsymbol{q}\i=\left(a\boldsymbol{p}-b\boldsymbol{q}\right)+\i\left(a\boldsymbol{q}+b\boldsymbol{p}\right)
\]

\begin{align*}
A\boldsymbol{p} & =a\boldsymbol{p}-b\boldsymbol{q}\\
A\boldsymbol{q} & =a\boldsymbol{q}+b\boldsymbol{p}
\end{align*}

\[
A\left(\boldsymbol{p},\boldsymbol{q}\right)=\left(\boldsymbol{p},\boldsymbol{q}\right)\left(\begin{array}{cc}
a & b\\
-b & a
\end{array}\right)
\]

\[
P^{-1}AP=\left(\begin{array}{cc}
a & b\\
-b & a
\end{array}\right)
\]

\[
\dot{\boldsymbol{u}}=\left(\begin{array}{cc}
a & b\\
-b & a
\end{array}\right)\boldsymbol{u}
\]

\begin{align*}
u\left(t\right) & =\e^{at}\left(u_{0}\cos bt-v_{0}\sin bt\right)\\
v\left(t\right) & =\e^{at}\left(u_{0}\sin bt+v_{0}\cos bt\right)
\end{align*}

\end{enumerate}

\paragraph{三次元連立微分方程式}

\[
A=\left(\begin{array}{ccc}
a_{11} & a_{12} & a_{13}\\
a_{21} & a_{22} & a_{23}\\
a_{31} & a_{32} & a_{33}
\end{array}\right),\boldsymbol{u}=\left(\begin{array}{c}
u\\
v\\
w
\end{array}\right)
\]

\begin{enumerate}
\item 異なる3つの解を持つ

\[
P^{-1}AP=\left(\begin{array}{ccc}
a & 0 & 0\\
0 & b & 0\\
0 & 0 & c
\end{array}\right)
\]

\[
u\left(t\right)=\e^{at}u_{01}v\left(t\right)=\e^{bt}v_{01}w\left(t\right)=\e^{ct}v_{0}
\]

\item 1つの実数と1つの2重根
\begin{enumerate}
\item $\rank\left(aIA\right)=1$のとき

\[
P^{-1}AP=\left(\begin{array}{ccc}
a & 0 & 0\\
0 & a & 0\\
0 & 0 & c
\end{array}\right)
\]

\[
u\left(t\right)=\e^{at}u_{01}v\left(t\right)=\e^{at}v_{0}'w\left(t\right)=\e^{ct}w_{0}
\]

\item $\rank\left(aI-A\right)=2$のとき

\[
P^{-1}AP=\left(\begin{array}{ccc}
a & 1 & 0\\
0 & a & 0\\
0 & 0 & c
\end{array}\right)
\]

\[
u\left(t\right)=\e^{at}\left(u_{0}+tv_{0}\right)_{1}v\left(t\right)=\e^{at}v_{0},w\left(t\right)=\e^{ct}w_{0}
\]

\end{enumerate}
\item 1つの実根と複素根

\[
P^{-1}AP=\left(\begin{array}{ccc}
a & b & 0\\
-b & a & 0\\
0 & 0 & c
\end{array}\right)
\]

\begin{align*}
u\left(t\right) & =\e^{at}\left(u_{0}\cos bt-v_{0}\sin bt\right)\\
v\left(t\right) & =\e^{at}\left(u_{0}\sin bt+v_{0}\cos bt\right),w\left(t\right)=\e^{ct}w_{0}
\end{align*}

\item 1つの三重根
\begin{enumerate}
\item $\rank\left(aI-A\right)=0\Rightarrow P^{-1}AP=\left(\begin{array}{ccc}
a & 0 & 0\\
0 & a & 0\\
0 & 0 & a
\end{array}\right)$
\item $\rank\left(aI-A\right)=1\Rightarrow P^{-1}AP=\left(\begin{array}{ccc}
a & 1 & 0\\
0 & a & 0\\
0 & 0 & a
\end{array}\right)$
\item $\rank\left(aI-A\right)=2\Rightarrow P^{-1}AP=\left(\begin{array}{ccc}
a & 1 & 0\\
0 & a & 1\\
0 & 0 & a
\end{array}\right)$
\end{enumerate}
\begin{align*}
u\left(t\right) & =\e^{at}\left(u_{0}+tv_{0}+\frac{t^{2}}{2}w_{0}\right)\\
v\left(t\right) & =\e^{at}\left(v_{0}+tw_{0}\right)\\
w\left(t\right) & =\e^{at}w_{0}
\end{align*}

\end{enumerate}

\section*{第6回}

11月16日 授業なし

11月23日 授業あり

11月30日 授業なし

\paragraph{定数係数n次元連立常微分方程式}

$A$: 定数の$N\times N$の正方行列

\[
\dot{\boldsymbol{x}}=A\boldsymbol{x}\Rightarrow\boldsymbol{x}\left(t\right)=\e^{At}\boldsymbol{c}
\]

($\e^{At}=1+At+\frac{1}{2}\left(At\right)^{2}+\frac{1}{3!}\left(At\right)^{3}+\cdots$)

\paragraph{Jordan標準形}

\[
\left(\begin{array}{cc}
\lambda & 1\\
0 & \lambda
\end{array}\right),\left(\begin{array}{ccc}
\lambda & 1 & 0\\
0 & \lambda & 1\\
0 & 0 & \lambda
\end{array}\right),\left(\begin{array}{cccc}
\lambda & 1 & 0 & 0\\
0 & \lambda & 1 & 0\\
0 & 0 & \lambda & 1\\
0 & 0 & 0 & \lambda
\end{array}\right)
\]

Jordan block細胞である。

c.f. $\left(\begin{array}{ccc}
a & 0 & 0\\
0 & b & 0\\
0 & 0 & c
\end{array}\right)$
\begin{center}
\includegraphics{images/MathematicsID/6-1}
\par\end{center}

\[
P^{-1}AP=J
\]

\[
\boldsymbol{x}=P\boldsymbol{u}
\]

\[
\dot{\boldsymbol{u}}=P^{-1}\dot{\boldsymbol{x}}=P^{-1}AP\boldsymbol{u}=J\boldsymbol{u}
\]

\[
\boldsymbol{u}\left(t\right)=\e^{Jt}\boldsymbol{C}\Rightarrow\boldsymbol{x}\left(t\right)=P\boldsymbol{u}\left(t\right)=P\e^{Jt}\boldsymbol{C}
\]

\[
A=\left(\begin{array}{cc}
\lambda_{1} & 0\\
0 & \lambda_{2}
\end{array}\right)\Rightarrow\e^{At}=\left(\begin{array}{cc}
\e^{\lambda_{1}t} & 0\\
0 & \e^{\lambda_{2}t}
\end{array}\right)
\]

1) 対角Block

固有値 $\lambda_{1},\lambda_{2},\cdots,\lambda_{m}$

固有ベクトル $\boldsymbol{v}_{1},\boldsymbol{v}_{2},\cdots,\boldsymbol{v}_{m}$

単位ベクトル $\boldsymbol{e}_{1},\boldsymbol{e}_{2},\cdots,\boldsymbol{e}_{m}$

\[
J=\left(\lambda_{1}\boldsymbol{e}_{1},\lambda_{2a}\boldsymbol{e}_{2},\cdots,\lambda_{m}\boldsymbol{e}_{m}\right)
\]

\[
P=\left(\boldsymbol{v}_{1},\boldsymbol{v}_{2},\cdots,\boldsymbol{v}_{m}\right)
\]

\[
\dot{\boldsymbol{u}}=J\boldsymbol{u}\Rightarrow\boldsymbol{u}_{i}=\e^{\lambda_{i}t}\boldsymbol{e}_{i}
\]

\begin{align*}
\boldsymbol{x}_{i}\left(t\right) & =P\boldsymbol{u}_{i}\left(t\right)=P\e^{\lambda_{i}t}\boldsymbol{e}_{i}\\
 & =\e^{\lambda_{i}t}P\boldsymbol{e}_{i}=\e^{\lambda_{i}t}\boldsymbol{v}_{i}
\end{align*}

基本行列
\begin{align*}
\Phi\left(t\right) & =\left(x_{1},x_{2},\cdots,x_{m}\right)\\
\Phi\left(0\right) & =P
\end{align*}

\[
\boldsymbol{x}\left(t\right)=\sum_{j=1}^{m}C_{j}x_{j}\left(t\right)=\Phi\left(t\right)\boldsymbol{C}
\]

\[
\Phi\left(t\right)=\e^{At}\left(\boldsymbol{v}_{1},\boldsymbol{v}_{2},\cdots,\boldsymbol{v}_{m}\right)
\]

\[
\e^{At}=\Phi\left(t\right)\Phi\left(0\right)^{-1}
\]

(i) 対角化できない場合

固有値$\lambda$が$n$重に縮退

Jordan Block $J\left(\lambda,n\right)$

\[
\left(\begin{array}{cccc}
\lambda & 1\\
 & \lambda & 1\\
 &  & \ddots & 1\\
 &  &  & \lambda
\end{array}\right)\leftarrow n\times n
\]

\[
J\left(\lambda,n\right)=P^{-1}AP
\]

\[
P=\left(\boldsymbol{v}_{1},\boldsymbol{v}_{2},\cdots,\boldsymbol{v}_{m}\right)
\]

\begin{align*}
\left(A-\lambda I\right)\boldsymbol{v}_{1} & =0\\
\left(A-\lambda I\right)\boldsymbol{v}_{2} & =\boldsymbol{v}_{1}\\
\left(A-\lambda I\right)\boldsymbol{v}_{3} & =\boldsymbol{v}_{2}
\end{align*}

\[
\left(A-\lambda I\right)\boldsymbol{v}_{n}=\boldsymbol{v}_{n-1}
\]

\begin{align*}
\boldsymbol{x}_{1} & =\e^{\lambda t}\boldsymbol{v}_{1}\\
\boldsymbol{x}_{2} & =\e^{\lambda t}\left(\boldsymbol{v}_{1}+t\boldsymbol{v}_{1}\right)\\
\boldsymbol{x}_{3} & =\e^{\lambda t}\left(\boldsymbol{v}_{3}+t\boldsymbol{v}_{2}+\frac{t^{2}}{2!}\boldsymbol{v}_{1}\right)
\end{align*}

\[
\boldsymbol{x}_{n}=\e^{\lambda t}\left(\boldsymbol{v}_{n}+t\boldsymbol{v}_{n-1}+\frac{t^{2}}{2!}\boldsymbol{v}_{n-2}+\cdots+\frac{t^{n}}{n!}\boldsymbol{v}_{1}\right)
\]

$n=4$のとき、
\[
J\left(\lambda,4\right)=P^{-1}AP=\left(\begin{array}{cccc}
\lambda & 1 & 0 & 0\\
0 & \lambda & 1 & 0\\
0 & 0 & \lambda & 1\\
0 & 0 & 0 & \lambda
\end{array}\right)
\]

\[
P=\left(\boldsymbol{v}_{1},\boldsymbol{v}_{2},\boldsymbol{v}_{3},\boldsymbol{v}_{4}\right)
\]

\[
P^{-1}\left(A-\lambda\right)P=J-\lambda
\]

\[
\left(A-\lambda\right)P=P\left(J-\lambda\right)
\]

\begin{align*}
\left(A-\lambda\right)\left(\boldsymbol{v}_{1},\boldsymbol{v}_{2},\boldsymbol{v}_{3},\boldsymbol{v}_{4}\right) & =\left(\boldsymbol{v}_{1},\boldsymbol{v}_{2},\boldsymbol{v}_{3},\boldsymbol{v}_{4}\right)\left(\begin{array}{cccc}
0 & 1 & 0 & 0\\
0 & 0 & 1 & 0\\
0 & 0 & 0 & 1\\
0 & 0 & 0 & 0
\end{array}\right)\\
 & =\left(0,\boldsymbol{v}_{1},\boldsymbol{v}_{2},\boldsymbol{v}_{3}\right)
\end{align*}

\begin{align*}
\left(A-\lambda I\right)\boldsymbol{v}_{1} & =0\\
\left(A-\lambda I\right)\boldsymbol{v}_{2} & =\boldsymbol{v}_{1}\\
\left(A-\lambda I\right)\boldsymbol{v}_{3} & =\boldsymbol{v}_{2}\\
\left(A-\lambda I\right)\boldsymbol{v}_{4} & =\boldsymbol{v}_{3}
\end{align*}

$\boldsymbol{u}=P^{-1}x$とおく。

\[
\dot{\boldsymbol{u}}=J\boldsymbol{u}
\]

\begin{align*}
\dot{u}_{1} & =\lambda u_{1}+u_{2}\\
\dot{u}_{2} & =\lambda u_{2}+u_{3}\\
\dot{u}_{3} & =\lambda u_{3}+u_{4}\\
\dot{u}_{4} & =\lambda u_{4}
\end{align*}

\begin{align*}
u_{4} & =C_{4}\e^{\lambda t}\\
u_{3} & =\left(C_{3}+C_{4}t\right)\e^{\lambda t}\\
u_{2} & =\left(C_{2}+C_{3}t+\frac{1}{2!}C_{4}t^{2}\right)\e^{\lambda t}\\
u_{1} & =\left(C_{1}+C_{2}t+\frac{1}{2!}C_{3}t^{2}+\frac{1}{3!}C_{4}t^{3}\right)\e^{\lambda t}
\end{align*}

$P\boldsymbol{e}_{i}=\boldsymbol{v}_{i}$を用いて、
\[
\boldsymbol{u}=\sum u_{i}\boldsymbol{e}_{i}
\]

\begin{align*}
\boldsymbol{x} & =\e^{\lambda t}P\left(\begin{array}{c}
C_{1}+C_{2}t+\frac{1}{2!}C_{3}t^{2}+\frac{1}{3!}C_{4}t^{3}\\
C_{2}+C_{3}t+\frac{1}{2!}C_{4}t^{2}\\
C_{3}+C_{4}t\\
C_{4}
\end{array}\right)\\
 & =C_{1}\e^{\lambda t}P\left(\begin{array}{c}
1\\
0\\
0\\
0
\end{array}\right)+C_{2}\e^{\lambda t}P\left(\begin{array}{c}
t\\
1\\
0\\
0
\end{array}\right)+C_{3}\e^{\lambda t}P\left(\begin{array}{c}
\frac{1}{2!}t^{2}\\
t\\
1\\
0
\end{array}\right)+C_{4}\e^{\lambda t}P\left(\begin{array}{c}
\frac{1}{3!}t^{3}\\
\frac{1}{2!}t^{2}\\
t\\
1
\end{array}\right)\\
 & =C_{1}\e^{\lambda t}\boldsymbol{v}_{1}+C_{2}\e^{\lambda t}\left(\boldsymbol{v}_{2}+t\boldsymbol{v}_{1}\right)+C_{3}\e^{\lambda t}\left(\boldsymbol{v}_{3}+t\boldsymbol{v}_{2}+\frac{t^{2}}{2!}\boldsymbol{v}_{1}\right)+C_{4}\e^{\lambda t}\left(\boldsymbol{v}_{4}+t\boldsymbol{v}_{3}+\frac{t^{2}}{2!}\boldsymbol{v}_{2}+\frac{t^{3}}{3!}\boldsymbol{v}_{1}\right)
\end{align*}


\paragraph{例}

\[
\dot{\boldsymbol{x}}=Ax
\]

\[
A=\left(\begin{array}{cc}
2 & 2\\
1 & 3
\end{array}\right)
\]

\[
\left|A-\lambda\right|=\left|\begin{array}{cc}
2-\lambda & 2\\
1 & 3-\lambda
\end{array}\right|=\left(\lambda-1\right)\left(\lambda-4\right)=0
\]

\[
\left(\begin{array}{cc}
2 & 2\\
1 & 3
\end{array}\right)\left(\begin{array}{c}
2\\
-1
\end{array}\right)=1\left(\begin{array}{c}
2\\
-1
\end{array}\right),\left(\begin{array}{cc}
2 & 2\\
1 & 3
\end{array}\right)\left(\begin{array}{c}
1\\
1
\end{array}\right)=4\left(\begin{array}{c}
1\\
1
\end{array}\right)
\]

\[
\boldsymbol{v}_{1}=\left(\begin{array}{c}
2\\
-1
\end{array}\right),\boldsymbol{v}_{2}=\left(\begin{array}{c}
1\\
1
\end{array}\right),P=\left(\begin{array}{cc}
2 & 1\\
-1 & 1
\end{array}\right)
\]

\[
P^{-1}AP=\left(\begin{array}{cc}
1 & 0\\
0 & 4
\end{array}\right)
\]

\[
\boldsymbol{x}=C_{1}\e^{t}\left(\begin{array}{c}
2\\
-1
\end{array}\right)+C_{2}\e^{4t}\left(\begin{array}{c}
1\\
1
\end{array}\right)
\]


\paragraph{例}

\[
\dot{\boldsymbol{x}}=Ax,A=\left(\begin{array}{cc}
1 & 1\\
-1 & 3
\end{array}\right)
\]

\[
\left|A-\lambda\right|=\left|\begin{array}{cc}
1-\lambda & 1\\
-1 & 3-\lambda
\end{array}\right|=\left(\lambda-2\right)^{2}=0
\]

\[
\left(A-2\right)\boldsymbol{v}=\left(\begin{array}{cc}
-1 & 1\\
-1 & 1
\end{array}\right)\boldsymbol{v}_{1}=0
\]

\[
\boldsymbol{v}_{1}=\left(\begin{array}{c}
1\\
1
\end{array}\right)
\]

\[
\left(A-2\right)\boldsymbol{v}_{2}=\left(\begin{array}{cc}
-1 & 1\\
-1 & 1
\end{array}\right)\boldsymbol{v}_{2}=\boldsymbol{v}_{1}=\left(\begin{array}{c}
1\\
1
\end{array}\right)
\]

\[
\boldsymbol{v}_{2}=\left(\begin{array}{c}
-1\\
0
\end{array}\right)
\]

\[
P=\left(\boldsymbol{v}_{1},\boldsymbol{v}_{2}\right)=\left(\begin{array}{cc}
1 & -1\\
1 & 0
\end{array}\right)
\]

\[
J=P^{-1}AP=\left(\begin{array}{cc}
2 & 1\\
0 & 2
\end{array}\right)
\]

\[
\boldsymbol{x}_{1}=\e^{2t}\boldsymbol{v}_{1}=\e^{2-t}\left(\begin{array}{c}
1\\
1
\end{array}\right)
\]

\[
\boldsymbol{x}_{2}=\e^{2t}\left(\boldsymbol{v}_{2}+t\boldsymbol{v}_{1}\right)=\e^{2t}\left(\begin{array}{c}
t-1\\
t
\end{array}\right)
\]

一般解は$\boldsymbol{x}=C_{1}x_{1}+C_{2}x_{2}$なので、
\begin{align*}
x & =\e^{2t}C_{1}+\e^{2t}\left(t-1\right)C_{2}\\
y & \e^{2t}C_{1}+tC_{2}
\end{align*}


\paragraph{例}

\[
\dot{\boldsymbol{x}}=Ax,A=\left(\begin{array}{ccc}
2 & 1 & 1\\
1 & 2 & 1\\
1 & 1 & 4
\end{array}\right)
\]

\[
\left|A-\lambda\right|=\left|\begin{array}{ccc}
2-\lambda & 1 & 1\\
1 & 2-\lambda & 1\\
1 & 1 & 4-\lambda
\end{array}\right|=-\left(\lambda-1\right)\left(\lambda-2\right)\left(\lambda-5\right)=0
\]

\begin{align*}
A\left(\begin{array}{c}
-1\\
1\\
0
\end{array}\right) & =1\left(\begin{array}{c}
-1\\
1\\
0
\end{array}\right)\\
A\left(\begin{array}{c}
-1\\
-1\\
1
\end{array}\right) & =2\left(\begin{array}{c}
-1\\
-1\\
1
\end{array}\right)\\
A\left(\begin{array}{c}
1\\
1\\
2
\end{array}\right) & =5\left(\begin{array}{c}
1\\
1\\
2
\end{array}\right)
\end{align*}

\[
\left(\begin{array}{c}
x\\
y\\
z
\end{array}\right)=C_{1}\left(\begin{array}{c}
1\\
1\\
2
\end{array}\right)\e^{5t}+C_{2}\left(\begin{array}{c}
-1\\
-1\\
1
\end{array}\right)\e^{2t}+C_{3}\left(\begin{array}{c}
-1\\
1\\
0
\end{array}\right)\e^{t}
\]


\paragraph{例}

\[
\dot{\boldsymbol{x}}=A\boldsymbol{x},A=\left(\begin{array}{ccc}
2 & 1 & 3\\
0 & 2 & -1\\
0 & 0 & 2
\end{array}\right)
\]

\[
\left|A-\lambda\right|=\left|\begin{array}{ccc}
2-\lambda & 1 & -3\\
0 & 2-\lambda & -1\\
0 & 0 & 2-\lambda
\end{array}\right|=-\left(\lambda-2\right)^{3}
\]

\[
\left(A-2\right)\boldsymbol{v}_{1}=0
\]

\[
\left(\begin{array}{ccc}
0 & -1 & -3\\
0 & 0 & -1\\
0 & 0 & 0
\end{array}\right)\boldsymbol{v}_{1}=\left(\begin{array}{c}
0\\
0\\
0
\end{array}\right)\Rightarrow\boldsymbol{v}_{1}=\left(\begin{array}{c}
1\\
0\\
0
\end{array}\right)
\]

\[
\left(A-2\right)\boldsymbol{v}_{2}=\boldsymbol{v}_{1}\Rightarrow\left(\begin{array}{ccc}
0 & 1 & -3\\
0 & 0 & -1\\
0 & 0 & 0
\end{array}\right)\boldsymbol{v}_{2}=\left(\begin{array}{c}
1\\
0\\
0
\end{array}\right)
\]

\[
\boldsymbol{v}_{2}=\left(\begin{array}{c}
0\\
1\\
0
\end{array}\right)
\]

\[
\left(A-2\right)\boldsymbol{v}_{3}=\boldsymbol{v}_{2}\Rightarrow\left(\begin{array}{ccc}
0 & 1 & -3\\
0 & 0 & -1\\
0 & 0 & 0
\end{array}\right)\boldsymbol{v}_{3}=\left(\begin{array}{c}
0\\
1\\
0
\end{array}\right)
\]

\[
\boldsymbol{v}_{3}=\left(\begin{array}{c}
0\\
-3\\
-1
\end{array}\right)
\]

\[
P=\left(\begin{array}{ccc}
1 & 0 & 0\\
0 & 1 & -3\\
0 & 0 & -1
\end{array}\right)
\]

\[
J=P^{-1}AP=\left(\begin{array}{ccc}
2 & 1 & 0\\
0 & 2 & 1\\
0 & 0 & 2
\end{array}\right)
\]

\[
x_{1}=\e^{2t}\boldsymbol{v}_{1}=\e^{2t}\left(\begin{array}{c}
1\\
0\\
0
\end{array}\right)
\]

\begin{align*}
x_{2} & =\e^{2t}\left(\boldsymbol{v}_{2}+t\boldsymbol{v}_{1}\right)\\
 & =\e^{2t}\left[\left(\begin{array}{c}
0\\
1\\
0
\end{array}\right)+t\left(\begin{array}{c}
1\\
0\\
0
\end{array}\right)\right]=\e^{2t}\left(\begin{array}{c}
t\\
1\\
0
\end{array}\right)
\end{align*}

\begin{align*}
\boldsymbol{x}_{3} & =\e^{2t}\left(\boldsymbol{v}_{3}+t\boldsymbol{v}_{2}+\frac{t^{2}}{2}\boldsymbol{v}_{1}\right)\\
 & =\e^{2t}\left(\begin{array}{c}
\frac{t^{2}}{2}\\
t-3\\
-1
\end{array}\right)
\end{align*}

一般解は
\[
\left(\begin{array}{c}
x\\
y\\
z
\end{array}\right)=\e^{2t}\left[C_{1}\left(\begin{array}{c}
1\\
0\\
0
\end{array}\right)+C_{2}\left(\begin{array}{c}
t\\
1\\
0
\end{array}\right)+C_{3}\left(\begin{array}{c}
\frac{t^{2}}{2}\\
t-3\\
-1
\end{array}\right)\right]
\]


\paragraph{来週の内容}

冪級数展開

\section*{第7回}

\paragraph{レポート}

\[
\dot{\boldsymbol{x}}=A\boldsymbol{x},A=\left(\begin{array}{cccc}
2 & 2 & 1 & 1\\
0 & 1 & 2 & 1\\
0 & 0 & 1 & 2\\
0 & 0 & 0 & 2
\end{array}\right)
\]

{[}一部紛失{]}

\[
\rho=\lim_{n\rightarrow\infty}\frac{\left|C_{2n}\right|}{\left|C_{2n+2}\right|}=\lim_{n\rightarrow\infty}\frac{2n+1}{2n-1}=1
\]


\chapter{第2章 ベクトル解析}

\paragraph{デカルト座標形}

\begin{align*}
\boldsymbol{e}_{x} & =\boldsymbol{e}_{1}=\left(1,0,0\right)\\
\boldsymbol{e}_{y} & =\boldsymbol{e}_{2}=\left(0,1,0\right)\\
\boldsymbol{e}_{z} & =\boldsymbol{e}_{3}=\left(0,0,1\right)
\end{align*}

\[
\boldsymbol{r}=x\boldsymbol{e}_{x}+y\boldsymbol{e}_{y}+z\boldsymbol{e}_{z}
\]


\paragraph{Kroneckerの$\delta_{ij}$}

\[
\delta_{11}=\delta_{22}=\delta_{33}=1
\]

\[
\delta_{12}=\delta_{21}=\delta_{13}=\delta_{31}=\delta_{23}=\delta_{32}=0
\]


\paragraph{Leri-Civitaの完全反対性テンソル}

\[
\varepsilon_{123}=\varepsilon_{231}=\varepsilon_{312}=1
\]

\[
\varepsilon_{132}=\varepsilon_{213}=\varepsilon_{321}=-1
\]

それ以外は
\[
\varepsilon_{ijk}=0
\]

\begin{align*}
\varepsilon_{ijk}\varepsilon_{lmn} & =\det\left[\begin{array}{ccc}
\delta_{il} & \delta_{im} & \delta_{in}\\
\delta_{jl} & \delta_{jm} & \delta_{jn}\\
\delta_{kl} & \delta_{km} & \delta_{kn}
\end{array}\right]\\
 & =\delta_{il}\left(\delta_{jm}\delta_{kn}-\delta_{jn}\delta_{km}\right)+\delta_{im}\left(\delta_{jn}\delta_{kl}-\delta_{jl}\delta_{kn}\right)+\delta_{in}\left(\delta_{jl}\delta_{km}-\delta_{jm}\delta_{kl}\right)
\end{align*}

\begin{align*}
\varepsilon_{ijk}\varepsilon_{lmk} & =\delta_{il}\delta_{jm}-\delta_{im}\delta_{jl}\\
 & =\det\left[\begin{array}{cc}
\delta_{il} & \delta_{im}\\
\delta_{jl} & \delta_{jm}
\end{array}\right]
\end{align*}

\[
\varepsilon_{ijk}\varepsilon_{ljk}=2\delta_{il}
\]

\[
\varepsilon_{ijk}\varepsilon_{ijk}=6-\left(\delta_{ij}+\delta_{jk}+\delta_{ki}\right)
\]


\paragraph{内積}

単位ベクトル$\left(\boldsymbol{e}_{x},\boldsymbol{e}_{y},\boldsymbol{e}_{z}\right)$

\[
\boldsymbol{e}_{i}\cdot\boldsymbol{e}_{j}=\delta_{ij},\boldsymbol{A}=A_{x}\boldsymbol{e}_{x}+A_{y}\boldsymbol{e}_{y}+A_{z}\boldsymbol{e_{z}}
\]

\begin{align*}
\boldsymbol{A}\cdot\boldsymbol{B} & =\left(A_{i}\boldsymbol{e}_{i}\right)\cdot\left(B_{j}\boldsymbol{e}_{j}\right)=A_{i}B_{j}\delta_{ij}=A_{i}B_{j}\\
 & =A_{x}B_{x}+A_{y}B_{y}+A_{z}B_{z}
\end{align*}


\paragraph{外積}

\[
\boldsymbol{e}_{i}\times\boldsymbol{e}_{j}=\varepsilon_{ijk}\boldsymbol{e}_{k}
\]

\[
\boldsymbol{e}_{x}\times\boldsymbol{e}_{x}=\boldsymbol{e}_{y}\times\boldsymbol{e}_{y}=\boldsymbol{e}_{z}\times\boldsymbol{e}_{z}=0
\]

\begin{align*}
\boldsymbol{e}_{x}\times\boldsymbol{e}_{y} & =-\boldsymbol{e}_{y}\times\boldsymbol{e}_{x}=\boldsymbol{e}_{z}\\
\boldsymbol{e}_{y}\times\boldsymbol{e}_{z} & =-\boldsymbol{e}_{z}\times\boldsymbol{e}_{y}=\boldsymbol{e}_{x}\\
\boldsymbol{e}_{z}\times\boldsymbol{e}_{x} & =-\boldsymbol{e}_{x}\times\boldsymbol{e}_{z}=\boldsymbol{e}_{y}
\end{align*}

\begin{align*}
\boldsymbol{A}\times\boldsymbol{B} & =A_{i}\boldsymbol{e}_{i}\times B_{j}\boldsymbol{e}_{j}\\
 & =A_{i}B_{j}\varepsilon_{ijk}\boldsymbol{e}_{k}\\
 & =\left(A_{y}B_{z}-A_{z}B_{y}\right)\boldsymbol{e}_{x}\\
 & +\left(A_{z}B_{x}-A_{x}B_{z}\right)\boldsymbol{e}_{y}\\
 & +\left(A_{x}B_{y}-A_{y}B_{x}\right)\boldsymbol{e}_{z}\\
 & =\left|\begin{array}{ccc}
\boldsymbol{e}_{x} & \boldsymbol{e}_{y} & \boldsymbol{e}_{z}\\
A_{x} & A_{y} & A_{z}\\
B_{x} & B_{y} & B_{z}
\end{array}\right|\\
 & =\left(\left|\boldsymbol{A}\right|\left|\boldsymbol{B}\right|\sin\theta\right)^{2}
\end{align*}

\[
\boldsymbol{A}\times\boldsymbol{B}=-\boldsymbol{B}\times\boldsymbol{A}
\]

\[
\boldsymbol{A}\times\left(\boldsymbol{B}+\boldsymbol{C}\right)=\boldsymbol{A}\times\boldsymbol{B}+\boldsymbol{A}\times\boldsymbol{C}
\]

スカラー三十石
\[
c\boldsymbol{A}\times\boldsymbol{B}=\boldsymbol{A}\times c\boldsymbol{B}=c\left(\boldsymbol{A}\times\boldsymbol{B}\right)
\]

\begin{align*}
\left(\boldsymbol{A}\times\boldsymbol{B}\right)\cdot\boldsymbol{C} & =\varepsilon_{ijk}A_{i}B_{j}\boldsymbol{e}_{k}\cdot C_{m}\boldsymbol{e}_{m}\\
 & =\varepsilon_{ijk}A_{i}B_{j}C_{k}\delta_{km}C_{m}=\varepsilon_{ijk}A_{i}B_{j}C_{k}\\
 & =\boldsymbol{A}\cdot\left(\boldsymbol{B}\times\boldsymbol{C}\right)=\boldsymbol{B}\cdot\left(\boldsymbol{C}\times\boldsymbol{A}\right)=\boldsymbol{C}\cdot\left(\boldsymbol{A}\times\boldsymbol{B}\right)
\end{align*}

ベクトル三重積

\begin{align*}
\boldsymbol{A}\times\left(\boldsymbol{B}\times\boldsymbol{C}\right) & =\left(A_{j}\boldsymbol{e}_{j}\right)\times\left(\varepsilon_{klm}B_{l}C_{m}\boldsymbol{e}_{k}\right)\\
 & =\varepsilon_{klm}\varepsilon_{jki}A_{j}B_{l}C_{m}\boldsymbol{e}_{i}\\
 & =\left(\delta_{il}\delta_{jm}-\delta_{im}\delta_{jl}\right)A_{j}B_{l}C_{m}\boldsymbol{e}_{i}
\end{align*}

\begin{align*}
\left|\boldsymbol{A}\times\boldsymbol{B}\right|^{2} & =\left(\boldsymbol{A}\times\boldsymbol{B}\right)\cdot\left(\boldsymbol{A}\times\boldsymbol{B}\right)\\
 & =\varepsilon_{iab}A_{a}B_{b}\varepsilon_{icd}A_{c}B_{d}\\
 & =\left(\delta_{ac}\delta_{bd}-\delta_{ad}-\delta_{bc}\right)A_{a}B_{b}A_{c}B_{d}\\
 & =\left|A\right|^{2}\left|B\right|^{2}-\left(\boldsymbol{A}\cdot\boldsymbol{B}\right)^{2}=\left|\boldsymbol{A}\right|^{2}\left|\boldsymbol{B}\right|^{2}\left(1-\cos^{2}\theta\right)\\
 & =\left(A_{j}B_{i}C_{j}-A_{j}B_{j}C_{i}\right)\boldsymbol{e}_{i}\text{ta}\\
 & =\left(\boldsymbol{A}\cdot\boldsymbol{C}\right)\boldsymbol{B}-\left(\boldsymbol{A}\cdot\boldsymbol{B}\right)\boldsymbol{C}
\end{align*}

テンソル積

\[
\boldsymbol{A}\otimes\boldsymbol{B}=A_{i}B_{j}=\left(\begin{array}{ccc}
A_{x}B_{x} & A_{x}B_{y} & A_{x}B_{z}\\
A_{y}B_{x} & A_{y}B_{y} & A_{y}B_{z}\\
A_{z}B_{x} & A_{z}B_{y} & A_{z}B_{z}
\end{array}\right)
\]


\section*{第8回}

来週休み

\paragraph{レポート}

\[
\dot{\boldsymbol{x}}=\left(\begin{array}{cc}
2 & 1\\
1 & 2
\end{array}\right)\boldsymbol{x}+\left(\begin{array}{c}
\mathrm{e}^{t}\\
t
\end{array}\right)\text{を解け。}
\]
を解け。

\paragraph{ベクトルの微分}

\[
\partial_{x}f=\pd fx
\]

ナブラ演算子
\begin{align*}
\nabla & =\boldsymbol{e}_{x}\partial_{x}+\boldsymbol{e}_{y}\partial_{y}+\boldsymbol{e}_{z}\partial_{z}\\
 & \equiv\boldsymbol{e}_{i}\partial_{i}\\
 & =\left(\begin{array}{c}
\partial_{x}\\
\partial_{y}\\
\partial_{z}
\end{array}\right)
\end{align*}


\paragraph{傾斜 gradient}

\begin{align*}
\grad f & =\nabla f\\
 & =\boldsymbol{e}_{i}\partial_{i}f\\
 & =\left(\begin{array}{c}
\partial_{x}f\\
\partial_{y}f\\
\partial_{z}f
\end{array}\right)
\end{align*}


\paragraph{発散 divergence}

\begin{align*}
\div\boldsymbol{F} & =\nabla\cdot\boldsymbol{F}=\left(\boldsymbol{e}_{i}\partial_{i}\right)\cdot\left(F_{j}\partial_{j}\right)\\
 & =\left(\boldsymbol{e}_{i}\cdot\boldsymbol{e}_{j}\right)\partial_{i}F_{j}=\delta_{ij}\partial_{i}F_{j}\\
 & =\partial_{i}F_{i}+\partial_{y}F_{y}+\partial_{z}F_{z}
\end{align*}


\paragraph{回転 rotation}

\begin{align*}
\rot\boldsymbol{F} & =\nabla\times\boldsymbol{F}=\left(\boldsymbol{e}_{i}\partial_{i}\right)\times\left(F_{j}\boldsymbol{e}_{j}\right)\\
 & =\left(\boldsymbol{e}_{i}\times\boldsymbol{e}_{j}\right)\partial_{i}F_{j}=\varepsilon_{ijk}\boldsymbol{e}_{k}\partial_{i}F_{j}\\
 & =\varepsilon_{ijk}\partial_{j}F_{j}\boldsymbol{e}_{k}\\
 & =\left(\begin{array}{c}
\partial_{y}F_{z}-\partial_{z}F_{y}\\
\partial_{z}F_{x}-\partial_{x}F_{z}\\
\partial_{x}F_{y}-\partial_{y}F_{x}
\end{array}\right)
\end{align*}


\paragraph{ラプラス演算子 Laplacian}

\begin{align*}
\Delta f & =\nabla^{2}f\\
 & =\nabla\cdot\nabla f\\
 & =\left(\boldsymbol{e}_{i}\partial_{i}\right)\cdot\left(\boldsymbol{e}_{j}\partial_{j}\right)f=\left(\boldsymbol{e}_{i}\cdot\boldsymbol{e}_{j}\right)\partial_{i}\partial_{j}f\\
 & =\delta_{ij}\partial_{i}\partial_{j}f=\partial_{i}\partial_{i}f=\partial_{x}^{2}f+\partial_{y}^{2}f+\partial_{z}^{2}f
\end{align*}

\[
\Delta f=0
\]


\paragraph{公式}

\begin{align*}
\nabla\left(\phi\psi\right) & =\boldsymbol{e}_{i}\partial_{i}\left(\phi\psi\right)\\
 & =\boldsymbol{e}_{i}\left(\partial_{i}\phi\right)\psi+\boldsymbol{e}_{i}\phi\left(\partial_{i}\psi\right)\\
 & =\left(\nabla\phi\right)\psi+\phi\nabla\psi
\end{align*}

\begin{align*}
\nabla\cdot\left(\phi\boldsymbol{A}\right) & =\left(\boldsymbol{e}_{i}\partial_{i}\right)\cdot\left(\phi\boldsymbol{A}\right)\\
 & =\boldsymbol{e}_{i}\cdot\left[\left(\partial_{i}\phi\right)\boldsymbol{A}+\phi\left(\partial_{i}\boldsymbol{A}\right)\right]\\
 & =\left(\nabla\phi\right)\cdot\boldsymbol{A}+\phi\left(\nabla\cdot\boldsymbol{A}\right)
\end{align*}

\begin{align*}
\nabla\times\left(\phi\boldsymbol{A}\right) & =\varepsilon_{ijk}\partial_{i}\left(\phi A_{j}\right)\boldsymbol{e}_{k}\\
 & =\left[\varepsilon_{ijk}\left(\partial_{i}\phi\right)A_{j}+\varepsilon_{ijk}\phi\left(\partial_{i}A_{j}\right)\right]\boldsymbol{e}_{k}\\
 & =\left(\nabla\phi\right)\times\boldsymbol{A}+\phi\left(\nabla\times\boldsymbol{A}\right)
\end{align*}

\begin{align*}
\nabla\cdot\left(\boldsymbol{A}\times\boldsymbol{B}\right) & =\left(\boldsymbol{e}_{i}\partial_{i}\right)\cdot\left(\varepsilon_{jkl}A_{j}B_{k}\boldsymbol{e}_{l}\right)\\
 & =\left(\boldsymbol{e}_{i}\cdot\boldsymbol{e}_{l}\right)\varepsilon_{jkl}\partial_{i}A_{j}B_{k}\\
 & =\varepsilon_{ijk}\partial_{i}\left(A_{j}B_{k}\right)\\
 & =\varepsilon_{ijk}\left[\left(\partial_{i}A_{j}\right)B_{k}+A_{j}\left(\partial_{i}B_{k}\right)\right]\\
 & =\boldsymbol{B}\cdot\left(\nabla\times\boldsymbol{A}\right)+\boldsymbol{A}\cdot\left(\nabla\times\boldsymbol{B}\right)
\end{align*}

\begin{align*}
\nabla\times\left(\boldsymbol{A}\times\boldsymbol{B}\right) & =\partial_{j}\left(A_{i}B_{j}\right)\boldsymbol{e}_{i}-\partial_{j}\left(A_{j}B_{i}\right)\boldsymbol{e}_{i}\\
 & =\boldsymbol{e}_{i}B_{j}\partial_{j}A_{i}-\boldsymbol{e}_{i}B_{i}\partial_{j}A_{j}-\boldsymbol{e}_{i}A_{j}\partial_{j}B_{i}+\boldsymbol{e}_{i}A_{j}\partial_{j}B_{i}\\
 & =\left(\boldsymbol{B}\cdot\nabla\right)\boldsymbol{A}-\boldsymbol{B}\left(\nabla\cdot\boldsymbol{A}\right)-\left(\boldsymbol{A}\cdot\nabla\right)\boldsymbol{B}+\boldsymbol{A}\left(\nabla\cdot\boldsymbol{B}\right)
\end{align*}

\begin{align*}
\nabla\left(\boldsymbol{A}\cdot\boldsymbol{B}\right) & =\left(\boldsymbol{B}\cdot\nabla\right)\boldsymbol{A}+\left(\boldsymbol{A}\cdot\nabla\right)\boldsymbol{B}+\boldsymbol{B}\times\left(\nabla\times\boldsymbol{A}\right)+\boldsymbol{A}\times\left(\nabla\times\boldsymbol{B}\right)\\
 & =B_{j}\partial_{j}A_{i}\boldsymbol{e}_{i}+A_{j}\partial_{j}B_{i}\boldsymbol{e}_{i}+\left(B_{j}\partial_{i}A_{j}-B_{j}\partial_{j}A_{i}\right)\boldsymbol{e}_{i}+\left(A_{j}\partial_{i}B_{j}-A_{j}\partial_{j}B_{i}\right)\boldsymbol{e}_{i}\\
 & =B_{j}\partial_{i}A_{j}\boldsymbol{e}_{i}+A_{j}\partial_{i}B_{j}\boldsymbol{e}_{i}\\
 & =\boldsymbol{e}_{i}\partial_{i}\left(A_{i}B_{j}\right)
\end{align*}

\begin{align*}
\rot\grad\phi & =\nabla\times\nabla\phi=\boldsymbol{e}_{j}\partial_{j}\times\boldsymbol{e}_{k}\partial_{k}\phi\text{ta}\\
 & =\boldsymbol{e}_{i}\varepsilon_{ijk}\partial_{j}\partial_{k}\phi & \left(\partial_{j}\partial_{k}=\partial_{k}\partial_{j}\right)\\
 & =0 & \left(\varepsilon_{ijk}=-\varepsilon_{ikj}\right)
\end{align*}

\[
\left(\boldsymbol{e}_{x}\varepsilon_{xyz}\partial_{y}\partial_{z}+\boldsymbol{e}_{x}\varepsilon_{xzy}\partial_{z}\partial_{y}+\cdots=\boldsymbol{e}_{x}\left(\partial_{y}\partial_{z}-\partial_{z}\partial_{y}\right)+\cdots=0\right)
\]

\begin{align*}
\div\rot\boldsymbol{A} & =\boldsymbol{e}_{l}\partial_{l}\cdot\boldsymbol{e}_{i}\left(\varepsilon_{ijk}\partial_{j}\partial_{k}\right)\\
 & =\partial_{i}\left(\varepsilon_{ijk}\partial_{j}A_{k}\right)=\varepsilon_{ijk}\partial_{i}\partial_{j}A_{k}=0
\end{align*}


\paragraph{ヘルツホルムの法則}

\[
\boldsymbol{F}\left(\boldsymbol{r}\right)=\boldsymbol{F}^{L}\left(\boldsymbol{r}\right)+\boldsymbol{F}^{T}\left(\boldsymbol{r}\right)
\]

\[
\nabla\times\boldsymbol{F}^{L}\left(\boldsymbol{r}\right)=0\Rightarrow\boldsymbol{F}^{L}\left(\boldsymbol{r}\right)=-\nabla\phi\left(\boldsymbol{r}\right)
\]

\[
\nabla\cdot\boldsymbol{F}^{T}\left(\boldsymbol{r}\right)=0\Rightarrow\boldsymbol{F}^{T}\left(\boldsymbol{r}\right)=\nabla\times\boldsymbol{A}\left(\boldsymbol{r}\right)
\]

$\phi\left(\boldsymbol{r}\right)$: スカラーポテンシャル

$\boldsymbol{A}\left(\boldsymbol{r}\right)$: ベクトルポテンシャル

\paragraph{証明}

フーリエ表示

\begin{align*}
\boldsymbol{F}\left(\boldsymbol{r}\right) & =\int\e^{i\boldsymbol{k}\cdot\boldsymbol{r}}\boldsymbol{F}\left[\boldsymbol{K}\right]\mathrm{d}\boldsymbol{K}\\
\boldsymbol{F}^{L}\left(\boldsymbol{r}\right) & =\int\e^{i\boldsymbol{k}\cdot\boldsymbol{r}}\boldsymbol{F}^{L}\left[\boldsymbol{K}\right]\mathrm{d}\boldsymbol{K}\\
\boldsymbol{F}^{T}\left(\boldsymbol{r}\right) & =\int\e^{i\boldsymbol{k}\cdot\boldsymbol{r}}\boldsymbol{F}^{T}\left[\boldsymbol{K}\right]\mathrm{d}\boldsymbol{K}
\end{align*}

まず$\boldsymbol{F}\left[\boldsymbol{k}\right]$から縦成分$\left(\frac{\boldsymbol{k}}{k}\right)$を取り出し、縦成分ベクトル$\boldsymbol{F}^{L}\left[\boldsymbol{K}\right]=\frac{\boldsymbol{K}\cdot\boldsymbol{F}\left[\boldsymbol{K}\right]}{K^{2}}\cdot\boldsymbol{K}$を定義する。

横成分ベクトル$\boldsymbol{F}^{T}\left[\boldsymbol{K}\right]=\boldsymbol{F}\left[\boldsymbol{K}\right]-\boldsymbol{F}^{L}\left[\boldsymbol{K}\right]$

\begin{align*}
\nabla\times\boldsymbol{F}^{L}\left(\boldsymbol{r}\right) & =\boldsymbol{e}_{i}\varepsilon_{ijm}\pd{F_{m}^{L}\left(\boldsymbol{r}\right)}{r_{j}}\\
 & =\boldsymbol{e}_{i}\varepsilon_{ijm}\int\pd{\e^{ikr}}{r_{j}}\frac{\boldsymbol{K}\cdot\boldsymbol{F}\left[\boldsymbol{K}\right]}{K^{2}}K_{m}\mathrm{d}K\\
 & =i\boldsymbol{e}_{i}\int\e^{i\boldsymbol{k}\cdot\boldsymbol{r}}\left(\varepsilon_{ijm}K_{j}K_{m}\right)\frac{\boldsymbol{K}\cdot\boldsymbol{F}\left[\boldsymbol{K}\right]}{K^{2}}\mathrm{d}\boldsymbol{K}=0
\end{align*}

\begin{align*}
\nabla\cdot\boldsymbol{F}^{T}\left(\boldsymbol{r}\right) & =\pd{}{r_{j}}\int\e^{i\boldsymbol{k}\cdot\boldsymbol{r}}\left[K_{j}\left[\boldsymbol{K}\right]\cdot\frac{\boldsymbol{K}\cdot\boldsymbol{F}\left[\boldsymbol{K}\right]}{X^{2}}K_{j}\right]\mathrm{d}\boldsymbol{K}\\
 & =i\int\boldsymbol{e}^{i\boldsymbol{k}\cdot\boldsymbol{r}}K_{j}\left[F_{j}\left[\boldsymbol{K}\right]\cdot\frac{\boldsymbol{K}\cdot\boldsymbol{F}\left[\boldsymbol{K}\right]}{K^{2}}K_{j}\right]\mathrm{d}\boldsymbol{K}\\
 & =i\int\boldsymbol{e}^{i\boldsymbol{k}\cdot\boldsymbol{r}}\left[\boldsymbol{K}\cdot\boldsymbol{F}\left[\boldsymbol{K}\right]-\boldsymbol{K}\cdot\boldsymbol{F}\left[\boldsymbol{K}\right]\right]\mathrm{d}\boldsymbol{K}\\
 & =0
\end{align*}

\begin{align*}
\phi\left(\boldsymbol{r}\right) & =\int\e^{i\boldsymbol{k}\cdot\boldsymbol{r}}\phi\left[\boldsymbol{K}\right]\mathrm{d}\boldsymbol{K}\\
-\nabla\phi\left(\boldsymbol{r}\right) & =-i\int\e^{i\boldsymbol{k}\cdot\boldsymbol{r}}\boldsymbol{K}\phi\left[\boldsymbol{K}\right]\mathrm{d}\boldsymbol{K}\\
\boldsymbol{F}^{L}\left(\boldsymbol{r}\right) & =\int\e^{i\boldsymbol{k}\cdot\boldsymbol{r}}\frac{\boldsymbol{K}\cdot\boldsymbol{F}\left[\boldsymbol{K}\right]}{K^{2}}\boldsymbol{K}\mathrm{d}\boldsymbol{K}\\
\phi\left[\boldsymbol{K}\right] & =i\frac{\boldsymbol{K}\cdot\boldsymbol{F}\left[\boldsymbol{K}\right]}{K^{2}}\\
\boldsymbol{A}\left(\boldsymbol{r}\right) & =\int\e^{i\boldsymbol{k}\cdot\boldsymbol{r}}\boldsymbol{A}\left[\boldsymbol{K}\right]\mathrm{d}\boldsymbol{K}\\
\nabla\times\boldsymbol{A}\left(\boldsymbol{r}\right) & =i\boldsymbol{e}_{i}\varepsilon_{ijm}\int\e^{i\boldsymbol{k}\cdot\boldsymbol{r}}K_{j}A_{m}\left[\boldsymbol{K}\right]\mathrm{d}\boldsymbol{K}\\
F^{T}\left(\boldsymbol{r}\right) & =\int\e^{i\boldsymbol{k}\cdot\boldsymbol{r}}\boldsymbol{F}^{T}\left[\boldsymbol{K}\right]\mathrm{d}\boldsymbol{K}\\
i\varepsilon_{ijm}K_{j}A_{m}\left[\boldsymbol{K}\right] & =F_{i}^{T}\left[\boldsymbol{K}\right]
\end{align*}

$\boldsymbol{F}^{T}$は2つの自由度$\left(A_{x},A_{y},A_{z}\right)$

\[
\nabla\cdot\boldsymbol{F}^{T}\left(\boldsymbol{r}\right)=k_{x}F_{x}^{T}+k_{y}F_{y}^{T}+k_{z}F_{z}^{T}=0
\]

\begin{align*}
k_{x}A_{y}-x_{y}A_{x} & =-iF_{z}^{T}\\
k_{y}A_{z}-x_{z}A_{y} & =-iF_{x}^{T}\\
k_{z}A_{x}-x_{x}A_{z} & =-iF_{y}^{T}
\end{align*}

$C\left[\boldsymbol{K}\right]$を任意関数として、
\begin{align*}
A_{x}\left[\boldsymbol{K}\right] & =\frac{1}{K_{y}}\left(K_{x}C\left[\boldsymbol{K}\right]+iF_{z}^{T}\left[\boldsymbol{K}\right]\right)\\
A_{y}\left[\boldsymbol{K}\right] & =C\left[\boldsymbol{K}\right]\\
A_{z}\left[\boldsymbol{K}\right] & =\frac{1}{K_{y}}\left(K_{z}C\left[\boldsymbol{K}\right]-iF_{x}^{T}\left[\boldsymbol{K}\right]\right)
\end{align*}

\begin{align*}
K_{z}A_{x}-K_{x}A_{z} & =\frac{K_{z}}{K_{y}}\left(K_{x}C+iF_{z}^{T}\right)-\frac{K_{x}}{K_{y}}\left(K_{z}C-iF_{x}^{T}\right)\\
 & =\frac{i}{K_{y}}\left(K_{z}F_{z}^{T}+K_{x}F_{x}^{T}\right)=-iF_{y}^{T}
\end{align*}


\paragraph{ゲージ変換}

ベクトル場: $\boldsymbol{F}\left(\boldsymbol{r}\right)=\left(F_{x},F_{y}F_{z}\right)$

ポテンシャル $\left(\phi,A_{x},A_{y}A_{z}\right)$

\[
\boldsymbol{F}\left(\boldsymbol{r}\right)=-\nabla\phi\left(\boldsymbol{r}\right)+\nabla\times\boldsymbol{A}\left(\boldsymbol{r}\right)
\]

ゲージ変換
\[
\boldsymbol{A}\left(\boldsymbol{r}\right)\mapsto\boldsymbol{A}'\left(\boldsymbol{r}\right)=\boldsymbol{A}\left(\boldsymbol{r}\right)+\nabla f\left(\boldsymbol{r}\right)
\]

\begin{align*}
\boldsymbol{F}'\left(\boldsymbol{r}\right) & =-\nabla\phi\left(\boldsymbol{r}\right)+\nabla\times\boldsymbol{A}'\left(\boldsymbol{r}\right)\\
 & =-\nabla\phi\left(\boldsymbol{r}\right)+\nabla\times\boldsymbol{A}\left(\boldsymbol{r}\right)+\nabla\times\nabla f\left(\boldsymbol{r}\right)\\
 & =\boldsymbol{F}\left(\boldsymbol{r}\right)
\end{align*}


\paragraph{磁場}

\[
\boldsymbol{B}=\nabla\times\boldsymbol{A}
\]

1) 対称ゲージ

\[
\boldsymbol{A}=B\left(-\frac{y}{2},\frac{x}{2},0\right)
\]

\[
\nabla\times\boldsymbol{A}=\varepsilon_{ijm}\boldsymbol{e}_{i}\partial_{j}A_{m}=\left(\begin{array}{c}
\partial_{y}A_{z}-\partial_{z}A_{y}\\
\partial_{z}A_{x}-\partial_{x}A_{z}\\
\partial_{x}A_{y}-\partial_{y}A_{x}
\end{array}\right)=\left(\begin{array}{c}
0\\
0\\
B
\end{array}\right)=\boldsymbol{e}_{z}B
\]

2) ランダウゲージ

\[
\boldsymbol{A}'=B\left(0,x,0\right)
\]

\begin{align*}
\nabla\times\boldsymbol{A}' & =\boldsymbol{e}_{z}B\\
\boldsymbol{A}-\boldsymbol{A}' & =\nabla f\left(\boldsymbol{r}\right)\\
\boldsymbol{A}-\boldsymbol{A}' & =-B\left(\frac{y}{2},\frac{x}{2},0\right)=-B\left(\partial_{x}f,\partial_{y}f,\partial_{z}f\right)\\
\Rightarrow f & =-B\frac{xy}{2}
\end{align*}


\section*{第9回}

ベクトル場$\boldsymbol{F}\left(\boldsymbol{r}\right)$が、$\nabla\times\boldsymbol{F}\left(\boldsymbol{r}\right)=\boldsymbol{0}$を満たせば、スカラーポテンシャル$\phi\left(\boldsymbol{r}\right)$が存在し、$\boldsymbol{F}\left(\boldsymbol{r}\right)=-\nabla\phi\left(\boldsymbol{r}\right)$となる。

\begin{align*}
\phi\left(x,y,z\right)= & \int_{x}^{x_{0}}F_{x}\left(x,y,z\right)\mathrm{d}x\\
 & +\int_{y}^{y_{0}}F_{y}\left(x_{0},y,z\right)\mathrm{d}y\\
 & +\int_{z}^{z_{0}}F_{z}\left(x_{0},y,z\right)\mathrm{d}z
\end{align*}


\paragraph{証明}

\[
\nabla\times\boldsymbol{F}\left(\boldsymbol{r}\right)=\boldsymbol{0}
\]

\begin{align*}
\Rightarrow\partial_{x}F_{y}-\partial_{y}F_{x} & =0\\
\partial_{y}F_{z}-\partial_{z}F_{y} & =0\\
\partial_{z}F_{x}-\partial_{x}F_{z} & =0
\end{align*}

\begin{align*}
-\partial_{x}\phi & =F_{x}\left(x,y,z\right)\\
-\partial_{y}\phi & =-\int_{x}^{x_{0}}\left[\partial_{y}F_{x}\left(x,y,z\right)\right]\mathrm{d}x+F_{y}\left(x_{0},y,z\right)\\
 & =-\int_{x}^{x_{0}}\left[\partial_{x}F_{y}\left(x,y,z\right)\right]\mathrm{d}x+F_{y}\left(x_{0},y,z\right)\\
 & =-F_{y}\left(x_{0},y,z\right)+F_{y}\left(x,y,z\right)+F_{y}\left(x_{0},y,z\right)\\
 & =F_{y}\left(x,y,z\right)\\
-\partial_{z}\phi & =-\int_{x}^{x_{0}}\left[\partial_{z}F_{x}\left(x,y,z\right)\right]\mathrm{d}x-\int_{y}^{y_{0}}\left[\partial_{z}F_{y}\left(x_{0},y,z\right)\right]\mathrm{d}y+F_{z}\left(x_{0},y_{0},z\right)\\
 & =-\int_{x}^{x_{0}}\left[\partial_{x}F_{z}\left(x,y,z\right)\right]\mathrm{d}x-\int_{y}^{y_{0}}\left[\partial_{y}F_{z}\left(x_{0},y,z\right)\right]\mathrm{d}y+F_{z}\left(x_{0},y_{0},z\right)\\
 & =-F_{z}\left(x_{0},y,z\right)+F_{z}\left(x,y,z\right)-F_{z}\left(x_{0},y_{0},z\right)+F_{z}\left(x_{0},y,z\right)+F_{z}\left(x_{0},y_{0},z\right)\\
 & =F_{z}\left(x,y,z\right)
\end{align*}

\[
\therefore-\partial_{i}\phi=F_{i}\left(x,y,z\right)
\]

\[
\boldsymbol{F}\left(\boldsymbol{r}\right)=-\nabla\phi\left(\boldsymbol{r}\right)
\]


\paragraph{例}

\[
\boldsymbol{F}=\left(y+\sin z\right)\boldsymbol{e}_{x}+x\boldsymbol{e}_{y}+x\cos\boldsymbol{e}_{z}
\]

\begin{align*}
\nabla\times\boldsymbol{F}= & \boldsymbol{e}_{x}\left(\partial_{y}F_{z}-\partial_{z}F_{y}\right)\\
 & +\boldsymbol{e}_{y}\left(\partial_{z}F_{x}-\partial_{x}F_{z}\right)\\
 & +\boldsymbol{e}_{z}\left(\partial_{x}F_{y}-\partial_{y}F_{x}\right)\\
= & \boldsymbol{e}_{y}\left(\cos z-\cos z\right)+\boldsymbol{e}_{z}\left(1-1\right)=0
\end{align*}

\begin{align*}
\phi\left(x,y,z\right)= & \int_{x}^{0}F_{x}\left(x,y,z\right)\mathrm{d}x\\
 & +\int_{y}^{0}F_{y}\left(x,y,z\right)\mathrm{d}y\\
 & +\int_{z}^{0}F_{z}\left(x,y,z\right)\mathrm{d}z\\
= & -\int_{0}^{x}\left(y+\sin z\right)\mathrm{d}x=-x\left(y+\sin z\right)
\end{align*}

\rule[0.5ex]{1\columnwidth}{1pt}

ベクトル場$\boldsymbol{F}\left(\boldsymbol{r}\right)$が$\nabla\cdot\boldsymbol{F}\left(\boldsymbol{r}\right)=0$を満たせば、ベクトルポテンシャル$\boldsymbol{A}$が存在し、$\boldsymbol{F}\left(\boldsymbol{r}\right)=\nabla\times\boldsymbol{A}\left(\boldsymbol{r}\right)$

一例として、
\begin{align*}
A_{x}\left(x,y,z\right) & =0\\
A_{y}\left(x,y,z\right) & =\int_{x_{0}}^{x}F_{z}\left(x,y,z\right)\mathrm{d}x\\
A_{z}\left(x,y,z\right) & =-\int_{x_{0}}^{x}F_{y}\left(x,y,z\right)\mathrm{d}x+\int+\int_{y_{0}}^{y}F_{x}\left(x_{0},y,z\right)\mathrm{d}y
\end{align*}

\[
\nabla\cdot\boldsymbol{F}\left(\boldsymbol{r}\right)=\boldsymbol{0}\Rightarrow\partial_{x}F_{x}+\partial_{y}F_{y}+\partial_{z}F_{z}=0
\]

\begin{align*}
\partial_{y}A_{z}-\partial_{z}A_{y}= & -\int_{x_{0}}^{x}\left[\partial_{y}F_{y}\left(x,y,z\right)\right]\mathrm{d}x\\
 & +F_{x}\left(x_{0},y,z\right)-\int_{x_{0}}^{x}\left[\partial_{z}F_{z}\left(x,y,z\right)\right]\mathrm{d}x\\
= & \int_{x_{0}}^{x}\left[\partial_{x}F_{x}\left(x,y,z\right)\right]\mathrm{d}x+F_{x}\left(x_{0},y,z\right)\\
= & F_{x}\left(x,y,z\right)
\end{align*}

\begin{align*}
\partial_{z}A_{x}-\partial_{x}A_{z} & =F_{y}\left(x,y,z\right)\\
\partial_{x}A_{y}-\partial_{y}A_{x} & =F_{z}\left(x,y,z\right)
\end{align*}

\[
\boldsymbol{F}\left(\boldsymbol{r}\right)=\nabla\times\boldsymbol{A}\left(\boldsymbol{r}\right)
\]


\paragraph{例}

\begin{align*}
\boldsymbol{F} & =2x^{2}y\boldsymbol{e}_{x}+\left(3yz^{2}-2xy^{2}\right)\boldsymbol{e}_{y}-z^{3}\boldsymbol{e}_{z}
\end{align*}

\begin{align*}
\nabla\times\boldsymbol{F} & =\partial_{x}\left(2x^{2}y\right)+\partial_{y}\left(3yz^{2}-2xy^{2}\right)+\partial_{x}\left(-z^{3}\right)\\
 & =4xy+3z^{2}-4xy-3z^{2}=0
\end{align*}

\begin{align*}
A_{x} & =0\\
A_{y} & =\int_{0}^{x}F_{z}\left(x,y,z\right)=-xz^{3}\\
A_{z} & =-\int_{0}^{x}F_{y}\left(x,y,z\right)+\int_{0}^{y}F_{x}\left(0,y,z\right)\mathrm{d}y\\
 & =\int_{0}^{x}\left(3yz^{2}-2xy^{2}\right)\mathrm{d}x=x^{2}y-3xyz^{2}\\
\boldsymbol{A} & =-xz^{3}\boldsymbol{e}_{y}+\left(x^{2}y^{2}-3xyz^{2}\right)\boldsymbol{e}_{z}
\end{align*}


\paragraph{線素}

\[
\boldsymbol{r}=\left(x\left(t\right),y\left(t\right),z\left(t\right)\right)
\]

\begin{align*}
\mathrm{d}\boldsymbol{r} & =\left(\mathrm{d}x,\mathrm{d}y,\mathrm{d}z\right)\\
 & =\boldsymbol{e}_{x}\mathrm{d}x+\boldsymbol{e}_{y}\mathrm{d}y+\boldsymbol{e}_{z}\mathrm{d}z
\end{align*}

\[
\mathrm{d}r=\left|\mathrm{d}\boldsymbol{r}\right|=\sqrt{\left(\mathrm{d}x\right)^{2}+\left(\mathrm{d}y\right)^{2}+\left(\mathrm{d}z\right)^{2}}
\]

\[
\mathrm{d}\boldsymbol{r}=\d{\boldsymbol{r}}t\mathrm{d}t
\]


\paragraph{長さ}

\[
s\left(t\right)=\int^{t}\mathrm{d}r=\int^{t}\sqrt{\left(\d xt\right)^{2}+\left(\d yt\right)^{2}+\left(\d zt\right)^{2}}\mathrm{d}t
\]


\paragraph{線積分}

\begin{align*}
\int_{C}^{t}\boldsymbol{F}\cdot\mathrm{d}\boldsymbol{r} & =\int_{C}^{t}\left(F_{x}\mathrm{d}x+F_{y}\mathrm{d}y+F_{z}\mathrm{d}z\right)\\
 & =\int_{C}^{t}\left(F_{x}\d xt+F_{y}\d yt+F_{z}\d zt\right)\mathrm{d}t
\end{align*}

ベクトル場$\boldsymbol{F}$がスカラーポテンシャル$\boldsymbol{f}$をもち、$\boldsymbol{F}=-\nabla f$なら、点$A$と$B$を結ぶ任意の曲線に沿っての線積分の値は等しく、
\begin{align*}
\int_{A}^{B}\boldsymbol{F}\cdot\mathrm{d}\boldsymbol{r} & =-\int_{A}^{B}\nabla f\cdot\mathrm{d}\boldsymbol{r}=-\int_{A}^{B}\mathrm{d}f\\
 & =f\left(A\right)-f\left(B\right)
\end{align*}


\paragraph{例}

原点から点$\left(1,2,2\right)$に向かう積分路$C$をとり、
\[
\boldsymbol{F}=x\boldsymbol{e}_{x}+2\left(x+z\right)\boldsymbol{e}_{y}+y\boldsymbol{e}_{z}
\]

\[
\boldsymbol{r}=\left(t,2t,2t\right)\text{ for }0\leqq t\leqq1
\]

\[
\boldsymbol{F}=\left(t,2\left(t+2t\right),2t\right)=\left(t,6t,2t\right)
\]

\[
\mathrm{d}\boldsymbol{r}=\left(1,2,2\right)\mathrm{d}t,\mathrm{d}r=\sqrt{1+2^{2}+2^{2}}\mathrm{d}t=3\mathrm{d}t
\]

\begin{align*}
\int_{C}\boldsymbol{F}\cdot\mathrm{d}\boldsymbol{r} & =\int_{0}^{1}\left(t,6t,2t\right)=\left(1,2,2\right)\mathrm{d}t\\
 & =\int_{0}^{1}\left(t+12t+4t\right)\mathrm{d}t=\frac{17}{2}
\end{align*}


\paragraph{例}

\[
\boldsymbol{F}=\left(3x^{2}y-y^{2}+yz\right)\boldsymbol{e}_{x}+\left(x^{3}-2xy+xz\right)\boldsymbol{e}_{y}+\left(xy-1\right)\boldsymbol{e}_{z}
\]
の線積が経路$C$に依存しないことを示す。

$A=\left(0,0,0\right),B=\left(2,2,2\right)$として積分値を求めよ

答え: $\nabla\times\boldsymbol{F}=\boldsymbol{0}$を言えば良い。

\begin{align*}
\partial_{y}F_{z}-\partial_{z}F_{y} & =x-x=0\\
\partial_{z}F_{x}-\partial_{x}F_{z} & =y-y=0\\
\partial_{x}F_{y}-\partial_{y}F_{x} & =3x^{2}-2y+z-\left(3y^{2}-2y+z\right)=0
\end{align*}

\begin{align*}
f\left(x,y,z\right)= & \int_{x}^{0}F_{x}\left(x,y,z\right)\mathrm{d}x\\
 & +\int_{y}^{0}F_{y}\left(0,y,z\right)\mathrm{d}y\\
 & +\int_{z}^{0}F_{z}\left(0,0,z\right)\mathrm{d}z\\
= & \left[x^{3}y,xy^{2}+xyz\right]_{x}^{0}-\left[z\right]_{z}^{0}\\
= & -x^{3}y+xy^{2}-xyz+z
\end{align*}

\begin{align*}
\int_{A}^{B}\boldsymbol{F}\cdot\mathrm{d}\boldsymbol{r} & =f\left(A\right)-f\left(B\right)\\
 & =\left[x^{3}y-xy^{2}+xyz-z\right]_{A}^{B}\\
 & =2^{4}-2^{3}+2^{3}-2=14
\end{align*}


\paragraph{面素}

局面は2つのパラメータ$\left(s,t\right)$表示できる。

\[
\boldsymbol{r}=\left(x\left(s,t\right),y\left(s,t\right),z\left(s,t\right)\right)
\]


\paragraph{接戦ベクトル(線素)}

パラメータの増分$\delta s,\delta t$に対して、$\delta_{s}\boldsymbol{r}=\pd{\boldsymbol{r}}s\mathrm{d}s,\delta_{t}\boldsymbol{r}=\pd{\boldsymbol{r}}t\mathrm{d}t$この2つの接戦ベクトルを含む平面を接平面と呼ぶ。

面素$\mathrm{d}\boldsymbol{s}=\delta_{s}\boldsymbol{r}\times\delta_{t}\boldsymbol{r}$

第1基本量 $E,F,G$

\[
E\left(s,t\right)=\pd{\boldsymbol{r}}s\cdot\pd{\boldsymbol{r}}s,F\left(s,t\right)=\pd{\boldsymbol{r}}s\cdot\pd{\boldsymbol{r}}t,G\left(s,t\right)=\pd{\boldsymbol{r}}t\cdot\pd{\boldsymbol{r}}t
\]

面素$\mathrm{d}\boldsymbol{s}$は大きさが面積$\mathrm{d}s=\left|\mathrm{d}\boldsymbol{s}\right|$で、方向が接平面の法線$\boldsymbol{n}$とするベクトルである。

\[
\boldsymbol{n}=\frac{\pd{\boldsymbol{r}}s\times\pd{\boldsymbol{r}}t}{\left|\pd{\boldsymbol{r}}s\times\pd{\boldsymbol{r}}t\right|},\mathrm{d}\boldsymbol{s}=\boldsymbol{r}\mathrm{d}s
\]

\[
\mathrm{d}s=\left|\pd{\boldsymbol{r}}s\times\pd{\boldsymbol{r}}t\right|\mathrm{d}s\mathrm{d}t
\]

\[
\left|\boldsymbol{A}\times\boldsymbol{B}\right|^{2}=\left|\boldsymbol{A}\right|^{2}\left|\boldsymbol{B}\right|^{2}-\left(\boldsymbol{A}\cdot\boldsymbol{B}\right)^{2}
\]

\begin{align*}
\mathrm{d}s & =\sqrt{\left|\pd{\boldsymbol{r}}s\right|^{2}\left|\pd{\boldsymbol{r}}t\right|^{2}-\left(\pd{\boldsymbol{r}}s\cdot\pd{\boldsymbol{r}}t\right)^{2}}\\
 & =\sqrt{EG-F^{2}}\mathrm{d}s\mathrm{d}t
\end{align*}

曲面の面積

\[
S=\iint_{D}\mathrm{d}s=\iint\sqrt{EG-F^{2}}\mathrm{d}s\mathrm{d}t
\]


\paragraph{例}

単位球の表面積

単位球面上の点は天頂面$\theta$と方位角$\phi$を用いて$\boldsymbol{r}=\left(x,y,z\right)=\left(\sin\theta\cos\phi,\sin\theta\sin\phi,\cos\theta\right)$とパラメータ表示できる。

$s=\theta,t=\phi$とおくと、
\begin{align*}
E\left(\theta,\phi\right)= & \pd{\boldsymbol{r}}{\theta}\cdot\pd{\boldsymbol{r}}{\theta}\\
= & \left|\left(\cos\theta\cos\phi,\cos\theta\sin\phi,-\sin\theta\right)\right|^{2}\\
= & 1\\
F\left(\theta,\phi\right)= & \pd{\boldsymbol{r}}{\theta}\cdot\pd{\boldsymbol{r}}{\phi}\\
= & \left(\cos\theta\cos\phi,\cos\theta\sin\phi,-\sin\theta\right)\cdot\\
 & \left(-\sin\theta\sin\phi,\sin\theta\cos\phi,0\right)\\
= & 0\\
G\left(\theta,\phi\right)= & \left|\left(-\sin\theta\sin\phi,\sin\theta\cos\phi,0\right)\right|^{2}\\
= & \sin\theta
\end{align*}

\begin{align*}
S & =\iint\sqrt{EG-F^{2}}\mathrm{d}\theta\mathrm{d}\phi\\
 & =\int_{0}^{2\pi}\mathrm{d}\phi\int_{0}^{\pi}\sin\theta\mathrm{d}\theta=4\pi
\end{align*}


\section*{第10回}

{[}一部紛失{]}

\paragraph{例}

曲面$z=f\left(x,y\right)$の面素ベクトル($z$が1価に解けてる)

\[
\boldsymbol{r}=x\boldsymbol{e}_{x}+y\boldsymbol{e}_{y}+f\left(x,y\right)\boldsymbol{e}_{z}
\]

\begin{align*}
\pd{\boldsymbol{r}}x\times\pd{\boldsymbol{r}}y & =\left(\boldsymbol{e}_{x}+\left(\partial_{x}f\right)\boldsymbol{e}_{z}\right)\times\left(\boldsymbol{e}_{y}+\left(\partial_{y}f\right)\boldsymbol{e}_{z}\right)\\
 & =-\partial_{x}f\boldsymbol{e}_{x}-\partial_{y}f\boldsymbol{e}_{y}+\boldsymbol{e}_{z}
\end{align*}

\[
\frac{\pd{\boldsymbol{r}}x\times\pd{\boldsymbol{r}}y}{\left|\pd{\boldsymbol{r}}x\times\pd{\boldsymbol{r}}y\right|}=\frac{-\partial_{x}f\boldsymbol{e}_{x}-\partial_{y}f\boldsymbol{e}_{y}+\boldsymbol{e}_{z}}{\sqrt{\left(\partial_{x}f\right)^{2}+\left(\partial_{y}f\right)^{2}+1}}
\]

\[
\mathrm{d}s=\sqrt{\left(\partial_{x}f\right)^{2}+\left(\partial_{y}f\right)^{2}+1}\mathrm{d}x\mathrm{d}y
\]

\[
\mathrm{d}\boldsymbol{s}=\boldsymbol{n}\mathrm{d}s
\]


\paragraph{例}

曲面$F\left(x,y,z\right)$の法線ベクトル$\boldsymbol{n}$

$F\left(x,y,z\right)=0\Rightarrow z=f\left(x,y\right)$と解く。

\[
\mathrm{d}F=\partial_{x}F\mathrm{d}x+\partial_{y}F\mathrm{d}y+\partial_{z}F\mathrm{d}z
\]
\[
\Rightarrow\mathrm{d}z-\frac{\partial_{x}F}{\partial_{z}F}\mathrm{d}x-\frac{\partial_{y}F}{\partial_{z}F}\mathrm{d}y
\]

一方$\mathrm{d}z=\partial_{x}f\mathrm{d}x+\partial_{y}f\mathrm{d}y$なので、
\[
\partial_{x}f=-\frac{\partial_{x}F}{\partial_{z}F},\partial_{y}f=-\frac{\partial_{y}F}{\partial_{z}F}
\]
となる。

\begin{align*}
\boldsymbol{n} & =\frac{-\partial_{x}f\boldsymbol{e}_{x}-\partial_{y}f\boldsymbol{e}_{y}+\boldsymbol{e}_{z}}{\sqrt{\left(\partial_{x}f\right)^{2}+\left(\partial_{y}f\right)^{2}+1}}\\
 & =\frac{\partial_{x}F\boldsymbol{e}_{x}+\partial_{y}F\boldsymbol{e}_{y}+\partial_{z}F\boldsymbol{e}_{z}}{\sqrt{\left(\partial_{x}F\right)^{2}+\left(\partial_{y}F\right)^{2}+\left(\partial_{z}F\right)^{2}}}
\end{align*}


\paragraph{例}

曲面$x^{2}y+y^{2}x+z^{2}y=3$上の点$\boldsymbol{P}=\left(0,3,1\right)$における単位法線ベクトルと接平面を求めよ

\[
f\left(x,y,z\right)=x^{2}y+y^{2}x+z^{2}y=3
\]

\[
\nabla f=\left(2xy+y^{2}\right)\boldsymbol{e}_{x}+\left(x^{2}+2xy+z^{2}\right)\boldsymbol{e}_{y}+2yz\boldsymbol{e}_{z}
\]

\[
\when{\nabla f}_{P=\left(0,3,1\right)}=9\boldsymbol{e}_{x}+\boldsymbol{e}_{y}+6\boldsymbol{e}_{z}
\]

規格化して
\[
\boldsymbol{n}=\frac{9\boldsymbol{e}_{x}+\boldsymbol{e}_{y}+6\boldsymbol{e}_{z}}{\sqrt{118}}
\]

接平面は法線ベクトルに垂直なので、接平面上の点を$\boldsymbol{r}$とすると、
\[
\boldsymbol{n}\cdot\left(\boldsymbol{r}-\boldsymbol{P}\right)=0\Rightarrow\left(9\boldsymbol{e}_{x}+\boldsymbol{e}_{y}+6\boldsymbol{e}_{z}\right)\cdot\left(x\boldsymbol{e}_{x}+\left(y-3\right)\boldsymbol{e}_{y}+\left(z-1\right)\boldsymbol{e}_{z}\right)
\]

\[
9x+\left(y-3\right)+6\left(z-1\right)=0
\]

\[
9x+y+6z=9
\]


\paragraph{例}

半径$a$の球面上でのベクトル場$\nabla\left(\frac{1}{r}\right)$の面積分

\[
\nabla\frac{1}{r}=\boldsymbol{e}_{x}\partial_{x}\frac{1}{r}+\boldsymbol{e}_{y}\partial_{y}\frac{1}{r}+\boldsymbol{e}_{z}\partial_{z}\frac{1}{r}
\]

\[
\left(\partial_{x}\frac{1}{r}=\partial_{x}\frac{1}{\sqrt{x^{2}+y^{2}+z^{2}}}=\frac{-\frac{1}{2}2x}{\left(x^{2}+y^{2}+z^{2}\right)^{\frac{3}{2}}}\right)
\]

\[
\nabla\frac{1}{r}=-\frac{x}{r^{3}}\boldsymbol{e}_{x}-\frac{y}{r^{3}}\boldsymbol{e}_{y}-\frac{z}{r^{3}}\boldsymbol{e}_{z}=-\frac{\boldsymbol{r}}{r^{3}}
\]

\[
\boldsymbol{n}=\frac{\boldsymbol{r}}{r}
\]

\begin{align*}
\int_{S}\mathrm{d}\boldsymbol{s}\cdot\nabla\frac{1}{r} & =-\int_{s}\frac{\boldsymbol{r}}{r^{3}}\mathrm{d}\boldsymbol{s}=-\int_{s}\frac{1}{r^{2}}\mathrm{d}s=-\frac{1}{a^{2}}\int_{s}\mathrm{d}s\\
 & =-\frac{1}{a^{2}}4\pi a^{2}=-4\pi
\end{align*}


\paragraph{例}

曲面$\left(s\cos t,s\sin t,t\right)$の単位法線ベクトル

\begin{align*}
\partial_{s}\boldsymbol{r} & =\boldsymbol{e}_{x}\cos t+\boldsymbol{e}_{y}\sin t\\
\partial_{t}\boldsymbol{r} & =-\boldsymbol{e}_{x}s\sin t+\boldsymbol{e}_{y}s\cos t+\boldsymbol{e}_{z}
\end{align*}

\begin{align*}
\pd{\boldsymbol{r}}s\times\pd{\boldsymbol{r}}t & =\left|\begin{array}{ccc}
\boldsymbol{e}_{x} & \boldsymbol{e}_{y} & \boldsymbol{e}_{z}\\
\cos t & \sin t & 0\\
-s\sin t & s\cos t & 1
\end{array}\right|\\
 & =\boldsymbol{e}_{x}\sin t-\boldsymbol{e}_{y}s\cos t+s\boldsymbol{e}_{z}
\end{align*}

\[
\boldsymbol{n}=\frac{\pd{\boldsymbol{r}}s\times\pd{\boldsymbol{r}}t}{\left|\pd{\boldsymbol{r}}s\times\pd{\boldsymbol{r}}t\right|}=\frac{\boldsymbol{e}_{x}\sin t-\boldsymbol{e}_{y}s\cos t+s\boldsymbol{e}_{z}}{\sqrt{1+s^{2}}}
\]


\paragraph{Greenの定理}

\begin{align*}
\iint_{s}\partial_{x}f\mathrm{d}x\mathrm{d}y & =\int_{\partial s}\mathrm{d}y\left[f\left(x_{\text{右}}\left(y\right),y\right)-f\left(x_{\text{左}}\left(y\right),y\right)\right]\\
 & =\oint_{\partial s}f\left(x,y\right)\mathrm{d}y
\end{align*}

\begin{align*}
\iint_{s}\partial_{y}f\mathrm{d}x\mathrm{d}y & =\int_{\partial s}\left[f\left(x,y_{\text{上}}\left(x\right)\right)-f\left(x,y_{\text{下}}\left(x\right)\right)\right]\\
 & =-\oint_{\partial s}f\left(x,y\right)\mathrm{d}x
\end{align*}


\paragraph{Stokesの定理}

\begin{align*}
\int_{s}\left(\nabla\times\boldsymbol{F}\right)\cdot\mathrm{d}\boldsymbol{s} & =\sum\int_{-\Delta s}\left(\nabla\times\boldsymbol{F}\right)\cdot\mathrm{d}\boldsymbol{s}\\
 & =\sum\int_{\Delta s}\left(\partial_{x}F_{y}-\partial_{y}F_{x}\right)\mathrm{d}x\mathrm{d}y\\
 & =\sum\oint_{\partial\left(\Delta s\right)}\left(F_{x}\mathrm{d}x+F_{y}\mathrm{d}y\right)\quad\left(\because\text{Greenの定理}\right)\\
 & =\sum\oint_{\partial\left(\Delta s\right)}\boldsymbol{F}\cdot\mathrm{d}r=\oint_{\partial s}\boldsymbol{F}\cdot\mathrm{d}\boldsymbol{r}
\end{align*}


\paragraph{アンペールの法則}

電流密度$I\left(\boldsymbol{x}\right)$があると磁場$\boldsymbol{B}\left(\boldsymbol{x}\right)$が発生する。

\[
\mu=\int_{s}\boldsymbol{I}\cdot\mathrm{d}\boldsymbol{s}=\oint_{\partial s}\boldsymbol{B}\cdot\mathrm{d}\boldsymbol{r}=\int_{s}\nabla\times\boldsymbol{B}\cdot\mathrm{d}\boldsymbol{s}
\]

\[
\Rightarrow\nabla\times\boldsymbol{B}=\mu_{0}\boldsymbol{I}
\]

(Maxwellの方程式)

\paragraph{例}

円周$x^{2}+y^{2}=4$に沿って
\[
\boldsymbol{F}=\left(x^{2}+y\right)\boldsymbol{e}_{x}+\left(x^{2}+2z\right)\boldsymbol{e}_{y}+2y\boldsymbol{e}_{z}
\]
の線積分を求めよ。

\[
\nabla\times\boldsymbol{F}=\left(2x-1\right)\boldsymbol{e}_{z}
\]

$\mathrm{d}\boldsymbol{s}=\boldsymbol{e}_{z}\mathrm{d}x\mathrm{d}y$なので、
\begin{align*}
\oint_{\partial s}\boldsymbol{F}\mathrm{d}\boldsymbol{s} & =\int_{s}\nabla\times\boldsymbol{F}\cdot\mathrm{d}\boldsymbol{s}=\int_{s}\left(2x+1\right)\mathrm{d}x\mathrm{d}y\\
 & =\int_{0}^{2\pi}\mathrm{d}\theta\int_{0}^{2}r\mathrm{d}r\left(2r\cos\theta-1\right)\\
 & =-2\pi\int_{0}^{2}r\mathrm{d}r=-4\pi
\end{align*}


\paragraph{Gaussの定理}

3次元の閉曲面$\partial V$に囲まれた領域$V$に対して
\begin{align*}
\int_{V}\nabla\cdot\boldsymbol{E}\mathrm{d}^{3}r & =\int_{V}\left(\partial_{x}E_{x}+\partial_{y}E_{y}+\partial_{z}E_{z}\right)\mathrm{d}x\mathrm{d}y\mathrm{d}z\\
 & =\int_{\partial V_{\text{左}}}E_{x}\mathrm{d}y\mathrm{d}z-\int_{\partial V_{\text{右}}}E_{x}\mathrm{d}y\mathrm{d}z+\cdots\\
 & =\oint_{\partial V}E_{x}\mathrm{d}s_{x}+E_{y}\mathrm{d}s_{y}+E_{z}\mathrm{d}s_{z}\\
 & =\oint_{\partial V}\boldsymbol{E}\cdot\mathrm{d}\boldsymbol{s}
\end{align*}


\paragraph{クーロンの法則}

電荷$Q_{i}$があると電場$\boldsymbol{E}\left(\boldsymbol{r}\right)$が発生する。

\[
\varepsilon_{0}\boldsymbol{E}\left(\boldsymbol{r}\right)=\frac{1}{4\pi}\sum_{i}\frac{Q_{i}}{\left|\boldsymbol{r}-x_{i}\right|^{2}}\frac{\boldsymbol{r}-x_{i}}{\left|\boldsymbol{r}-x_{i}\right|}
\]

領域$V$ の中の総電荷を$Q_{V}=\sum_{i\left(x_{i}\in V\right)}Q_{i}$とする。

Gaussの閉塞

\begin{align*}
\varepsilon_{0}\int_{V}\nabla\cdot\boldsymbol{E}\mathrm{d}^{3}\boldsymbol{r} & =\varepsilon_{0}\oint_{\partial r}\boldsymbol{E}\cdot\mathrm{d}\boldsymbol{s}/\frac{1}{4\pi}\sum_{i=1}^{n}Q_{i}\oint_{\partial r}\frac{\boldsymbol{r}-x_{i}}{\left|\boldsymbol{r}-x_{i}\right|^{3}}\mathrm{d}\boldsymbol{s}\\
 & =Q_{r}
\end{align*}

電荷密度$\rho\left(\boldsymbol{r}\right),Q_{V}=\int_{V}\rho\left(\boldsymbol{r}\right)\mathrm{d}^{3}\boldsymbol{r}$

\[
\varepsilon_{0}\int_{V}\nabla\cdot\boldsymbol{E}\mathrm{d}^{3}\boldsymbol{r}=Q_{V}=\int_{V}\rho\left(\boldsymbol{r}\right)\mathrm{d}^{3}\boldsymbol{r}
\]

\[
\Rightarrow\varepsilon_{0}\nabla\cdot\boldsymbol{E}\left(\boldsymbol{r}\right)=\rho\left(\boldsymbol{r}\right)
\]

(Maxwellの方程式)

\paragraph{例}

半径$a$の球面上で
\[
\boldsymbol{F}=x^{3}\boldsymbol{e}_{x}+y^{3}\boldsymbol{e}_{y}+z^{3}\boldsymbol{e}_{z}
\]
の面積分

\[
\nabla\cdot\boldsymbol{F}=3\left(x^{2}+y^{2}+z^{2}\right)
\]

\begin{align*}
\int_{S}\boldsymbol{F}\cdot\mathrm{d}\boldsymbol{s} & =\int_{V}\nabla\cdot\boldsymbol{F}\mathrm{d}x\mathrm{d}y\mathrm{d}z\\
 & =\int_{V}3\left(x_{2}+y^{2}+z^{2}\right)\mathrm{d}x\mathrm{d}y\mathrm{d}z\\
 & =4\pi\times3\int_{0}^{a}r^{2}\cdot r^{2}\mathrm{d}r=\frac{12}{5}\pi a^{5}
\end{align*}

\end{document}
