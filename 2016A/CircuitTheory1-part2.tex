%% LyX 2.2.2 created this file.  For more info, see http://www.lyx.org/.
%% Do not edit unless you really know what you are doing.
\documentclass[english]{article}
\usepackage[T1]{fontenc}
\usepackage[utf8]{inputenc}
\usepackage[a5paper]{geometry}
\geometry{verbose,tmargin=2cm,bmargin=2cm,lmargin=1cm,rmargin=1cm}
\setlength{\parskip}{\smallskipamount}
\setlength{\parindent}{0pt}
\usepackage{textcomp}
\usepackage{amsmath}
\usepackage{amssymb}
\usepackage{graphicx}

\makeatletter
%%%%%%%%%%%%%%%%%%%%%%%%%%%%%% User specified LaTeX commands.
\usepackage[dvipdfmx]{hyperref}
\usepackage[dvipdfmx]{pxjahyper}

\makeatother

\usepackage{babel}
\begin{document}

\title{2016-A 電気回路理論第一 後半}

\author{教員: 小関泰之 入力: 高橋光輝}

\maketitle
\global\long\def\pd#1#2{\frac{\partial#1}{\partial#2}}
\global\long\def\d#1#2{\frac{\mathrm{d}#1}{\mathrm{d}#2}}
\global\long\def\pdd#1#2{\frac{\partial^{2}#1}{\partial#2^{2}}}
\global\long\def\dd#1#2{\frac{\mathrm{d}^{2}#1}{\mathrm{d}#2^{2}}}
\global\long\def\e{\mathrm{e}}
\global\long\def\i{\mathrm{i}}
\global\long\def\j{\jmath}
\global\long\def\grad{\mathrm{grad}}
\global\long\def\rot{\mathrm{rot}}
\global\long\def\div{\mathrm{div}}
\global\long\def\diag{\mathrm{diag}}
\global\long\def\when#1{\left.#1\right|}


\section*{第1回}

\paragraph{講義ページ}

https://sites.google.com/site/ysozeki/lecture

\section{過渡現象}

\subsection{過渡現象とは?}
\begin{center}
\includegraphics{images/CircuitTheory1-part2/1-1}
\par\end{center}

応用: ディジタル回路・アナログ回路・制御・物理現象

\paragraph{4つの見方}
\begin{itemize}
\item 微分方程式(今日)
\item 周波数特性
\item インパルス応答・ステップ応答
\item 伝達関数(複素周波数平面)
\end{itemize}
ラプラス変換・フーリエ変換がこの4つを繋げてくれる。

→複雑な回路へ

\subsection{回路方程式から微分方程式へ}
\begin{center}
\includegraphics{images/CircuitTheory1-part2/1-2}
\par\end{center}

\subsubsection{基本: KVL (キルヒホッフの電圧則)}
\begin{center}
\includegraphics{images/CircuitTheory1-part2/1-3}
\par\end{center}

\[
v_{R}+v_{L}+v_{C}=v\left(t\right)
\]

\[
Ri+L\d it+\frac{1}{C}\int i\mathrm{d}t=v\left(t\right)
\]

→微分

\[
R\d it+L\dd it+\frac{i}{C}=\d{v\left(t\right)}t
\]


\subsubsection{複雑な回路}

連立微分方程式
\begin{center}
\includegraphics{images/CircuitTheory1-part2/1-4}
\par\end{center}

\[
\begin{cases}
Ri_{1}+L\d{}t\left(i_{1}+i_{2}\right)=v\left(t\right)\\
\frac{1}{C}\int i_{2}\mathrm{d}t+R_{2}i_{2}=L\d{}t\left(i_{1}+i_{2}\right)
\end{cases}
\]


\subsubsection{相互インダクタンス}

向きに注意
\begin{center}
\includegraphics{images/CircuitTheory1-part2/1-5}
\par\end{center}

\[
\begin{cases}
Ri_{1}+L_{1}\d{i_{1}}t-M\d{i_{2}}t=v\left(t\right)\\
L_{2}\d{i_{2}}t-M\d{i_{1}}t+\frac{1}{L}\int i_{2}\mathrm{d}t=0
\end{cases}
\]


\subsection{RC回路}

(あとでRL回路・RLC回路)
\begin{center}
\includegraphics{images/CircuitTheory1-part2/1-6}
\par\end{center}

\[
Ri+\frac{1}{C}\int i\mathrm{d}t=V\left(t\geqq0\right)
\]

微分して、
\[
R\d it+\frac{1}{C}i=0
\]

変数分離して、
\[
\frac{\mathrm{d}i}{i}=-\frac{1}{RC}\mathrm{d}t
\]

積分して、
\[
\log i=-\frac{1}{RC}+K\rightarrow i\left(t\right)=k\e^{-\frac{t}{RC}}
\]

ただし$k=\e^{K}$。

スイッチを入れた直後、$v_{c}\left(0+\right)=0$より初期条件
\[
i\left(0+\right)=\frac{V}{R}=k
\]

\[
i\left(t\right)=\frac{V}{R}\e^{-\frac{1}{RC}}\left(t\geqq0\right)
\]

\[
v_{R}\left(t\right)=Ri\left(t\right)=V\e^{-\frac{t}{RC}}
\]
\[
v_{c}\left(t\right)=V-v_{R}\left(t\right)=V\left(1-\e^{-\frac{t}{RC}}\right)
\]

\begin{center}
\includegraphics{images/CircuitTheory1-part2/1-7}
\par\end{center}

・$C$に供給されるパワー

\begin{align*}
P_{C}\left(t\right) & =v_{C}\left(t\right)i\left(t\right)\\
 & =\frac{V^{2}}{R}\e^{-\frac{t}{RC}}\left(1-\e^{-\frac{t}{RC}}\right)
\end{align*}

・時刻$t$までに蓄えられるエネルギー

\begin{align*}
W_{C}\left(t\right) & =\int_{0}^{t}P_{C}\left(t\right)\mathrm{d}t\\
 & =\frac{1}{2}CV^{2}\left(1-2\e^{-\frac{t}{\tau}}+\e^{-\frac{2t}{\tau}}\right)
\end{align*}

$t\rightarrow\infty$で$W_{C}\rightarrow\frac{1}{c}CV^{2}$

\paragraph{電源を短絡すると?}
\begin{center}
\includegraphics{images/CircuitTheory1-part2/1-8}
\par\end{center}

\[
i\left(t\right)=k\e^{-\frac{t}{RC}}
\]

$v_{C}\left(0+\right)=V$より、
\[
i\left(0+\right)=-\frac{V}{R}
\]

\[
i\left(t\right)=-\frac{V}{R}\e^{-\frac{t}{RC}}\left(t\geqq0\right)
\]

\begin{center}
\includegraphics{images/CircuitTheory1-part2/1-9}
\par\end{center}

\subsection{RL回路}
\begin{center}
\includegraphics{images/CircuitTheory1-part2/1-10}
\par\end{center}

\[
Ri+L\d it=V\left(t\geqq0\right)
\]

変数分離して、
\[
i+\frac{L}{R}\d it=\frac{V}{R}
\]
\[
i-\frac{V}{R}=-\frac{L}{R}\d it
\]
\[
\frac{\mathrm{d}i}{i-\frac{V}{R}}=-\frac{R}{L}\mathrm{d}t
\]
\[
\log\left(i-\frac{V}{R}\right)=-\frac{R}{L}t
\]
\[
\therefore i\left(t\right)=\frac{V}{R}+k\e^{\frac{R}{L}t}
\]

インダクタの電流は連続的に変化するため、初期条件$i\left(0+\right)=0=\frac{V}{R}+k$

\[
i\left(t\right)=\frac{V}{R}\left(1-\e^{-\frac{R}{L}t}\right)
\]

\begin{center}
\includegraphics{images/CircuitTheory1-part2/1-11}
\par\end{center}

・$L$に供給されるパワー

\begin{align*}
P_{L}\left(t\right) & =v_{L}\left(t\right)i\left(t\right)\\
 & =\frac{V^{2}}{R}\e^{-\frac{R}{L}t}\left(1-\e^{-\frac{R}{L}t}\right)
\end{align*}

・時刻$t$までに蓄えられるエネルギー

\begin{align*}
W_{L}\left(t\right) & =\int_{0}^{t}P_{L}\left(t\right)\mathrm{d}t\\
 & \xrightarrow{t\rightarrow\infty}\frac{L}{2}\left(\frac{V}{R}\right)^{2}
\end{align*}

・電源短絡時

\[
Ri+L\d it=0
\]

\[
i\left(t\right)=k\e^{-\frac{R}{L}t}
\]

$i\left(0+\right)=\frac{V}{R}$より、
\[
i\left(t\right)=\frac{V}{R}\e^{-\frac{R}{L}t}
\]

\begin{center}
\includegraphics{images/CircuitTheory1-part2/1-12}
\par\end{center}

\subsection{RLC回路}
\begin{center}
\includegraphics{images/CircuitTheory1-part2/1-13}
\par\end{center}

\[
Ri+\frac{1}{C}\int i\mathrm{d}t+L\d it=V\left(t\geqq0\right)
\]

微分して、
\[
L\dd it+R\d it+\frac{1}{C}i=0
\]

試行解$k=\e^{st}$($s$: 複素数)を代入

\[
Ls^{2}+Rs+\frac{1}{C}=0
\]

\begin{align*}
s & =-\frac{R}{2L}\pm\sqrt{\left(\frac{R}{2L}\right)^{2}-\frac{1}{LC}}\\
 & =s_{1},s_{2}
\end{align*}
とし、場合分けする。
\begin{enumerate}
\item $s$: 2つの実数解
\item $s$: 重解
\item $s$: 複素数解
\end{enumerate}

\paragraph{下準備}

i) オイラーの公式

\[
\e^{\j\theta}=\cos\theta+\j\sin\theta
\]

覚え方: $f\left(\theta\right)=\e^{\j\theta}$とすると、
\[
\begin{cases}
f\left(0\right)=1\\
\d{}{\theta}f\left(\theta\right)=\j f\left(\theta\right)
\end{cases}
\]

\begin{center}
\includegraphics{images/CircuitTheory1-part2/1-14}
\par\end{center}

ii) 三角関数

\[
\cos\theta=\frac{\e^{\j\theta}-\e^{-\j\theta}}{2},\sin\theta=\frac{\e^{\j\theta}-\e^{-\j\theta}}{2\j}
\]

\begin{center}
\includegraphics{images/CircuitTheory1-part2/1-15}
\par\end{center}

iii) 双曲線関数

\[
\cosh\theta=\frac{\e^{\theta}+\e^{-\theta}}{2},\sinh\theta=\frac{\e^{\theta}-\e^{-\theta}}{2}
\]

\begin{enumerate}
\item $\left(\frac{R}{2L}\right)^{2}>\frac{1}{LC}$は$S$は2つの実数解

\begin{align*}
s_{1} & =-\frac{R}{2L}+\sqrt{\left(\frac{R}{2L}\right)^{2}-\frac{1}{LC}}\equiv-a+b\\
s_{2} & =-\frac{R}{2L}-\sqrt{\left(\frac{R}{2L}\right)^{2}-\frac{1}{LC}}\equiv-a-b
\end{align*}

ただし
\begin{align*}
a & =\frac{R}{2L}\\
b & =\sqrt{\left(\frac{R}{2L}\right)^{2}-\frac{1}{LC}}
\end{align*}

として、$i\left(t\right)=k_{1}\e^{s_{1}t}+k_{2}\e^{s_{2}t}$(一般解)とおく。

初期条件から$k_{1},k_{2}$をて決定

$t=0+$の
\begin{itemize}
\item 電流: $i\left(0+\right)=0\rightarrow k_{1}+k_{2}=0$
\item 電圧: $v_{L}\left(0+\right)=L\d it=V\rightarrow L\left(k_{1}s_{1}+k_{2}s_{2}\right)=V$
\end{itemize}
から、
\[
k_{1}=-k_{2}=\frac{V}{L}\frac{1}{s_{1}-s_{2}}=\frac{V}{2bL}
\]

\begin{align*}
i\left(t\right) & =\frac{L}{2bL}\left(\e^{\left(-a+b\right)t}-\e^{\left(-a-b\right)t}\right)\\
 & =\frac{V}{bL}\e^{-at}\sinh bt
\end{align*}

\begin{align*}
v_{R} & =Ri\left(t\right)\\
v_{L} & =L\d it\\
v_{C} & =V-v_{R}-v_{L}
\end{align*}

\begin{center}
\includegraphics{images/CircuitTheory1-part2/1-16}
\par\end{center}
\item $\left(\frac{R}{2L}\right)^{2}=\frac{1}{LC}$のとき、$s$は重根

$s_{1}=s_{2}=-\frac{R}{2L}=-a$

一般解
\[
i\left(t\right)=k_{1}\e^{-at}+k_{2}t\e^{-at}
\]

(重根のとき$t\e^{-at}$も解)

\paragraph{証明}

\begin{align*}
L\dd it+R\d it+\frac{1}{L}i & =L\left(\d{}t-s_{1}\right)\left(\d{}t-s_{2}\right)i\\
 & =L\left(\d{}t+a\right)\left(\d{}t+a\right)i
\end{align*}

ここで$i=t^{n}\e^{-at}$とおくと
\[
\left(\d{}t+a\right)i=nt^{n-1}\e^{-at}-at^{n}\e^{-at}+at^{n}\e^{-at}
\]

従って$n=1$として、$i=t\e^{-at}$とすると、
\[
\left(\d{}t+a\right)\left(\d{}t+a\right)i=\left(\d{}t+a\right)\left(\d{}t+a\right)t\e^{-at}=\left(\d{}t+a\right)\e^{-at}=0
\]

初期条件から$k_{1},k_{2}$を決定する。

$t=0+$における
\begin{itemize}
\item 電流: $i\left(0+\right)=0\rightarrow k_{1}=0$
\item 電圧: $L\d it=V\rightarrow k_{2}=\frac{V}{C}$
\end{itemize}
\[
i\left(t\right)=\frac{V}{L}t\e^{-at}
\]

\begin{center}
\includegraphics{images/CircuitTheory1-part2/1-17}
\par\end{center}

\paragraph{別解}

$b\rightarrow0$とすると$\sinh bt\sim bt$とでき、
\[
i\left(t\right)=\frac{V}{L}t\e^{-at}
\]

\item $\left(\frac{R}{2L}\right)^{2}<\frac{1}{LC}$のとき(複素数根)

\begin{align*}
s_{1} & =-\frac{R}{2L}+\j\sqrt{\frac{1}{LC}-\left(\frac{R}{2L}\right)^{2}}\equiv-a+\j\beta\\
s_{2} & =-\frac{R}{2L}-\j\sqrt{\frac{1}{LC}-\left(\frac{R}{2L}\right)^{2}}\equiv-a-\j\beta
\end{align*}
として、一般解
\begin{align*}
i\left(t\right) & =k_{1}\e^{s_{1}t}+k_{2}\e^{s_{2}t}\\
 & =\e^{-at}\left(k_{1}\e^{\j\beta t}+k_{2}\e^{-\j\beta t}\right)
\end{align*}

初期条件
\[
\begin{cases}
i\left(0+\right)=0 & \rightarrow k_{1}+k_{2}=0\\
L\when{\d it}_{t=0}=V & \rightarrow L\left(k_{1}s_{1}+k_{2}s_{2}\right)=L\left(-a\left(k_{1}-k_{2}\right)+\j\beta\left(k_{1}-k_{2}\right)\right)=V
\end{cases}
\]

\[
i\left(t\right)=\e^{-at}\frac{V}{2\j\beta L}\left(\e^{\j\beta t}-\e^{-\j\beta t}\right)=\frac{V}{\beta L}\e^{-at}\sin\beta t
\]

\begin{center}
\includegraphics{images/CircuitTheory1-part2/1-18}
\par\end{center}

1. で$b=\j\beta$としても同じ。
\end{enumerate}
定常状態のあと、$t=0$で電源を短絡

\[
\begin{cases}
i\left(0+\right)=0\\
v_{L}\left(0+\right)=L\when{\d it}_{t=0}=-V
\end{cases}
\]


\subsection{LC回路}
\begin{center}
\includegraphics{images/CircuitTheory1-part2/1-19}
\par\end{center}

RLC回路で$R\rightarrow0$($a\rightarrow0$)

\[
i\left(t\right)=\frac{V}{\beta L}\sin\beta t
\]

\[
\beta=\sqrt{\frac{1}{LC}}\equiv\omega_{0}
\]

$\omega_{0}$: 固有角周波数

とすると、
\[
i\left(t\right)=\frac{V}{\omega_{0}L}\sin\omega_{0}t
\]

\begin{center}
\includegraphics{images/CircuitTheory1-part2/1-20}
\par\end{center}

\subsection{まとめ}
\begin{enumerate}
\item 回路は過渡応答を持つ
\item 微分方程式で解析できる
\item RC、RL回路: 時定数$\tau\left(=RC,\frac{L}{R}\right)$で指数減少
\item RLC$\begin{cases}
\text{減衰的}\\
\text{臨界的}\\
\text{振動的}
\end{cases}$
\item 余談1: RCL回路と工学
\begin{center}
\includegraphics{images/CircuitTheory1-part2/1-21}
\par\end{center}

$Q=\int i\mathrm{d}t$を用いたKVL

\[
L\dd{}tQ+R\d Qt+\frac{1}{C}Q=V
\]
\[
\underbrace{m\dd xt}_{\text{慣性力}}+\underbrace{P\d xt}_{\text{摩擦}}+\underbrace{kt}_{\text{ばね}}=F
\]

電荷$Q$⇔位置$x$

電流$i=\d Qt$⇔速度$\d xt$

$L$⇔$m$

$\frac{1}{C}$⇔$k$

パワー$vi$⇔$F\d xt$
\item 余談2

今日の話はステップ入力に対応する応答
\begin{center}
\includegraphics{images/CircuitTheory1-part2/1-22}
\par\end{center}

\begin{align*}
v\left(t\right) & =\begin{cases}
v & \left(t\geqq0\right)\\
0 & \left(t<0\right)
\end{cases}\\
 & =\underbrace{\frac{V}{2}}_{\text{直流}}+\underbrace{\int_{0}^{\infty}\frac{1}{\pi\omega}\sin\omega t\mathrm{d}\omega}_{\text{様々な周波数の正弦波の和}}
\end{align*}

それぞれの周波数に対する回路の応答を計算すればよい。

→フーリエ変換、ラプラス変換
\end{enumerate}

\end{document}
