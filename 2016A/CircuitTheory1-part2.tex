%% LyX 2.2.2 created this file.  For more info, see http://www.lyx.org/.
%% Do not edit unless you really know what you are doing.
\documentclass[english]{article}
\usepackage[LGR,T1]{fontenc}
\usepackage[utf8]{inputenc}
\usepackage[a5paper]{geometry}
\geometry{verbose,tmargin=2cm,bmargin=2cm,lmargin=1cm,rmargin=1cm}
\setlength{\parskip}{\smallskipamount}
\setlength{\parindent}{0pt}
\usepackage{textcomp}
\usepackage{amsmath}
\usepackage{amssymb}
\usepackage{graphicx}

\makeatletter

%%%%%%%%%%%%%%%%%%%%%%%%%%%%%% LyX specific LaTeX commands.
\DeclareRobustCommand{\greektext}{%
  \fontencoding{LGR}\selectfont\def\encodingdefault{LGR}}
\DeclareRobustCommand{\textgreek}[1]{\leavevmode{\greektext #1}}
\ProvideTextCommand{\~}{LGR}[1]{\char126#1}


%%%%%%%%%%%%%%%%%%%%%%%%%%%%%% User specified LaTeX commands.
\usepackage[dvipdfmx]{hyperref}
\usepackage[dvipdfmx]{pxjahyper}

% http://tex.stackexchange.com/a/192428/116656
\AtBeginDocument{\let\origref\ref
   \renewcommand{\ref}[1]{(\origref{#1})}}

\makeatother

\usepackage{babel}
\begin{document}

\title{2016-A 電気回路理論第一 後半}

\author{教員: 小関泰之 入力: 高橋光輝}

\maketitle
\global\long\def\pd#1#2{\frac{\partial#1}{\partial#2}}
\global\long\def\d#1#2{\frac{\mathrm{d}#1}{\mathrm{d}#2}}
\global\long\def\pdd#1#2{\frac{\partial^{2}#1}{\partial#2^{2}}}
\global\long\def\dd#1#2{\frac{\mathrm{d}^{2}#1}{\mathrm{d}#2^{2}}}
\global\long\def\ddd#1#2{\frac{\mathrm{d}^{3}#1}{\mathrm{d}#2^{3}}}
\global\long\def\e{\mathrm{e}}
\global\long\def\i{\mathrm{i}}
\global\long\def\j{\mathrm{j}}
\global\long\def\grad{\operatorname{grad}}
\global\long\def\rot{\operatorname{rot}}
\global\long\def\div{\operatorname{div}}
\global\long\def\diag{\operatorname{diag}}
\global\long\def\rank{\operatorname{rank}}
\global\long\def\prob{\operatorname{Prob}}
\global\long\def\cov{\operatorname{Cov}}
\global\long\def\when#1{\left.#1\right|}
\global\long\def\laplace#1{\mathcal{L}\left[#1\right]}


\section*{第1回}

\paragraph{講義ページ}

https://sites.google.com/site/ysozeki/lecture

\section{過渡現象}

\subsection{過渡現象とは?}
\begin{center}
\includegraphics{images/CircuitTheory1-part2/1-1}
\par\end{center}

応用: ディジタル回路・アナログ回路・制御・物理現象

\paragraph{4つの見方}
\begin{itemize}
\item 微分方程式(今日)
\item 周波数特性
\item インパルス応答・ステップ応答
\item 伝達関数(複素周波数平面)
\end{itemize}
ラプラス変換・フーリエ変換がこの4つを繋げてくれる。

→複雑な回路へ

\subsection{回路方程式から微分方程式へ}
\begin{center}
\includegraphics{images/CircuitTheory1-part2/1-2}
\par\end{center}

\subsubsection{基本: KVL (キルヒホッフの電圧則)}
\begin{center}
\includegraphics{images/CircuitTheory1-part2/1-3}
\par\end{center}

\[
v_{R}+v_{L}+v_{C}=v\left(t\right)
\]

\[
Ri+L\d it+\frac{1}{C}\int i\mathrm{d}t=v\left(t\right)
\]

→微分

\[
R\d it+L\dd it+\frac{i}{C}=\d{v\left(t\right)}t
\]


\subsubsection{複雑な回路}

連立微分方程式
\begin{center}
\includegraphics{images/CircuitTheory1-part2/1-4}
\par\end{center}

\[
\begin{cases}
Ri_{1}+L\d{}t\left(i_{1}+i_{2}\right)=v\left(t\right)\\
\frac{1}{C}\int i_{2}\mathrm{d}t+R_{2}i_{2}=L\d{}t\left(i_{1}+i_{2}\right)
\end{cases}
\]


\subsubsection{相互インダクタンス}

向きに注意
\begin{center}
\includegraphics{images/CircuitTheory1-part2/1-5}
\par\end{center}

\[
\begin{cases}
Ri_{1}+L_{1}\d{i_{1}}t-M\d{i_{2}}t=v\left(t\right)\\
L_{2}\d{i_{2}}t-M\d{i_{1}}t+\frac{1}{L}\int i_{2}\mathrm{d}t=0
\end{cases}
\]


\subsection{RC回路}

(あとでRL回路・RLC回路)
\begin{center}
\includegraphics{images/CircuitTheory1-part2/1-6}
\par\end{center}

\[
Ri+\frac{1}{C}\int i\mathrm{d}t=V\left(t\geqq0\right)
\]

微分して、
\[
R\d it+\frac{1}{C}i=0
\]

変数分離して、
\[
\frac{\mathrm{d}i}{i}=-\frac{1}{RC}\mathrm{d}t
\]

積分して、
\[
\log i=-\frac{1}{RC}+K\rightarrow i\left(t\right)=k\e^{-\frac{t}{RC}}
\]

ただし$k=\e^{K}$。

スイッチを入れた直後、$v_{c}\left(0+\right)=0$より初期条件
\[
i\left(0+\right)=\frac{V}{R}=k
\]

\[
i\left(t\right)=\frac{V}{R}\e^{-\frac{1}{RC}}\left(t\geqq0\right)
\]

\[
v_{R}\left(t\right)=Ri\left(t\right)=V\e^{-\frac{t}{RC}}
\]
\[
v_{c}\left(t\right)=V-v_{R}\left(t\right)=V\left(1-\e^{-\frac{t}{RC}}\right)
\]

\begin{center}
\includegraphics{images/CircuitTheory1-part2/1-7}
\par\end{center}

・$C$に供給されるパワー

\begin{align*}
P_{C}\left(t\right) & =v_{C}\left(t\right)i\left(t\right)\\
 & =\frac{V^{2}}{R}\e^{-\frac{t}{RC}}\left(1-\e^{-\frac{t}{RC}}\right)
\end{align*}

・時刻$t$までに蓄えられるエネルギー

\begin{align*}
W_{C}\left(t\right) & =\int_{0}^{t}P_{C}\left(t\right)\mathrm{d}t\\
 & =\frac{1}{2}CV^{2}\left(1-2\e^{-\frac{t}{\tau}}+\e^{-\frac{2t}{\tau}}\right)
\end{align*}

$t\rightarrow\infty$で$W_{C}\rightarrow\frac{1}{c}CV^{2}$

\paragraph{電源を短絡すると?}
\begin{center}
\includegraphics{images/CircuitTheory1-part2/1-8}
\par\end{center}

\[
i\left(t\right)=k\e^{-\frac{t}{RC}}
\]

$v_{C}\left(0+\right)=V$より、
\[
i\left(0+\right)=-\frac{V}{R}
\]

\[
i\left(t\right)=-\frac{V}{R}\e^{-\frac{t}{RC}}\left(t\geqq0\right)
\]

\begin{center}
\includegraphics{images/CircuitTheory1-part2/1-9}
\par\end{center}

\subsection{RL回路}
\begin{center}
\includegraphics{images/CircuitTheory1-part2/1-10}
\par\end{center}

\[
Ri+L\d it=V\left(t\geqq0\right)
\]

変数分離して、
\[
i+\frac{L}{R}\d it=\frac{V}{R}
\]
\[
i-\frac{V}{R}=-\frac{L}{R}\d it
\]
\[
\frac{\mathrm{d}i}{i-\frac{V}{R}}=-\frac{R}{L}\mathrm{d}t
\]
\[
\log\left(i-\frac{V}{R}\right)=-\frac{R}{L}t
\]
\[
\therefore i\left(t\right)=\frac{V}{R}+k\e^{\frac{R}{L}t}
\]

インダクタの電流は連続的に変化するため、初期条件$i\left(0+\right)=0=\frac{V}{R}+k$

\[
i\left(t\right)=\frac{V}{R}\left(1-\e^{-\frac{R}{L}t}\right)
\]

\begin{center}
\includegraphics{images/CircuitTheory1-part2/1-11}
\par\end{center}

・$L$に供給されるパワー

\begin{align*}
P_{L}\left(t\right) & =v_{L}\left(t\right)i\left(t\right)\\
 & =\frac{V^{2}}{R}\e^{-\frac{R}{L}t}\left(1-\e^{-\frac{R}{L}t}\right)
\end{align*}

・時刻$t$までに蓄えられるエネルギー

\begin{align*}
W_{L}\left(t\right) & =\int_{0}^{t}P_{L}\left(t\right)\mathrm{d}t\\
 & \xrightarrow{t\rightarrow\infty}\frac{L}{2}\left(\frac{V}{R}\right)^{2}
\end{align*}

・電源短絡時

\[
Ri+L\d it=0
\]

\[
i\left(t\right)=k\e^{-\frac{R}{L}t}
\]

$i\left(0+\right)=\frac{V}{R}$より、
\[
i\left(t\right)=\frac{V}{R}\e^{-\frac{R}{L}t}
\]

\begin{center}
\includegraphics{images/CircuitTheory1-part2/1-12}
\par\end{center}

\subsection{RLC回路}
\begin{center}
\includegraphics{images/CircuitTheory1-part2/1-13}
\par\end{center}

\[
Ri+\frac{1}{C}\int i\mathrm{d}t+L\d it=V\left(t\geqq0\right)
\]

微分して、
\[
L\dd it+R\d it+\frac{1}{C}i=0
\]

試行解$k=\e^{st}$($s$: 複素数)を代入

\[
Ls^{2}+Rs+\frac{1}{C}=0
\]

\begin{align*}
s & =-\frac{R}{2L}\pm\sqrt{\left(\frac{R}{2L}\right)^{2}-\frac{1}{LC}}\\
 & =s_{1},s_{2}
\end{align*}
とし、場合分けする。
\begin{enumerate}
\item $s$: 2つの実数解
\item $s$: 重解
\item $s$: 複素数解
\end{enumerate}

\paragraph{下準備}

i) オイラーの公式

\[
\e^{\j\theta}=\cos\theta+\j\sin\theta
\]

覚え方: $f\left(\theta\right)=\e^{\j\theta}$とすると、
\[
\begin{cases}
f\left(0\right)=1\\
\d{}{\theta}f\left(\theta\right)=\j f\left(\theta\right)
\end{cases}
\]

\begin{center}
\includegraphics{images/CircuitTheory1-part2/1-14}
\par\end{center}

ii) 三角関数

\[
\cos\theta=\frac{\e^{\j\theta}-\e^{-\j\theta}}{2},\sin\theta=\frac{\e^{\j\theta}-\e^{-\j\theta}}{2\j}
\]

\begin{center}
\includegraphics{images/CircuitTheory1-part2/1-15}
\par\end{center}

iii) 双曲線関数

\[
\cosh\theta=\frac{\e^{\theta}+\e^{-\theta}}{2},\sinh\theta=\frac{\e^{\theta}-\e^{-\theta}}{2}
\]

\begin{enumerate}
\item $\left(\frac{R}{2L}\right)^{2}>\frac{1}{LC}$は$S$は2つの実数解

\begin{align*}
s_{1} & =-\frac{R}{2L}+\sqrt{\left(\frac{R}{2L}\right)^{2}-\frac{1}{LC}}\equiv-a+b\\
s_{2} & =-\frac{R}{2L}-\sqrt{\left(\frac{R}{2L}\right)^{2}-\frac{1}{LC}}\equiv-a-b
\end{align*}

ただし
\begin{align*}
a & =\frac{R}{2L}\\
b & =\sqrt{\left(\frac{R}{2L}\right)^{2}-\frac{1}{LC}}
\end{align*}

として、$i\left(t\right)=k_{1}\e^{s_{1}t}+k_{2}\e^{s_{2}t}$(一般解)とおく。

初期条件から$k_{1},k_{2}$をて決定

$t=0+$の
\begin{itemize}
\item 電流: $i\left(0+\right)=0\rightarrow k_{1}+k_{2}=0$
\item 電圧: $v_{L}\left(0+\right)=L\d it=V\rightarrow L\left(k_{1}s_{1}+k_{2}s_{2}\right)=V$
\end{itemize}
から、
\[
k_{1}=-k_{2}=\frac{V}{L}\frac{1}{s_{1}-s_{2}}=\frac{V}{2bL}
\]

\begin{align*}
i\left(t\right) & =\frac{L}{2bL}\left(\e^{\left(-a+b\right)t}-\e^{\left(-a-b\right)t}\right)\\
 & =\frac{V}{bL}\e^{-at}\sinh bt
\end{align*}

\begin{align*}
v_{R} & =Ri\left(t\right)\\
v_{L} & =L\d it\\
v_{C} & =V-v_{R}-v_{L}
\end{align*}

\begin{center}
\includegraphics{images/CircuitTheory1-part2/1-16}
\par\end{center}
\item $\left(\frac{R}{2L}\right)^{2}=\frac{1}{LC}$のとき、$s$は重根

$s_{1}=s_{2}=-\frac{R}{2L}=-a$

一般解
\[
i\left(t\right)=k_{1}\e^{-at}+k_{2}t\e^{-at}
\]

(重根のとき$t\e^{-at}$も解)

\paragraph{証明}

\begin{align*}
L\dd it+R\d it+\frac{1}{L}i & =L\left(\d{}t-s_{1}\right)\left(\d{}t-s_{2}\right)i\\
 & =L\left(\d{}t+a\right)\left(\d{}t+a\right)i
\end{align*}

ここで$i=t^{n}\e^{-at}$とおくと
\[
\left(\d{}t+a\right)i=nt^{n-1}\e^{-at}-at^{n}\e^{-at}+at^{n}\e^{-at}
\]

従って$n=1$として、$i=t\e^{-at}$とすると、
\[
\left(\d{}t+a\right)\left(\d{}t+a\right)i=\left(\d{}t+a\right)\left(\d{}t+a\right)t\e^{-at}=\left(\d{}t+a\right)\e^{-at}=0
\]

初期条件から$k_{1},k_{2}$を決定する。

$t=0+$における
\begin{itemize}
\item 電流: $i\left(0+\right)=0\rightarrow k_{1}=0$
\item 電圧: $L\d it=V\rightarrow k_{2}=\frac{V}{C}$
\end{itemize}
\[
i\left(t\right)=\frac{V}{L}t\e^{-at}
\]

\begin{center}
\includegraphics{images/CircuitTheory1-part2/1-17}
\par\end{center}

\paragraph{別解}

$b\rightarrow0$とすると$\sinh bt\sim bt$とでき、
\[
i\left(t\right)=\frac{V}{L}t\e^{-at}
\]

\item $\left(\frac{R}{2L}\right)^{2}<\frac{1}{LC}$のとき(複素数根)

\begin{align*}
s_{1} & =-\frac{R}{2L}+\j\sqrt{\frac{1}{LC}-\left(\frac{R}{2L}\right)^{2}}\equiv-a+\j\beta\\
s_{2} & =-\frac{R}{2L}-\j\sqrt{\frac{1}{LC}-\left(\frac{R}{2L}\right)^{2}}\equiv-a-\j\beta
\end{align*}
として、一般解
\begin{align*}
i\left(t\right) & =k_{1}\e^{s_{1}t}+k_{2}\e^{s_{2}t}\\
 & =\e^{-at}\left(k_{1}\e^{\j\beta t}+k_{2}\e^{-\j\beta t}\right)
\end{align*}

初期条件
\[
\begin{cases}
i\left(0+\right)=0 & \rightarrow k_{1}+k_{2}=0\\
L\when{\d it}_{t=0}=V & \rightarrow L\left(k_{1}s_{1}+k_{2}s_{2}\right)=L\left(-a\left(k_{1}-k_{2}\right)+\j\beta\left(k_{1}-k_{2}\right)\right)=V
\end{cases}
\]

\[
i\left(t\right)=\e^{-at}\frac{V}{2\j\beta L}\left(\e^{\j\beta t}-\e^{-\j\beta t}\right)=\frac{V}{\beta L}\e^{-at}\sin\beta t
\]

\begin{center}
\includegraphics{images/CircuitTheory1-part2/1-18}
\par\end{center}

1. で$b=\j\beta$としても同じ。
\end{enumerate}
定常状態のあと、$t=0$で電源を短絡

\[
\begin{cases}
i\left(0+\right)=0\\
v_{L}\left(0+\right)=L\when{\d it}_{t=0}=-V
\end{cases}
\]


\subsection{LC回路}
\begin{center}
\includegraphics{images/CircuitTheory1-part2/1-19}
\par\end{center}

RLC回路で$R\rightarrow0$($a\rightarrow0$)

\[
i\left(t\right)=\frac{V}{\beta L}\sin\beta t
\]

\[
\beta=\sqrt{\frac{1}{LC}}\equiv\omega_{0}
\]

$\omega_{0}$: 固有角周波数

とすると、
\[
i\left(t\right)=\frac{V}{\omega_{0}L}\sin\omega_{0}t
\]

\begin{center}
\includegraphics{images/CircuitTheory1-part2/1-20}
\par\end{center}

\subsection{まとめ}
\begin{enumerate}
\item 回路は過渡応答を持つ
\item 微分方程式で解析できる
\item RC、RL回路: 時定数$\tau\left(=RC,\frac{L}{R}\right)$で指数減少
\item RLC$\begin{cases}
\text{減衰的}\\
\text{臨界的}\\
\text{振動的}
\end{cases}$
\item 余談1: RCL回路と工学
\begin{center}
\includegraphics{images/CircuitTheory1-part2/1-21}
\par\end{center}

$Q=\int i\mathrm{d}t$を用いたKVL

\[
L\dd{}tQ+R\d Qt+\frac{1}{C}Q=V
\]
\[
\underbrace{m\dd xt}_{\text{慣性力}}+\underbrace{P\d xt}_{\text{摩擦}}+\underbrace{kt}_{\text{ばね}}=F
\]

電荷$Q$⇔位置$x$

電流$i=\d Qt$⇔速度$\d xt$

$L$⇔$m$

$\frac{1}{C}$⇔$k$

パワー$vi$⇔$F\d xt$
\item 余談2

今日の話はステップ入力に対応する応答
\begin{center}
\includegraphics{images/CircuitTheory1-part2/1-22}
\par\end{center}

\begin{align*}
v\left(t\right) & =\begin{cases}
v & \left(t\geqq0\right)\\
0 & \left(t<0\right)
\end{cases}\\
 & =\underbrace{\frac{V}{2}}_{\text{直流}}+\underbrace{\int_{0}^{\infty}\frac{1}{\pi\omega}\sin\omega t\mathrm{d}\omega}_{\text{様々な周波数の正弦波の和}}
\end{align*}

それぞれの周波数に対する回路の応答を計算すればよい。

→フーリエ変換、ラプラス変換
\end{enumerate}

\section*{第2回}

\paragraph{評価について}
\begin{itemize}
\item 課題とテストの問題が「そっくり」になるようにしている
\item 電気系なら必ず身につけていてほしい問題
\end{itemize}

\paragraph{本日の内容}
\begin{itemize}
\item レポートの解説
\item 2. 交流回路の過渡現象
\item 2.1. 交流RC回路
\item 2.2. 交流RL回路
\item 2.3. 交流RLC回路
\item 2.4. 交流LC回路
\item 2.5. まとめ
\item 3. ラプラス変換
\item 3.1. 線形時不変系
\end{itemize}

\paragraph{レポート解説}

(1)

KVLより、
\[
Ri+\frac{1}{C}\int i\mathrm{d}t+L\d it=0
\]

微分して、
\[
L\dd it+R\d it+\frac{1}{C}i=0
\]

(2)

$i\left(t\right)=k\e^{st}$を代入
\[
Ls^{2}+Rs+\frac{1}{C}=0
\]

\begin{align*}
S=-\frac{R}{2L}\pm\sqrt{\left(\frac{R}{2L}\right)^{2}-\frac{1}{LC}} & =-\frac{R}{2L}\pm\j\sqrt{\frac{1}{LC}-\left(\frac{R}{2L}\right)^{2}}\\
 & \equiv-a\pm\j\beta
\end{align*}

基本解: 
\[
i\left(t\right)=\e^{-at}\left(k_{1}\e^{\j\beta t}+k\e^{-\j\beta t}\right)
\]

初期条件$\begin{cases}
i\left(0_{+}\right)=0\\
L\when{\d it}_{t=0_{+}}=-V
\end{cases}$より、$i\left(t\right)=-\frac{V}{\beta L}\e^{-at}\sin\beta t$

(3)

$t=0$で、エネルギーを蓄積しているのはキャパシタのみ。$\frac{1}{2}CV^{2}$。

$t>0$で、エネルギーを蓄積するのは抵抗のみ。$\frac{1}{2}CV^{2}$

\[
W=\int_{0}^{\infty}Ri^{2}\left(t\right)\mathrm{d}t=\cdots=\frac{1}{2}CV^{2}
\]

\begin{center}
\includegraphics{images/CircuitTheory1-part2/2-1}
\par\end{center}

※
\begin{center}
\includegraphics{images/CircuitTheory1-part2/2-2}
\par\end{center}

\section{交流回路の過渡現象}
\begin{center}
\includegraphics{images/CircuitTheory1-part2/2-3}
\par\end{center}

$t>0$で、KVL
\[
Ri+\frac{1}{C}\int i\mathrm{d}t+L\d it=V\sin\left(\omega t+\varphi\right)
\]

微分して、
\[
L\dd it+R\d it+\frac{1}{C}i=\omega V\cos\left(\omega t+\varphi\right)
\]
となり、右辺が0にならない。

交流回路: 非斉次微分方程式を解く必要。

\subsection{交流RC回路}
\begin{center}
\includegraphics{images/CircuitTheory1-part2/2-4}
\par\end{center}

KVL 
\[
Ri+\frac{1}{C}\int i\mathrm{d}t=V\sin\left(\omega t+\varphi\right)
\]

\[
R\d it+\frac{i}{C}=\omega V\cos\left(\omega t+\varphi\right)
\]

別紙式(9)を使って特解を求める。$F\left(D\right)=RD+\frac{1}{C}$とおく。

\[
i\left(t\right)=\Re\left[\frac{\omega V}{F\left(\j\omega\right)}\e^{\j\left(\omega t+\varphi\right)}\right]=\Re\left[\frac{\omega V\e^{\j\left(\omega t+\varphi\right)}}{\j\omega R+\frac{1}{C}}\right]=\underbrace{\Re\left[\frac{1}{\j}\frac{V\e^{\j\left(\omega t+\varphi\right)}}{R+\frac{1}{\j\omega C}}\right]}_{\text{三角関数}}
\]

\[
R+\frac{1}{\j\omega C}=R-\frac{\j}{\omega C}\equiv Z\e^{-\j\theta}
\]

\begin{center}
\includegraphics{images/CircuitTheory1-part2/2-5}
\par\end{center}

\begin{align*}
i_{s} & =\Re\left[\frac{1}{\j}\frac{V\e^{\j m(\omega t+\varphi}}{Z\e^{-\j\theta}}\right]\\
 & =\frac{V}{Z}\sin\left(\omega t+\varphi+\theta\right)\:\left(\text{特解}\right)
\end{align*}

\begin{itemize}
\item 基本解: $i_{t}=k\e^{-\frac{t}{RC}}$
\item 一般解: 
\begin{align*}
i\left(t\right) & =i_{s}\left(t\right)+i_{t}\left(t\right)\\
 & =\frac{V}{Z}\sin\left(\omega t+\varphi+\theta\right)+k\e^{-\frac{t}{RC}}
\end{align*}
\item 初期条件: $i\left(0_{+}\right)=\frac{V}{R}\sin\varphi$
\end{itemize}
初期条件より、$k=V\left(\frac{\sin\varphi}{R}-\frac{\sin\left(\varphi+\theta\right)}{Z}\right)$

\[
i\left(t\right)=V\left[\underbrace{\frac{\sin\left(\omega t+\varphi+\theta\right)}{Z}}_{\text{特解(定常解)}}+\underbrace{\left(\frac{\sin\varphi}{R}-\frac{\sin\left(\varphi+\theta\right)}{Z}\right)\e^{-\frac{t}{RC}}}_{\text{過渡応答}}\right]
\]


\subsection{交流RL回路}
\begin{center}
\includegraphics{images/CircuitTheory1-part2/2-7}
\par\end{center}

KVL: 
\[
L\d it+Ri=V\sin\left(\omega t+\varphi\right)
\]

別紙式(11)を適用。

$F\left(D\right)=LD+R$とおく。
\[
i_{s}\left(t\right)=\Re\left[\frac{V\e^{\j\left(\omega t+\varphi\right)}}{\j F\left(\j\omega\right)}\right]=\Re\left[\frac{1}{\j}\frac{V\e^{\j\left(\omega t+\varphi\right)}}{R+\j\omega L}\right]
\]

$R+\j\omega L=Z\e^{\j\theta}$とおく。
\begin{center}
\includegraphics{images/CircuitTheory1-part2/2-8}
\par\end{center}

\begin{align*}
i_{s}\left(t\right) & =\Re\left[\frac{V\e^{\j\left(\omega t+\varphi\right)}}{\j Z\e^{\j\theta}}\right]=\Re\left[\frac{V}{\j Z}\e^{\j\left(\omega t+\varphi-\theta\right)}\right]\\
 & =\frac{V}{Z}\sin\left(\omega+\varphi-\theta\right)\:\left(\text{特解}\right)
\end{align*}

\begin{itemize}
\item 基本解: $i_{t}\left(t\right)=k\e^{-\frac{R}{L}t}$
\item 一般解: $i\left(t\right)=i_{s}\left(t\right)+i_{t}\left(t\right)$
\item 初期条件: $i\left(0_{+}\right)=0$
\end{itemize}
\[
i\left(t\right)=\frac{V}{Z}\left(\underbrace{\sin\left(\omega t+\varphi-\theta\right)}_{\text{定常解}}-\underbrace{\sin\left(\varphi-\theta\right)\e^{-\frac{R}{L}t}}_{\text{過渡応答}}\right)
\]


\subsection{交流RCL回路 (振動的 $\frac{1}{LC}>\left(\frac{R}{2L}\right)^{2}$)}

KVL+微分: 
\[
L\dd it+R\d it+\frac{i}{C}=\omega V\cos\left(\omega t+\varphi\right)
\]

別紙式(9)。$F\left(D\right)=LD^{2}+RD+\frac{1}{C}$とおき、
\[
i_{s}\left(t\right)=\Re\left[\frac{\omega V\e^{\j\left(\omega t+\varphi\right)}}{F\left(\j\omega\right)}\right]=\Re\left[\frac{\omega V\e^{\j\left(\omega t+\varphi\right)}}{-\omega^{2}L+\j\omega R+\frac{1}{C}}\right]=\Re\left[\frac{1}{\j}\frac{V\e^{\j\left(\omega t+\varphi\right)}}{R+\j\left(\omega L-\frac{1}{\omega C}\right)}\right]
\]

$R+\j\left(\omega L-\frac{1}{\omega C}\right)\equiv Z\e^{\j\theta}$とする。

\[
i_{s}\left(t\right)=\Re\left[\frac{V\e^{\j\left(\omega t+\varphi\right)}}{\j Z\e^{\j\theta}}\right]=\frac{V}{Z}\sin\left(\omega t+\varphi-\theta\right)
\]
 

基本解: 
\begin{align*}
i_{t}\left(t\right) & =\e^{-at}\left(k_{1}\e^{\j\beta t}+k_{2}\e^{-\j\beta t}\right)\\
 & =\e^{-at}\left(\left(k_{1}+k_{2}\right)\cos\beta t+\j\left(k_{1}-k_{2}\right)\sin\beta t\right)\\
 & =\e^{-at}\left(K_{1}\cos\beta t+K_{2}\sin\beta t\right)
\end{align*}

初期条件: 
\[
\begin{cases}
i\left(0_{+}\right)=0\\
L\when{\d it}_{t=0_{0}}=V\sin\varphi & t=0\text{における電源電圧}
\end{cases}
\]

\begin{align*}
K_{1} & =-\frac{V}{Z}\sin\left(\varphi-\theta\right)\\
K_{2} & =\frac{V}{Z}\left(\frac{Z}{\beta L}\sin\varphi-\frac{\omega}{\beta}\cos\left(\varphi-\theta\right)-\frac{a}{\beta}\sin\left(\varphi-\theta\right)\right)
\end{align*}

以上合わせて、
\[
i\left(t\right)=\frac{V}{Z}\left(\underbrace{\sin\left(\omega t+\varphi-\theta\right)}_{\text{定常解}(\text{特解})}+\underbrace{\e^{-at}\left(K_{1}\cos\beta t+K_{2}\sin\beta t\right)}_{\text{減衰振動}}\right)
\]

$\omega$: 電源周波数

$\beta$: 回路の固有振動数

\subsection{交流LC回路}

RLC回路で$R\rightarrow0$とする。

\begin{align*}
a & =\frac{R}{2L}\rightarrow0\\
\beta & =\sqrt{\frac{1}{LC}-\left(\frac{R}{2L}\right)^{2}}\rightarrow\sqrt{\frac{1}{LC}}\equiv\omega_{0}\:\left(\text{共振周波数}\right)\\
Z & =\sqrt{R^{2}+\left(\omega L-\frac{1}{\omega C}\right)^{2}}\rightarrow\left|\omega L-\frac{1}{\omega C}\right|\\
\theta & =\begin{cases}
\frac{\pi}{2} & \omega>\omega_{0}\\
-\frac{\pi}{2} & \omega<\omega_{0}
\end{cases}
\end{align*}

$\omega>\omega_{0}$のとき、$Z=\omega L-\frac{1}{\omega C},\theta=\frac{\pi}{2}$

\begin{align*}
K_{1} & =-\frac{V}{Z}\sin\left(\varphi-\frac{\pi}{2}\right)=\frac{V}{Z}\cos\varphi\\
K_{2} & =\frac{V}{Z}\left(\frac{\omega L-\frac{1}{\omega C}}{\omega_{0}L}\sin\varphi-\frac{\omega}{\omega_{0}}\cos\left(\varphi-\frac{\pi}{2}\right)\right)=-\frac{V}{Z}\frac{\sin\varphi}{\omega_{0}\omega LC}\\
 & =-\frac{V}{Z}\frac{\omega_{0}}{\omega}\sin\varphi
\end{align*}

\[
i\left(t\right)=\frac{V}{Z}\left(\underbrace{\sin\left(\omega t+\varphi-\frac{\pi}{2}\right)}_{\text{定常解}}+\underbrace{\cos\varphi\cos\omega_{0}-\frac{\omega_{0}}{\omega}\sin\varphi\sin\omega_{0}t}_{\text{周波数}\omega_{0}\text{(減衰しない)}}\right)
\]

$\omega=\omega_{0}$のとき(電源周波数と回路の共振周波数が等しい)、$F\left(\j\omega\right)=0$なので、別紙式(10)を用いる。

$\varphi=0$のとき、$i\left(t\right)=\frac{V}{2L}t\sin\omega_{0}t$
\begin{center}
\includegraphics{images/CircuitTheory1-part2/2-9}
\par\end{center}
\begin{itemize}
\item 発散する
\item 徐々に増える
\item 定常状態解析ではわからない。過渡応答解析で初めて分かる。
\end{itemize}

\subsection{まとめ}

交流回路の過渡応答: 非斉次微分方程式

直流・交流ともに計算が面倒→ラプラス変換

\section{ラプラス変換(前半)}

\subsection{線形時不変系 (インパルス応答とたたみこみ)}
\begin{center}
\includegraphics{images/CircuitTheory1-part2/2-10}
\par\end{center}

RC, RL, RLC, LCはみな線形時不変

\paragraph{インパルス応答$h\left(t\right)$: デルタ関数に対する、$H$の応答。}
\begin{center}
\includegraphics{images/CircuitTheory1-part2/2-11}
\par\end{center}

デルタ関数$\delta\left(t\right)$→$H$→インパルス応答$h\left(t\right)$
\begin{center}
\includegraphics{images/CircuitTheory1-part2/2-12}
\par\end{center}

\begin{align*}
f\left(t\right) & =\begin{cases}
\frac{1}{\varepsilon} & \left(0\leqq t\leqq\varepsilon\right)\\
0 & \left(t<0,t>\varepsilon\right)
\end{cases}\\
\delta\left(t\right) & =\lim_{\varepsilon\rightarrow0}f\left(t\right)
\end{align*}

\begin{center}
\includegraphics{images/CircuitTheory1-part2/2-13}
\par\end{center}

$h\left(t\right)$がわかると、$H$の性質がすべてわかる。

\paragraph{デルタ関数の性質}
\begin{enumerate}
\item $\int f\left(t\right)\delta\left(1-\tau\right)\mathrm{d}t=f\left(\tau\right)$

$f$: 上の$f\left(t\right)$とは異なる一般的な関数
\begin{center}
\includegraphics{images/CircuitTheory1-part2/2-14}
\par\end{center}
\item $\int f\left(\tau\right)\delta\left(t-\tau\right)\mathrm{d}t=f\left(t\right)$

1. で$t$と$\tau$を入れ替え、$\delta\left(t\right)=\delta\left(-t\right)$を使う。
\begin{center}
\includegraphics{images/CircuitTheory1-part2/2-15}
\par\end{center}

様々な時間$\tau$におけるデルタ関数$\delta\left(t-\tau\right)$に重み付け$f\left(\tau\right)$をかけて足し合わせると、$f\left(t\right)$を表せる。
\end{enumerate}
$y\left(t\right)$は$x\left(t\right)$と$h\left(t\right)$のたたみこみ。
\begin{center}
\includegraphics{images/CircuitTheory1-part2/2-16}
\par\end{center}
\begin{itemize}
\item ラプラス変換
\item フーリエ変換
\end{itemize}
はたたみこみを計算するツール。

注: たたみこみは順序によらない。

\begin{align*}
x\left(t\right)*h\left(t\right) & =\int x\left(\tau\right)h\left(t-\tau\right)\mathrm{d}\tau\\
 & =-\int_{-\infty}^{\infty}x\left(t-t'\right)h\left(t'\right)\mathrm{d}t'\\
 & =\int_{-\infty}^{\infty}h\left(t'\right)x\left(t-t'\right)\mathrm{d}t'\\
 & =h\left(t\right)*x\left(t\right)
\end{align*}


\paragraph{インパルス応答の例}
\begin{center}
\includegraphics{images/CircuitTheory1-part2/2-17}
\par\end{center}

KVL
\[
L\d it+Ri=v\left(t\right)
\]

$v\left(t\right)=\delta\left(t\right),i\left(t\right)=h\left(t\right)$とする。
\begin{itemize}
\item $t<0\rightarrow h\left(t\right)=0$
\item $t=0\rightarrow L\when{\d jk}_{t=0}=\delta\left(0\right)=\infty$

\begin{align*}
h\left(0_{+}\right) & =\lim_{t\rightarrow0}\int_{0}^{\varepsilon}\d ht\mathrm{d}t\\
 & =\lim_{\varepsilon\rightarrow0}\int_{0}^{\varepsilon}\frac{1}{L}\delta\left(t\right)\mathrm{d}t=\frac{1}{L}
\end{align*}

\item $t>0$

$h\left(t\right)=k\e^{-\frac{R}{L}t}$
\end{itemize}
したがって、
\[
h\left(t\right)=\begin{cases}
0 & \left(t<0\right)\\
\frac{1}{L}\e^{-\frac{R}{L}t} & \left(t\geqq0\right)
\end{cases}
\]

\begin{center}
\includegraphics{images/CircuitTheory1-part2/2-18}
\par\end{center}

RL回路のコンダクタンスのインパルス応答
\begin{center}
\includegraphics{images/CircuitTheory1-part2/2-19}
\par\end{center}

\begin{align*}
i\left(t\right) & =v_{0}\cos\omega t*h\left(t\right)=h\left(t\right)*v_{0}\cos\omega t\\
 & =\frac{v_{0}}{L}\int_{0}^{\infty}\e^{-\frac{R}{L}\tau}\cos\omega\left(t-\tau\right)\mathrm{d}\tau\\
 & =\frac{v_{0}}{L}\int_{0}^{\infty}\e^{-\frac{R}{L}\tau}\frac{1}{2}\left(\e^{j\omega\left(t-\tau\right)}+\e^{-\j\omega\left(t-\tau\right)}\right)\mathrm{d}\tau\\
 & =\cdots=\frac{v_{0}}{2}\left[\frac{\e^{\j\omega t}}{R+\j\omega L}+\frac{\e^{-\j\omega t}}{R-\j\omega L}\right]=\Re\left[\frac{v_{0}\e^{\j\omega t}}{R+\j\omega L}\right]
\end{align*}

$R+\j\omega L$の部分が$R$と$L$の直列インピーダンスとなっている。
\begin{center}
\includegraphics{images/CircuitTheory1-part2/2-20}
\par\end{center}
\begin{itemize}
\item 交流回路さえも、インパルス応答のたたみこみで計算できる。
\item 回路は周波数ごとに異なる応答をしているわけではない。
\item 常にインパルス応答で応答しているだけ。
\end{itemize}
続きは次回(12/8)

\section*{第3回}

\paragraph{レポート解}

\[
i\left(t\right)=\frac{V}{2\omega_{0}L}\sin\varphi\sin\omega_{0}t+\frac{Vt}{2L}\sin\left(\omega_{0}t+\varphi\right)
\]


\paragraph{本日の内容}
\begin{itemize}
\item 3 ラプラス変換
\begin{itemize}
\item 3.1 線形時不変系(復習)
\item 3.2 線形代数のおさらい
\item 3.3 フーリエ変換
\item 3.4 ラプラス変換
\item 3.5 微分積分のラプラス変換
\item 3.6 まとめ
\end{itemize}
\end{itemize}
たたみこみをいかに計算するか→線形代数を活用

\subsection{線形代数のおさらい}

(1) $n$次元の空間を$n$個のベクトルの線形結合で表す。\textbf{基底}
\begin{center}
\includegraphics{images/CircuitTheory1-part2/3-2}
\par\end{center}

(2) ベクトルの内積

\[
\vec{r}_{1}\cdot\vec{r}_{2}=\left|\vec{r}_{1}\right|\left|\vec{r}_{2}\right|\cos\theta
\]

実数成分の時、
\begin{align*}
\vec{r}_{1} & =\left(x_{1},y_{1},z_{1},\cdots\right)\\
\vec{r}_{2} & =\left(x_{2},y_{2},z_{2},\cdots\right)
\end{align*}
とすると、
\[
\vec{r}_{1}\cdot\vec{r}_{2}=x_{1}x_{2}+y_{1}y_{2}+z_{1}z_{2}+\cdots
\]

(3) 複素数の内積

\begin{align*}
\vec{a} & =\left(a_{1},a_{2},a_{3},\cdots\right)\\
\vec{b} & =\left(b_{1},b_{2},b_{3},\cdots\right)
\end{align*}

$a_{n},b_{n}$: 複素数

\[
\vec{a}\cdot\vec{b}=a_{1}b_{1}^{*}+a_{2}b_{2}^{*}+a_{3}b_{3}^{*}+\cdots
\]

この定義によって、
\begin{align*}
\vec{a}\cdot\vec{a} & =a_{1}a_{1}^{*}+a_{2}a_{2}^{*}+a_{3}a_{3}^{*}+\cdots\\
 & =\left|a_{1}\right|^{2}+\left|a_{2}\right|^{2}+\left|a_{3}\right|^{2}+\cdots\geqq0
\end{align*}

長さの2乗となる。

(4) 連続関数の内積

\[
\int f\left(t\right)g^{*}\left(t\right)\mathrm{d}t
\]

\begin{center}
\includegraphics{images/CircuitTheory1-part2/3-4}
\par\end{center}

(5) 正規直交基底

基底のとり方のうち、長さが1で互いに直交しているもの。
\begin{center}
\includegraphics{images/CircuitTheory1-part2/3-5}
\par\end{center}

(6) 正規直交基底入っする基底変換
\begin{center}
\includegraphics{images/CircuitTheory1-part2/3-6}
\par\end{center}

適当なベクトル$\vec{r}$を正規直交基底の線形結合で表す。

\[
\vec{r}=c_{1}\vec{e}_{1}+c_{2}\vec{e}_{2}
\]

重み付けの係数は、上式と$\vec{e}_{1},\vec{e}_{2}$の内積で求まる。

\begin{align*}
\vec{r}\cdot\vec{e}_{1} & =c_{1}\vec{e}_{1}\cdot\vec{e}_{1}+c_{2}\vec{e}_{2}\cdot\vec{e}_{1}=c_{1}\\
\vec{r}\cdot\vec{e}_{2} & =c_{1}\vec{e}_{1}\cdot\vec{e}_{2}+c_{2}\vec{e}_{2}\cdot\vec{e}_{2}=c_{2}
\end{align*}

(7) 固有ベクトル

\[
A\vec{r}=\lambda\vec{r}
\]

$A$: 線形変換(行列)

$\lambda$: 固有値

線形変換の結果が、自身の定数倍

\subsection{フーリエ変換}

$f\left(t\right)$を無限次元のベクトル空間と捉える。

\[
f\left(t\right)=\int_{-\infty}^{\infty}f\left(\tau\right)\delta\left(t-\tau\right)\mathrm{d}\tau
\]

$f\left(\tau\right)$: 重み付け

$\delta\left(t-\tau\right)$: 基底
\begin{center}
\includegraphics{images/CircuitTheory1-part2/3-5}
\par\end{center}

\paragraph{$F\left(\omega\right)$を求めるには?: 基底変換}

$\frac{1}{\sqrt{2\pi}}\e^{\j\omega t}$が正規直交基底であることを利用。

\[
\int_{-\infty}^{\infty}\frac{\e^{\j\omega t}}{\sqrt{2\pi}}\left(\frac{\e^{\j\omega t}}{\sqrt{2\pi}}\right)^{*}\mathrm{d}t=\cdots=\delta\left(\omega_{1}-\omega_{2}\right)
\]

$\omega_{1}\neq\omega_{2}$のとき、内積が0

$\omega_{1}=\omega_{2}$のとき、平行。

\paragraph{証明したい人へ}

$\omega=\omega_{1}-\omega_{2}$とおく。

\begin{align*}
\left(\text{左辺}\right) & =\frac{1}{2\pi}\int_{-\infty}^{\infty}\e^{\j\omega t}\mathrm{d}t\\
 & =\lim_{T\rightarrow\infty}\frac{1}{2\pi}\int_{-\infty}^{\infty}\e^{-\frac{t^{2}}{2T^{2}}}\e^{\j\omega t}\mathrm{d}t\\
 & =\cdots=\lim_{T\rightarrow\infty}\frac{T}{\sqrt{2\pi}}\e^{-\frac{T^{2}\omega^{2}}{2}}\\
 & =\delta\left(\omega\right)
\end{align*}


\paragraph{ある関数$g\left(t\right)$を基底変換}

\[
\int g\left(t\right)\left(\frac{1}{\sqrt{2\pi}}\e^{\j\omega t}\right)^{*}\equiv G\left(\omega\right)
\]
とすると、$\frac{1}{\sqrt{2\pi}}\e^{\j\omega t}$の線形結合で$g\left(t\right)$を表せる。

\[
g\left(t\right)=\int G\left(\omega\right)\frac{1}{\sqrt{2\pi}}\e^{\j\omega t}\mathrm{d}\omega
\]

正規直交基底を使ったフーリエ変換(対称性を大事にする場合)

\paragraph{電気系の流儀}

$f\left(t\right)=g\left(t\right),F\left(\omega\right)=\sqrt{2\pi}G\left(\omega\right)$

\[
\begin{cases}
F\left(\omega\right)=\int_{-\infty}^{\infty}f\left(t\right)\e^{-\j\omega t}\mathrm{d}t & \text{フーリエ変換は基底との内積}\\
f\left(t\right)=\frac{1}{2\pi}\int_{-\infty}^{\infty}F\left(\omega\right)\e^{\j\omega t}\mathrm{d}\omega & \text{フーリエ逆変換は基底との線形結合}
\end{cases}
\]


\paragraph{なぜ$\protect\e^{\protect\j\omega t}$を基底に使うの?}

$\e^{\j\omega t}$は、畳込みに対する固有ベクトル

\begin{align*}
y\left(t\right) & =x\left(t\right)*h\left(t\right)=h\left(t\right)*x\left(t\right)\\
 & =\int h\left(\tau\right)x\left(t-\tau\right)\mathrm{d}\tau
\end{align*}

$x\left(t\right)=\e^{\j\omega t}$とする。

\begin{align*}
y\left(t\right) & =\int h\left(t\right)\e^{\j\omega\left(t-\tau\right)}\mathrm{d}t\\
 & =\e^{\j\omega t}\int h\left(\tau\right)\e^{-\j\omega\tau}\mathrm{d}\tau\\
 & =\e^{\j\omega t}h\left(\omega\right)
\end{align*}

$x\left(t\right)=\frac{1}{2\pi}\int X\left(\omega\right)\e^{\j\omega t}\mathrm{d}\omega$とすると、様々な$\omega$の正弦波

\begin{align*}
y\left(t\right) & =\frac{1}{2\pi}\int X\left(\omega\right)\left(\e^{\j\omega t}*h\left(t\right)\right)\mathrm{d}\omega\\
 & =\frac{1}{2\pi}\int\underbrace{X\left(\omega\right)H\left(\omega\right)}_{y\left(t\right)\text{のフーリエ変換}}\e^{\j\omega t}\mathrm{d}\omega
\end{align*}

\[
Y\left(\omega\right)=X\left(\omega\right)H\left(\omega\right)
\]

\begin{center}
\includegraphics{images/CircuitTheory1-part2/3-8}
\par\end{center}

\paragraph{フーリエ変換}

たたみこみのための基底変換

注1: $t<0$で$h\left(t\right)=0$ (因果律)
\begin{center}
\includegraphics{images/CircuitTheory1-part2/3-9}
\par\end{center}

注2: $\delta\left(t\right)$のフーリエ変換=1

\[
h\left(t\right)=\delta\left(t\right)\Rightarrow H\left(\omega\right)=1
\]

$H\left(\omega\right)$が$\omega$に対して一定でない→$h\left(t\right)\neq\delta\left(t\right)$

$x\left(t\right)$に$h\left(t\right)$が畳み込まれ、過渡応答が生じる。

\subsection{ラプラス変換}

フーリエ変換の拡張(歴史的には逆)

フーリエ変換

\[
F\left(\omega\right)=\int_{-\infty}^{\infty}f\left(t\right)\e^{-\j\omega t}\mathrm{d}t
\]

\[
f\left(t\right)=\frac{1}{2\pi}\int_{-\infty}^{\infty}F\left(\omega\right)\e^{\j\omega t}\mathrm{d}\omega
\]

\begin{itemize}
\item $t<0$で$f\left(t\right)=0$
\item $S=\j\omega$
\end{itemize}
ラプラス変換

\[
F\left(s\right)=\int_{0}^{\infty}f\left(t\right)\e^{-st}\mathrm{d}t
\]

\[
f\left(t\right)=\frac{1}{2\pi}\int_{-\j\infty}^{\j\infty}F\left(s\right)\e^{st}\frac{\mathrm{d}s}{\j}
\]

ラプラス変換を、時間領域のたたみこみを$s$領域の積にできる。

$t<0$で$f\left(t\right)=0$となる関数のフーリエ変換は、ラプラス変換で$s=\j\omega$と置いたものに等しい。(積分が収束すれば)
\begin{center}
\includegraphics{images/CircuitTheory1-part2/3-10}
\par\end{center}

\begin{center}
\includegraphics{images/CircuitTheory1-part2/3-11}
\par\end{center}

\[
\int_{-\infty}^{\infty}\delta\left(t\right)\e^{-\j\omega t}\mathrm{d}t=\e^{0}=1\stackrel{\text{等しい}}{\Leftrightarrow}\int_{0}^{\infty}\delta\left(t\right)\e^{-st}\mathrm{d}t=\e^{0}=1
\]

\[
\int_{-\infty}^{\infty}f\left(t\right)\e^{-\j\omega t}\mathrm{d}t=\int_{0}^{\infty}\e^{at}\e^{-\j\omega t}\mathrm{d}t=\frac{1}{\j\omega-a}\stackrel{\text{等しい}}{\Leftrightarrow}\int_{0}^{\infty}f\left(t\right)\e^{-st}\mathrm{d}t=\int_{0}^{\infty}\e^{at}\e^{-st}\mathrm{d}t=\frac{1}{s-a}
\]

\begin{center}
\includegraphics{images/CircuitTheory1-part2/3-12}
\par\end{center}

フーリエ変換

\begin{align*}
\int_{-\infty}^{\infty}f\left(t\right)\e^{\j\omega t}\mathrm{d}t & =\int_{0}^{\infty}\e^{at}\e^{\j\omega t}\mathrm{d}t\\
 & =\left[\frac{\e^{\left(a-\j\omega\right)t}}{a-\j\omega}\right]_{0}^{\infty}\rightarrow\infty
\end{align*}

ラプラス変換

\begin{align*}
\int_{0}^{\infty}f\left(t\right)\e^{-st}\mathrm{d}t & =\int_{0}^{\infty}\e^{-st}\mathrm{d}t\\
 & =\left[\frac{\e^{\left(a-s\right)t}}{a-s}\right]_{0}^{\infty}=\frac{1}{s-a}\left(\Re\left[s-a\right]>0\right)
\end{align*}

となり、異なる。
\begin{center}
\includegraphics{images/CircuitTheory1-part2/3-13}
\par\end{center}

フーリエ変換

\begin{align*}
\int_{-\infty}^{\infty}f\left(t\right)\e^{-\j\omega t}\mathrm{d}t & =\int_{0}^{\infty}\e^{-\j\omega t}\mathrm{d}t\\
 & =\cdots=\frac{1}{\j\omega}+\pi\delta\left(\omega\right)
\end{align*}

ラプラス変換

\begin{align*}
\int_{0}^{\infty}f\left(t\right)\e^{-st}\mathrm{d}t & =\int_{0}^{\infty}\e^{-st}\mathrm{d}t\\
 & =\frac{1}{s}\left(\Re s>0\right)
\end{align*}

となり、異なる。

ラプラス変換: 周波数$s$を複素数に拡張。$s=\sigma+\j\omega$

このため$f\left(t\right)$が$t\rightarrow\infty$で発散 or 0に収束しないときでも、ラプラス変換が可能。フーリエ変換は不可能
or 難しい。回路にはラプラス。

\paragraph{$s=\sigma+\protect\j\omega$ってどういうこと?}

\[
F\left(\sigma+\j\omega\right)=\int_{0}^{\infty}f\left(t\right)\e^{-\sigma t}\e^{-\j\omega t}\mathrm{d}t
\]

$f\left(t\right)\e^{-\sigma t}\left(t>0\right)$のフーリエ変換

\begin{align*}
\frac{1}{2\pi\j}\int_{\sigma-\j\omega}^{\sigma+\j\omega}F\left(s\right)\e^{st}\mathrm{d}s & =\frac{1}{2\pi\j}\int_{-\infty}^{\infty}F\left(\sigma+\j\omega\right)\e^{\sigma t}\e^{\j\omega t}\j\mathrm{d}\omega\\
 & =\e^{\sigma t}\underbrace{\frac{1}{2\pi}\int F\left(\sigma+\j\omega\right)\e^{\j\omega t}\mathrm{d}\omega}_{f\left(t\right)\e^{-\sigma t}}\\
 & =f\left(t\right)
\end{align*}

\begin{center}
\includegraphics{images/CircuitTheory1-part2/3-13}
\par\end{center}

$F\left(s\right)$の形(極や零点)を見るだけで、いろいろな事がわかる。(回路理論第二)
\begin{center}
\includegraphics{images/CircuitTheory1-part2/3-14}
\par\end{center}

ラプラス変換
\begin{itemize}
\item $t\geqq0$のみを問題にする
\item フーリエ変換を計算できない関数も扱える
\item たたみこみを積にできる
\end{itemize}

\subsection{微分・積分のラプラス変換}

$t<0$で$f\left(t\right)=0$となる関数の場合

\[
f\left(t\right)=\frac{1}{2\pi\j}\int_{\sigma-\j\infty}^{\sigma+\j\infty}F\left(s\right)\e^{st}\mathrm{d}s
\]

$\e^{st}$: 様々な$s$に対する基底$\e^{st}$の線形結合 $\e^{st}=\e^{\sigma t}\e^{\j\omega t}$
\begin{center}
\includegraphics{images/CircuitTheory1-part2/3-20}
\par\end{center}

微分・積分は線形なので、
\begin{align*}
\d{}tf\left(t\right) & =\frac{1}{2\pi\j}\int_{\sigma-\j\infty}^{\sigma+\j\infty}F\left(s\right)s\e^{st}\mathrm{d}s\Rightarrow\laplace{\d{}tf\left(t\right)}=sF\left(s\right)\\
\int f\left(t\right)\mathrm{d}t & =\frac{1}{2\pi\j}\int_{\sigma-\j\infty}^{\sigma+\j\infty}F\left(s\right)\frac{1}{s}\e^{st}\mathrm{d}s\Rightarrow\laplace{\int f\left(t\right)\mathrm{d}t}=\frac{1}{s}F\left(s\right)
\end{align*}

$\mathcal{L}$: ラブラス変換

微分の伝達関数→$s$

積分の伝達関数→$\frac{1}{s}$

\paragraph{例 指数減衰関数}

\[
f\left(t\right)=\begin{cases}
\e^{-t} & \left(t\geqq0\right)\\
0 & \left(t<0\right)
\end{cases}
\]

\begin{center}
\includegraphics{images/CircuitTheory1-part2/3-17}
\par\end{center}

\[
\laplace{f\left(t\right)}=\int_{0}^{\infty}\e^{t}\e^{-st}\mathrm{d}t=\frac{1}{s+1}
\]

微分

\[
\d{}tf\left(t\right)=\begin{cases}
\delta\left(t\right)-\e^{-t}\\
0 & \left(t<0\right)
\end{cases}
\]

\begin{center}
\includegraphics{images/CircuitTheory1-part2/3-18}
\par\end{center}

\begin{align*}
\laplace{\d{}tf\left(t\right)} & =1-\frac{1}{s+1}\\
 & =\frac{s}{s+1}
\end{align*}

$s$倍されている。

積分

\[
\int_{-\infty}^{t}f\left(\tau\right)\mathrm{d}\tau=\begin{cases}
1-\e^{-t} & \left(t\geqq0\right)\\
0 & \left(t<0\right)
\end{cases}
\]

\begin{center}
\includegraphics{images/CircuitTheory1-part2/3-19}
\par\end{center}

\[
\laplace{\int_{-\infty}^{t}f\left(\tau\right)\mathrm{d}\tau}=\frac{1}{s}-\frac{1}{s+1}=\frac{1}{s\left(s+1\right)}
\]

$\frac{1}{s}$倍されている。

\paragraph{注意}

$t<0$で$f\left(t\right)-0$とならない関数の場合
\begin{center}
\includegraphics{images/CircuitTheory1-part2/3-21}
\par\end{center}

微分

\begin{align*}
\laplace{\d{}tf\left(t\right)} & =\int_{0}^{\infty}\d ft\e^{-st}\mathrm{d}t\\
 & =\left[f\left(t\right)\e^{-st}\right]_{0}^{\infty}\int_{0}^{\infty}f\left(t\right)\left(-s\e^{-st}\right)\mathrm{d}t\\
 & =sF\left(s\right)-f\left(0+\right)
\end{align*}

\[
F\left(s\right)=\frac{1}{s+1}-1=-\frac{1}{s+1}
\]

積分

\[
f^{\left(-1\right)}\left(t\right)=\int_{-\infty}^{t}f\left(\tau\right)\mathrm{d}\tau
\]

\begin{align*}
\laplace{f^{\left(-1\right)}\left(t\right)} & =\laplace{\int_{-\infty}^{0}f\left(\tau\right)\mathrm{d}\tau+\int_{0}^{\infty}f\left(\tau\right)\mathrm{d}\tau}\\
 & =\frac{f^{\left(-1\right)}\left(0\right)}{s}+\frac{F\left(s\right)}{s}
\end{align*}


\subsection{まとめ}
\begin{itemize}
\item 線形時不変系: $x\left(t\right)\rightarrow H\rightarrow y\left(t\right)=x\left(t\right)*h\left(t\right)$
($h\left(t\right)$: インパルス応答)
\item 線形代数: 基底変換で計算を容易に
\item フーリエ変換: 複素正弦波$\e^{\j\omega t}$への基底変換→たたみこみが掛け算
\item ラプラス変換: $t\geqq0$、周波数を複素数に拡張→たたみこみが掛け算
\begin{itemize}
\item $h\left(t\right)$のフーリエ変換: 周波数応答(複素数)←$s=\j\omega$
\item $h\left(t\right)$のラプラス変換
\end{itemize}
\item 微分積分の伝達関数: $s,\frac{1}{s}$
\end{itemize}

\section*{第4回}

\paragraph{レポート+α}

(1)(a)

\begin{align*}
\laplace{\e^{at}} & =\int_{0}^{\infty}\e^{at}\e^{-st}\mathrm{d}t\\
 & =\frac{1}{s-a}\quad\left(\Re\left(s-a\right)>0\right)\\
f\left(t\right) & =\begin{cases}
\e^{at} & \left(t>0\right)\\
0 & \left(t<0\right)
\end{cases}
\end{align*}

図回後4-1

ローパスフィルタのインパルス応答

(b)

\[
\laplace{t\e^{at}}=\cdots=\frac{1}{\left(s-a\right)^{2}}\quad\Re\left(s-a\right)>0
\]

図回後4-2

ローパスフィルタを2回通った時のインパルス応答

参考:

\begin{align*}
\laplace{t^{n}\e^{at}} & =\int_{0}^{\infty}t^{n}\e^{at}\e^{-st}\mathrm{d}t\\
 & =\left[t^{n}\frac{\e^{\left(a-s\right)t}}{a-s}\right]_{0}^{\infty}+\frac{n}{s-a}\int t^{\left(n-1\right)}\e^{\left(a-s\right)t}\mathrm{d}t\\
 & =\frac{n\left(n-1\right)\cdots1}{\left(s-a\right)^{n}}\int_{0}^{\infty}\e^{\left(a-s\right)t}\mathrm{d}t=\frac{n!}{\left(s-a\right)^{n+1}}
\end{align*}

図回後4-3

$a=0$のとき
\[
\laplace{t^{n}}=\frac{n!}{s^{n+1}}
\]

(c)

\[
\laplace{\e^{at}\sin\omega t}=\cdots=\frac{\omega}{\left(s-a\right)^{2}+\omega^{2}}
\]

同様に$\laplace{\e^{at}\cos\omega t}=\frac{s-a}{\left(s-a\right)^{2}+\omega^{2}}$

\[
\e^{at}\sin\omega t\quad\left(t>0\right)
\]

図回後4-4

(2)(a)

\[
f\left(t\right)=\begin{cases}
0 & t<0\\
\e^{at} & t\geqq0
\end{cases}\quad\left(a>0\right)
\]

\[
\int_{-\infty}^{\infty}f\left(t\right)\e^{-\j\omega t}\mathrm{d}t=\frac{1}{\j\omega-a}
\]

図回後4-5

(b)

\[
f\left(t\right)=\begin{cases}
-\e^{at} & t\leqq0\\
0 & t>0
\end{cases}\quad a>0
\]

\[
\int_{-\infty}^{\infty}f\left(t\right)\e^{-\j\omega t}\mathrm{d}t=\frac{1}{\j\omega-a}
\]

図回後4-6

注) ラプラス変換

\[
F\left(s\right)=\frac{1}{s-a}\Leftrightarrow f\left(t\right)=\e^{at}\left(t\geqq0\right)
\]

図回後4-7

図回後4-8

図回後4-9

\paragraph{本日の内容}
\begin{itemize}
\item 4 ラプラス変換(後半)
\begin{itemize}
\item 4.1 ラプラス変換による回路の過渡応答解析
\item S領域のインピーダンス
\item 4.3 部分分数展開
\item 4.4 ラプラス変換の性質
\item 4.5 周波数応答と極・零点の関係
\item 4.6 周波数応答の実部と虚部の関係
\item 4.7 まとめ
\end{itemize}
\end{itemize}

\section{ラプラス変換(後半)}

\subsection{ラプラス変換による回路の過渡応答解析}
\begin{enumerate}
\item 直流RC回路

図回後4-10

\[
q\left(0+\right)=\int_{-\infty}^{\infty}i\left(t\right)\mathrm{d}t=0
\]

\[
\frac{1}{C}\int i\left(t\right)\mathrm{d}t+Ri\left(t\right)=Vu\left(t\right)
\]

\[
\xrightarrow{\mathcal{L}}\frac{1}{C}\left[\frac{I\left(s\right)}{s}+\frac{q\left(0+\right)}{s}\right]+RI\left(s\right)=V\frac{1}{s}
\]

$I\left(s\right)$について解く。

\begin{align*}
I\left(s\right) & =\frac{V}{s\left(\frac{1}{sC}+R\right)}\\
 & =\frac{V}{R}\frac{1}{s+\frac{1}{RC}}
\end{align*}

\[
i\left(t\right)=\frac{V}{R}\e^{-\frac{1}{RC}}\quad\left(t\geqq0\right)
\]

\item 交流RL回路

図回後4-11

\[
L\d it+Ri=\frac{V}{\j}\e^{\j\omega t}\quad\left(t\geqq0\right)
\]

\[
\xrightarrow{\mathcal{L}}L\left(SI\left(s\right)-i\left(0+\right)\right)+RI\left(s\right)=\frac{V}{\j}\frac{1}{s-j}\omega
\]

\begin{align*}
I\left(s\right) & =\frac{V}{\j}\frac{1}{s-\j\omega}\frac{1}{LS+R}=\frac{V}{\j L}\left(\frac{1}{s-\j\omega}-\frac{1}{s+\frac{R}{L}}\right)\\
 & =\frac{V}{\j}\frac{1}{R+\j\omega L}\left(\frac{1}{s-\j\omega}-\frac{1}{s+\frac{R}{L}}\right)
\end{align*}

\[
i\left(t\right)=\mathcal{L}^{-1}\left[I\left(s\right)\right]=\frac{V}{\j}\frac{1}{R+\j\omega L}\left(\e^{\j\omega t}-\e^{-\frac{R}{L}t}\right)
\]

ここで$R+\j\omega L=Z\e^{\j\theta}$とする。

\[
i\left(t\right)=\frac{V}{\j Z}\left(\e^{\j\left(\omega t-\theta\right)}-\e^{-\j\theta}\e^{-\frac{R}{L}t}\right)
\]

\[
\Re i\left(t\right)=\frac{V}{Z}\left(\sin\left(\omega t-\theta\right)+\sin\theta\e^{-\frac{R}{L}t}\right)
\]

\end{enumerate}

\subsection{S領域のインピーダンス}

微分方程式を使わず解析が行える。

図回後4-12

\[
e\left(t\right)=L\d it
\]

\[
\xrightarrow{\mathcal{L}}E\left(s\right)=L\left(SI\left(s\right)-i\left(0+\right)\right)
\]

\[
E\left(s\right)=SLI\left(s\right)
\]

$SL$: $L$のS領域のインピーダンス

\[
i\left(t\right)=\frac{1}{L}\int e\left(t\right)\mathrm{d}t
\]

\[
\xrightarrow{\mathcal{L}}I\left(s\right)=\frac{E\left(s\right)}{LS}+\frac{i\left(0+\right)}{s}
\]

$i\left(0+\right)=0$の場合

図回後4-13

$t=0$におけるインダクタの電流は、電流源として考える。

図回後4-14

$e\left(0+\right)=\frac{q\left(0+\right)}{c}=0$のとき、
\[
E\left(s\right)=\frac{1}{sC}I\left(s\right)
\]

$\frac{1}{sC}$: $C$のS領域のインピーダンス

\[
e\left(t\right)=\frac{1}{C}\int i\left(t\right)\mathrm{d}t
\]

\[
\xrightarrow{\mathcal{L}}\frac{I\left(s\right)}{sC}+\frac{e\left(0+\right)}{s}
\]

\[
i\left(t\right)=C\d{e\left(t\right)}t
\]

\[
\xrightarrow{\mathcal{L}}I\left(s\right)=C\left(sE\left(s\right)-e\left(0+\right)\right)
\]

図回後4-15

\subsection{部分分数展開}

ラプラス逆変換のコツ

求める応答(s領域)

\[
E\left(s\right)=\frac{P\left(s\right)}{Q\left(s\right)}
\]

$P\left(s\right)$: 次数$m$

$Q\left(s\right)$: 次数$n$

$E\left(s\right)$が過渡応答のとき$m<n$

$E\left(s\right)$が伝達応答のときは$m\geqq n$であってよい。

※$m\geqq n$と考えると、$E\left(s\right)=b_{0}+b_{1}s=\cdots+\frac{P\left(s\right)}{Q\left(s\right)}\xrightarrow{\mathcal{L}^{-1}}e\left(t\right)=b_{0}\delta\left(t\right)+b_{1}\d{}t\delta\left(t\right)+\cdots$

まず、分母$Q\left(s\right)$を因数分解 $Q\left(s\right)=a_{0}\left(s-s_{1}\right)\left(s-s_{2}\right)\cdots\left(s-s_{n}\right)$

i) 重根がないとき

$\frac{P\left(s\right)}{Q\left(s\right)}=\frac{k_{1}}{s-s_{1}}+\frac{k_{2}}{s-s_{2}}+\cdots+\frac{k_{n}}{s-s_{n}}$とおき$k_{1},k_{n}$を求める。

\[
\xrightarrow{\mathcal{L}^{-1}}k_{1}\e^{s_{1}t}+k_{2}\e^{s_{2}t}+\cdots+k_{n}\e^{s_{n}t}
\]

ii) 重根があるとき

\[
\frac{P\left(s\right)}{\left(s-s_{1}\right)r}=\frac{k_{11}}{s-s_{1}}+\frac{k_{12}}{\left(s-s_{1}\right)^{2}}+\cdots+\frac{k_{1r}}{\left(s-s_{1}\right)^{r}}
\]

\[
\xrightarrow{\mathcal{L}^{-1}}k_{11}\e^{s_{1}t}+k_{12}t\e^{s_{1}t}+\cdots+\frac{k_{1r}}{\left(r-1\right)!}t^{r-1}\e^{s_{1}t}
\]

iii) 2根が複素共役のとき

\[
\frac{P\left(s\right)}{Q_{1}\left(s\right)\left(\left(s-\alpha\right)^{2}+\omega^{2}\right)}=k_{1}\frac{s-\alpha}{\left(s-\alpha\right)^{2}+\omega^{2}}+k_{2}\frac{\omega}{\left(s-\alpha\right)^{2}+\omega^{2}}+\cdots
\]

\[
\xrightarrow{\mathcal{L}^{-1}}k_{1}\e^{\alpha t}\cos\omega t+k_{2}\e^{\alpha t}\sin\omega t+\cdots
\]


\paragraph{部分分数展開の例}

\begin{align}
E\left(s\right) & =\frac{4}{s^{4}+4s^{3}+8s^{2}+8s}=\frac{4}{s\left(s+2\right)\left(s^{2}+2s+4\right)}\nonumber \\
 & =\frac{k_{1}}{s}+\frac{k_{2}}{s+2}+\frac{sk_{3}+k_{4}}{s^{2}+2s+4}\label{4-1}
\end{align}

両辺に$s$を掛ける
\[
k_{1}+\frac{sk_{2}}{s+2}+\frac{s\left(sk_{3}+k_{4}\right)}{s^{2}+2s+4}=\frac{4}{\left(s+2\right)\left(s^{2}+2s+4\right)}
\]

$s=0$を代入 $k_{1}=\frac{1}{2}$

$s+2$を両辺に掛ける
\[
\frac{\left(s+2\right)}{s}k_{1}+k_{2}+\frac{\left(s+2\right)\left(sk_{3}+k_{4}\right)}{s^{2}+2s+4}=\frac{4}{s\left(s^{2}+2s+4\right)}
\]

$s=-2$を代入 $k_{2}=-\frac{1}{2}$

$s^{2}+2s+4$を両辺に掛け、$s^{2}+2s+4=0\:\left(s=-1\pm\sqrt{3}\right)$とおく→$k_{3}=0,k_{4}=-1$

\[
E\left(s\right)=\frac{1}{2}\left(\frac{1}{s}-\frac{1}{s+2}\right)-\frac{1}{s^{2}+2s+4}
\]

\[
\xrightarrow{\mathcal{L}}e\left(t\right)=\frac{1}{2}\left(1-\e^{-2t}\right)-\frac{1}{\sqrt{3}}\e^{-t}\sin\sqrt{3}t
\]

図回後4-16

\subsection{ラプラス変換の性質}

4つ
\begin{enumerate}
\item 初期値定理

\[
f\left(0+\right)=\lim_{s\rightarrow\infty}sF\left(s\right)
\]

\item 最終値定理

\[
f\left(\infty\right)=\lim_{s\rightarrow0}sF\left(s\right)
\]

←発散しないとき
\end{enumerate}

\paragraph{証明}

\begin{align*}
\int_{0+}^{\infty}\d ft\e^{-st}\mathrm{d}t & =\left[f\left(t\right)\e^{-st}\right]_{0+}^{\infty}-\int f\left(t\right)\left(-s\right)\e^{-st}\mathrm{d}t\\
 & =-f\left(0+\right)+sF\left(s\right)
\end{align*}

$s\rightarrow0$
\[
f\left(0+\right)+\int_{0+}^{\infty}\d ft\mathrm{d}t=f\left(\infty\right)=\lim_{s\rightarrow0}sF\left(s\right)
\]

$s\rightarrow\infty$
\[
f\left(0+\right)=\lim_{s\rightarrow\infty}sF\left(s\right)
\]

3. 相似定理

\[
\laplace{f\left(\frac{t}{a}\right)}=aF\left(as\right)
\]

\begin{align*}
\laplace{f\left(\frac{t}{a}\right)} & =\int_{0}^{\infty}f\left(\frac{t}{a}\right)\e^{-st}\mathrm{d}t\quad\leftarrow t'=\frac{t}{a}\\
 & =\int_{0}^{\infty}f\left(t'\right)\e^{-sat''}a\mathrm{d}t'=aF\left(as\right)
\end{align*}

図回後4-17

面積$a$倍、周波数$\frac{1}{a}$倍

4. 推移定理

\[
\laplace{f\left(t-T\right)}=\e^{-Ts}F\left(s\right)
\]

\[
\because f\left(t-T\right)=f\left(t\right)*\delta\left(t-T\right)
\]

\begin{align*}
\laplace{\delta\left(t-T\right)} & =\int_{0}^{\infty}\delta\left(t-T\right)\e^{-st}\mathrm{d}t\\
 & =\e^{-sT}
\end{align*}

←遅延$T$の伝達関数

図回後4-18

\paragraph{推移定理の応用}

方形波

図回後4-19

\[
f\left(t\right)=u\left(t\right)-u\left(t-\tau\right)
\]

\begin{align*}
F\left(s\right) & =\frac{1}{s}-\frac{\e^{-s\tau}}{s}\\
 & =\frac{1}{s}\left(1-\e^{-s\tau}\right)
\end{align*}

同期的な方形波

図回後4-20

\begin{align*}
F\left(s\right) & =\frac{1}{s}\left(1-\e^{-s\tau}\right)\underbrace{\left(1+\e^{-sT}+\e^{-2sT}+\cdots\right)}_{\text{無限級数}}\\
 & =\frac{1}{s}\frac{1-\e^{-s\tau}}{1-\e^{-sT}}
\end{align*}


\subsection{伝達関数の極・零点と周波数特性の関係}

\begin{align*}
H\left(s\right) & =\frac{1}{s+a}\quad\left(a>0\right)\\
h\left(s\right) & =\e^{-at}\quad\left(t\geqq0\right)
\end{align*}

図回後4-21

図回後4-22

図回後4-23

\begin{align*}
H\left(s\right) & =s+a\quad\left(a>0\right)\\
h\left(t\right) & =\d{}t\delta\left(t\right)+a\delta\left(t\right)
\end{align*}

図回後4-24

図回後4-25

図回後4-26

\subsection{周波数特性の実部と虚部の関係}

$f\left(t\right)$が実数、$t<0$で$f\left(t\right)=0$を満たすとする。$f\left(t\right)$のフーリエ変換を$F\left(\omega\right)=\int_{-\infty}^{\infty}f\left(t\right)\e^{-\j\omega t}\mathrm{d}t$とすると、$\Re F\left(\omega\right)$がわかれば、$\Im F\left(\omega\right)$もわかる。

言い換え: $f\left(t\right)$が因果律を満たすインパルス応答であるとき、周波数応答の実部・虚部に関係がある

\paragraph{証明}

\begin{align*}
f_{S}\left(t\right) & =\frac{1}{2}\left(f\left(t\right)+f\left(-t\right)\right)\\
f_{AS}\left(t\right) & =\frac{1}{2}\left(f\left(t\right)-f\left(-t\right)\right)
\end{align*}

※S→symmetric、AS→anti-symmetric

\[
f\left(t\right)=f_{S}\left(t\right)+f_{AS}\left(t\right)
\]

\begin{align*}
\mathcal{F}\left[f_{S}\left(t\right)\right] & =\int_{-\infty}^{\infty}f_{S}\left(t\right)\e^{-\j\omega t}\mathrm{d}t=\frac{1}{2}\left[\int_{-\infty}^{\infty}f\left(t\right)\e^{-\j\omega t}\mathrm{d}t+\int_{-\infty}^{\infty}f\left(-t\right)\e^{-\j\omega t}\mathrm{d}t\right]\\
 & =\frac{1}{2}\left[\int_{-\infty}^{\infty}f\left(t\right)\left(\e^{-j\omega t}+\e^{\j\omega t}\right)\right]=\int_{-\infty}^{\infty}\underbrace{f\left(t\right)}_{\text{実数}}\cos\omega t\mathrm{d}t=\Re\left[F\left(\omega\right)\right]\\
\mathcal{F}\left[f_{AS}\left(t\right)\right] & =\int_{-\infty}^{\infty}f_{AS}\left(t\right)\e^{-\j\omega t}\mathrm{d}t=\frac{1}{2}\left[\int_{-\infty}^{\infty}f\left(t\right)\e^{-\j\omega t}\mathrm{d}t-\int_{-\infty}^{\infty}f\left(-t\right)\e^{-\j\omega t}\mathrm{d}t\right]\\
 & =\frac{1}{2}\int_{-\infty}^{\infty}f\left(t\right)\left(\e^{-\j\omega t}-\e^{\j\omega t}\right)\mathrm{d}t=\j\int_{-\infty}^{\infty}f\left(t\right)\sin\left(-\omega t\right)\mathrm{d}t=\j\Im F\left(\omega\right)
\end{align*}

$\Re F\left(\omega\right)$がわかる→$f_{S}\left(t\right)$がわかる→$f\left(t\right)$も$f_{AS}\left(t\right)$もわかる→$\Im F\left(\omega\right)$もわかる

周波数応答が実数のみ→インパルス応答が$t=0$に対して対称

因果律を満たさない(例外: $\delta\left(t\right)$)

虚部があって初めて因果律を満たせる。

\subsection{ラプラス変換(後半) まとめ}
\begin{itemize}
\item 微積分方程式が、s領域で多項式になる。
\begin{itemize}
\item 代数計算計算で伝達関数が得られる。
\end{itemize}
\item s領域のインピーダンス: $sL,\frac{1}{sC},R$
\item ラプラス逆変換のコツ: 部分分数展開
\item 色々な性質: 時間波形と絡めながら、理解してほしい
\end{itemize}

\section*{第5回}

\paragraph{レポート解説}

図回後5-1

(1)
\[
Ri+L\d it+\frac{1}{C}\int i\mathrm{d}t=Vu\left(t\right)
\]

$u\left(t\right)$: ステップ関数

(2)
\[
RI\left(s\right)+LsI\left(s\right)+\frac{1}{C}\frac{I\left(s\right)}{s}=V\frac{1}{s}
\]

\[
I\left(s\right)=\frac{V}{s}\frac{1}{R+sL+\frac{1}{sC}}=\frac{V}{Ls^{2}+Rs+\frac{1}{C}}
\]

ここで、$\frac{V}{s}$が電圧波形、$\frac{1}{R+sL+\frac{1}{sC}}$が伝達関数、$R+sL+\frac{1}{sC}$がRLC直列インピーダンスに対応していることに注目。

(3) $Ls^{2}+Rs+\frac{1}{C}=0$の解: 
\begin{align*}
s_{1},s_{2} & =-\frac{R}{2L}\pm\sqrt{\left(\frac{R}{2L}\right)^{2}-\frac{1}{LC}}\\
 & =-a\pm b=-a\pm\j\beta
\end{align*}

(a) $\left(\frac{R}{2L}\right)^{2}>\frac{1}{LC}$

\[
I\left(s\right)=\frac{V}{L}\frac{1}{\left(s-s_{1}\right)\left(s-s_{2}\right)}=\frac{V}{L}\left(\frac{1}{s-s_{1}}-\frac{1}{s-s_{2}}\right)\frac{1}{s_{1}-s_{2}}=\frac{V}{2bL}\left(\frac{1}{s-s_{1}}-\frac{1}{s-s_{2}}\right)
\]

\[
i\left(s\right)=\frac{V}{2bL}\left(\e^{s_{1}t}-\e^{s_{2}t}\right)=\frac{V}{bL}\e^{-at}\left(\frac{\e^{bt}-\e^{-bt}}{2}\right)=\frac{V}{bL}\e^{-at}\sinh bt
\]

(b) $\left(\frac{R}{2L}\right)^{2}=\frac{1}{LC}$のとき

\[
I\left(s\right)=\frac{V}{L\left(s+a\right)^{2}}\rightarrow i\left(t\right)=\frac{V}{L}t\e^{-at}
\]

(c) $\left(\frac{R}{2L}\right)^{2}<\frac{1}{LC}$のとき

\[
I\left(s\right)=\frac{V}{2\j\beta L}\left(\frac{1}{s-s_{1}}-\frac{1}{s-s_{2}}\right)
\]

\[
i\left(t\right)=\frac{V}{2\j\beta L}\left(\e^{s_{1}t}-\e^{s_{2}t}\right)=\frac{V}{\beta L}\e^{-at}\frac{\e^{\j\beta t}-\e^{-\j\beta t}}{2}=\frac{V}{\beta L}\e^{-at}\sin\beta t
\]

(4)
\[
\when{\d it}_{i=0+}=\lim_{s\rightarrow\infty}ssI\left(s\right)=\frac{V}{L}
\]

図回後5-2

\[
\int_{0}^{\infty}i\left(t\right)\mathrm{d}t=\lim_{s\rightarrow0}s\frac{I\left(s\right)}{s}=C
\]

$I\left(s\right)$から色んな情報が引き出せる。

\paragraph{本日の内容}
\begin{itemize}
\item 5. 4端子網
\item 5.1. 4端子網と基本行列(F行列)
\item 5.2. Z行列, Y行列
\item 5.3. F, Z, Yの間c
\item 5.4. 等価変換
\item 5.5. 4端子網の応用
\item 5.6. 影像パラメータ
\item 5.7. まとめ
\end{itemize}

\section{4端子網}

\subsection{4端子網と基本行列}

図回後5-3
\begin{itemize}
\item 大きな回路を小さな回路の組み合わせで考えるときの標準的な回路の単位。
\item 入力→出力(作用)だけでなく、出力→入力(反作用)も表せる。(伝達関数は1方向しか表せない)
\item 行列で表す
\end{itemize}

\paragraph{基本行列}

\[
\left(\begin{array}{c}
V_{1}\\
I_{1}
\end{array}\right)=\left(\begin{array}{cc}
A & B\\
C & D
\end{array}\right)\left(\begin{array}{c}
V_{2}\\
I_{2}
\end{array}\right)
\]

図回後5-4

\[
\left(\begin{array}{c}
V_{1}\\
I_{1}
\end{array}\right)=\boldsymbol{F}_{1}\left(\begin{array}{c}
V_{2}\\
I_{2}
\end{array}\right)=\boldsymbol{F}_{1}\boldsymbol{F}_{2}\left(\begin{array}{c}
V_{3}\\
I_{3}
\end{array}\right)
\]

縦続接続の計算に便利な表現

\paragraph{F行列の計算}

2つ覚える。

図回後5-5

\begin{align*}
V_{1} & =V_{2}\\
I_{1} & =\frac{V_{2}}{Z}+I_{2}
\end{align*}

\[
\left(\begin{array}{c}
V_{1}\\
I_{1}
\end{array}\right)=\left(\begin{array}{cc}
1 & 0\\
\frac{1}{Z} & 1
\end{array}\right)\left(\begin{array}{c}
V_{2}\\
I_{2}
\end{array}\right)
\]

図回後5-6

\begin{align*}
V_{1} & =V_{2}+ZI_{2}\\
I_{1} & =I_{2}
\end{align*}

\[
\left(\begin{array}{c}
V_{1}\\
I_{1}
\end{array}\right)=\left(\begin{array}{cc}
1 & Z\\
0 & 1
\end{array}\right)\left(\begin{array}{c}
V_{2}\\
I_{2}
\end{array}\right)
\]


\paragraph{例}

図回後5-7

\[
\left(\begin{array}{c}
V_{1}\\
I_{1}
\end{array}\right)=\left(\begin{array}{cc}
1 & 0\\
\frac{1}{R} & 1
\end{array}\right)\left(\begin{array}{cc}
1 & sL\\
0 & 1
\end{array}\right)\left(\begin{array}{cc}
1 & 0\\
sC & 1
\end{array}\right)\left(\begin{array}{cc}
1 & sL\\
0 & 1
\end{array}\right)\left(\begin{array}{cc}
1 & 0\\
\frac{1}{R} & 1
\end{array}\right)\left(\begin{array}{c}
V_{2}\\
I_{2}
\end{array}\right)
\]


\subsection{Z行列とY行列}

Z行列

\[
\left(\begin{array}{c}
V_{1}\\
V_{2}
\end{array}\right)=\left(\begin{array}{cc}
Z_{11} & Z_{12}\\
Z_{21} & Z_{22}
\end{array}\right)\left(\begin{array}{c}
I_{1}\\
-I_{2}
\end{array}\right)=\boldsymbol{Z}\left(\begin{array}{c}
I_{1}\\
-I_{2}
\end{array}\right)
\]

\begin{align*}
V_{1} & =Z_{11}I_{1}-Z_{12}I_{2}\\
V_{2} & =Z_{21}I_{1}-Z_{22}I_{2}
\end{align*}

図回後5-8

\begin{align*}
\left(\begin{array}{c}
V_{1}\\
V_{2}
\end{array}\right) & =Z_{1}\left(\begin{array}{c}
I_{1}\\
-I_{2}
\end{array}\right)\\
\left(\begin{array}{c}
V_{3}\\
V_{4}
\end{array}\right) & =Z_{2}\left(\begin{array}{c}
I_{1}\\
-I_{2}
\end{array}\right)
\end{align*}

\[
\left(\begin{array}{c}
V_{1}+V_{3}\\
V_{2}+V_{4}
\end{array}\right)=\left(Z_{1}+Z_{2}\right)\left(\begin{array}{c}
I_{1}\\
-I_{2}
\end{array}\right)
\]


\paragraph{Z行列を回路から読み解く}

\begin{align*}
Z_{11} & =\when{\frac{V_{1}}{I_{1}}}_{I_{2}=0}\\
Z_{21} & =\when{\frac{V_{2}}{I_{1}}}_{I_{2}=0}
\end{align*}

(参考) $Z_{11}$: 開放駆動点インピーダンス、$Z_{21}$: 開放伝達インピーダンス と呼ぶ。

図回後5-9

\begin{align*}
Z_{12} & =\when{\frac{V_{1}}{-I_{2}}}_{I_{1}=0}\\
Z_{22} & =\when{\frac{V_{2}}{-I_{2}}}_{I_{1}=0}
\end{align*}


\paragraph{例}

図回後5-10

\begin{align*}
V_{11} & =\when{\frac{V_{1}}{I_{1}}}_{I_{2}=0}=Z_{a}+Z_{b}\\
V_{21} & =\when{\frac{V_{2}}{I_{1}}}_{I_{2}=0}=Z_{b}
\end{align*}

同様に、$Z_{12}=Z_{b},Z_{22}=Z_{b}+Z_{c}$

\paragraph{Y行列}

\[
\left(\begin{array}{c}
I_{1}\\
-I_{2}
\end{array}\right)=\left(\begin{array}{cc}
Y_{11} & Y_{12}\\
Y_{21} & Y_{22}
\end{array}\right)\left(\begin{array}{c}
V_{1}\\
V_{2}
\end{array}\right)=Y\left(\begin{array}{c}
V_{1}\\
V_{2}
\end{array}\right)
\]

\begin{align*}
I_{1} & =Y_{11}V_{1}+Y_{12}V_{2}\\
-I_{2} & =Y_{21}V_{1}+Y_{22}V_{2}
\end{align*}

図回後5-11

\begin{align*}
\left(\begin{array}{c}
I_{1}\\
-I_{2}
\end{array}\right) & =Y_{1}\left(\begin{array}{c}
V_{1}\\
V_{2}
\end{array}\right)\\
\left(\begin{array}{c}
I_{3}\\
-I_{4}
\end{array}\right) & =Y_{2}\left(\begin{array}{c}
V_{1}\\
V_{2}
\end{array}\right)
\end{align*}

\[
\left(\begin{array}{c}
I_{1}+I_{3}\\
-\left(I_{2}+I_{4}\right)
\end{array}\right)=\left(Y_{1}+Y_{2}\right)\left(\begin{array}{c}
V_{1}\\
V_{2}
\end{array}\right)
\]

並列接続された回路の計算に便利。

\paragraph{Y行列を回路から読み解く}

\begin{align*}
Y_{11} & =\when{\frac{I_{1}}{V_{1}}}_{V_{2}=0}\\
Y_{21} & =\when{\frac{-I_{2}}{V_{1}}}_{V_{2}=0}
\end{align*}

図回後5-12

(参考) $Y_{11}$: 短絡駆動点アドミッタンス、$Y_{21}$: 短絡伝達アドミッタンス と呼ぶ。

\begin{align*}
Y_{12} & =\when{\frac{I_{1}}{V_{2}}}_{V_{1}=0}\\
Y_{22} & =\when{\frac{-I_{2}}{V_{2}}}_{V_{1}=0}
\end{align*}


\paragraph{例}

図回後5-13

\begin{align*}
Y_{11} & =\when{\frac{I_{1}}{V_{1}}}_{V_{2}=0}=Y_{1}+Y_{2}\\
Y_{21} & =\when{\frac{-I_{2}}{V_{1}}}_{V_{2}=0}=-Y_{2}
\end{align*}

同様に$Y_{12}=-Y_{2},Y_{22}=Y_{2}+Y_{3}$

\subsection{F, Z, Y行列の関係}

←総ゴリ変換しながら、縦続、直列、並列を計算

\[
Y=Z^{-1}=\frac{1}{\left|Z\right|}\left(\begin{array}{cc}
Z_{22} & -Z_{12}\\
-Z_{21} & Z_{11}
\end{array}\right)\quad\left(\left|Z\right|=Z_{11}Z_{22}-Z_{12}Z_{21}\right)
\]

F行列について、
\begin{align*}
V_{1} & =AV_{2}+BI_{2}\\
I_{1} & =CV_{2}+DI_{2}
\end{align*}
をよく見る。

\begin{align}
A & =\when{\frac{V_{1}}{V_{2}}}_{I_{2}=0}=\frac{Z_{11}}{Z_{21}}=-\frac{Y_{22}}{Y_{21}}\label{5.3-1}\\
B & =\when{\frac{V_{1}}{I_{2}}}_{V_{2}=0}=-\frac{1}{Y_{21}}=\frac{\left|Z\right|}{Z_{21}}\label{5.3-3}\\
C & =\when{\frac{I_{1}}{V_{2}}}_{I_{2}=0}=\frac{1}{Z_{21}}=-\frac{\left|Y\right|}{Y_{21}}\label{5.3-2}\\
D & =\when{\frac{I_{1}}{I_{2}}}_{V_{2}=0}=-\frac{Y_{11}}{Y_{21}}=\frac{Z_{22}}{Z_{21}}\label{5.3-4}
\end{align}

逆に解くと、ZとFの関係が得られる。

\ref{5.3-1}\textdiv \ref{5.3-2}より、
\[
Z_{11}=\frac{A}{C}
\]

\ref{5.3-2}より、
\[
Z_{21}=\frac{1}{C}
\]

\ref{5.3-1}\ref{5.3-2}\ref{5.3-4}を\ref{5.3-3}に代入して、

\[
Z_{12}=\frac{AD-BC}{C}
\]

\ref{5.3-4}\textdiv \ref{5.3-2}より、
\[
Z_{22}=\frac{D}{C}
\]

\[
Z=\frac{1}{C}\left(\begin{array}{cc}
A & AD-BC\\
1 & D
\end{array}\right)
\]

注1) 回路が相反性を有する→$Z_{12}=Z_{21},Y_{12}=Y_{21},AD-BC=1$

行列計算のチェックに役立つ。

注2) 回路が対称→$Z_{11}=Z_{22},Y_{11}=Y_{22},A=D$

\subsection{等価変換}

異なる回路構成で同じ特性を実現

図回後5-14

\begin{align*}
F_{T} & =\left(\begin{array}{cc}
1 & Z_{a}\\
0 & 1
\end{array}\right)\left(\begin{array}{cc}
1 & 0\\
\frac{1}{Z_{b}} & 1
\end{array}\right)\left(\begin{array}{cc}
1 & Z_{C}\\
0 & 1
\end{array}\right)\\
F_{\pi} & =\left(\begin{array}{cc}
1 & 0\\
Y_{a} & 1
\end{array}\right)\left(\begin{array}{cc}
1 & \frac{1}{Y_{b}}\\
0 & 1
\end{array}\right)\left(\begin{array}{cc}
1 & 0\\
Y_{C} & 1
\end{array}\right)
\end{align*}

$F_{T}=F_{\pi}$とおくと、
\begin{align*}
Y_{a} & =\frac{Z_{c}}{Z_{0}^{2}}\\
Y_{b} & =\frac{Z_{b}}{Z_{0}^{2}}\\
Y_{c} & =\frac{Z_{a}}{Z_{0}^{2}}
\end{align*}

\[
Z_{0}^{2}=Z_{a}Z_{b}+Z_{b}Z_{c}+Z_{c}Z_{a}
\]


\paragraph{対称$\pi$型回路↔対称格子回路}

Z行列で考える。

図回後5-15

左

\begin{align*}
Z_{11} & =Z_{22}=Z_{a}+Z_{b}\\
Z_{21} & =Z_{12}=Z_{b}
\end{align*}

右

\begin{align*}
Z_{11} & =Z_{22}=\frac{1}{2}\left(Z_{\alpha}+Z_{\beta}\right)\\
Z_{12} & =Z_{21}=\frac{1}{2}\left(Z_{\beta}-Z_{\alpha}\right)
\end{align*}

\[
\begin{cases}
Z_{a}=Z_{\alpha}\\
Z_{b}=\frac{1}{2}\left(Z_{\beta}-Z_{\alpha}\right)
\end{cases}
\]

図回後5-16

左

\begin{align*}
Y_{11} & =Y_{22}=Y_{1}+Y_{2}\\
Y_{12} & =Y_{21}=\when{\frac{-I_{2}}{V_{1}}}_{V_{2}=0}=-Y_{2}
\end{align*}

右

\begin{align*}
Y_{11} & =Y_{22}=\frac{1}{2}\left(Y_{\alpha}+Y_{\beta}\right)\\
Y_{12} & =Y_{21}=\when{\frac{-I_{2}}{V_{1}}}_{V_{2}=0}=\frac{1}{2}\left(Y_{\beta}-Y_{\alpha}\right)
\end{align*}

\[
\begin{cases}
Y_{1}=Y_{\beta}\\
Y_{2}=\frac{1}{2}\left(Y_{\alpha}-Y_{\beta}\right)
\end{cases}
\]


\subsection{4端子網の応用}

\paragraph{インピーダンス変換}

図回後5-17

\[
Z_{1}=\frac{V_{1}}{I_{1}}=\frac{AV_{2}+B_{2}}{CV_{2}+DI_{2}}=\frac{AZ_{L}+B}{CZ_{L}+D}
\]

$Z_{L}$が$F$によって$Z_{1}$に変換される。

図回後5-18

最大の電力を取り出す条件

\[
Z_{1}=\frac{V_{1}}{I_{1}}=Z_{0}^{*}
\]


\paragraph{フィルタ}

周波数に応じて信号を通過・阻止する。

\paragraph{low-pass filter (LPF)}

図回後5-19

\[
\frac{V_{2}}{V_{1}}=\frac{1}{1+sRC}
\]

$s\rightarrow0$
\[
\frac{V_{2}}{V_{1}}=1
\]

$s\rightarrow\infty$
\[
\frac{V_{2}}{V_{1}}=0
\]


\paragraph{high-pass filter (HPF)}

図回後5-20

\[
\frac{V_{2}}{V_{1}}=\frac{sRC}{1+sRC}
\]

$s\rightarrow0$
\[
\frac{V_{2}}{V_{1}}=0
\]

$s\rightarrow\infty$
\[
\frac{V_{2}}{V_{1}}=1
\]


\paragraph{band pass filter}

図回後5-21

\[
\frac{V_{2}}{V_{1}}=\frac{RS}{Ls^{2}+Rs+\frac{1}{C}}
\]

特定周波数だけ抽出

F行列を使って計算

\subsection{影像パラメータ}

縦続接続した回路の計算方法

図回後5-22

インピーダンス整合した回路のみ表現

図回後5-23

$\theta_{1},\theta_{2},\theta_{3}$: 影像パラメータ(複素数)

$\theta=\theta_{1}+\theta_{2}+\theta_{3}$: 複素数の足し算

「インピーダンスが整合する」←「影像インピーダンス」を合わせる

影像インピーダンス: 各4端子網に固有の値

\paragraph{影像インピーダンスの求め方}

図回後5-24

\[
Z_{1}=\frac{V_{1}}{I_{1}},Z_{2}=\frac{V_{2}}{I_{2}}
\]

図回後5-25

\[
Z_{1}=\frac{V_{1}'}{-I_{1}'},Z_{2}=\frac{V_{2}'}{-I_{2}'}
\]

この4条件を満たす$Z_{1},Z_{2}$: 影像インピーダンス

\paragraph{$Z_{1},Z_{2}$とF行列の関係}

\[
\left(\begin{array}{c}
V_{1}\\
I_{1}
\end{array}\right)=\left(\begin{array}{cc}
A & B\\
C & D
\end{array}\right)\left(\begin{array}{c}
V_{2}\\
I_{2}
\end{array}\right)
\]

\begin{align*}
V_{1} & =Z_{1}I_{1}\\
V_{2} & =Z_{2}I_{2}
\end{align*}

\[
\left(\begin{array}{c}
V_{2}'\\
I_{2}'
\end{array}\right)=\frac{1}{AD-BC}\left(\begin{array}{cc}
D & -B\\
-C & A
\end{array}\right)\left(\begin{array}{c}
V_{1}'\\
I_{1}'
\end{array}\right)
\]

\begin{align*}
V_{1}' & =-Z_{1}I_{1}'\\
V_{2}' & =-Z_{2}I_{2}'
\end{align*}

\begin{equation}
\frac{Z_{1}I_{1}}{I_{1}}=\frac{AZ_{2}+B}{CZ_{2}+D}\label{5.6-1}
\end{equation}

\begin{equation}
\frac{-Z_{2}I_{2}'}{I_{2}'}=\frac{-DZ_{1}-B}{CZ_{1}+D}\label{5.6-2}
\end{equation}

(\ref{5.6-1})、(\ref{5.6-2})より、
\[
Z_{1}=\sqrt{\frac{AB}{CD}}=\sqrt{\when{\frac{V_{1}}{I_{1}}}_{I_{2}=0}\when{\frac{V_{1}}{I_{1}}}_{V_{2}=0}},Z_{2}=\sqrt{\frac{BD}{AC}}=\sqrt{\when{\frac{V_{2}}{-I_{2}}}_{I_{1}=0}\when{\frac{V_{2}}{-I_{2}}}_{V_{1}=0}}
\]

→開放駆動点インピーダンスと短絡駆動点インピーダンスの相乗平均

\paragraph{影像パラメータ$\theta$}

\[
\e^{\theta}=\sqrt{\frac{V_{1}I_{1}}{V_{2}I_{2}}}=\sqrt{\frac{\text{入力パワー}}{\text{出力パワー}}}=\frac{\text{入力振幅}}{\text{出力振幅}}
\]

\[
\theta=\log\sqrt{\frac{V_{1}I_{1}}{V_{2}I_{2}}}=\alpha+\j\beta
\]

$\alpha$: 減衰定数

$\beta$: 位相定数

\[
\e^{\theta}=\sqrt{\frac{V_{1}I_{1}}{V_{2}I_{2}}}=\cdots=\sqrt{AD}+\sqrt{BC}
\]


\subsection{まとめ}
\begin{itemize}
\item 4端子網: 回路をブロックに分けて考える単位
\item 表現方法4つ
\begin{itemize}
\item F行列、Z行列、Y行列 ←自由度4 (相反性があるので3)
\item 影像パラメータ ($Z_{1},Z_{2},\theta$) ←自由度3
\end{itemize}
\item 回路形式の変換、インピーダンス変換、フィルタ設計で活用
\end{itemize}

\end{document}
