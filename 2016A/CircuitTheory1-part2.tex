%% LyX 2.2.2 created this file.  For more info, see http://www.lyx.org/.
%% Do not edit unless you really know what you are doing.
\documentclass[english]{article}
\usepackage[T1]{fontenc}
\usepackage[utf8]{inputenc}
\usepackage[a5paper]{geometry}
\geometry{verbose,tmargin=2cm,bmargin=2cm,lmargin=1cm,rmargin=1cm}
\setlength{\parskip}{\smallskipamount}
\setlength{\parindent}{0pt}
\usepackage{textcomp}
\usepackage{amsmath}
\usepackage{amssymb}

\makeatletter
%%%%%%%%%%%%%%%%%%%%%%%%%%%%%% User specified LaTeX commands.
\usepackage[dvipdfmx]{hyperref}
\usepackage[dvipdfmx]{pxjahyper}

\makeatother

\usepackage{babel}
\begin{document}

\title{2016-A 電気回路理論第一 後半}

\author{教員: 小関泰之 入力: 高橋光輝}

\maketitle
\global\long\def\pd#1#2{\frac{\partial#1}{\partial#2}}
\global\long\def\d#1#2{\frac{\mathrm{d}#1}{\mathrm{d}#2}}
\global\long\def\pdd#1#2{\frac{\partial^{2}#1}{\partial#2^{2}}}
\global\long\def\dd#1#2{\frac{\mathrm{d}^{2}#1}{\mathrm{d}#2^{2}}}
\global\long\def\e{\mathrm{e}}
\global\long\def\i{\mathrm{i}}
\global\long\def\j{\jmath}
\global\long\def\grad{\mathrm{grad}}
\global\long\def\rot{\mathrm{rot}}
\global\long\def\div{\mathrm{div}}
\global\long\def\diag{\mathrm{diag}}


\section*{第1回}

\paragraph{講義ページ}

https://sites.google.com/site/ysozeki/lecture

\section{過渡現象}

\subsection{過渡現象とは?}

図電回後1-1

応用: ディジタル回路・アナログ回路・制御・物理現象

\paragraph{4つの見方}
\begin{itemize}
\item 微分方程式(今日)
\item 周波数特性
\item インパルス応答・ステップ応答
\item 伝達関数(複素周波数平面)
\end{itemize}
ラプラス変換・フーリエ変換がこの4つを繋げてくれる。

→複雑な回路へ

\subsection{回路方程式から微分方程式へ}

図電回後1-2

\subsubsection{基本: KVL (キルヒホッフの電圧則)}

図電回後1-3

\[
v_{R}+v_{L}+v_{C}=v\left(t\right)
\]

\[
Ri+L\d it+\frac{1}{C}\int i\mathrm{d}t=v\left(t\right)
\]

→微分

\[
R\d it+L\dd it+\frac{i}{C}=\d{v\left(t\right)}t
\]


\subsubsection{複雑な回路}

連立微分方程式

図電回後1-4

\[
\begin{cases}
Ri_{1}+L\d{}t\left(i_{1}+i_{2}\right)=v\left(t\right)\\
\frac{1}{C}\int i_{2}\mathrm{d}t+R_{2}i_{2}=L\d{}t\left(i_{1}+i_{2}\right)
\end{cases}
\]


\subsubsection{相互インダクタンス}

向きに注意

図電回後1-5

\[
\begin{cases}
Ri_{1}+L_{1}\d{i_{1}}t-M\d{i_{2}}t=v\left(t\right)\\
L_{2}\d{i_{2}}t-M\d{i_{1}}t+\frac{1}{L}\int i_{2}\mathrm{d}t=0
\end{cases}
\]


\subsection{RC回路}

(あとでRL回路・RLC回路)

図電回後1-6

\[
Ri+\frac{1}{C}\int i\mathrm{d}t=V\left(t\geqq0\right)
\]

微分して、
\[
R\d it+\frac{1}{C}i=0
\]

変数分離して、
\[
\frac{\mathrm{d}i}{i}=-\frac{1}{RC}\mathrm{d}t
\]

積分して、
\[
\log i=-\frac{1}{RC}+K\rightarrow i\left(t\right)=k\e^{-\frac{t}{RC}}
\]

ただし$k=\e^{K}$。

スイッチを入れた直後、$v_{c}\left(0+\right)=0$より初期条件
\[
i\left(0+\right)=\frac{V}{R}=k
\]

\[
i\left(t\right)=\frac{V}{R}\e^{-\frac{1}{RC}}\left(t\geqq0\right)
\]

\[
v_{R}\left(t\right)=Ri\left(t\right)=V\e^{-\frac{t}{RC}}
\]
\[
v_{c}\left(t\right)=V-v_{R}\left(t\right)=V\left(1-\e^{-\frac{t}{RC}}\right)
\]

図電回後半1-7

・$C$に供給されるパワー

\begin{align*}
P_{C}\left(t\right) & =v_{C}\left(t\right)i\left(t\right)\\
 & =\frac{V^{2}}{R}\e^{-\frac{t}{RC}}\left(1-\e^{-\frac{t}{RC}}\right)
\end{align*}

・時刻$t$までに蓄えられるエネルギー

\begin{align*}
W_{C}\left(t\right) & =\int_{0}^{t}P_{C}\left(t\right)\mathrm{d}t\\
 & =\frac{1}{2}CV^{2}\left(1-2\e^{-\frac{t}{\tau}}+\e^{-\frac{2t}{\tau}}\right)
\end{align*}

$t\rightarrow\infty$で$W_{C}\rightarrow\frac{1}{c}CV^{2}$

\paragraph{電源を短絡すると?}

図電回後半1-8

\[
i\left(t\right)=k\e^{-\frac{t}{RC}}
\]

$v_{C}\left(0+\right)=V$より、
\[
i\left(0+\right)=-\frac{V}{R}
\]

\[
i\left(t\right)=-\frac{V}{R}\e^{-\frac{t}{RC}}\left(t\geqq0\right)
\]

図電回後半1-9

\subsection{RL回路}

図電回後半1-10

\[
Ri+L\d it=V\left(t\geqq0\right)
\]

変数分離して、
\[
i+\frac{L}{R}\d it=\frac{V}{R}
\]
\[
i-\frac{V}{R}=-\frac{L}{R}\d it
\]
\[
\frac{\mathrm{d}i}{i-\frac{V}{R}}=-\frac{R}{L}\mathrm{d}t
\]
\[
\log\left(i-\frac{V}{R}\right)=-\frac{R}{L}t
\]
\[
\therefore i\left(t\right)=\frac{V}{R}+k\e^{\frac{R}{L}t}
\]

インダクタの電流は連続的に変化するため、初期条件$i\left(0+\right)=0=\frac{V}{R}+k$

\[
i\left(t\right)=\frac{V}{R}\left(1-\e^{-\frac{R}{L}t}\right)
\]

図電回後半1-11

・$L$に供給されるパワー

\begin{align*}
P_{L}\left(t\right) & =v_{L}\left(t\right)i\left(t\right)\\
 & =\frac{V^{2}}{R}\e^{-\frac{R}{L}t}\left(1-\e^{-\frac{R}{L}t}\right)
\end{align*}

・時刻$t$までに蓄えられるエネルギー

\begin{align*}
W_{L}\left(t\right) & =\int_{0}^{t}P_{L}\left(t\right)\mathrm{d}t\\
 & \xrightarrow{t\rightarrow\infty}\frac{L}{2}\left(\frac{V}{R}\right)^{2}
\end{align*}

・電源短絡時

\[
Ri+L\d it=0
\]

\[
i\left(t\right)=k\e^{-\frac{R}{L}t}
\]

$i\left(0+\right)=\frac{V}{R}$より、
\[
i\left(t\right)=\frac{V}{R}\e^{-\frac{R}{L}t}
\]

図電回後半1-12

\subsection{RLC回路}

図電回後半1-13

\[
Ri+\frac{1}{C}\int i\mathrm{d}t+L\d it=V\left(t\geqq0\right)
\]

微分して、
\[
L\dd it+R\d it+\frac{1}{C}i=0
\]

試行解$k=\e^{st}$($s$: 複素数)を代入

\[
Ls^{2}Rs+\frac{1}{C}=0
\]

\begin{align*}
S & =-\frac{R}{2L}\pm\sqrt{\left(\frac{R}{2L}\right)^{2}-\frac{1}{LC}}\\
 & =S_{1},S_{2}
\end{align*}
とし、場合分けする。
\begin{enumerate}
\item $s$: 2つの実数解
\item $s$: 重解
\item $s$: 複素数解
\end{enumerate}

\paragraph{下準備}

i) オイラーの公式

\[
\e^{\j\theta}=\cos\theta+\j\sin\theta
\]

覚え方: $f\left(\theta\right)=\e^{\j\theta}$とすると、
\[
\begin{cases}
f\left(0\right)=1\\
\d{}{\theta}f\left(\theta\right)=\j f\left(\theta\right)
\end{cases}
\]

図電回後半1-14

ii) 三角関数

\[
\cos\theta=\frac{\e^{\j\theta}-\e^{-\j\theta}}{2},\sin\theta=\frac{\e^{\j\theta}-\e^{-\j\theta}}{2\j}
\]

図電回後半1-15

iii) 双曲線関数

\[
\cosh\theta=\frac{\e^{\theta}+\e^{-\theta}}{2},\sinh\theta=\frac{\e^{\theta}-\e^{-\theta}}{2}
\]

\end{document}
