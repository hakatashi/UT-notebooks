%% LyX 2.2.2 created this file.  For more info, see http://www.lyx.org/.
%% Do not edit unless you really know what you are doing.
\documentclass[english]{article}
\usepackage[T1]{fontenc}
\usepackage[utf8]{inputenc}
\usepackage[a5paper]{geometry}
\geometry{verbose,tmargin=2cm,bmargin=2cm,lmargin=1cm,rmargin=1cm}
\setlength{\parskip}{\smallskipamount}
\setlength{\parindent}{0pt}
\usepackage{textcomp}
\usepackage{amsmath}
\usepackage{amssymb}
\usepackage{graphicx}

\makeatletter
%%%%%%%%%%%%%%%%%%%%%%%%%%%%%% User specified LaTeX commands.
\usepackage[dvipdfmx]{hyperref}
\usepackage[dvipdfmx]{pxjahyper}

\makeatother

\usepackage{babel}
\begin{document}

\title{2016-A 電子デバイス基礎}

\author{教員: 竹中充・染谷隆夫 入力: 高橋光輝}

\maketitle
\global\long\def\pd#1#2{\frac{\partial#1}{\partial#2}}
\global\long\def\d#1#2{\frac{\mathrm{d}#1}{\mathrm{d}#2}}
\global\long\def\pdd#1#2{\frac{\partial^{2}#1}{\partial#2^{2}}}
\global\long\def\dd#1#2{\frac{\mathrm{d}^{2}#1}{\mathrm{d}#2^{2}}}
\global\long\def\e{\mathrm{e}}
\global\long\def\i{\mathrm{i}}
\global\long\def\j{\mathrm{j}}
\global\long\def\grad{\mathrm{grad}}
\global\long\def\rot{\mathrm{rot}}
\global\long\def\div{\mathrm{div}}
\global\long\def\diag{\mathrm{diag}}


\section*{第1回}

\paragraph{期末試験}

関数電卓持ち込み可

\paragraph{参考書}
\begin{itemize}
\item 『半導体デバイス入門』柴田直 数理工社
\item 『半導体デバイス - 基礎理論とプロセス』S.M.ジャー 産業図書
\item 『ファインマン物理学 量子力学』
\end{itemize}

\section{結晶中の電子}

\paragraph{自由空間中の電子(自由電子)}

質量$m=9.11\times10^{-31}\text{kg}$

電荷$q=1.6\times10^{-19}\text{c}$

\paragraph{結晶中の電子}

有効質量$m^{*}$の自由電子として振る舞う。

$m^{*}<m$の場合もある。

→電子の波動性による。

de Broglie (ド・ブロイ)波長$\lambda=\frac{h}{p}$

$h$: プランク定数

$p$: 運動量

電子を1Vで加速した場合を考える。

\begin{align*}
\frac{1}{2}mv^{2} & =\text{1V}\times q\\
v & =5.2\times10^{6}\text{\ensuremath{\frac{m}{s}}}\\
\therefore\lambda & =1.2\text{nm}
\end{align*}

電子の波には実体がない。その本質は「複素数の波」である。

\[
A\mathrm{e}^{\mathrm{i}\left(kx-\omega t\right)}
\]

$A$: 振幅

ところでオイラーの公式より
\[
\mathrm{e}^{\mathrm{i}\theta}=\cos\theta+\mathrm{i}\sin\theta
\]


\section*{第2回}

\subsection{波の複素数表示}

\[
u\left(x,t\right)=A\e^{\i\underbrace{\left(kx-\omega t\right)}_{\text{位相(phase)}}}
\]

\begin{center}
\includegraphics{images/ElectronicDevice/2-1}
\par\end{center}

$x=x_{0}$で見ると、
\begin{center}
\includegraphics{images/ElectronicDevice/2-2}
\par\end{center}

$T$: 周期

\[
T=\frac{2\pi}{\omega}
\]

$\omega$: 角周波数

$t=t_{0}$で見ると、
\begin{center}
\includegraphics{images/ElectronicDevice/2-3}
\par\end{center}

$\lambda$: 波長

\[
\lambda=\frac{2\pi}{k}
\]

$k$: 波数
\begin{center}
\includegraphics{images/ElectronicDevice/2-4}
\par\end{center}

同位相の位置

\[
kx-\omega t=k\left(x+\Delta x\right)-\omega\left(t+\Delta t\right)
\]

→位相速度$v_{p}=\frac{\Delta x}{\Delta t}=\frac{\omega}{k}$

\subsection{電子の波}

$\Psi\left(x,t\right)$: 波動関数(複素数)

$\left|\Psi\right|^{2}$: 電子の存在確率

シュレディンガー方程式
\[
H\Psi\left(x,t\right)=\i\hbar\pd{\Psi}t
\]

$\hbar-\frac{h}{2\pi}$

$H$: ハミルトニアン

\[
H=-\frac{\hbar^{2}}{2m}\pdd{}x+V\left(x\right)
\]

$\Psi\left(x,t\right)=\varphi\left(x\right)\e^{-\i\omega t}$

\[
H\varphi\left(x\right)\e^{-\i\omega t}=\i\hbar\pd{}t\left[\varphi\left(x\right)\e^{-\i\omega t}\right]=\hbar\omega\varphi\left(x\right)\e^{-\i\omega x}
\]

$H\varphi\left(x\right)=\hbar\omega\varphi\left(x\right)$: 定常状態

$\hbar\omega$: エネルギー固有値 $E=\hbar\omega$の定常状態

\paragraph{平面波}

\[
\Psi\left(x,t\right)=A\e^{\i\left(kx-\omega t\right)}
\]

エネルギー$E=\hbar\omega$

運動量$p=\frac{h}{\lambda}=\hbar k$

\subsection{自由電子モデルによるエネルギーバンド}

有効質量$m^{*}$、$V\left(x\right)\sim0$

\[
-\frac{\hbar^{2}}{2m^{*}}\cdot\pdd{}x\varphi\left(x\right)=E\varphi\left(x\right)
\]

\[
\varphi\left(x\right)=A\e^{\i kx}
\]

\[
E=\frac{\left(\hbar k\right)^{2}}{2m^{*}}
\]

\begin{center}
\includegraphics{images/ElectronicDevice/2-5}
\par\end{center}

$k$と$E$の関係: 分散関係

ex. Siの有効質量$m^{*}=0.19m$

\paragraph{結晶中}

$k\sim0$の場合
\begin{center}
\includegraphics{images/ElectronicDevice/2-6}
\par\end{center}

$\lambda=\frac{2\pi}{k}\gg0$

$k=\frac{\pi}{a}\Rightarrow\lambda=2a$の場合
\begin{center}
\includegraphics{images/ElectronicDevice/2-7}
\par\end{center}

\begin{align*}
\varphi^{+}\left(x\right) & =A\left(\e^{\i kx}+\e^{-\i kx}\right)=2A\cos kx\\
\varphi^{-}\left(x\right) & =A\left(\e^{\i kx}-\e^{-\i kx}\right)=2\i A\sin kx
\end{align*}

\begin{center}
\includegraphics{images/ElectronicDevice/2-8}
\par\end{center}

$k=\pm\left(\frac{\pi}{a}\right)\underbrace{n}_{\text{整数}}$でもエネルギーが分裂する。
\begin{center}
\includegraphics{images/ElectronicDevice/2-9}
\par\end{center}

\subsection{パウリの排他律}
\begin{center}
\includegraphics{images/ElectronicDevice/2-10}
\par\end{center}

電子は全く同じ量子状態を持つことはできない。

\subsection{状態密度}

$\Delta N$: $E\sim E+\Delta E$の間にある状態数

\[
n\left(E\right)=\frac{\Delta N}{\Delta E}
\]

$n\left(E\right)$: 状態密度(Density of States, DOS)

\paragraph{3次元}
\begin{center}
\includegraphics{images/ElectronicDevice/2-11}
\par\end{center}

\[
\Psi\left(x,t\right)=\e^{\i\left(kx-\omega t\right)}
\]

\textbf{同期的境界条件}を考える。

\[
\Psi\left(x+L\right)=\Psi\left(x\right)
\]

\[
\e^{\i k\left(x+L\right)}=\e^{\i kx}
\]

\[
\e^{\i kL}=1
\]

すなわち、
\[
kL=2\pi\times\text{整数}
\]

\[
k_{x}=\frac{2\pi}{L}\times\text{整数}
\]

\[
k_{y}=\frac{2\pi}{L}\times\text{整数}
\]

\[
k_{z}=\frac{2\pi}{L}\times\text{整数}
\]

\begin{center}
\includegraphics{images/ElectronicDevice/2-12}
\par\end{center}

半径$k$の球({*})の状態数
\begin{align*}
N\left(k\right) & =\frac{\frac{4}{3}\pi k^{3}\times2\left(\text{スピン}\right)}{\left(\frac{2\pi}{L}\right)^{3}}\\
 & =\frac{L^{3}}{3\pi^{2}}k^{3}
\end{align*}

$E=\frac{\left(\hbar k\right)^{2}}{2m^{*}}$を仮定→
\[
k=\left(\frac{2m^{*}E}{\hbar^{2}}\right)^{\frac{1}{2}}
\]

∴
\[
N\left(E\right)=\frac{L^{3}}{3\pi^{2}}\left(\frac{2m^{*}E}{\hbar^{2}}\right)^{\frac{3}{2}}
\]

\[
n\left(E\right)=\d NE\cdot\frac{1}{L^{3}}
\]

\[
n\left(E\right)=\frac{1}{2\pi^{2}}\left(\frac{2m^{*}}{\hbar^{2}}\right)^{\frac{3}{2}}\cdot\left(E\right)^{\frac{1}{2}}
\]

最終的に計算される状態数は$L$によらない。
\begin{center}
\includegraphics{images/ElectronicDevice/2-13}
\par\end{center}

\section*{第3回}

\subsection{絶縁体・半導体・金属}
\begin{center}
\includegraphics{images/ElectronicDevice/3-1}
\par\end{center}

\begin{center}
\includegraphics{images/ElectronicDevice/3-2}
\par\end{center}

取りうる状態の数は、
\[
\frac{\frac{2\pi}{a}}{\frac{2\pi}{L}}=\frac{L}{a}=N
\]

スピンの状態も考慮して、$2N$個の電子状態をとることができる。

1原子あたり1個の電子を持つことができる。
\begin{center}
\includegraphics{images/ElectronicDevice/3-3}
\par\end{center}

\begin{center}
\includegraphics{images/ElectronicDevice/3-4}
\par\end{center}

\begin{center}
\includegraphics{images/ElectronicDevice/3-5}
\par\end{center}

\subsection{原子準位とバンド}
\begin{center}
\includegraphics{images/ElectronicDevice/3-6}
\par\end{center}

\paragraph{$N$個の電子}
\begin{center}
\includegraphics{images/ElectronicDevice/3-7}
\par\end{center}

\subsection{原子構造}
\begin{center}
\includegraphics{images/ElectronicDevice/3-8}
\par\end{center}

\[
\left[-\frac{\pi^{2}}{2m}\nabla^{2}+V\left(\boldsymbol{r}\right)\right]\varphi\left(\boldsymbol{r}\right)=E\varphi\left(\boldsymbol{r}\right)
\]

\[
\nabla^{2}=\pdd{}x+\pdd{}y+\pdd{}z
\]

\[
V\left(\boldsymbol{r}\right)=-\frac{2q^{2}}{4\pi\varepsilon_{0}r}
\]

極座標上でこれを解くと
\[
\varphi_{n,r,m}\left(r,\theta,\phi\right)=R_{n,l}\left(r\right)Y_{l}^{n}\left(\theta,\phi\right)
\]

$n$: 主量子数、1, 2, 3 … → K, L, M 殻

$l$: 軌道量子数 $0\leq l\leq n-1$ → 軌道の形

$m$: 磁気量子数 $-l\leq m\leq l$

\paragraph{原子の軌道}

$l=0$: s軌道

$l=1$: p軌道

$l=2$: d軌道

$l=3$: f軌道

sp混成軌道

\section*{第4回}

一部欠損

\section{???}

\subsection{???}

\subsection{真性導体}
\begin{center}
\includegraphics{images/ElectronicDevice/4-1}
\par\end{center}

\[
E_{i}\sim\frac{E_{e}+E_{r}}{2}
\]

電子濃度: $n$

ホール濃度: $p$

\[
n=p=n_{i}
\]

$n_{i}$: 真性キャリア濃度 会陰。Si($R_{1}T_{1}$) $10^{10}\mathrm{cm}^{3}$

$E\sim E+\mathrm{d}E\text{\text{の電子数}(\text{単位体積})}=\underbrace{n_{e}\left(E\right)}_{\text{状態密度}}\mathrm{d}E\times f\left(e\right)$
\begin{center}
\includegraphics{images/ElectronicDevice/4-2}
\par\end{center}

\[
E-E_{F}>\sim2kT
\]

\[
f\left(E\right)\sim\exp\left(-\frac{E-E_{F}}{kT}\right)
\]

ボルツマン分布をとる。

\subsection{N型半導体}

P, As(V族)などの不純物(dopant)を添加(doping)
\begin{center}
\includegraphics{images/ElectronicDevice/4-3}
\par\end{center}

\begin{center}
\includegraphics{images/ElectronicDevice/4-4}
\par\end{center}

\subsection{P型半導体}

B(III族)をドーピング
\begin{center}
\includegraphics{images/ElectronicDevice/4-5}
\par\end{center}

\begin{center}
\includegraphics{images/ElectronicDevice/4-6}
\par\end{center}

\paragraph{質量作用の法則}
\begin{center}
\includegraphics{images/ElectronicDevice/4-7}
\par\end{center}

生成速度$g=A\left(T\right)$ ($T$のみに依存)

再結合速度$r=B\left(T\right)np$

熱平衡状態では$g=r$なので
\[
np=\frac{A\left(T\right)}{B\left(T\right)}=n_{i}^{2}
\]

ex. N型: $n=N_{D},p=\frac{n_{i}^{2}}{N_{D}}$

P型: $p=N_{A},n=\frac{n_{i}^{2}}{N_{A}}$

$N_{D}$: $10^{20}\mathrm{cm^{-3}}$ のとき、

$n$: $10^{20}\mathrm{cm^{-3}}$: 多数キャリア(majority carrier)

$p$: $1\mathrm{cm^{-3}}$: 少数キャリア(minority carrier)

\subsection{半導体中のキャリア濃度}

伝導帯中の電子濃度
\[
n=\int_{E_{c}}^{E_{\mathrm{top}}}n_{e}\left(E\right)f\left(E\right)\mathrm{d}E
\]

価電子帯中のホール濃度
\[
p=\int_{E_{\mathrm{bottom}}}^{E_{r}}n_{p}\left(E\right)\left(1-f\left(E\right)\right)\mathrm{d}E
\]

\[
n_{e}\left(E\right)=\frac{1}{2\pi^{2}}\left(\frac{2m_{e}}{\hbar^{2}}\right)^{\frac{3}{2}}\sqrt{E-E_{c}}
\]

\begin{center}
\includegraphics{images/ElectronicDevice/4-8}
\par\end{center}

\begin{align*}
n & \sim\int_{E_{c}}^{\infty}\frac{1}{2\pi^{2}}\left(\frac{2m_{e}}{\hbar^{2}}\right)^{\frac{3}{2}}\sqrt{E-E_{c}}\exp\left(-\frac{E-E_{F}}{kT}\right)\mathrm{d}E\\
 & =\frac{1}{2\pi^{2}}\left(\frac{2m_{e}}{\hbar^{2}}\right)^{\frac{3}{2}}\exp\left(-\frac{E_{c}-E_{F}}{kT}\right)\int_{E_{c}}^{\infty}\sqrt{E-E_{c}}\exp\left(-\frac{E-E_{c}}{kT}\right)\mathrm{d}E\\
 & =2\left(\frac{m_{e}kT}{2\pi\hbar^{2}}\right)^{\frac{3}{2}}\exp\left(-\frac{E_{c}-E_{F}}{kT}\right)
\end{align*}

←$\int_{0}^{\infty}\sqrt{x}\exp\left(-\frac{x}{kT}\right)\mathrm{d}E=\frac{1}{2}\sqrt{\pi\left(kT\right)^{3}}$

\section*{第5回}
\begin{center}
\includegraphics{images/ElectronicDevice/5-1}
\par\end{center}

\begin{align*}
n & =\int_{E_{c}}^{E_{\text{top}}\rightarrow\infty}n_{e}\left(E\right)f\left(E\right)\mathrm{d}E\\
 & =2\left(\frac{m_{e}kT}{2\pi k^{2}}\right)^{\frac{3}{2}}\exp\left(-\frac{E_{C}-E_{F}}{kT}\right)
\end{align*}

$N_{C}\equiv2\left(\frac{m_{e}kT}{2\pi k^{2}}\right)^{\frac{3}{2}}$:
実効状態密度という。

\[
p=N_{V}\exp\left(-\frac{E_{F}-E_{V}}{kT}\right)
\]

\[
N_{V}=2\left(\frac{m_{k}kT}{2\pi k^{2}}\right)^{\frac{3}{2}}
\]

ex. Si (室温) $N_{C}=2.86\times10^{19}\mathrm{cm^{-3}},N_{V}=2.66\times^{19}\mathrm{cm^{-3}}$

\subsection{真性キャリア濃度}

\[
np=N_{C}N_{V}\exp\left(-\frac{E_{C}-E_{V}}{kT}\right)=n_{i}^{2}
\]

\[
n_{i}=\sqrt{N_{C}N_{V}}\exp\left(-\frac{\frac{E_{g}}{2}}{kT}\right)
\]

$E_{g}=E_{C}-E_{V}$: バンドギャップエネルギー

\paragraph{真性半導体}
\begin{center}
\includegraphics{images/ElectronicDevice/5-2}
\par\end{center}

\begin{align*}
n & =n_{i}=N_{C}\exp\left(-\frac{E_{C}-E_{i}}{xT}\right)\\
p & =n_{i}=N_{V}\exp\left(-\frac{E_{i}-E_{V}}{kT}\right)
\end{align*}

\[
\therefore E_{i}=\frac{E_{C}+E_{V}}{2}+\frac{kT}{2}\ln\frac{N_{V}}{N_{C}}
\]

\begin{align*}
n & =N_{C}\exp\left(-\frac{E_{C}-E_{F}}{kT}\right)=N_{C}\exp\left(-\frac{E_{C}-E_{i}}{kT}\right)\exp\left(\frac{E_{F}-E_{i}}{kT}\right)\\
 & =n_{i}\exp\left(\frac{E_{F}-E_{i}}{kT}\right)\\
p & =n_{i}\exp\left(\frac{E_{i}-E_{F}}{kT}\right)
\end{align*}

\begin{center}
\includegraphics{images/ElectronicDevice/5-3}
\par\end{center}

\begin{align*}
E_{F}-E_{i} & =kT\ln\left(\frac{n}{n_{i}}\right)\\
E_{F}-E_{i} & =-kT\ln\left(\frac{p}{n_{i}}\right)
\end{align*}


\subsection{キャリア濃度の温度依存性}
\begin{center}
\includegraphics{images/ElectronicDevice/5-4}
\par\end{center}

アレニウス・プロットという。

\begin{align*}
k & =A\exp\left(-\frac{E_{a}}{kT}\right)\\
\ln k & =\ln A-\frac{E_{a}}{kT}
\end{align*}

\begin{center}
\includegraphics{images/ElectronicDevice/5-5}
\par\end{center}

\begin{center}
\includegraphics{images/ElectronicDevice/5-6}
\par\end{center}

\subsection{不純物の補償}

ドナー、アクセプタを同時に含む

$N_{D}\gg N_{A}$→実効的なドナー濃度$N_{D}-N_{A}$

$N_{D}\sim N_{A}$の場合、
\[
np=n_{i}^{2},n+N_{A}=p+N_{D}
\]
を満たすことから、
\begin{align*}
n & =\frac{N_{D}-N_{A}}{2}+\sqrt{\left(\frac{N_{D}-N_{A}}{2}\right)^{2}+n_{i}^{2}}\\
p & =\frac{N_{A}-N_{D}}{2}+\sqrt{\left(\frac{N_{D}-N_{A}}{2}\right)^{2}+n_{i}^{2}}
\end{align*}


\subsection{縮退半導体}
\begin{center}
\includegraphics{images/ElectronicDevice/5-7}
\par\end{center}

ボルツマン近似が成り立たない。フェルミ・ディラック関数に立ち返って計算する必要がある。

\[
n=N_{C}\exp\left(-\frac{E_{C}-E_{F}}{kT}\right)
\]

\[
N_{D}\sim N_{C}\Rightarrow10^{19}\mathrm{cm^{-3}}
\]


\section{ファン導体中の電気伝導}

電流の媒体: 電子+ホール

電流の機構: ドリフト+拡散

\subsection{ドリフト電流}

\[
\frac{1}{2}m_{n}v_{th}^{2}=\frac{3}{2}kT
\]

$v_{th}$: 熱速度

Si (R.T.) の場合、$v_{tn}\sim10^{7}\mathrm{cm/s}$
\begin{center}
\includegraphics{images/ElectronicDevice/5-8}
\par\end{center}

\[
v_{th}=\frac{\ell}{\tau}
\]

$\ell$: 平均自由行程

$\tau$: 平均緩和時間

散乱には、
\begin{itemize}
\item フォノン散乱
\item クーロン散乱
\end{itemize}
などがある。
\begin{center}
\includegraphics{images/ElectronicDevice/5-9}
\par\end{center}

衝突時、電子の運動量~0

\[
m^{*}v_{n}=-q\vec{\varepsilon}\tau
\]

\[
\therefore v_{n}=-\frac{q\tau}{m^{*}}\vec{\varepsilon}\equiv\mu\text{(移動度})
\]

ex. Si (R.T.) $\mu_{e}\sim1600\mathrm{cm^{2}/V\cdot s}$

\paragraph{ドリフト電流}

電子電流密度 $J_{n}=-qnv_{m}=qn\mu_{n}\vec{\varepsilon}$
\end{document}
