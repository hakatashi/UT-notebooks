%% LyX 2.2.2 created this file.  For more info, see http://www.lyx.org/.
%% Do not edit unless you really know what you are doing.
\documentclass[english]{article}
\usepackage[T1]{fontenc}
\usepackage[utf8]{inputenc}
\usepackage[a5paper]{geometry}
\geometry{verbose,tmargin=2cm,bmargin=2cm,lmargin=1cm,rmargin=1cm}
\setlength{\parskip}{\smallskipamount}
\setlength{\parindent}{0pt}
\usepackage{textcomp}
\usepackage{amsmath}
\usepackage{amssymb}
\usepackage{graphicx}
\PassOptionsToPackage{normalem}{ulem}
\usepackage{ulem}

\makeatletter
%%%%%%%%%%%%%%%%%%%%%%%%%%%%%% User specified LaTeX commands.
\usepackage[dvipdfmx]{hyperref}
\usepackage[dvipdfmx]{pxjahyper}

\makeatother

\usepackage{babel}
\begin{document}

\title{2016-A 電子デバイス基礎}

\author{教員: 竹中充・染谷隆夫 入力: 高橋光輝}

\maketitle
\global\long\def\pd#1#2{\frac{\partial#1}{\partial#2}}
\global\long\def\d#1#2{\frac{\mathrm{d}#1}{\mathrm{d}#2}}
\global\long\def\pdd#1#2{\frac{\partial^{2}#1}{\partial#2^{2}}}
\global\long\def\dd#1#2{\frac{\mathrm{d}^{2}#1}{\mathrm{d}#2^{2}}}
\global\long\def\ddd#1#2{\frac{\mathrm{d}^{3}#1}{\mathrm{d}#2^{3}}}
\global\long\def\e{\mathrm{e}}
\global\long\def\i{\mathrm{i}}
\global\long\def\j{\mathrm{j}}
\global\long\def\grad{\operatorname{grad}}
\global\long\def\rot{\operatorname{rot}}
\global\long\def\div{\operatorname{div}}
\global\long\def\diag{\operatorname{diag}}
\global\long\def\rank{\operatorname{rank}}
\global\long\def\prob{\operatorname{Prob}}
\global\long\def\cov{\operatorname{Cov}}
\global\long\def\when#1{\left.#1\right|}


\section*{第1回}

\paragraph{期末試験}

関数電卓持ち込み可

\paragraph{参考書}
\begin{itemize}
\item 『半導体デバイス入門』柴田直 数理工社
\item 『半導体デバイス - 基礎理論とプロセス』S.M.ジャー 産業図書
\item 『ファインマン物理学 量子力学』
\end{itemize}

\section{結晶中の電子}

\paragraph{自由空間中の電子(自由電子)}

質量$m=9.11\times10^{-31}\text{kg}$

電荷$q=1.6\times10^{-19}\text{c}$

\paragraph{結晶中の電子}

有効質量$m^{*}$の自由電子として振る舞う。

$m^{*}<m$の場合もある。

→電子の波動性による。

de Broglie (ド・ブロイ)波長$\lambda=\frac{h}{p}$

$h$: プランク定数

$p$: 運動量

電子を1Vで加速した場合を考える。

\begin{align*}
\frac{1}{2}mv^{2} & =\text{1V}\times q\\
v & =5.2\times10^{6}\text{\ensuremath{\frac{m}{s}}}\\
\therefore\lambda & =1.2\text{nm}
\end{align*}

電子の波には実体がない。その本質は「複素数の波」である。

\[
A\mathrm{e}^{\mathrm{i}\left(kx-\omega t\right)}
\]

$A$: 振幅

ところでオイラーの公式より
\[
\mathrm{e}^{\mathrm{i}\theta}=\cos\theta+\mathrm{i}\sin\theta
\]


\section*{第2回}

\subsection{波の複素数表示}

\[
u\left(x,t\right)=A\e^{\i\underbrace{\left(kx-\omega t\right)}_{\text{位相(phase)}}}
\]

\begin{center}
\includegraphics{images/ElectronicDevice/2-1}
\par\end{center}

$x=x_{0}$で見ると、
\begin{center}
\includegraphics{images/ElectronicDevice/2-2}
\par\end{center}

$T$: 周期

\[
T=\frac{2\pi}{\omega}
\]

$\omega$: 角周波数

$t=t_{0}$で見ると、
\begin{center}
\includegraphics{images/ElectronicDevice/2-3}
\par\end{center}

$\lambda$: 波長

\[
\lambda=\frac{2\pi}{k}
\]

$k$: 波数
\begin{center}
\includegraphics{images/ElectronicDevice/2-4}
\par\end{center}

同位相の位置

\[
kx-\omega t=k\left(x+\Delta x\right)-\omega\left(t+\Delta t\right)
\]

→位相速度$v_{p}=\frac{\Delta x}{\Delta t}=\frac{\omega}{k}$

\subsection{電子の波}

$\Psi\left(x,t\right)$: 波動関数(複素数)

$\left|\Psi\right|^{2}$: 電子の存在確率

シュレディンガー方程式
\[
H\Psi\left(x,t\right)=\i\hbar\pd{\Psi}t
\]

$\hbar-\frac{h}{2\pi}$

$H$: ハミルトニアン

\[
H=-\frac{\hbar^{2}}{2m}\pdd{}x+V\left(x\right)
\]

$\Psi\left(x,t\right)=\varphi\left(x\right)\e^{-\i\omega t}$

\[
H\varphi\left(x\right)\e^{-\i\omega t}=\i\hbar\pd{}t\left[\varphi\left(x\right)\e^{-\i\omega t}\right]=\hbar\omega\varphi\left(x\right)\e^{-\i\omega x}
\]

$H\varphi\left(x\right)=\hbar\omega\varphi\left(x\right)$: 定常状態

$\hbar\omega$: エネルギー固有値 $E=\hbar\omega$の定常状態

\paragraph{平面波}

\[
\Psi\left(x,t\right)=A\e^{\i\left(kx-\omega t\right)}
\]

エネルギー$E=\hbar\omega$

運動量$p=\frac{h}{\lambda}=\hbar k$

\subsection{自由電子モデルによるエネルギーバンド}

有効質量$m^{*}$、$V\left(x\right)\sim0$

\[
-\frac{\hbar^{2}}{2m^{*}}\cdot\pdd{}x\varphi\left(x\right)=E\varphi\left(x\right)
\]

\[
\varphi\left(x\right)=A\e^{\i kx}
\]

\[
E=\frac{\left(\hbar k\right)^{2}}{2m^{*}}
\]

\begin{center}
\includegraphics{images/ElectronicDevice/2-5}
\par\end{center}

$k$と$E$の関係: 分散関係

ex. Siの有効質量$m^{*}=0.19m$

\paragraph{結晶中}

$k\sim0$の場合
\begin{center}
\includegraphics{images/ElectronicDevice/2-6}
\par\end{center}

$\lambda=\frac{2\pi}{k}\gg0$

$k=\frac{\pi}{a}\Rightarrow\lambda=2a$の場合
\begin{center}
\includegraphics{images/ElectronicDevice/2-7}
\par\end{center}

\begin{align*}
\varphi^{+}\left(x\right) & =A\left(\e^{\i kx}+\e^{-\i kx}\right)=2A\cos kx\\
\varphi^{-}\left(x\right) & =A\left(\e^{\i kx}-\e^{-\i kx}\right)=2\i A\sin kx
\end{align*}

\begin{center}
\includegraphics{images/ElectronicDevice/2-8}
\par\end{center}

$k=\pm\left(\frac{\pi}{a}\right)\underbrace{n}_{\text{整数}}$でもエネルギーが分裂する。
\begin{center}
\includegraphics{images/ElectronicDevice/2-9}
\par\end{center}

\subsection{パウリの排他律}
\begin{center}
\includegraphics{images/ElectronicDevice/2-10}
\par\end{center}

電子は全く同じ量子状態を持つことはできない。

\subsection{状態密度}

$\Delta N$: $E\sim E+\Delta E$の間にある状態数

\[
n\left(E\right)=\frac{\Delta N}{\Delta E}
\]

$n\left(E\right)$: 状態密度(Density of States, DOS)

\paragraph{3次元}
\begin{center}
\includegraphics{images/ElectronicDevice/2-11}
\par\end{center}

\[
\Psi\left(x,t\right)=\e^{\i\left(kx-\omega t\right)}
\]

\textbf{同期的境界条件}を考える。

\[
\Psi\left(x+L\right)=\Psi\left(x\right)
\]

\[
\e^{\i k\left(x+L\right)}=\e^{\i kx}
\]

\[
\e^{\i kL}=1
\]

すなわち、
\[
kL=2\pi\times\text{整数}
\]

\[
k_{x}=\frac{2\pi}{L}\times\text{整数}
\]

\[
k_{y}=\frac{2\pi}{L}\times\text{整数}
\]

\[
k_{z}=\frac{2\pi}{L}\times\text{整数}
\]

\begin{center}
\includegraphics{images/ElectronicDevice/2-12}
\par\end{center}

半径$k$の球({*})の状態数
\begin{align*}
N\left(k\right) & =\frac{\frac{4}{3}\pi k^{3}\times2\left(\text{スピン}\right)}{\left(\frac{2\pi}{L}\right)^{3}}\\
 & =\frac{L^{3}}{3\pi^{2}}k^{3}
\end{align*}

$E=\frac{\left(\hbar k\right)^{2}}{2m^{*}}$を仮定→
\[
k=\left(\frac{2m^{*}E}{\hbar^{2}}\right)^{\frac{1}{2}}
\]

∴
\[
N\left(E\right)=\frac{L^{3}}{3\pi^{2}}\left(\frac{2m^{*}E}{\hbar^{2}}\right)^{\frac{3}{2}}
\]

\[
n\left(E\right)=\d NE\cdot\frac{1}{L^{3}}
\]

\[
n\left(E\right)=\frac{1}{2\pi^{2}}\left(\frac{2m^{*}}{\hbar^{2}}\right)^{\frac{3}{2}}\cdot\left(E\right)^{\frac{1}{2}}
\]

最終的に計算される状態数は$L$によらない。
\begin{center}
\includegraphics{images/ElectronicDevice/2-13}
\par\end{center}

\section*{第3回}

\subsection{絶縁体・半導体・金属}
\begin{center}
\includegraphics{images/ElectronicDevice/3-1}
\par\end{center}

\begin{center}
\includegraphics{images/ElectronicDevice/3-2}
\par\end{center}

取りうる状態の数は、
\[
\frac{\frac{2\pi}{a}}{\frac{2\pi}{L}}=\frac{L}{a}=N
\]

スピンの状態も考慮して、$2N$個の電子状態をとることができる。

1原子あたり1個の電子を持つことができる。
\begin{center}
\includegraphics{images/ElectronicDevice/3-3}
\par\end{center}

\begin{center}
\includegraphics{images/ElectronicDevice/3-4}
\par\end{center}

\begin{center}
\includegraphics{images/ElectronicDevice/3-5}
\par\end{center}

\subsection{原子準位とバンド}
\begin{center}
\includegraphics{images/ElectronicDevice/3-6}
\par\end{center}

\paragraph{$N$個の電子}
\begin{center}
\includegraphics{images/ElectronicDevice/3-7}
\par\end{center}

\subsection{原子構造}
\begin{center}
\includegraphics{images/ElectronicDevice/3-8}
\par\end{center}

\[
\left[-\frac{\pi^{2}}{2m}\nabla^{2}+V\left(\boldsymbol{r}\right)\right]\varphi\left(\boldsymbol{r}\right)=E\varphi\left(\boldsymbol{r}\right)
\]

\[
\nabla^{2}=\pdd{}x+\pdd{}y+\pdd{}z
\]

\[
V\left(\boldsymbol{r}\right)=-\frac{2q^{2}}{4\pi\varepsilon_{0}r}
\]

極座標上でこれを解くと
\[
\varphi_{n,r,m}\left(r,\theta,\phi\right)=R_{n,l}\left(r\right)Y_{l}^{n}\left(\theta,\phi\right)
\]

$n$: 主量子数、1, 2, 3 … → K, L, M 殻

$l$: 軌道量子数 $0\leq l\leq n-1$ → 軌道の形

$m$: 磁気量子数 $-l\leq m\leq l$

\paragraph{原子の軌道}

$l=0$: s軌道

$l=1$: p軌道

$l=2$: d軌道

$l=3$: f軌道

sp混成軌道

\section*{第4回}

一部欠損

\section{???}

\subsection{???}

\subsection{真性導体}
\begin{center}
\includegraphics{images/ElectronicDevice/4-1}
\par\end{center}

\[
E_{i}\sim\frac{E_{e}+E_{r}}{2}
\]

電子濃度: $n$

ホール濃度: $p$

\[
n=p=n_{i}
\]

$n_{i}$: 真性キャリア濃度 会陰。Si($R_{1}T_{1}$) $10^{10}\mathrm{cm}^{3}$

$E\sim E+\mathrm{d}E\text{\text{の電子数}(\text{単位体積})}=\underbrace{n_{e}\left(E\right)}_{\text{状態密度}}\mathrm{d}E\times f\left(e\right)$
\begin{center}
\includegraphics{images/ElectronicDevice/4-2}
\par\end{center}

\[
E-E_{F}>\sim2kT
\]

\[
f\left(E\right)\sim\exp\left(-\frac{E-E_{F}}{kT}\right)
\]

ボルツマン分布をとる。

\subsection{N型半導体}

P, As(V族)などの不純物(dopant)を添加(doping)
\begin{center}
\includegraphics{images/ElectronicDevice/4-3}
\par\end{center}

\begin{center}
\includegraphics{images/ElectronicDevice/4-4}
\par\end{center}

\subsection{P型半導体}

B(III族)をドーピング
\begin{center}
\includegraphics{images/ElectronicDevice/4-5}
\par\end{center}

\begin{center}
\includegraphics{images/ElectronicDevice/4-6}
\par\end{center}

\paragraph{質量作用の法則}
\begin{center}
\includegraphics{images/ElectronicDevice/4-7}
\par\end{center}

生成速度$g=A\left(T\right)$ ($T$のみに依存)

再結合速度$r=B\left(T\right)np$

熱平衡状態では$g=r$なので
\[
np=\frac{A\left(T\right)}{B\left(T\right)}=n_{i}^{2}
\]

ex. N型: $n=N_{D},p=\frac{n_{i}^{2}}{N_{D}}$

P型: $p=N_{A},n=\frac{n_{i}^{2}}{N_{A}}$

$N_{D}$: $10^{20}\mathrm{cm^{-3}}$ のとき、

$n$: $10^{20}\mathrm{cm^{-3}}$: 多数キャリア(majority carrier)

$p$: $1\mathrm{cm^{-3}}$: 少数キャリア(minority carrier)

\subsection{半導体中のキャリア濃度}

伝導帯中の電子濃度
\[
n=\int_{E_{c}}^{E_{\mathrm{top}}}n_{e}\left(E\right)f\left(E\right)\mathrm{d}E
\]

価電子帯中のホール濃度
\[
p=\int_{E_{\mathrm{bottom}}}^{E_{r}}n_{p}\left(E\right)\left(1-f\left(E\right)\right)\mathrm{d}E
\]

\[
n_{e}\left(E\right)=\frac{1}{2\pi^{2}}\left(\frac{2m_{e}}{\hbar^{2}}\right)^{\frac{3}{2}}\sqrt{E-E_{c}}
\]

\begin{center}
\includegraphics{images/ElectronicDevice/4-8}
\par\end{center}

\begin{align*}
n & \sim\int_{E_{c}}^{\infty}\frac{1}{2\pi^{2}}\left(\frac{2m_{e}}{\hbar^{2}}\right)^{\frac{3}{2}}\sqrt{E-E_{c}}\exp\left(-\frac{E-E_{F}}{kT}\right)\mathrm{d}E\\
 & =\frac{1}{2\pi^{2}}\left(\frac{2m_{e}}{\hbar^{2}}\right)^{\frac{3}{2}}\exp\left(-\frac{E_{c}-E_{F}}{kT}\right)\int_{E_{c}}^{\infty}\sqrt{E-E_{c}}\exp\left(-\frac{E-E_{c}}{kT}\right)\mathrm{d}E\\
 & =2\left(\frac{m_{e}kT}{2\pi\hbar^{2}}\right)^{\frac{3}{2}}\exp\left(-\frac{E_{c}-E_{F}}{kT}\right)
\end{align*}

←$\int_{0}^{\infty}\sqrt{x}\exp\left(-\frac{x}{kT}\right)\mathrm{d}E=\frac{1}{2}\sqrt{\pi\left(kT\right)^{3}}$

\section*{第5回}
\begin{center}
\includegraphics{images/ElectronicDevice/5-1}
\par\end{center}

\begin{align*}
n & =\int_{E_{c}}^{E_{\text{top}}\rightarrow\infty}n_{e}\left(E\right)f\left(E\right)\mathrm{d}E\\
 & =2\left(\frac{m_{e}kT}{2\pi k^{2}}\right)^{\frac{3}{2}}\exp\left(-\frac{E_{C}-E_{F}}{kT}\right)
\end{align*}

$N_{C}\equiv2\left(\frac{m_{e}kT}{2\pi k^{2}}\right)^{\frac{3}{2}}$:
実効状態密度という。

\[
p=N_{V}\exp\left(-\frac{E_{F}-E_{V}}{kT}\right)
\]

\[
N_{V}=2\left(\frac{m_{k}kT}{2\pi k^{2}}\right)^{\frac{3}{2}}
\]

ex. Si (室温) $N_{C}=2.86\times10^{19}\mathrm{cm^{-3}},N_{V}=2.66\times^{19}\mathrm{cm^{-3}}$

\subsection{真性キャリア濃度}

\[
np=N_{C}N_{V}\exp\left(-\frac{E_{C}-E_{V}}{kT}\right)=n_{i}^{2}
\]

\[
n_{i}=\sqrt{N_{C}N_{V}}\exp\left(-\frac{\frac{E_{g}}{2}}{kT}\right)
\]

$E_{g}=E_{C}-E_{V}$: バンドギャップエネルギー

\paragraph{真性半導体}
\begin{center}
\includegraphics{images/ElectronicDevice/5-2}
\par\end{center}

\begin{align*}
n & =n_{i}=N_{C}\exp\left(-\frac{E_{C}-E_{i}}{xT}\right)\\
p & =n_{i}=N_{V}\exp\left(-\frac{E_{i}-E_{V}}{kT}\right)
\end{align*}

\[
\therefore E_{i}=\frac{E_{C}+E_{V}}{2}+\frac{kT}{2}\ln\frac{N_{V}}{N_{C}}
\]

\begin{align*}
n & =N_{C}\exp\left(-\frac{E_{C}-E_{F}}{kT}\right)=N_{C}\exp\left(-\frac{E_{C}-E_{i}}{kT}\right)\exp\left(\frac{E_{F}-E_{i}}{kT}\right)\\
 & =n_{i}\exp\left(\frac{E_{F}-E_{i}}{kT}\right)\\
p & =n_{i}\exp\left(\frac{E_{i}-E_{F}}{kT}\right)
\end{align*}

\begin{center}
\includegraphics{images/ElectronicDevice/5-3}
\par\end{center}

\begin{align*}
E_{F}-E_{i} & =kT\ln\left(\frac{n}{n_{i}}\right)\\
E_{F}-E_{i} & =-kT\ln\left(\frac{p}{n_{i}}\right)
\end{align*}


\subsection{キャリア濃度の温度依存性}
\begin{center}
\includegraphics{images/ElectronicDevice/5-4}
\par\end{center}

アレニウス・プロットという。

\begin{align*}
k & =A\exp\left(-\frac{E_{a}}{kT}\right)\\
\ln k & =\ln A-\frac{E_{a}}{kT}
\end{align*}

\begin{center}
\includegraphics{images/ElectronicDevice/5-5}
\par\end{center}

\begin{center}
\includegraphics{images/ElectronicDevice/5-6}
\par\end{center}

\subsection{不純物の補償}

ドナー、アクセプタを同時に含む

$N_{D}\gg N_{A}$→実効的なドナー濃度$N_{D}-N_{A}$

$N_{D}\sim N_{A}$の場合、
\[
np=n_{i}^{2},n+N_{A}=p+N_{D}
\]
を満たすことから、
\begin{align*}
n & =\frac{N_{D}-N_{A}}{2}+\sqrt{\left(\frac{N_{D}-N_{A}}{2}\right)^{2}+n_{i}^{2}}\\
p & =\frac{N_{A}-N_{D}}{2}+\sqrt{\left(\frac{N_{D}-N_{A}}{2}\right)^{2}+n_{i}^{2}}
\end{align*}


\subsection{縮退半導体}
\begin{center}
\includegraphics{images/ElectronicDevice/5-7}
\par\end{center}

ボルツマン近似が成り立たない。フェルミ・ディラック関数に立ち返って計算する必要がある。

\[
n=N_{C}\exp\left(-\frac{E_{C}-E_{F}}{kT}\right)
\]

\[
N_{D}\sim N_{C}\Rightarrow10^{19}\mathrm{cm^{-3}}
\]


\section{ファン導体中の電気伝導}

電流の媒体: 電子+ホール

電流の機構: ドリフト+拡散

\subsection{ドリフト電流}

\[
\frac{1}{2}m_{n}v_{th}^{2}=\frac{3}{2}kT
\]

$v_{th}$: 熱速度

Si (R.T.) の場合、$v_{tn}\sim10^{7}\mathrm{cm/s}$
\begin{center}
\includegraphics{images/ElectronicDevice/5-8}
\par\end{center}

\[
v_{th}=\frac{\ell}{\tau}
\]

$\ell$: 平均自由行程

$\tau$: 平均緩和時間

散乱には、
\begin{itemize}
\item フォノン散乱
\item クーロン散乱
\end{itemize}
などがある。
\begin{center}
\includegraphics{images/ElectronicDevice/5-9}
\par\end{center}

衝突時、電子の運動量~0

\[
m^{*}v_{n}=-q\vec{\varepsilon}\tau
\]

\[
\therefore v_{n}=-\frac{q\tau}{m^{*}}\vec{\varepsilon}\equiv\mu\text{(移動度})
\]

ex. Si (R.T.) $\mu_{e}\sim1600\mathrm{cm^{2}/V\cdot s}$

\paragraph{ドリフト電流}

電子電流密度 $J_{n}=-qnv_{m}=qn\mu_{n}\vec{\varepsilon}$

\section*{第6回}

\subsection{拡散電流}
\begin{center}
\includegraphics{images/ElectronicDevice/6-1}
\par\end{center}

\begin{align*}
n\left(-l\right) & \sim n\left(0\right)-l\d nx\\
n\left(l\right) & \sim n\left(0\right)+\d nx
\end{align*}
MF\_1 =
\begin{align*}
F_{1} & =\frac{\frac{1}{2}n\left(-l\right)\times l}{\tau}=\frac{1}{2}n\left(-l\right)v_{\text{in}}\\
F_{2} & =\frac{1}{2}n\left(l\right)v_{\text{in}}
\end{align*}

\[
F=F_{1}-F_{2}=\frac{1}{2}v_{\text{in}}\left[n\left(-l\right)-n\left(l\right)\right]=-v_{\text{in}}l\d nx\equiv-D_{n}\d nx
\]

$D_{n}$: 拡散係数

\[
J_{n}=-qF=qD_{n}\d nx
\]
\[
J_{p}=-qD_{p}\d px
\]


\paragraph{アインシュタインの関係数}

1次元の

\subsection{直接キャリア結合}
\begin{center}
\includegraphics{images/ElectronicDevice/6-2}
\par\end{center}

\[
\d{\left(n+\Delta n\right)}t=g-r=g^{2}+g_{0}-r
\]

定常状態
\[
g^{*}=r-g_{0}\equiv U
\]

$U$: 実効再結合レート

\begin{align*}
U & =r-g_{0}=Bnp_{0}-Bn_{0}p_{0}\\
 & =BP_{0}\left(n-n_{0}\right)\\
 & =\frac{1}{\frac{1}{BP_{0}}}\Delta n\\
 & =\frac{1}{\tau_{n}}\Delta n
\end{align*}

$\tau_{n}$: 少数キャリア寿命

\[
\d{\Delta n}t=g^{*}-U=g^{*}-\frac{\Delta n}{\tau_{n}}
\]

\begin{center}
\includegraphics{images/ElectronicDevice/6-3}
\par\end{center}

定常状態
\[
\Delta n=g^{*}\tau_{n}
\]


\paragraph{間接再結合}
\begin{center}
\includegraphics{images/ElectronicDevice/6-4}
\par\end{center}

\[
U=\frac{pn-n_{i}^{2}}{p+n+2n_{i}\cosh\left(\frac{E_{t}-E_{i}}{kT}\right)}\times v_{\text{th}}\alpha N_{t}
\]

$\alpha$: 捕獲断面積

$N_{t}$: 欠陥密度

Shochley-Read-Hall(SRH)の式

\paragraph{表面再結合}

\subsection{連続の式}
\begin{center}
\includegraphics{images/ElectronicDevice/6-4}
\par\end{center}

\[
\pd ntA\cdot\mathrm{d}x=\left[\frac{J_{n}\left(x\right)A}{-q}-\frac{J_{n}\left(x+\mathrm{d}x\right)A}{-q}\right]+\left(q-r\right)A\mathrm{d}x
\]


\subsection{静電電位とバンド図}

図電づ6-5

$E_{c},E_{i},E_{v}$と$\phi$は対応

\[
\phi=-\frac{E_{i}}{q}
\]

\[
\vec{\varepsilon}=\frac{1}{q}\cdot\pd{E_{i}}x
\]

\begin{enumerate}
\item 
\[
\pd nt=\frac{1}{q}\pd{J_{n}}x+\left(g-r\right)=\mu_{n}\pd{\left(n\vec{\varepsilon}\right)}x+D_{n}\pdd nx+\left(g-r\right)
\]
\item 
\[
\pd Pt=-\frac{1}{q}\pd{J_{p}}x+\left(g-r\right)=-\mu_{p}\pd{\left(p\vec{\varepsilon}\right)}x+D_{p}\pdd px+\left(g-r\right)
\]

$\vec{\varepsilon}=-\pd{}x\phi$: 静電電位
\item 
\[
\pdd{}x\phi=-\frac{\rho}{\varepsilon}
\]
: ポアソン方程式
\end{enumerate}

\section*{第7回}

3.9 熱平衡次のフェルミ準位

\[
J_{n}=J_{p}=0
\]

\begin{align*}
J_{n} & =q\mu_{n}n\vec{\varepsilon}+qD_{n}\d nx\\
 & =q\mu_{n}n\left(\frac{1}{q}\d{E_{i}}x\right)+kT\mu_{n}\d nx
\end{align*}

\[
n=n_{i}\exp\left(\frac{E_{F}-E_{i}}{kT}\right)
\]

\begin{align*}
\therefore\d nx & =n_{i}\exp\left(\frac{E_{F}-E_{i}}{kT}\right)\left[\frac{1}{kT}\cdot\d{E_{F}}x-\frac{1}{kT}\d{E_{i}}x\right]\\
 & =n\frac{1}{kT}\left(\d{E_{f}}x-\d{E_{i}}x\right)
\end{align*}

\begin{align*}
J_{n} & =q\mu_{n}n\left(\frac{1}{q}\cdot\d{E_{i}}x\right)+n\mu_{n}\left(\d{E_{F}}x-\d{E_{i}}x\right)\\
 & =n\mu_{n}\d{E_{F}}x=0
\end{align*}

\[
\therefore\d{E_{F}}x=0
\]

⇒$E_{F}$が一定

3.10 擬フェルミ準位

熱平衡
\[
n=n_{i}\exp\left(\frac{E_{F}-E_{i}}{kT}\right),p=n_{i}\exp\left(\frac{E_{i}-E_{F}}{kT}\right)
\]

非平衡
\[
n=n_{i}\exp\left(\frac{E_{F_{n}}-E_{i}}{kT}\right),p=n_{i}\exp\left(\frac{E_{i}-E_{FP}}{kT}\right)
\]

\begin{center}
\includegraphics{images/ElectronicDevice/7-1}
\par\end{center}

\begin{align*}
J_{n} & =\mu_{n}n\d{E_{F_{n}}}x\\
J_{P} & =\mu_{P}p\d{E_{F_{P}}}x
\end{align*}


\section{PN接合(junction)}
\begin{center}
\includegraphics{images/ElectronicDevice/7-2}
\par\end{center}

\subsection{階段結合}
\begin{center}
\includegraphics{images/ElectronicDevice/7-3}
\par\end{center}

\begin{center}
\includegraphics{images/ElectronicDevice/7-4}
\par\end{center}

\[
\pdd{\phi}x=-\frac{\rho}{\varepsilon_{S}}
\]

空気層$n=0,p\sim0$ (空乏近似)

境界条件
\[
\vec{\varepsilon}\left(x=-l_{n}\right)=0,\vec{\varepsilon}\left(x=l_{p}\right)=0
\]

\[
\phi\left(x=0\right)=0
\]

(i) $-l_{n}\leq x\leq0$

\[
\pdd{\phi}x=-\frac{qN_{D}}{\varepsilon_{S}}
\]

\[
\pd{\phi}x=-\frac{qN_{D}}{\varepsilon_{S}}\left(x+l_{n}\right)
\]

\[
\therefore\phi=-\frac{qN_{D}}{2\varepsilon_{S}}\left[\left(x+l_{n}\right)^{2}-l_{n}^{2}\right]
\]

(ii) $0\leq x\leq l_{P}$

\[
\pdd{\phi}x=\frac{qN_{A}}{\varepsilon_{S}}
\]

\[
\phi=\frac{qN_{A}}{2\varepsilon_{S}}\left[\left(x-l_{P}\right)^{2}-l_{P}^{2}\right]
\]

\begin{center}
\includegraphics{images/ElectronicDevice/7-5}
\par\end{center}

\begin{align*}
V_{bi} & =\phi\left(x=l_{n}\right)-\phi\left(x=l_{p}\right)\\
 & =\frac{q}{2\varepsilon_{S}}\left[N_{D}l_{n}^{2}+N_{A}l_{p}^{2}\right]
\end{align*}

$l_{n}N_{D}=l_{p}N_{A}$

\[
\therefore l_{n}=\sqrt{\frac{2\varepsilon_{S}V_{bi}}{qN_{D}\left(1+\frac{N_{D}}{N_{A}}\right)}},l_{p}=\sqrt{\frac{2\varepsilon_{S}V_{bi}}{qN_{A}\left(1+\frac{N_{D}}{N_{A}}\right)}}
\]

\[
W=l_{n}+l_{p}=\sqrt{\frac{2\varepsilon_{S}V_{bi}}{q}\left(\frac{1}{N_{D}}+\frac{1}{N_{A}}\right)}
\]

・$V>0$ (順バイアス)
\begin{center}
\includegraphics{images/ElectronicDevice/7-6}
\par\end{center}

拡散電流↑

少数キャリア注入
\begin{center}
\includegraphics{images/ElectronicDevice/7-7}
\par\end{center}

拡散電流~0

空乏層幅

\[
V_{bi}\rightarrow V_{bi}-V
\]

\[
l_{n}=\sqrt{\frac{2\varepsilon_{S}\left(V_{bi}-V\right)}{qN_{D}\left(1+\frac{N_{D}}{N_{A}}\right)}},l_{p}=\sqrt{\frac{2\varepsilon_{S}\left(V_{bi}-V\right)}{qN_{A}\left(1+\frac{N_{D}}{N_{A}}\right)}}
\]

\[
W=l_{n}+l_{p}=\sqrt{\frac{2\varepsilon_{S}\left(V_{bi}-V\right)}{q}\left(\frac{1}{N_{D}}+\frac{1}{N_{A}}\right)}
\]

$V<0$の時$W$↑
\begin{center}
\includegraphics{images/ElectronicDevice/7-8}
\par\end{center}

\section*{第8回}

自主休講

\section*{第9回}

\section{MOSトランジスタ(MOS-FET)}

Metal-Oxide-Semiconductor: 金属-SiO2(絶縁体)-半導体(Si)

Field-Effect-Transistor: 電界効果トランジスタ

\subsection{どのような構造をしているか}

図電デ9-1

図電デ9-2

\subsection{動作原理のポイント}
\begin{enumerate}
\item 水(電子)の流れをどう作るのか?

→ドレイン(排水口)の推移を下げれば良い。

外部から電界を印加する。
\item 水門(ゲート)をどうつくるのか?

NPN構造を使う。(PNP構造でも良い)
\item 水門(ゲート)をどう動かすのか?

MON構造を使い電界効果で動かす。
\end{enumerate}

\paragraph{水門の作り方}

図電デ9-3

図電デ9-4

\paragraph{MOSFETの2つのタイプ}
\begin{itemize}
\item NPN構造
\begin{itemize}
\item NチャネルMOSFET(NMoS)
\item キャリアは電子
\end{itemize}
\item PNP構造
\begin{itemize}
\item PチャネルMOSFET(PMOS)
\item キャリアは正孔
\end{itemize}
\end{itemize}

\subsection{MOS構造}

図電デ9-5

反転層の電荷量$Q_{I}$

\[
Q_{I}=qN_{I}=\mathrm{Cox}\left(V-V_{TH}\right)
\]

$I$: Inversion

$N_{I}$: キャリア濃度

$\mathrm{Cox}$: 酸化膜反応

$V$: 外部印加電圧

$V_{TH}$ しきい(電圧)

\section*{第10回}

\paragraph{MOSのバンド図}

電子エネルギーを表す

電位は下向きが正

図電デ10-1

図電デ10-2

図電デ10-3

熱平衡状態においては
\[
\psi_{S}-\psi_{M}=V_{\mathrm{OX}}+\phi_{S}
\]

$V_{\mathrm{OX}}$: 酸化膜($\mathrm{SiO_{2}}$)にかかる電圧

$\phi_{S}$: 半導体の空乏層にかかる電圧→「バンド曲がり」

図電デ10-4

\[
V_{\mathrm{FB}}=-\left(\psi_{S}-\psi_{M}\right)=\psi_{M}-\psi_{S}=\psi_{MS}
\]

$\psi_{M}-\psi_{S}$: 金属と半導体の仕事関数の差

図電デ10-5

深さ$x$のキャリア密度は?

電子濃度 
\[
n\left(x\right)=n_{i}\exp\frac{q\left[\phi_{i}\left(x\right)-\phi_{F}\right]}{kT}
\]

正孔濃度
\[
p\left(x\right)=n_{i}\exp\frac{q\left[\phi_{F}-\phi_{i}\left(x\right)\right]}{kT}
\]

$\phi_{F}$は$x>W$のP型Siのフェルミ準位

$\phi_{i}\left(x\right)$は$x$の位置の$\phi_{i}$

半導体表面($x=0$)の電子濃度$n_{n}$
\[
n_{s}=n\left(0\right)=n_{i}\exp\frac{q\left[\phi_{i}\left(x=0\right)-\phi_{F}\right]}{kT}
\]

P型半導体の基盤の正孔濃度$P_{\mathrm{PO}}$
\[
P_{\mathrm{PO}}=P\left(x>W\right)=n_{i}\exp\frac{q\left[\phi_{F}-\phi_{i}\left(x=0\right)\right]}{kT}
\]
となる。

ここで、
\[
\phi_{F}-\phi_{i}\left(x>W\right)=\phi\left(x=0\right)-\phi_{F}=\phi_{FP}
\]
のとき、
\begin{align*}
n_{s} & =n_{i}\exp\frac{q\phi_{FP}}{kT}\\
p_{P0} & =n_{i}\exp\frac{q\phi_{FP}}{kT}
\end{align*}
となり両者が等しくなる。

\textbf{もともとP型半導体の表面(界面)がN型半導体のようになった瞬間!! →反転}

\[
2\phi_{F}=\phi_{i}\left(x=0\right)-\phi_{i}\left(x>W\right)=\phi_{S}
\]

$2\phi_{F}=\phi_{S}$のときに反転する。
\begin{itemize}
\item 電子がたまり始めてから$2\phi_{F}$まで: 弱反転
\item それ以上: 強反転
\end{itemize}
ひとたび\uline{反転すると}、$n_{S}$は指数関数的に増加。

→\uline{それ以上、空乏層は広がらない。}

→$Q=C_{\mathrm{OX}}\left(V_{G}-V_{TH}\right)$ ($V_{TH}$: しきい値電圧)

\section*{第11回}

\paragraph{反転(復習と訂正)}

図電デ11-1

\paragraph{深さ方向のキャリア密度}

電子密度 $n\left(x\right)=n_{i}\exp\frac{q\left[\phi_{i}\left(x\right)-\phi_{F}\right]}{kT}$

正孔密度 $p\left(x\right)=n_{i}\exp\frac{q\left[\phi_{F}-\phi_{i}\left(x\right)\right]}{kT}$

表面の電子密度 $n\left(x=0\right)=n_{i}\exp\frac{q\left[\phi_{i}\left(x=0\right)-\phi_{F}\right]}{kT}$

P型半導体の電子密度 $p\left(x>W\right)=n_{i}\exp\frac{q\left[\phi_{F}-\phi_{i}\left(x>W\right)\right]}{kT}$ 

\[
\phi_{S}=\phi_{i}\left(x=0\right)-\phi_{i}\left(x>W\right)=2\phi_{FP}
\]

図電デ11-2

図電デ11-3

図電デ11-4

\paragraph{MOSFETのIV特性を求めてみよう}

図電デ11-5

MOS構造はすべて反転しており、電荷からの電気力線は全てゲートで終端している。その後でドレインソース間の電圧を考慮する。これをグラ樹あるチャネル近似という。

(1) チャネルの位置$x$における(単位長さあたりの)電荷密度$Q\left(x\right)$は?

(答) 
\[
Q\left(x\right)=C_{OX}W\left(V_{GS}-\phi\left(x\right)-V_{TH}\right)
\]

ただし、$C_{OX}$はゲート酸化膜容量、$V_{TH}$はしきい値電圧(チャネルを形成するまでに必要な電圧)

(2) (ソース)ドレイン電流$I_{d}$を$\phi\left(x\right)$で表わせ。

(答) 
\[
I_{D}=-vSnq=-\mu\mu EQ\left(x\right)=\mu Q\left(x\right)\d{\phi\left(x\right)}x
\]

$\mu$: 移動度

\[
I_{D}=-\mu C_{OX}W\left(V_{GS}-V_{TH}-\phi\left(x\right)\right)\d{\phi\left(x\right)}x
\]

(3) (2)の式を0から$L$まで積分して$I_{D}$を求めてみよう。

$x:0\rightarrow L,\phi\left(x\right):0\rightarrow V_{DS}$にth木。さらに、$I_{D}$は一定。

\[
\int_{0}^{L}I_{D}\mathrm{d}x=\int_{0}^{V_{DS}}\mu C_{OX}\left(V_{GS}-V_{TH}-\phi\left(x\right)\right)\mathrm{d}\phi
\]

\[
I_{D}\cdot L=\mu C_{OX}\left\{ \left(V_{GS}-V_{TH}\right)V_{DS}-\frac{1}{2}V_{DS}^{2}\right\} 
\]

\[
I_{D}=\mu C_{OX}\frac{W}{L}\left\{ \left(V_{GS}-V_{TH}\right)V_{DS}-\frac{1}{2}V_{DS}^{2}\right\} 
\]

ここで$\left|V_{DS}\right|\ll\left|V_{GS}-V_{TH}\right|$の場合、
\[
I_{D}=\mu C_{OX}\frac{W}{L}\left(V_{GS}-V_{TH}\right)V_{DS}
\]

図電デ11-6

\[
I_{D}=\mu C_{OX}\frac{W}{L}\left\{ \left(V_{GS}-V_{TH}\right)V_{DS}-\frac{1}{2}V_{DS}^{2}\right\} 
\]

←放物線

この式が$V_{DS}=V_{GS}-V_{TH}$で最大になる。

図電デ11-7

$V_{DS}>V_{GS}-V_{TH}$で$I_{D}$は一定

\[
I_{D}=\frac{1}{2}C_{OX}\mu\frac{W}{L}\left(V_{GS}-V_{TH}\right)^{2}
\]

ピンチオフ点ではなにが起こっているのか?

\[
Q\left(x\right)=C_{OX}W\left(V_{GS}-V_{TH}-\phi\left(x\right)\right)
\]

$x=L\rightarrow\phi\left(x=L\right)=V_{DS}=V_{GS}-V_{TH}$←ドレイン端

\[
Q\left(x=L\right)=C_{OX}W\left(V_{HS}-V_{TH}-\left(V_{HS}-V_{TH}\right)\right)=0
\]

$x=L$で反転電子がなくなる

→ドレイン端でチャネルが消失!!

図電デ11-8
\end{document}
