%% LyX 2.0.0 created this file.  For more info, see http://www.lyx.org/.
%% Do not edit unless you really know what you are doing.
\documentclass[english]{article}
\usepackage[T1]{fontenc}
\setlength{\textwidth}{17cm}
\setlength{\textheight}{24cm}
\setlength{\leftmargin}{-1cm}
\setlength{\topmargin}{-2cm}
\setlength{\oddsidemargin}{0cm}
\setlength{\evensidemargin}{0cm}
\setlength{\parindent}{0pt}
\setlength{\parskip}{10pt}

\usepackage{xltxtra}
\setmainfont{RyuminPr5-Light}
\setsansfont{IPAPGothic}
\setmonofont{IPAGothic}
\XeTeXlinebreaklocale "ja"

\usepackage{textcomp}
\usepackage{amsmath}
\usepackage{amssymb}

\makeatletter

%%%%%%%%%%%%%%%%%%%%%%%%%%%%%% LyX specific LaTeX commands.
\DeclareRobustCommand{\greektext}{%
  \fontencoding{LGR}\selectfont\def\encodingdefault{LGR}}
\DeclareRobustCommand{\textgreek}[1]{\leavevmode{\greektext #1}}
\DeclareFontEncoding{LGR}{}{}
\DeclareTextSymbol{\~}{LGR}{126}

\makeatother

\usepackage{babel}
\usepackage{xunicode}
\begin{document}

\title{数学II 講義ノート}


\author{寺杣 友秀}


\date{2013夏学期}
\maketitle
\begin{abstract}
授業科目名: 数学Ⅱ①

時間割コード: 10520

曜限: 火3

開講区分: 夏学期

単位数: 2

教室: 531教室

対象クラス: 1年 理一(25-27)

科目区分: 基礎科目 数理科学

入力: 博多市
\end{abstract}
この講義で扱う内容: 線形代数
\begin{itemize}
\item ベクトル空間
\item 線形写像
\end{itemize}
教科書: http://gauss.ms.u-tokyo.ac.jp/


\subsection{講義(期末テスト)と演習(小テスト)}

一回目 一回目の内容(解答)

二回目 休み

三回目 一回目のテスト+2,3回目の内容(解答)


\subsection{今後の予定}


\subsubsection{前期}

一回目 平面、空間ベクトル・複素数

二回目 行列

三回目 線形写像

四回目 一次方程式の解法

五回目 行列式


\subsubsection{後期}

抽象的なベクトル空間

線形写像

固有値、固有空間、円積空間


\section{平面ベクトル、空間ベクトル、複素数}


\subsection{集合論の復習}

集合=ものの集まり

集合論はモノの集まりを記述する言葉を用いる

E.g. 1以上3以下の自然数 → $\left\{ 1,2,3\right\} $,$\left\{ x\text{は自然数}|1\text{≦}x\text{≦}3\right\} $

$\left\{ \right\} $: 集合を表すときに使う括弧

$\left\{ \text{集合の元}|\text{条件}\right\} $と表す


\subsubsection{よく使う記号}

$\mathbb{N}$: 自然数全体の集合 = $\left\{ 1,2,3\cdots\right\} $

$\mathbb{Z}$: 整数全体の集合 = $\left\{ 0,\pm1,\pm2\cdots\right\} $

$\mathbb{R}$: 実数全体の集合

$\mathbb{C}$: 複素数全体の集合 = $\left\{ x+yi|x,y\text{は実数、}i\text{は虚数単位}\right\} $


\subsubsection{属する}

$A$: 集合

$x$が$A$に属しているものとする=$x$は$A$の元である

→記号$x\in A$と書く

そうでないとき$x\notin A$と書く

e.g. $A=\left\{ 1,2,3\right\} $

$1\in A,4\notin A$


\subsubsection{包含関係}

$A=\left\{ 1,2,3\right\} ,B=\left\{ 2,3\right\} ,C=\left\{ 3,4\right\} $

この時、$B$は$A$に含まれる→$B\subset A$と書く

正確な定義: 「$B$が$A$に含まれる」ということは「$x\in B$ ならば$x\in A$」

そうでない時、$C$は$A$に含まれない→$C\not\subset A$と書く 

「$B$は$A$に入っている」などの曖昧な言葉は使わない


\subsubsection{ならば}

$A\Rightarrow B$: 「$A$ならば$B$」

「$A$が成立していれば$B$が成立する」

$A$が成立していない時は?

「$A$ならば$B$」は「$A$であって$B$でない」ことはないということ

$x$は実数とする。

「$x^{2}+1=0$ならば$x=3$である」は正しい命題

「$x^{2}+1=0$ならば$x\neq3$である」も正しい命題


\subsubsection{同値}

$A\Leftrightarrow B$: 「$A$ならば$B$」かつ「$B$ならば$A$」


\subsection{平面ベクトル、空間ベクトル}


\subsubsection{平面ベクトルの定義}

始点と終点が定まっているもの

記号$\vec{v}$、始点$P$、終点$Q$、記号$\vec{v}'$、始点$P'$、終点$Q'$とする。

$P$を$P'$に移し、$Q$を$Q'$に移す平行移動がある時は同一視する。

始点$P$、終点$Q$のベクトルは$\vec{PQ}$と表す。


\subsubsection{成分表示}

平面の座標を取り、$\vec{v}=\vec{PQ}$に対して始点が原点になるように平行移動する。その時の終点の座標を$\left(x_{1},y_{1}\right)$とすると、2つの実数の組$\left(x_{1},y_{1}\right)$が定まる。この$\left(x_{1},y_{1}\right)$を$\vec{v}$の成分表示という。

この対応で、「平面ベクトル全体の集合」と「2つの実数の組の2つの集合」は一対一に対応する。


\subsubsection{加法}

$\vec{v},\vec{w}$を2つの平面ベクトルとする。

$\vec{w}$の始点を$\vec{v}$の終点と一致させる。→$\vec{v}$の始点を始点として、$\vec{w}$の終点を終点とするベクトルが定まる。これを$\vec{v}+\vec{w}$と書く。

これにより、ベクトルの加法が定まる。


\subsubsection{スカラー倍}

$\vec{v}$をベクトル、$r$を実数とする。
\begin{enumerate}
\item $r>0$のとき、$r\vec{v}$は$\vec{v}$と同じ方向で長さ$r$倍のベクトル
\item $r=0$のとき、始点=終点のベクトルとして$r\vec{v}$を定める。
\item $r<0$のとき、$r\vec{v}$は$\vec{v}$と逆方向で長さ$\left|r\right|$倍のベクトル
\end{enumerate}

\subsubsection{成分表示の関係}


\paragraph{加法}

$\vec{v_{1}}=\left(x_{1},y_{1}\right),\vec{v_{2}}=\left(x_{2},y_{2}\right)$

\[
\vec{v_{1}}+\vec{v_{2}}=\left(x_{1}+x_{2},y_{1}+y_{2}\right)
\]



\paragraph{スカラー倍}

$r\in\mathbb{R},\vec{v_{1}}=\left(x_{1},y_{1}\right)$とする。

\[
r\vec{v_{1}}=\left(rx_{1},ry_{1}\right)
\]



\subsubsection{法則}

$\vec{v_{1}},\vec{v_{2}},\vec{v_{3}}$をベクトル、$r_{1},r_{2}\in\mathbb{R}$とする。
\begin{enumerate}
\item $\vec{v_{1}}+\vec{v_{2}}=\vec{v_{2}}+\vec{v_{1}}$
\item $\left(\vec{v_{1}}+\vec{v_{2}}\right)+\vec{v_{3}}=\vec{v_{1}}+\left(\vec{v_{2}}+\vec{v_{3}}\right)$
\item $\vec{0}+\vec{v_{1}}=\vec{v_{1}}$
\item $\left(-1\right)\cdot\vec{v_{1}}=-\vec{v_{1}}$とおくと、$\vec{v_{1}}+\left(-\vec{v_{1}}\right)=\vec{0}$
\item $1\cdot\vec{v_{1}}=\vec{v_{1}}$
\item $\left(r_{1}r_{2}\right)\vec{v_{1}}=r_{1}\left(r_{2}\vec{v_{1}}\right)$
\item $\left(r_{1}+r_{2}\right)\vec{v_{1}}=r_{1}\vec{v_{1}}+r_{2}\vec{v_{1}}$、$r_{1}\left(\vec{v_{1}}+\vec{v_{2}}\right)=r_{1}\vec{v_{1}}+r_{1}\vec{v_{2}}$
\end{enumerate}

\subsubsection{空間ベクトル}

空間ベクトルについても同様にできる。


\subsubsection{始点、終点、成分表示}

$\vec{v}$を空間ベクトルとし、座標を固定する。

$\vec{v}$の始点を原点→終点の座標$\left(x_{1},y_{1},z_{1}\right)$となるとき、$\left(x_{1},y_{1},z_{1}\right)$を$\vec{v}$の成分表示という。

他に、加法、スカラー倍も求まる。→1.~7.成立


\subsection{ベクトルの内積}


\subsubsection{定義}

$\vec{v},\vec{w}$を平面ベクトルとする。

$\vec{v}\cdot\vec{w}$あるいは$\left(\vec{v},\vec{w}\right)$を次のように定める。

$\left\Vert \vec{v}\right\Vert =\vec{v}\text{の長さ}$

$\theta=\text{2つのベクトルのなす角}$

\[
\left(\vec{v},\vec{w}\right)=\left\Vert \vec{v}\right\Vert \left\Vert \vec{w}\right\Vert \cos\theta
\]


ただし$\vec{v},\vec{w}$のいずれかが$\vec{0}$の場合$\left(\vec{v},\vec{w}\right)=\vec{0}$と定める。


\subsubsection{命題1.9}

$\vec{v_{1}}=\left(x_{1},y_{1}\right),\vec{v_{2}}=\left(x_{2},y_{2}\right)$とおく。

このとき、
\[
\left(\vec{v_{1}},\vec{v_{2}}\right)=x_{1}x_{2}+y_{1}y_{2}
\]



\subsubsection{証明}

$\left\Vert \vec{v_{1}}\right\Vert \neq0,\left\Vert \vec{v_{2}}\right\Vert \neq0$とする。

$\theta=\text{2つのベクトルのなす角}$

第二余弦定理より

\[
\cos\theta=\frac{\left\Vert \vec{v_{1}}\right\Vert ^{2}+\left\Vert \vec{v_{2}}\right\Vert ^{2}-\left\Vert \vec{v_{1}}-\vec{v_{2}}\right\Vert ^{2}}{2\left\Vert \vec{v_{1}}\right\Vert \left\Vert \vec{v_{2}}\right\Vert }
\]


$\vec{v_{1}}=\left(x_{1},y_{1}\right),\vec{v_{2}}=\left(x_{2},y_{2}\right),\vec{v_{1}}-\vec{v_{2}}=\left(x_{1}-x_{2},y_{1}-y_{2}\right)$より、

\begin{eqnarray*}
\cos\theta & = & \frac{\left\Vert \vec{v_{1}}\right\Vert ^{2}+\left\Vert \vec{v_{2}}\right\Vert ^{2}-\left\Vert \vec{v_{1}}-\vec{v_{2}}\right\Vert ^{2}}{2\left\Vert \vec{v_{1}}\right\Vert \left\Vert \vec{v_{2}}\right\Vert }\\
2\left\Vert \vec{v_{1}}\right\Vert \left\Vert \vec{v_{2}}\right\Vert \cos\theta & = & \left(x_{1}^{2}+y_{1}^{2}\right)+\left(x_{2}^{2}+y_{2}^{2}\right)-\left(\left(x_{1}^{2}-x_{2}^{2}\right)+\left(y_{1}^{2}-y_{2}^{2}\right)\right)\\
 & = & 2\left(x_{1}x_{2}+y_{1}y_{2}\right)
\end{eqnarray*}


\begin{eqnarray*}
\left(\vec{v_{1}},\vec{v_{2}}\right) & = & \left\Vert \vec{v_{1}}\right\Vert \left\Vert \vec{v_{2}}\right\Vert \cos\theta\\
 & = & x_{1}x_{2}+y_{1}y_{2}
\end{eqnarray*}


空間ベクトルの場合も同様に$\vec{v_{1}}=\left(x_{1},y_{1},z_{1}\right),\vec{v_{2}}=\left(x_{2},y_{2},z_{3}\right)$とおくと、
\[
\left(\vec{v_{1}},\vec{v_{2}}\right)=x_{1}x_{2}+y_{1}y_{2}+z_{1}z_{2}
\]


となる。


\subsection{複素数}


\subsubsection{極形式、ド・モアブルの定理}

$x,y\in\mathbb{R}$のとき、$x+y\mathrm{i}$の形のものを複素数という。(規則、$\mathrm{i}^{2}=-1$)(結合法則、分配法則)

この時、$z=x+y\mathrm{i}$に対して複素共役$\overline{z}=x-y\mathrm{i}$と定めた。(xはzの実部$\mathrm{Re}\left(z\right)$、yはzの虚部$\mathrm{Im}\left(z\right)$と書く)


\subsubsection{性質}

\[
z\cdot\overline{z}=\sqrt{z\cdot\overline{z}}
\]
として定める。これをzの絶対値という。

\[
z\left(\dfrac{\overline{z}}{|z|^{2}}\right)=1
\]
より$z\neq0$なら逆数がある。


\subsubsection{複素平面}

$\mathbb{C}$=複素数全体の集合

$z=x+y\mathrm{i}\in\mathbb{C}$に対して座標平面上の一点$\left(x,y\right)$を対応させる。

「複素数全体の集合⇔平面内の点の集合」で一対一で対応する。

x-y座標平面における$x$軸を実軸と呼び、$y$軸を虚軸と呼ぶ。

$z$は原点からの距離と実軸の正の部分のなす角$\theta$で表せる($z\neq0$とする)。

\[
r=\sqrt{x^{2}+y^{2}}+\sqrt{z\cdot\overline{z}}=|z|
\]


$\theta=z\text{の偏角}$(角は弧度法で実軸から反時計回りを正の方向とする。)

このとき、

\[
x=r\cos\theta
\]


\[
y=r\sin\theta
\]
なので、
\[
z=r\left(\cos\theta+i\sin\theta\right)
\]
となる。

$r$と$\theta$を用いて$z$を表した形を極形式という。

⇒極形式は、乗法に関する振る舞いがよい

\begin{eqnarray*}
z_{1} & = & x_{1}+y_{1}\mathrm{i}\\
 & = & r_{1}\left(\cos\theta_{1}+\mathrm{i}\sin\theta_{1}\right)
\end{eqnarray*}


\begin{eqnarray*}
z_{2} & = & x_{2}+y_{2}\mathrm{i}\\
 & = & r_{2}\left(\cos\theta_{2}+\mathrm{i}\sin\theta_{2}\right)
\end{eqnarray*}


と置くと、

\begin{eqnarray*}
z_{1}z_{2} & = & r_{1}r_{2}\left(\cos\theta_{1}+\mathrm{i}\sin\theta_{1}\right)\left(\cos\theta_{2}+\mathrm{i}\sin\theta_{2}\right)\\
 & = & r_{1}r_{2}\left\{ \left(\cos\theta\cos\theta_{2}-\sin\theta_{1}\sin\theta_{2}\right)+\left(\cos\theta_{1}\sin\theta_{2}+\sin\theta_{1}\cos\theta_{2}\right)\right\} \\
 & = & r_{1}r_{2}\left(\cos\left(\theta_{1}+\theta_{2}\right)+\sin\left(\theta_{1}+\theta_{2}\right)\mathrm{i}\right)\\
 & = & r_{1}r_{2}\left(\cos\left(\theta_{1}+\theta_{2}\right)+\mathrm{i}\sin\left(\theta_{1}+\theta_{2}\right)\right)
\end{eqnarray*}



\subsubsection{記号}

zの偏角を$\mathrm{ang}\left(z\right)$と書く($\mathrm{ang}\left(z\right)$は2πの整数倍の差を除いて定まる)

従って$z,\omega\in\mathbb{C}\Rightarrow\mathrm{ang}\left(z\right)+\mathrm{ang}\left(\omega\right)$も2πの整数倍を除いて定まる。

命題 1.20

(1) $\left|z_{1}z_{2}\right|=\left|z_{1}\right|\left|z_{2}\right|$

(2) $\mathrm{ang}\left(z_{1}z_{2}\right)=\mathrm{ang}\left(z_{1}\right)+\mathrm{ang}\left(z_{2}\right)$(ただし2πの整数倍の差は除く)


\subsubsection{系}

$n\geqq1$の整数として、

$z=r\left(\cos\theta+\mathrm{i}\sin\theta\right)$とすると、

\[
z^{n}=\overbrace{z\cdots z}^{n}=r^{n}\left(\cos\left(n\theta\right)+\mathrm{i}\sin\left(n\theta\right)\right)
\]
(ド・モアブルの定理)


\subsubsection{演習問題}

(1)$z\neq0,n>0,n\in\mathbb{N}$

\[
z^{-n}=\dfrac{1}{\underbrace{z\cdot z\cdots z}_{n}}\in\mathbb{C}
\]


と定めると、$-n\left(n>0\right)$の時もド・モアブルの定理が成立することを示せ。

(2) $n>1,n\in\mathbb{N}$

$z^{n}=1$となる$z\in\mathbb{C}$をすべて求めよ。


\subsubsection{オイラーの記法}

$\omega=a+b\mathrm{i}$

\[
e=\text{自然対数の底}=2.71828\cdots
\]


(ネーピアの定数)

$\mathrm{e}^{\omega}=\mathrm{e}^{a+b\mathrm{i}}$を
\[
\mathrm{e}^{a}\left(\cos b+\mathrm{i}\sin b\right)
\]
と定める。

\[
\omega_{1}=a_{1}+b_{1}\mathrm{i}
\]
\[
\omega_{2}=a_{2}+b_{2}\mathrm{i}
\]
とすると、

\begin{eqnarray*}
\mathrm{e}^{\omega_{1}}\mathrm{e}^{\omega_{2}} & = & \mathrm{e}^{a_{1}}\left(\cos b_{1}+\mathrm{i}\sin b_{1}\right)\mathrm{e}^{a_{2}}\left(\cos b_{2}+\mathrm{i}\sin b_{2}\right)\\
 & = & \mathrm{e}^{a_{1}+a_{2}}\left(\cos\left(b_{1}+b_{2}\right)+\mathrm{i}\sin\left(b_{1}+b_{2}\right)\right)\\
 & = & \mathrm{e}^{a_{1}+a_{2}+\mathrm{i}\left(b_{1}+b_{2}\right)}\\
 & = & \mathrm{e}^{\omega_{1}+\omega_{2}}
\end{eqnarray*}


\[
\mathrm{e}^{\omega_{1}}\mathrm{e}^{\omega_{2}}=\mathrm{e}^{\omega_{1}\omega_{2}}
\]
指数定理は複素数まで拡張できる。

これは自然な拡張ということができる。


\subsubsection{問題}

$z\in\mathbb{C}$に対して、

\[
\sin\left(z\right)=\dfrac{\mathrm{e}^{\mathrm{i}z}-\mathrm{e}^{-\mathrm{i}z}}{2i}
\]


\[
\cos\left(z\right)=\dfrac{\mathrm{e}^{\mathrm{i}z}+\mathrm{e}^{-\mathrm{i}z}}{2}
\]


と定めると$z\in\mathbb{R}$の時は今までの三角関数と一致することを示せ。


\section{行列}

\[
m,n\geqq1,m,n\in\mathbb{R}
\]


数が縦に$m$個、横に$n$個で長方形の形に並んでいるものを$\left(m,n\right)$行列という。

\[
\left(\begin{array}{ccccc}
a_{11} & a_{12} & a_{13} & \cdots & a_{1n}\\
a_{21} & a_{22} & a_{23} & \cdots & a_{2n}\\
\vdots & \vdots & \vdots & \ddots & \vdots\\
a_{m1} & a_{m2} & a_{m3} & \cdots & a_{mn}
\end{array}\right)
\]


$i$行目$\left(1\leqq i\leqq m\right)$$j$列目$\left(1\leqq j\leqq n\right)$を$\left(i,j\right)$成分という。


\subsection{行列の定め方}
\begin{itemize}
\item すべての要素を与える
\item $\left(i,j\right)$に来る数を規則によって与える\\
例 $\left(i,j\right)$成分に$\left(i+j\right)$が並んでいる$\left(2\times3\right)$行列\\
$\left(\begin{array}{ccc}
2 & 3 & 4\\
3 & 4 & 5
\end{array}\right)$
\end{itemize}

\subsection{記法}

$a_{ij}$が$\left(i,j\right)$成分に並んでいる行列を$\left(a_{ij}\right)_{ij}$と書く。
(例 $\left(i+j\right)_{ij}$)

\[
\left(a_{ij}\right)_{ij}=\left(\begin{array}{cccc}
a_{11} & a_{12} & a_{13} & \cdots\\
a_{21} & a_{22} & a_{23} & \cdots\\
\vdots & \vdots & \vdots & \ddots
\end{array}\right)
\]
なので、時にi,jを略して$\left(a_{ij}\right)$と書く


\subsection{色々な行列と呼称}

(1) $\left(\begin{array}{ccc}
0 & \cdots & 0\\
\vdots & \ddots & \vdots\\
0 & \cdots & 0
\end{array}\right)$を0行列という。記号$\mathbb{O}$と書く。

(2) $m=n$の時の$\left(m,n\right)$行列を$n$次正方行列という。

このとき$i=j$である$\left(i,j\right)$成分を対角成分という。

(3) 対角成分以外がすべて0である正方行列を対角行列という。

更に対角成分が1の行列を単位行列という。

$n$次の単位行列を$\mathbb{E}_{n}$と書く。

例 4次の単位行列$\mathbb{E}_{4}=\left(\begin{array}{cccc}
1 & 0 & 0 & 0\\
0 & 1 & 0 & 0\\
0 & 0 & 1 & 0\\
0 & 0 & 0 & 1
\end{array}\right)$

(4) 対角線の上側/下側にある成分が全て0の行列

$\left(a_{ij}\right)_{ij}$について$a_{ij}=0$$\left(i>j\right)$であるものを上半三角行列という。

$\left(a_{ij}\right)_{ij}$について$a_{ij}=0$$\left(i<j\right)$であるものを下半三角行列という。

例 3×3の上半三角行列$\left(\begin{array}{ccc}
a_{11} & a_{12} & a_{13}\\
0 & a_{22} & a_{23}\\
0 & 0 & a_{33}
\end{array}\right)$

(5) $m=1$のとき(1行)、行ベクトルという。

\[
\left(\begin{array}{cccc}
a_{11} & a_{12} & \cdots & a_{1n}\end{array}\right)\:(n\text{次元})
\]


$n=1$のとき(1列)、列ベクトルという。

\[
\left(\begin{array}{c}
a_{11}\\
a_{21}\\
\vdots\\
a_{m1}
\end{array}\right)\:(m\text{次元})
\]



\subsection{行列の和、スカラー倍}

$A,B$を$\left(m\times n\right)$行列とする。

$A$の$\left(i,j\right)$成分を$a_{ij}$とすると
\[
A=\left(a_{ij}\right)_{ij}
\]


以下このように表記する

$B=\left(b_{ij}\right)_{ij}$とする。


\subsubsection{和}

$A$と$B$の和$A+B$を
\[
A+B=\left(a_{ij}+b_{ij}\right)_{ij}
\]
で定める。(つまり$\left(i,j\right)$成分が$a_{ij}+b_{ij}$で与えられている)


\subsubsection{スカラー倍}

$r\in\mathbb{R}$

以下、数=実数(スカラー)とする。理論的には複素数、有理数などでも可である

$rA=\left(ra_{ij}\right)_{ij}$として定める。($\left(i,j\right)$成分に$ra_{ij}$を与えたもの)

例 $A=\left(\begin{array}{cc}
1 & 2\\
3 & 4
\end{array}\right)$ $B=\left(\begin{array}{cc}
2 & 3\\
4 & 1
\end{array}\right)$

\begin{eqnarray*}
3A+2B & = & \left(\begin{array}{cc}
3\cdot1 & 3\cdot2\\
3\cdot3 & 3\cdot4
\end{array}\right)+\left(\begin{array}{cc}
4 & 6\\
8 & 2
\end{array}\right)\\
 & = & \left(\begin{array}{cc}
7 & 12\\
17 & 14
\end{array}\right)
\end{eqnarray*}



\subsection{命題}
\begin{enumerate}
\item $A+\mathbb{O}=A$
\item $A+B=B+A$ ($A,B$は$\left(m\times n\right)$行列)
\[
\left(A+B\right)+C=A+\left(B+C\right)
\]

\item $1\cdot A=A$\\
$r,s\in\mathbb{R}$について
\[
\left(rs\right)A=r\left(sA\right)
\]

\item $r,s\in\mathbb{R}$\\
$A,B$を$\left(m\times n\right)$行列とすると
\[
\left(r+s\right)A=rA+sA
\]
\[
r\left(A+B\right)=rA+rB
\]

\end{enumerate}
これは各成分ごとに見れば実数について成り立つ性質から導かれる。


\section{--}


\subsection{--}


\subsection{行列の和・積、スカラー倍}

$A,B$を$\left(m\times n\right)$行列とする。

$A=\left(a_{ij}\right)_{ij},B=\left(b_{ij}\right)_{ij}$

つまり

\[
A=\left(\begin{array}{cccc}
a_{11} & a_{12} & \cdots & a_{1n}\\
\vdots & \vdots & \ddots & \vdots\\
a_{m1} & a_{m2} & \cdots & a_{mn}
\end{array}\right)
\]



\paragraph{和}

\[
A+B=\left(a_{ij}+b_{ij}\right)_{ij}\:\left(\left(i,j\right)\text{成分が}a_{ij}+b_{ij}\text{で与える}\right)
\]
で定める


\paragraph{スカラー倍}

$r\in\mathbb{R}\text{に対して}rA=\left(ra_{ij}\right)_{ij}\text{で定める}$


\subsubsection{行列の積}

$A:\left(m\times n\right)\text{行列}$

$B:\left(n\times l\right)\text{行列}$

のとき、積$AB$が定義される($A$の列数=$B$の行数)

$AB$は$\left(m\times l\right)$行列である。

\[
A=\left(a_{ij}\right)_{ij},B=\left(b_{jk}\right)_{jk}
\]


$AB\text{の}\left(i,k\right)\text{成分を}C_{ik}\text{とおくと、}$

\[
C_{ik}=\sum_{j=1}^{n}a_{ij}b_{jk}
\]


で定める…(1)


\subsubsection{例}

\[
A=\left(\begin{array}{cc}
a_{11} & a_{12}\\
a_{21} & a_{22}
\end{array}\right)
\]


\[
B=\left(\begin{array}{cc}
b_{11} & b_{12}\\
b_{21} & b_{22}
\end{array}\right)
\]


(1) 定義に従って

\[
c_{11}=\sum_{j=1}^{2}a_{1j}b_{j1}=a_{11}b_{11}+a_{12}b_{21}
\]


\[
c_{12}=\sum_{j=1}^{2}a_{1j}b_{j2}=a_{11}b_{12}+a_{12}b_{22}
\]


図形的に

\[
\left(\begin{array}{cc}
a_{11} & a_{12}\\
a_{21} & a_{22}
\end{array}\right)\left(\begin{array}{cc}
b_{11} & b_{12}\\
b_{21} & b_{22}
\end{array}\right)=\left(\begin{array}{cc}
a_{11}b_{11}+a_{12}b_{21} & a_{11}b_{12}+a_{12}b_{22}\\
a_{21}b_{11}+a_{22}b_{21} & a_{21}b_{12}+a_{22}b_{22}
\end{array}\right)
\]



\subsubsection{命題}

$A\text{を}\left(m\times n\right)\text{行列とする}$

\[
E_{m}A=A
\]


\[
AE_{n}=A
\]



\subsubsection{証明}

クロネッカーのデルタ記号$\delta_{ij}$

\[
\delta_{ij}=\begin{cases}
1 & \left(i=j\right)\\
0 & \left(i\neq j\right)
\end{cases}
\]


を用いると、$E_{m}=\left(\delta_{ij}\right)_{ij}$と書ける

$A=\left(a_{ij}\right)_{ij}=\left(a_{jk}\right)_{jk}$とおく。

$E_{m}A\text{の}\left(j,k\right)\text{成分を求める。}$

\[
\left(E_{m}A\right)_{ik}=\sum_{j=1}^{m}\delta_{ij}a_{jk}
\]


この和においてjがiでないところは寄与しない。

\begin{eqnarray*}
\left(E_{m}A\right)_{ik} & = & \delta_{ij}a_{ik}\\
 & = & a_{ik}\\
 & = & \left(A\right)_{ik}
\end{eqnarray*}


\[
\therefore E_{m}A=A
\]


(つまり$E_{m}$は左からの乗法に対する単位元)

$A\cdot E_{m}=A\text{も同様に示せる。}Q.E.D$


\subsubsection{行列の積に関する結合法則}

$A\left(m\times n\right)\text{行列、}B\left(n\times p\right)\text{行列、}C\left(p\times q\right)\text{行列とする}$

$AB\text{は}\left(m\times q\right)\text{行列、}BC\text{は}\left(n\times q\right)\text{行列}\text{となる。このとき、}$

\[
\left(AB\right)C=A\left(BC\right)\:(\text{ともに}\left(m\times q\right)\text{行列})
\]



\subsubsection{定理}

A,B,Cを上記のサイズの行列とすると

\[
\left(AB\right)C=A\left(BC\right)
\]


が成立する。


\subsubsection{証明}

\[
A=\left(a_{ij}\right)_{ij}
\]


\[
B=\left(b_{js}\right)_{js}
\]


\[
C=\left(c_{st}\right)_{st}
\]


$\left(1\leqq i\leqq m,\;1\leqq j\leqq n,\;1\leqq s\leqq p,\;1\leqq t\leqq q\right)$

(1) $\left(AB\right)C$を求める。

\[
\left(AB\right)_{is}=\sum_{j=1}^{n}a_{ij}b_{js}
\]


\begin{eqnarray*}
\left(\left(AB\right)C\right)_{it} & = & \sum_{s=1}^{p}\left(AB\right)_{is}c_{st}\\
 & = & \sum_{s=1}^{p}\left(\sum_{j=1}^{n}a_{ij}b_{js}\right)c_{st}\\
 & = & \sum_{s=1}^{p}\left(\sum_{j=1}^{n}a_{ij}b_{js}c_{st}\right)\\
 & = & \sum_{1\leqq s\leqq p,1\leqq j\leqq n}\left(a_{ij}b_{js}c_{st}\right)
\end{eqnarray*}
…(2)

(2) $a\left(BC\right)\text{を求める}\text{。}$

\[
\left(BC\right)_{jt}=\sum_{s=1}^{p}b_{js}c_{st}
\]


\begin{eqnarray*}
\left(A\left(BC\right)\right)_{it} & = & \sum_{j=1}^{n}a_{ij}\left(BC\right)_{jt}\\
 & = & \sum_{j=1}^{n}a_{ij}\left(\sum_{s=1}^{p}b_{js}c_{st}\right)\\
 & = & \sum_{j=1}^{n}\left(\sum_{j=1}^{p}a_{ij}b_{js}c_{st}\right)\\
 & = & \sum_{1\leqq s\leqq p,1\leqq j\leqq n}\left(a_{ij}b_{js}c_{st}\right)
\end{eqnarray*}


…(2)


\subsubsection{命題}

$A_{1},A_{2}\text{を}\left(m\times n\right)\text{行列}\text{、}B_{1},B_{2}\text{を}\left(n\times l\right)\text{行列}$とする。この時、

\[
\left(A_{1}+A_{2}\right)B_{1}=A_{1}B_{1}+A_{2}B_{1}
\]


\[
A_{1}\left(B_{1}+B_{2}\right)=A_{1}B_{1}+A_{1}B_{2}
\]



\subsubsection{証明}

\[
A_{1}=\left(a_{ij}\right)_{ij},A_{2}=\left(a_{ij}'\right)_{ij},B_{1}=\left(b_{jk}\right)_{jk}
\]
とおく。

\[
\left(A_{1}+A_{2}\right)_{ij}=\left(a_{ij}+a_{ij}'\right)_{ij}
\]


\begin{eqnarray*}
\therefore\left(\left(A_{1}+A_{2}\right)B\right)_{ik} & = & \sum_{j=1}^{n}\left(a_{ij}+a_{ij}'\right)b_{jk}\\
 & = & \sum_{j=1}^{n}a_{ij}b_{jk}+\sum_{j=1}^{n}a_{ij}'b_{jk}\\
 & = & \left(A_{1}B_{1}\right)_{ik}+\left(A_{2}B_{1}\right)_{ik}\\
 & = & \left(A_{1}B_{1}+A_{2}B_{1}\right)_{ik}
\end{eqnarray*}


\[
\therefore\left(A_{1}+A_{2}\right)B_{1}=A_{1}B_{1}+A_{2}B_{1}
\]


Q.E.D.


\subsubsection{転置行列}

$A\text{を}\left(m\times n\right)\text{行列とする}$

行と列を入れ替えた行列を転置行列という。

\[
A=\left(\begin{array}{ccc}
1 & 3 & 5\\
2 & 4 & 6
\end{array}\right)
\]
の転置は
\[
^{t}A=\left(\begin{array}{cc}
1 & 2\\
3 & 4\\
5 & 6
\end{array}\right)
\]
である。

Aの転置行列を$^{t}A$と書く。(トランポーズA)

$^{t}A\text{は}\left(n\times m\right)\text{行列}$


\subsubsection{命題}

$A\text{を}\left(m\times n\right)\text{行列}\text{、}B\text{を}\left(n\times l\right)\text{行列とする}$

\[
^{t}\left(AB\right)=^{t}B^{t}A
\]
が成立する。

証明は省略


\subsection{正則行列と逆行列}

Aをn次正方行列とする。


\subsubsection{定義}

あるn次正方行列Bで$AB=E_{n}=BA$を満たすものがあるとき、Aは正則行列であるという。

このとき
\[
A\text{が正則}\Leftrightarrow ad-bc\neq0
\]


さらにその時

\[
A^{-1}=\dfrac{1}{ad-bc}\left(\begin{array}{cc}
d & -b\\
-c & a
\end{array}\right)
\]


で与えられる。


\subsubsection{行列の分割}

$A\left(m\times n\right),B\left(n\times l\right)\text{行列}$

方針 これら(R1)~(R3)の同値変形を用いて

\[
\left(\begin{array}{ccccccc}
1 & 0 & \cdots & 0 & b_{1,r+1} & \cdots & b_{1,n}\\
0 & 1 & \cdots & 0 & \cdots & \cdots & \cdots\\
0 & \cdots & 0 & 1 & b_{r,r+1} & \cdots & b_{r,n}
\end{array}\right)
\]


の形に持って行きたい。

もしこの形に同値変形させられたら、……ともとの方程式は同値なので

$x_{r+1},\cdots,x_{n}$は自由に動くこれらを用いて$x_{1},\cdots,x_{r}$は表せる。

これでいつもうまくいくか?

←はじめにある変数について解く


\subsubsection{例}

\[
x+y+z+\omega=0
\]


\[
2x+2y+3z+4\omega=0
\]


\[
\left(\begin{array}{cccc}
1 & 1 & 1 & 1\\
2 & 2 & 3 & 4
\end{array}\right)\rightarrow\underbrace{\left(\begin{array}{cccc}
1 & 1 & 1 & 1\\
0 & 0 & 1 & 2
\end{array}\right)}_{x,y,z,\omega}\rightarrow\underbrace{\left(\begin{array}{cccc}
1 & 1 & 1 & 1\\
0 & 1 & 0 & 2
\end{array}\right)}_{x,z,y,\omega}\rightarrow\left(\begin{array}{cccc}
1 & 0 & 1 & -1\\
0 & 1 & 0 & 2
\end{array}\right)
\]


\[
x+y+\omega=0
\]


\[
z+\omega=0
\]


$y=s,\omega=t$と置く。

\[
x=-s+t
\]


\[
z=-t
\]


\[
\left(\begin{array}{c}
x\\
y\\
z\\
\omega
\end{array}\right)=\left(\begin{array}{c}
-s+t\\
s\\
-t\\
t
\end{array}\right)\cdots\text{一般解}
\]


変数の入れ替え⇔係数行列では列の入れ替え

(C3) i列目とj列目を入れ替える

以下$\left(\begin{array}{cc}
I_{r} & B\\
0 & 0
\end{array}\right)$タイプの行列を正規形という

方針 (R1)(R2)(R3)(C3)を用いて行列を正規形にする

2つ目の表し方

\begin{eqnarray*}
\left(\begin{array}{c}
x\\
y\\
z\\
\omega
\end{array}\right) & = & \left(\begin{array}{c}
-s+t\\
s\\
-t\\
t
\end{array}\right)\\
 & = & \left(\begin{array}{c}
-s\\
s\\
0\\
0
\end{array}\right)\\
 & = & \left(\begin{array}{c}
t\\
0\\
-t\\
t
\end{array}\right)\\
 & = & s\left(\begin{array}{c}
-1\\
1\\
0\\
0
\end{array}\right)+t\left(\begin{array}{c}
1\\
0\\
-1\\
1
\end{array}\right)\\
 & = & sv_{1}+tv_{2}
\end{eqnarray*}


つまり$v_{1}\text{と}v_{2}$の一次結合の形である。


\subsubsection{定理4.3}

A:(m×n)行列とする

この時変形(R1)\textasciitilde{}(R3)と(C3)を用いて正規形にすることができる


\subsubsection{証明}

mに関する帰納法 (m=行数)


\paragraph{1. m=1のとき}

\[
A=\left(a_{11},a_{12},\cdots a_{1n}\right)
\]



\subparagraph{(a) $A=\left(0,0,\cdots0\right)$}

r=0として正規型


\subparagraph{(b) $A\neq\left(0,0,\cdots0\right)$}
\begin{enumerate}
\item $a_{1i}\neq0$でないiがあるので、(C3)を用いて$a_{1i}$を一列目に持ってくる。
\item その結果$\left(a'_{11},a'_{12},\cdots a'_{1n}\right)$となったとして、$a'_{11}\left(\neq0\right)$で割る。
\[
\left(1,\dfrac{a'_{12}}{a'_{11}},\cdots,\dfrac{a'_{1n}}{a'_{11}}\right)
\]
正規形になる。
\end{enumerate}

\paragraph{2. (m-1)行の行列が(R1)\textasciitilde{}(R3)(C3)を用いて正規形にできると仮定する。...({*})}

A: (m×n)行列であると仮定する

Aの1\textasciitilde{}(m-1)行目に着目して、(R1)\textasciitilde{}(R3)(C3)を用いてAを変形する

({*})を用いて


\subparagraph{(a) $b'_{1}=\left(0.0.\cdots,0\right)$のとき}

これですでに正規形。


\subparagraph{(b)$b'_{2}+\left(0,0,\cdots,0\right)\text{のとき}$}

(r+1)列目\textasciitilde{}n列目を入れ替えて、$b'_{2}$の初めの成分≠0とする。結果$b''_{2}$になったとする

\[
b''_{2}=\left(b''_{2r+1},\cdots,b''_{2n}\right)
\]


なので、m行目を$b''_{2r+1}$で割る


\subsubsection{定義}

最終的に得られるrのことをAの階数(rank)と呼ぶ。

$r=\mathrm{rank}\left(A\right)$と書く。

(R3)i行目とj行目を入れ替える

(C3)i列目とj列目を入れ替える

$\mbox{\ensuremath{\left(\begin{array}{cc}
 I_{r}  &  B\\
0  &  0 
\end{array}\right)}}$の形に変形できる。

例

\[
2x+y+z+3w=0
\]


\[
-2x+y+z+2w=0
\]


\[
-10x-y-z-5w=0
\]


方程式の係数行列は

\[
\left(\begin{array}{cccc}
2 & 1 & 1 & 3\\
-2 & 1 & 1 & 2\\
-10 & -1 & -1 & -5
\end{array}\right)
\]


1を(1,1)にもっていく

(C3)で変数の順番を入れ替える

\[
\left(\begin{array}{cccc}
1 & 1 & 2 & 3\\
1 & 1 & -2 & 2\\
-1 & -1 & -10 & -5
\end{array}\right)
\]


\[
\rightarrow\left(\begin{array}{cccc}
1 & 1 & 2 & 3\\
0 & 0 & -4 & -1\\
0 & 0 & -8 & -2
\end{array}\right)
\]


\[
\rightarrow\left(\begin{array}{cccc}
1 & 3 & 2 & 1\\
0 & -1 & -4 & 0\\
0 & -2 & -8 & 0
\end{array}\right)
\]


\[
\rightarrow\left(\begin{array}{cccc}
1 & 3 & 2 & 1\\
0 & 1 & 4 & 0\\
0 & -2 & -8 & 0
\end{array}\right)
\]


\[
\rightarrow\left(\begin{array}{cccc}
1 & 0 & -10 & 1\\
0 & 1 & 4 & 0\\
0 & 0 & 0 & 0
\end{array}\right)
\]


\[
z-10x+y=0,\: w+4x=0
\]


\[
z=10x-y,\: w=-4x
\]


x,yは自由に動く

$s=s,\: y=t\text{とおいて、}$

\[
\left(\begin{array}{c}
x\\
y\\
z\\
w
\end{array}\right)=s\left(\begin{array}{c}
1\\
0\\
10\\
-4
\end{array}\right)+t\left(\begin{array}{c}
0\\
1\\
-1\\
0
\end{array}\right)
\]


これが一般の解の形


\subsection{線形写像の核}

f: $\mathbb{R}^{n}\rightarrow\mathbb{R}^{m}$ 線形写像

$\left\{ \mathbf{x}\in\mathbb{R}|f\left(\mathbf{x}\right)=\mathbb{O}\right\} $をfの核(kernel)と言い、$\ker\left(f\right)$と書く。

※$\mathbb{O}=\left(\begin{array}{c}
0\\
\vdots\\
0
\end{array}\right)$

例 f: $\mathbb{R}^{4}\rightarrow\mathbb{R}^{3}$ を線形写像で$\left(\begin{array}{cccc}
2 & 1 & 1 & 3\\
-2 & 1 & 1 & 2\\
-10 & -1 & -1 & -5
\end{array}\right)$に対応するものとする。

この時$\ker\left(f\right)$を求めよ。

$x=\left(\begin{array}{c}
x\\
y\\
z\\
w
\end{array}\right)$とおく。

\begin{eqnarray*}
f\left(\mathbf{x}\right)=\mathbb{O} & \Leftrightarrow & \left(\begin{array}{cccc}
2 & 1 & 1 & 3\\
-2 & 1 & 1 & 2\\
-10 & -1 & -1 & -5
\end{array}\right)\left(\begin{array}{c}
x\\
y\\
z\\
w
\end{array}\right)=\left(\begin{array}{c}
0\\
0\\
0
\end{array}\right)\\
 & \Leftrightarrow & \text{さっきやった方程式を}x\text{が満たす}\\
 & \Leftrightarrow & \mathbf{x}=s\left(\begin{array}{c}
1\\
0\\
10\\
-4
\end{array}\right)+t\left(\begin{array}{c}
0\\
1\\
-1\\
0
\end{array}\right)\\
 & \Leftrightarrow & x\text{は}\left(\begin{array}{c}
1\\
0\\
10\\
-1
\end{array}\right)\text{と}\left(\begin{array}{c}
0\\
1\\
-1\\
0
\end{array}\right)\text{の一次結合で表せる。}
\end{eqnarray*}



\section{---}


\section{---}


\subsection{---}


\subsection{---}

行基本変形(R1)(R2)(R3)と基本行列の関係


\subsection{基本行列}

(R1)i行目をC倍する(C!=0)

i行目を$C^{-1}$倍すると$\bar{x}$に戻る。

この操作の逆操作はCとなる。

(R2) i行目をC倍してj行目に加える

R2の操作は行列Qを左から書ける操作。

逆操作はi行目を-i倍してq行目に加える。

=$Q\left(-c,i,j\right)$を左からかける。

$Q\left(-c,i,j\right)=Q\left(c,i,j\right)^{-1}$で

Q(c,i,j)は正則行列である。

(R3) i行目とj行目を入れ替える操作は行列R(i,j)を左からかける操作と一致。

i行目とj行目を入れ替える操作の逆操作はi行目とj行目を入れ替える操作なので、

\[
R\left(i,j\right)^{-1}=R\left(i,j\right)
\]
でR(i,j)は正則行列


\subsubsection{定義}

\[
P\left(c,i\right)\:\left(C\neq0\right)
\]
\[
Q\left(c,i,j\right)\:\left(i\neq j\right)
\]
\[
R\left(i,j\right)\:\left(i\neq j\right)
\]


のタイプの行列を基本行列という。

全く同様にして列基本変形を

(C1)1つの列をc倍する($c\neq0$)

(C2)j列目のc倍をi列目に加える ($i\neq j$)

(C3)i列目とj列目を入れ替える

という操作によって定める。

この操作は同様に行列の掛け算として表せるか?

答えはYes.

(C2)を見る。

\[
A=\left(\mathbb{Q}_{1}\:\cdots\:\mathbb{Q}_{m}\right)
\]
列ベクトル分割する。

\[
A=\left(\begin{array}{ccccccc}
\mathbb{Q}_{1} & \cdots & \mathbb{Q}_{i} & \cdots & \mathbb{Q}_{j} & \cdots & \mathbb{Q}_{m}\end{array}\right)\left(\begin{array}{ccccccc}
1\\
 & \ddots\\
 &  & 1\\
 &  & \vdots & \ddots\\
 &  & c & \cdots & 1\\
 &  &  &  &  & \ddots\\
 &  &  &  &  &  & 1
\end{array}\right)\left(=Q\left(c,i,j\right)\right)
\]


(C2)はp(c,i,j)を右からかける操作に一致する。

(C1),(C3)はそれぞれp(c,i),R(i,j)を右からかける操作となる。

定理4.3を基本行列の言葉で言えば、行列Aはいくつかの基本行列を左から掛け、R(i,j)の形の行列を右からかけると$\left(\begin{array}{cc}
I_{r} & B\\
0 & 0
\end{array}\right)$の形になる


\subsubsection{逆行列を求めることへの応用}

$A=\left(n\times n\right)\text{行列}$

\[
\left(A,E_{n}\right)\rightarrow\left(\left(R1\right)\left(R2\right)\left(R3\right)\text{の繰り返し}\right)\rightarrow\left(E_{n},B\right)
\]


この時、いくつかの基本行列$P_{1},\: P_{2},\: P_{3}\cdots,\: P_{k}$が存在し、

\begin{eqnarray*}
\left(A,E_{n}\right) & \rightarrow & P_{1}\left(A,E_{n}\right)\\
 & \rightarrow & P_{2}P_{1}\left(A,E_{n}\right)\\
 & \rightarrow & P_{k}P_{k-1}\cdots P_{2}P_{1}\left(A,E_{n}\right)=\left(E_{n},B\right)
\end{eqnarray*}


$Q=P_{k}P_{k-1}\cdots P_{2}P_{1}$とおくと、

\begin{eqnarray*}
Q\left(A,E_{n}\right) & = & \left(E_{n},B\right)\\
 & = & \left(QA,QE_{n}\right)\\
 & = & \left(QA,Q\right)
\end{eqnarray*}


ブロック比較…$BA=E_{n}$

$QA=E_{n}$…B=Aの逆行列

\[
Q=B
\]


例 

$\left(\begin{array}{cc}
1 & 2\\
2 & 3
\end{array}\right)$の逆行列を求める。

\[
A=\left(\begin{array}{cc}
1 & 2\\
2 & 3
\end{array}\right)
\]


\[
\left(A,E_{n}\right)=\left(\begin{array}{cc}
1 & 2\\
2 & 3
\end{array}|\begin{array}{cc}
1 & 0\\
0 & 1
\end{array}\right)
\]


標準形の(C3)はダメ

\begin{eqnarray*}
\left(\begin{array}{cc}
1 & 2\\
2 & 3
\end{array}|\begin{array}{cc}
1 & 0\\
0 & 1
\end{array}\right) & \rightarrow & \left(\begin{array}{cc}
1 & 2\\
0 & -1
\end{array}|\begin{array}{cc}
1 & 0\\
-2 & 1
\end{array}\right)\\
 & \rightarrow & \left(\begin{array}{cc}
1 & 2\\
0 & 1
\end{array}|\begin{array}{cc}
1 & 0\\
2 & -1
\end{array}\right)\\
 & \rightarrow & \left(\begin{array}{cc}
1 & 0\\
0 & 1
\end{array}|\begin{array}{cc}
-3 & 2\\
2 & -1
\end{array}\right)
\end{eqnarray*}


\[
\therefore\left(\begin{array}{cc}
1 & 2\\
2 & 3
\end{array}\right)^{-1}=\left(\begin{array}{cc}
-3 & 2\\
2 & -1
\end{array}\right)
\]



\subsubsection{演習問題解説}

1. 係数行列=$\overbrace{\left(\begin{array}{cccc}
-2 & 4 & 1 & -5\\
-3 & 11 & 3 & -11\\
-3 & 8 & 2 & -9
\end{array}\right)}^{x,y,z,w}$

文字の入れ替えを積極的に使っていく

\[
\rightarrow\overbrace{\left(\begin{array}{cccc}
1 & 4 & -2 & -5\\
3 & 11 & -3 & -11\\
2 & 8 & -3 & -9
\end{array}\right)}^{z,y,x,w}\rightarrow\overbrace{\left(\begin{array}{cccc}
1 & 4 & -2 & -5\\
0 & -1 & 3 & 4\\
0 & 0 & 1 & 1
\end{array}\right)}^{z,y,x,w}\rightarrow\overbrace{\left(\begin{array}{cccc}
1 & 4 & -2 & -5\\
0 & 1 & -3 & -4\\
0 & 0 & 1 & 1
\end{array}\right)}^{z,y,x,w}\rightarrow\overbrace{\left(\begin{array}{cccc}
1 & 0 & 10 & 11\\
0 & 1 & -3 & -4\\
0 & 0 & 1 & 1
\end{array}\right)}^{z,y,x,w}\rightarrow\overbrace{\left(\begin{array}{cccc}
1 & 0 & 0 & 1\\
0 & 1 & 0 & -1\\
0 & 0 & 1 & 1
\end{array}\right)}^{z,y,x,w}
\]


\[
z=-w
\]


\[
y=w
\]


\[
z=-w
\]


$\therefore w=s$とおくと、
\[
\left(\begin{array}{c}
x\\
y\\
z\\
w
\end{array}\right)=s\left(\begin{array}{c}
-1\\
1\\
-1\\
1
\end{array}\right)
\]


2. 階数(rank)

(R1)(R2)(R3)(C3)を用いて$\left(\begin{array}{cc}
I_{r} & B\\
0 & 0
\end{array}\right)$の形にした時のr=rank

(1) $A=\left(\begin{array}{cccc}
1 & -1 & 2 & 1\\
-2 & 1 & 1 & 3\\
-4 & 1 & 9 & 11
\end{array}\right)\rightarrow\left(\begin{array}{cccc}
1 & -1 & 2 & 1\\
0 & -1 & 5 & 5\\
0 & -3 & 15 & 15
\end{array}\right)\rightarrow\left(\begin{array}{cccc}
1 & -1 & 2 & 1\\
0 & 1 & -5 & -5\\
0 & -3 & 15 & 15
\end{array}\right)\rightarrow\left(\begin{array}{cccc}
1 & 0 & -3 & -4\\
0 & 1 & -5 & -5\\
0 & 0 & 0 & 0
\end{array}\right)$

rankは2 rank(A)=2

(2) $\left(\begin{array}{cc}
3+a & *\\
* & *
\end{array}\right)$ 3+aが(1,1)に来ないようにする。吐き出しの要にしない。

\begin{eqnarray*}
\left(\begin{array}{cc}
3+a & -1\\
-1 & 2+b
\end{array}\right) & \rightarrow & \left(\begin{array}{cc}
-1 & 2+b\\
3+a & -1
\end{array}\right)\\
 & \rightarrow & \left(\begin{array}{cc}
1 & -2-b\\
3+a & -1
\end{array}\right)\\
 & \rightarrow & \left(\begin{array}{cc}
1 & -2-b\\
0 & \left(2+b\right)\left(3+a\right)-1
\end{array}\right)=\left(\begin{array}{cc}
1 & -2-b\\
0 & ab+2a+3b+5
\end{array}\right)
\end{eqnarray*}


(i) $ab+2a+3b+5=0$のとき、正規形でrank(A)=1

(ii) $ab+2a+3b+5\neq0$のとき、$\rightarrow\left(\begin{array}{cc}
1 & -2-b\\
0 & 1
\end{array}\right)\rightarrow\left(\begin{array}{cc}
1 & 0\\
0 & 1
\end{array}\right)$

この時rank(A)=2(変数が入った行列のrankは行基本変形の要になるべく文字が入らないように)

3.

$\left(\begin{array}{cccc}
2 & 1 & 0 & 3\\
1 & 1 & -1 & -2\\
-7 & -5 & 3 & 0
\end{array}\right)\left(\begin{array}{c}
x\\
y\\
z\\
w
\end{array}\right)=\mathbb{O}$を解く。

\begin{eqnarray*}
\left(\begin{array}{cccc}
2 & 1 & 0 & 3\\
1 & 1 & -1 & -2\\
-7 & -5 & 3 & 0
\end{array}\right) & \rightarrow & \left(\begin{array}{cccc}
1 & 1 & -1 & -2\\
2 & 1 & 0 & 3\\
-7 & -5 & 3 & 0
\end{array}\right)\\
 & \rightarrow & \left(\begin{array}{cccc}
1 & 1 & -1 & -2\\
0 & -1 & 2 & 7\\
0 & 2 & -4 & -14
\end{array}\right)\\
 & \rightarrow & \left(\begin{array}{cccc}
1 & 1 & -1 & -2\\
0 & 1 & -2 & -7\\
0 & 2 & -9 & -14
\end{array}\right)\\
 & \rightarrow & \left(\begin{array}{cccc}
1 & 0 & 1 & 5\\
0 & 1 & -2 & -7\\
0 & 0 & 0 & 0
\end{array}\right)
\end{eqnarray*}


\[
x=-z-5w
\]


\[
y=2z+7w
\]


\[
\left(\begin{array}{c}
x\\
y\\
z\\
w
\end{array}\right)=s\left(\begin{array}{c}
-1\\
2\\
1\\
0
\end{array}\right)+t\left(\begin{array}{c}
-5\\
7\\
0\\
1
\end{array}\right)
\]


確認

\[
\left(\begin{array}{cccc}
2 & 1 & 0 & 3\\
1 & 1 & -1 & -2\\
-7 & -5 & 3 & 0
\end{array}\right)\left(\begin{array}{cc}
-1 & -5\\
2 & 7\\
1 & 0\\
0 & 1
\end{array}\right)=\left(\begin{array}{cc}
0 & 0\\
0 & 0\\
0 & 0
\end{array}\right)
\]


O.K.

一般のker(f)の元は$s\left(\begin{array}{c}
-1\\
2\\
1\\
0
\end{array}\right)+t\left(\begin{array}{c}
-5\\
7\\
0\\
1
\end{array}\right)$の形$\left(s,t\in\mathbb{R}\right)$

4.

$\left(\begin{array}{ccc}
-1 & -3 & 3\\
-2 & -6 & 5\\
-2 & -7 & 6
\end{array}|\begin{array}{ccc}
1 & 0 & 0\\
0 & 1 & 0\\
0 & 0 & 1
\end{array}\right)\rightarrow\left(\begin{array}{ccc}
1 & 3 & -3\\
-2 & -6 & 5\\
-2 & -7 & 6
\end{array}|\begin{array}{ccc}
-1 & 0 & 0\\
0 & 1 & 0\\
0 & 0 & 1
\end{array}\right)\rightarrow\left(\begin{array}{ccc}
1 & 3 & -3\\
0 & 0 & -1\\
0 & -1 & 0
\end{array}|\begin{array}{ccc}
-1 & 0 & 0\\
-2 & 1 & 0\\
-2 & 0 & 1
\end{array}\right)\rightarrow\left(\begin{array}{ccc}
1 & 3 & -1\\
0 & 1 & 0\\
0 & 0 & 1
\end{array}|\begin{array}{ccc}
-1 & 0 & 0\\
2 & 0 & -1\\
2 & -1 & 0
\end{array}\right)\rightarrow\left(\begin{array}{ccc}
1 & 0 & 0\\
0 & 1 & 0\\
0 & 0 & 1
\end{array}|\begin{array}{ccc}
-1 & -3 & 3\\
2 & 0 & -1\\
2 & -1 & 0
\end{array}\right)$

答 $\left(\begin{array}{ccc}
-1 & -3 & 3\\
2 & 0 & -1\\
3 & -1 & 0
\end{array}\right)$

5.

\[
A_{1}=\left(\begin{array}{cc}
3 & 5\\
4 & 7
\end{array}\right)
\]


\[
A_{2}=\left(\begin{array}{cc}
1 & 0\\
-1 & 1
\end{array}\right)A_{1}=\left(\begin{array}{cc}
3 & 5\\
1 & 2
\end{array}\right)
\]


\[
A_{3}=\left(\begin{array}{cc}
1 & -3\\
0 & 1
\end{array}\right)A_{2}=\left(\begin{array}{cc}
0 & -1\\
1 & 2
\end{array}\right)
\]


\[
A_{4}=\left(\begin{array}{cc}
1 & 0\\
2 & 1
\end{array}\right)A_{3}=\left(\begin{array}{cc}
0 & -1\\
1 & 0
\end{array}\right)
\]


\[
A_{5}=\left(\begin{array}{cc}
0 & 1\\
1 & 0
\end{array}\right)A_{4}=\left(\begin{array}{cc}
1 & 0\\
0 & -1
\end{array}\right)
\]


\[
A_{6}=\left(\begin{array}{cc}
1 & 0\\
0 & -1
\end{array}\right)A_{5}=\left(\begin{array}{cc}
1 & 0\\
0 & 1
\end{array}\right)=E_{2}
\]


\[
E_{2}=\left(\begin{array}{cc}
1 & 0\\
0 & -1
\end{array}\right)A_{5}=\left(\begin{array}{cc}
1 & 0\\
0 & -1
\end{array}\right)\left(\begin{array}{cc}
0 & 1\\
1 & 0
\end{array}\right)A_{4}=\cdots=\left(\begin{array}{cc}
1 & 0\\
0 & -1
\end{array}\right)\left(\begin{array}{cc}
0 & 1\\
1 & 0
\end{array}\right)\left(\begin{array}{cc}
1 & 0\\
2 & 1
\end{array}\right)\left(\begin{array}{cc}
1 & -3\\
0 & 1
\end{array}\right)\left(\begin{array}{cc}
1 & 0\\
-1 & 1
\end{array}\right)A_{1}
\]


\[
A_{1}=\left\{ \left(\begin{array}{cc}
1 & 0\\
0 & -1
\end{array}\right)\left(\begin{array}{cc}
0 & 1\\
1 & 0
\end{array}\right)\cdots\left(\begin{array}{cc}
1 & 0\\
-1 & 1
\end{array}\right)\right\} ^{-1}=\left(\begin{array}{cc}
1 & 0\\
1 & 1
\end{array}\right)\left(\begin{array}{cc}
1 & 3\\
0 & 1
\end{array}\right)\left(\begin{array}{cc}
1 & 0\\
-2 & 1
\end{array}\right)\left(\begin{array}{cc}
0 & 1\\
1 & 0
\end{array}\right)\left(\begin{array}{cc}
1 & 0\\
0 & -1
\end{array}\right)
\]



\subsection{行列式}

A $\left(n\times n\right)$正方行列→行列式

$\left(2\times2\right)$の場合、図的に言えば平行四辺形の面積。

平面(xy平面)内の2津のベクトル$v_{1},\: v_{2}$で$v_{1}\neq0,\: v_{2}\neq0$、平行でない

$v_{2}$は$v_{1}$に対して反時計方向にある。

$S\left(v_{1},v_{2}\right)=v_{1}$と$v_{2}$で作られる平行四辺形の面積

(1) $r\in\mathbb{R},\: r>0$とすると、

\[
S\left(rv_{1},v_{2}\right)=rS\left(v_{1},v_{2}\right)=S\left(v_{1},rv_{2}\right)
\]


(2) $v_{2},v_{2}'$共に$v_{1}$に対して反時計方向。

\[
S\left(v_{1},v_{2}\right)+S\left(v_{1},v_{2}'\right)=S\left(v_{1},v_{2}+v_{2}'\right)
\]


(3) $C\in\mathbb{R}$とすると、

$v_{2}$が$v_{1}$に対して反時計回り

$\Rightarrow v_{2}+cv_{1}$も$v_{1}$に対して反時計回りであり、

\[
S\left(v_{1},v_{2}\right)=S\left(v_{1},v_{2}+cv_{1}\right)
\]


(1)(2)の制限を取りたい。

(2)でもし$v_{2}'$が$v_{1}$に対して時計回りなので、$S\left(v_{1},v_{2}+v_{2}'\right)$は面積に符号を付けて考える。(符号付き面積''向き付けされた図形の面積'')

一般に

\[
S\left(v_{1},v_{2}\right)=\begin{cases}
v_{1}\text{と}v_{2}\text{のなす平行四辺形の面積} & \left(v_{2}\text{が}v_{1}\text{に対して反時計回り}\right)\\
-v_{1}\text{と}v_{2}\text{のなす平行四辺形の面積} & \left(v_{2}\text{が}v_{1}\text{に対して時計回り}\right)
\end{cases}
\]


(1) 
\[
S\left(v_{1,}-v_{2}\right)=-S\left(v_{1},v_{2}\right)
\]


より、$r\in\mathbb{R}$ならば

\[
S\left(rv_{1},v_{2}\right)=rS\left(v_{1},v_{2}\right)=S\left(v_{1},rv_{2}\right)
\]


(2) $v_{1},v_{2},v_{2}'$

\[
S\left(v_{1},v_{2}+v_{2}'\right)=S\left(v_{1},v_{2}\right)+S\left(v_{1},v_{2}'\right)
\]


また、$v_{1}$と$v_{2}$を入れ替えると、

\[
S\left(v_{2},v_{1}\right)=-S\left(v_{1}v_{2}\right)
\]



\subsection{$S\left(v_{1},v_{2}\right)$の計算}

座標を一つ固定。

\[
\mathbb{C}_{1}=\left(\begin{array}{c}
1\\
0
\end{array}\right),\mathbb{C}_{2}=\left(\begin{array}{c}
1\\
0
\end{array}\right)
\]


を基本単位ベクトルとする。

上の性質から導かれる性質


\subsubsection{性質1 
\[
S\left(v_{1},v_{1}\right)=0
\]
}


\subsubsection{証明}

\[
S\left(v_{1},v_{1}\right)=-S\left(v_{1},v_{1}\right)
\]


移行して、
\[
2S\left(v_{1},v_{1}\right)=0
\]


\[
S\left(v_{1},v_{1}\right)=0
\]



\subsubsection{性質2}

\[
S\left(\mathbb{C}_{1},\mathbb{C}_{2}\right)=1
\]


\[
S\left(\mathbb{C}_{2},\mathbb{C}_{1}\right)=-1
\]


これらを用いて、$v_{1}=\left(\begin{array}{c}
a\\
b
\end{array}\right),v_{2}=\left(\begin{array}{c}
c\\
d
\end{array}\right)$として$S\left(v_{1},v_{2}\right)$を求める。

\[
v_{1}=a\mathbb{C}_{1}+b\mathbb{C}_{2}
\]


\[
v_{2}=c\mathbb{C}_{1}+d\mathbb{C}_{2}
\]


\begin{eqnarray*}
S\left(v_{1},v_{2}\right) & = & S\left(a\mathbb{C}_{1}+b\mathbb{C}_{2},v_{2}\right)\\
 & = & S\left(a\mathbb{C}_{1},v_{2}\right)+S\left(b\mathbb{C}_{2},v_{2}\right)\\
 & = & aS\left(\mathbb{C}_{1},v_{2}\right)+bS\left(\mathbb{C}_{2},v_{2}\right)\\
 & = & aS\left(\mathbb{C}_{1},c\mathbb{C}_{1}+d\mathbb{C}_{2}\right)+bS\left(\mathbb{C}_{2},c\mathbb{C}_{1}+d\mathbb{C}_{2}\right)\\
 & = & acS\left(\mathbb{C}_{1},\mathbb{C}_{1}\right)+adS\left(\mathbb{C}_{1},\mathbb{C}_{2}\right)+bcS\left(\mathbb{C}_{2},\mathbb{C}_{1}\right)+bdS\left(\mathbb{C}_{2},\mathbb{C}_{2}\right)\\
 & = & ad-bc
\end{eqnarray*}



\subsubsection{定義}

$A=\left(v_{1},v_{2}\right)$ $\left(2\times2\right)$行列

Aの行列式を$S\left(v_{1},v_{2}\right)$で定める。

$\det\left(A\right)$と書く。(determinant)

以下、章の目標は一般のnについての行列式を求めることである。(体積の満たすべき性質)


\subsubsection{まとめ}

(1) $S\left(v_{1},v_{2}\right)$は$v_{1},v_{2}$について線形写像

(2) $S\left(v_{1},v_{2}\right)=-S\left(v_{2},v_{1}\right)$

(3) $S\left(\left(\begin{array}{c}
1\\
0
\end{array}\right),\left(\begin{array}{c}
0\\
1
\end{array}\right)\right)=1$


\subsubsection{置換と符号}

記号$\left[1,n\right]=\left\{ 1,2,3,\cdots,n\right\} $

$1\leqq m<n$、m,nは自然数

$\left[m,n\right]=\left\{ m,m+1,m+2,\cdots,n\right\} $


\subsubsection{定義}

(1) $\left[1,n\right]$から$\left[1,n\right]$の全単射を(n次の)置換という。

例

n=3のとき

1→2, 2→3, 3→1 は{[}1,3{]}の置換。

n=3のとき$\sigma$写像が置換である→$\sigma\left(1\right),\sigma\left(2\right),\sigma\left(3\right)$が1~3の順列である。

従って{[}1,3{]}の置換は全部で3!=6個。

n=4の時は全部で24個。

(2) {[}1,n{]}の置換の集合を$\delta_{n}$と書く。


\subsubsection{置換の符号の定義}

数字と対応する写像を曲線で結ぶ。ただし下向き。

・3つの線は交わらない

・接することはできない

例えば上の例では交点の数=8


\subsubsection{命題}

交点の数の偶奇は結び方によらない。


\subsubsection{証明}

1つの結び方から外の結び方に変形する。(終点、始点は変わらないように変形させる)

変化の起こるパターン

(1) 交点変化なし

(2) 交点の数が2つ増えるもしくは2つ減る

(1)と(2)の繰り返しですべての結び方は移り合うので、偶奇は変わらない。


\subsubsection{定義}

$\sigma\in\mathfrak{S}_{n}$とする。

1と$\sigma\left(1\right)$、2と$\sigma\left(2\right)$、・・・、nと$\sigma\left(n\right)$を結んでできる交点が偶数の時、$sqn\left(\sigma\right)=1$、奇数の時、$sqn\left(\sigma\right)=-1$と定める。

$sqn\left(\sigma\right)$を$\sigma$の符号という。


\subsubsection{命題5.8}

$\sigma=\mathfrak{S}_{n}$

(1) $\sigma\left(1\right)=1,\sigma\left(2\right)=2,\cdots\sigma\left(n\right)=n$

となる$\sigma$を恒等置換という。これをeと書く。

$sqn\left(e\right)=1$

(2) $\sigma,\tau\in\mathfrak{S}_{n}$とする。

$\sigma$と$\tau$の合成$\tau\circ\sigma$を$\tau\sigma$と書く。

この$\tau\sigma\in\mathfrak{S}_{n}$について、

$sqn\left(\tau\sigma\right)=sqn\left(\tau\right)sqn\left(\sigma\right)$

(3) $i<j,\: i,j\in\left[1,n\right]$

\[
\sigma\left(k\right)=\begin{cases}
k & k\neq i,j\\
i & k=j\\
j & k=i
\end{cases}
\]


(つまり$\sigma$は$i$と$j$を入れ替えて他を動かさない)

これを$i$と$j$の互換という。

$sgn\left(\sigma\right)=-1$


\subsubsection{証明}

(1) 下向きにまっすぐ結ぶ→交点数0

$sgn\left(e\right)=1$

(2) $\tau\sigma$に対する結び方として$\sigma$に対する結び方と$\tau$に対する結び方をつなげたものが取れる。

全体の交点数=前半の交点数+後半の交点数→$sgn\left(\tau\sigma\right)=sgn\left(\tau\right)sgn\left(\sigma\right)$

(3) i-j以外の交点は全て2つずつペアになっている。→全体で奇数。


\subsubsection{命題}

\[
\mathrm{sgn}\left(\tau\sigma\right)=\mathrm{sgn}\left(\tau\right)\mathrm{sgn}\left(\sigma\right)
\]



\subsubsection{証明}

$i$と$\sigma\left(i\right)$を結び(交点a個)、さらに$\sigma\left(i\right)$と$\tau\sigma\left(i\right)$を結ぶ(交点b個)

→全体として$i$と$\tau\sigma\left(i\right)$が結ばれる交点の数は$\left(a+b\right)$個

この時、
\[
\mathrm{sgn}\left(\sigma\right)=\left(-1\right)^{a}
\]


\[
\mathrm{sgn}\left(\tau\right)=\left(-1\right)^{b}
\]


\begin{eqnarray*}
\mathrm{sgn}\left(\tau\sigma\right) & = & \left(-1\right)^{a+b}\\
 & = & \mathrm{sgn}\left(\tau\right)+\mathrm{sgn}\left(\sigma\right)
\end{eqnarray*}



\subsubsection{互換}

$1\sim n$のうち2つの元を入れ替えてほかは動かさないものを互換という。

2つの元を$i,j$とすると、

\[
\sigma\left(k\right)=\begin{cases}
k & k\neq i,j\\
j & k=i\\
i & k=j
\end{cases}
\]



\subsubsection{命題}

$\sigma\in\mathfrak{S}_{n}$、互換とする。

\[
\mathrm{sgn}\left(\sigma\right)=-1
\]



\subsubsection{例}

$\left[1,7\right]$ 2と5の互換

入れ替えた2つの線の交点以外は全て2つずつ組になる→交点の数は奇数


\subsubsection{互換の合成}

置換は互換の合成として書ける。

$\left(\sigma\left(1\right),\sigma\left(2\right),\sigma\left(3\right),\sigma\left(4\right),\sigma\left(5\right)\right)=\left(5,3,4,1,2\right)$

$\left(\tau_{1}\left(1\right),\tau_{1}\left(2\right),\tau_{1}\left(3\right),\tau_{1}\left(4\right),\tau_{1}\left(5\right)\right)=\left(1,3,4,5,2\right)$

1と5の互換を$\gamma\left(1,5\right)$と書く。

\[
\gamma\left(1,5\right)\tau_{1}=\sigma
\]


$\left(\tau_{2}\left(1\right),\tau_{2}\left(2\right),\tau_{2}\left(3\right),\tau_{2}\left(4\right),\tau_{2}\left(5\right)\right)=\left(1,2,4,5,3\right)$

\[
\gamma\left(2,3\right)\tau_{2}=\tau_{1}
\]


\[
\therefore\sigma=\gamma\left(1,5\right)\gamma\left(2,3\right)\tau_{2}
\]


繰り返して$\tau_{3},\tau_{4}\cdots$を作って、

$\tau_{i}=\left(1,2,3,4,5\right)$になるまで行う。

以下、$\tau_{i}$の推移。

\[
\left(5,3,4,1,2\right)
\]


\[
\left(1,3,4,5,2\right)
\]


\[
\left(1,2,4,5,3\right)
\]


\[
\left(1,2,3,5,4\right)
\]


\[
\left(1,2,3,4,5\right)
\]


よって
\[
\sigma=\gamma\left(1,5,\right)\gamma\left(2,3\right)\gamma\left(3,4\right)\gamma\left(4,5\right)
\]


$\therefore$σは4つの互換の積に書けた。

\[
\mathrm{sgn}\left(\gamma\left(1,5\right)\right)=-1
\]


\[
\mathrm{sgn}\left(\sigma\tau\right)=\mathrm{sgn}\left(\sigma\right)\mathrm{sgn}\left(\tau\right)
\]


\[
\therefore\mathrm{sgn}\left(\sigma\right)=\left(-1\right)^{4}
\]



\subsubsection{まとめ σの符号の求め方}
\begin{enumerate}
\item $\left(\sigma\left(1\right),\sigma\left(2\right),\cdots,\sigma\left(n\right)\right)$という順列を考える。
\item 数字の入れ替えを繰り返して$\left(1,2,3,\cdots,n\right)$の形に持ってくる。
\item かかった階数をaとすると、$\mathrm{sgn}\left(\sigma\right)=\left(-1\right)^{a}$
\end{enumerate}

\subsubsection{特別な置換}

$e\in\mathfrak{S}_{n}$

$e\left(1\right)=1,e\left(2\right)=2,\cdots,e\left(n\right)=n$

単位元、恒等写像という。

\[
\sigma e=\sigma=e\sigma
\]


$\sigma$は全単射なので逆写像$\sigma^{-1}$がある。

\[
\sigma\sigma^{-1}=e=\sigma^{-1}\sigma
\]


\begin{eqnarray*}
\mathrm{sgn}\left(\sigma\right)\mathrm{sgn}\left(\sigma^{-1}\right) & = & \mathrm{sgn}\left(\sigma\sigma^{-1}\right)\\
 & = & \mathrm{sgn}\left(e\right)\\
 & = & 1
\end{eqnarray*}


\[
\therefore\mathrm{sgn}\left(\sigma\right)=\mathrm{sgn}\left(\sigma^{-1}\right)
\]


今までの議論をそのまま用いて、$\left[1,n\right]$の代わりに$\left[m,n\right]=\left\{ m,m+1,\cdots,n\right\} $を用いると、集合$\left[m,n\right]$の置換、符号も考えることができる。

この置換の全体を$\mathfrak{S}\left[m,n\right]$と書く。

$\mathfrak{S}\left[3,5\right]$は$\left(\sigma\left(3\right),\sigma\left(4\right),\sigma\left(5\right)\right)$が3から5の数の置換。

さらに、$\mathfrak{S}\left[1,m\right]\in\sigma$と$\mathfrak{S}\left[m+1,n\right]\in\tau$を並べて$\left(\sigma,\tau\right)\in\mathfrak{S}\left[1,n\right]$が定義できる。


\subsubsection{例}

$\left(\sigma\left(1\right),\sigma\left(2\right),\sigma\left(3\right)\right)=\left(3,1,2\right)$

$\left(\tau\left(4\right),\tau\left(5\right)\right)=\left(5,4\right)$

$\Rightarrow\left(\sigma,\tau\right)=\gamma$と書くと、$\gamma\in\mathfrak{S}\left[1,5\right]$で、$\left(\gamma\left(1\right),\gamma\left(2\right),\gamma\left(3\right),\gamma\left(4\right),\gamma\left(5\right)\right)=\left(3,1,2,5,4\right)$となるものである。

図(略)より

\[
\mathrm{sqn}\left(\sigma,\tau\right)=\mathrm{sqn}\left(\sigma\right)\mathrm{sqn}\left(\tau\right)
\]



\subsection{行列式の定義}

$n=2$のとき、

\[
\left|\begin{array}{cc}
a_{1} & a_{2}\\
b_{1} & b_{2}
\end{array}\right|=a_{1}b_{2}-a_{2}b_{1}
\]


を、$\left(\begin{array}{cc}
a_{1} & a_{2}\\
b_{1} & b_{2}
\end{array}\right)$の行列式と定めた。


\subsubsection{一般のn次正方行列の行列式の定義}

A: n次正方行列

\begin{eqnarray*}
A & = & \left(a_{ij}\right)_{ij}\\
 & = & \left(\begin{array}{cccc}
a_{11} & a_{12} & \cdots & a_{1n}\\
a_{21} & a_{22} & \cdots & a_{2n}\\
\vdots & \vdots & \ddots & \vdots\\
a_{n1} & a_{n2} & \cdots & a_{nn}
\end{array}\right)
\end{eqnarray*}



\subsubsection{行列式の作り方}
\begin{enumerate}
\item $\sigma\in\mathfrak{S}_{n}$をとる。
\item 1行目から$\sigma\left(1\right)$列目をとり、2行目から$\sigma\left(2\right)$列目をとり、…、n行目から$\sigma\left(n\right)$列目をとり、掛け合わせる。
\item 2. で得られた結果に$\mathrm{sgn}\left(\sigma\right)$を掛ける。
\item 1. \textasciitilde{} 3. をすべての$\sigma$について行い、和をとる。
\end{enumerate}
こうして得られたものがAの行列式(determinant)である。

これを$\mathrm{det}\left(A\right)$と書く。

$n=3$のとき、

\[
\left|\begin{array}{ccc}
a_{11} & a_{12} & a_{13}\\
a_{21} & a_{22} & a_{23}\\
a_{31} & a_{32} & a_{33}
\end{array}\right|=+a_{11}a_{22}a_{33}-a_{11}a_{23}a_{32}-a_{12}a_{21}a_{33}+a_{12}a_{23}a_{31}+a_{13}a_{21}a_{32}-a_{12}a_{22}a_{31}
\]


これをサラスの公式と呼ぶ。

\begin{eqnarray*}
\left|\begin{array}{ccc}
1 & 4 & 2\\
1 & 3 & 3\\
2 & 1 & 2
\end{array}\right| & = & 1\times3\times2+3\times4\times2+1\times1\times2-1\times1\times3-2\times3\times2-1\times4\times2\\
 & = & 9
\end{eqnarray*}


※3つのベクトル$\left(\begin{array}{c}
1\\
1\\
2
\end{array}\right),\left(\begin{array}{c}
4\\
3\\
1
\end{array}\right),\left(\begin{array}{c}
2\\
3\\
2
\end{array}\right)$で作られる平行六面体の体積が9

行列式を作る操作を$\sum$記号で表す。(すべての順列を動く)

\[
\sum_{\sigma\in\mathfrak{S}_{n}}a_{1\sigma\left(1\right)}a_{2\sigma\left(2\right)}\cdots a_{n\sigma\left(n\right)}\mathrm{sgn}\left(\sigma\right)=\sum_{\sigma\in\mathfrak{S}_{n}}\mathrm{sgn}\left(\sigma\right)\prod_{j=1}^{n}a_{j\sigma\left(j\right)}
\]



\subsubsection{演習問題解説}


\paragraph{1.}

\[
v_{1}=\left(\begin{array}{c}
7\\
3
\end{array}\right)
\]


\[
v_{2}=\left(\begin{array}{c}
4\\
5
\end{array}\right)
\]


\begin{eqnarray*}
\det\left(v_{1},v_{2}\right) & = & \left(\begin{array}{cc}
7 & 4\\
3 & 5
\end{array}\right)\\
 & = & 35-12\\
 & = & 23
\end{eqnarray*}



\paragraph{2.}

\[
\det\left(v_{1},v_{2}\right)>0\Leftrightarrow v_{2}\text{は}v_{1}\text{に対して反時計回り}
\]


\begin{eqnarray*}
\det\left(\begin{array}{cc}
1 & -2\\
-3 & 5
\end{array}\right) & = & 5-6\\
 & = & -1\\
 & < & 0
\end{eqnarray*}


$v_{2}$は$v_{1}$に対して時計回り

$v_{1}$は$v_{2}$に対して反時計回り


\paragraph{3. (1)}

\begin{eqnarray*}
\left(\sigma\left(1\right),\sigma\left(2\right),\sigma\left(3\right),\sigma\left(4\right),\sigma\left(5\right)\right) & = & \left(5,1,4,3,2\right)\\
 & \rightarrow & \left(1,5,4,3,2\right)\\
 & \rightarrow & \left(1,2,4,3,5\right)\\
 & \rightarrow & \left(1,2,3,4,5\right)
\end{eqnarray*}


3回互換。
\begin{eqnarray*}
\mathrm{sqn}\left(\sigma\right) & = & \left(-1\right)^{3}\\
 & = & -1
\end{eqnarray*}



\paragraph{3. (2)}

\begin{eqnarray*}
\left(\sigma\left(1\right),\sigma\left(2\right),\cdots,\sigma\left(n\right)\right) & = & \left(2,3,4,\cdots,n,1\right)\\
 & \rightarrow & \left(2,3,4,\cdots,1,n\right)\\
 & \rightarrow & \left(2,3,4,\cdots,1,n-1,n\right)\\
 & \vdots & \vdots\\
 & \rightarrow & \left(1,2,3,4,\cdots,n-1,n\right)\:\left(n-1\right)\text{回目}
\end{eqnarray*}


\[
\mathrm{sqn}\left(\sigma\right)=\left(-1\right)^{n-1}
\]



\paragraph{3. (3)}

$\sigma\left(1\right)=n,\sigma\left(2\right)=n-1,\cdots,\sigma\left(n-1\right)=2,\sigma\left(n\right)=1$


\subparagraph{$n=2m$の時}

互換m個で$\left(1,2,3,\cdots,2m\right)$

\[
\mathrm{sgn}\left(\sigma\right)=\left(-1\right)^{m}
\]



\subparagraph{$n=2m+1$の時}

互換m個で$\left(1,2,3,\cdots,2m+1\right)$

\[
\mathrm{sgn}\left(\sigma\right)=\left(-1\right)^{m}
\]



\paragraph{4.}

\begin{eqnarray*}
\left(\sigma\left(1\right),\sigma\left(2\right),\sigma\left(3\right),\sigma\left(4\right)\right) & = & \left(4,1,3,2\right)\\
\left(\tau\left(1\right),\tau\left(2\right),\tau\left(3\right),\tau\left(4\right)\right) & = & \left(2,3,4,1\right)
\end{eqnarray*}
\[
\sigma\tau\left(1\right)=\sigma\left(2\right)=1,\sigma\tau\left(2\right)=3,\sigma\tau\left(3\right)=2,\sigma\tau\left(4\right)=4
\]


\[
\tau\sigma\left(1\right)=1,\tau\sigma\left(2\right)=2,\tau\sigma\left(3\right)=4,\tau\sigma\left(4\right)=3
\]



\paragraph{5. (1)}

\begin{eqnarray*}
\left|\begin{array}{cc}
6 & -1\\
8 & 3
\end{array}\right| & = & 18-\left(-8\right)\\
 & = & 26
\end{eqnarray*}



\paragraph{5. (2)}

\begin{eqnarray*}
\left|\begin{array}{ccc}
1 & 2 & 2\\
3 & 4 & 2\\
2 & 5 & 3
\end{array}\right| & = & 12+8+30-\left(16+10+18\right)\\
 & = & 50-44\\
 & = & 6
\end{eqnarray*}



\paragraph{6. (1)}

\begin{eqnarray*}
\left|\begin{array}{ccc}
a_{1} & b_{1} & c_{1}\\
a_{2} & b_{2} & c_{2}\\
a_{2} & b_{2} & c_{2}
\end{array}\right| & = & a_{1}b_{2}c_{2}+b_{1}c_{2}a_{2}+c_{1}a_{2}b_{2}-\left(a_{1}c_{2}b_{2}+b_{1}a_{2}c_{2}+c_{1}b_{2}a_{2}\right)\\
 & = & 0
\end{eqnarray*}



\paragraph{6. (2)}

\[
y=\frac{b_{1}-b_{2}}{a_{1}-a_{2}}\left(x-a_{1}\right)+b_{1}
\]


\[
\left(a_{1}-a_{2}\right)y-\left(b_{1}-b_{2}\right)x+b_{1}\left(a_{2}-b_{1}\right)=0
\]
\begin{eqnarray*}
\left|\begin{array}{ccc}
a_{1} & b_{1} & 1\\
a_{2} & b_{2} & 1\\
x & y & 1
\end{array}\right| & = & a_{1}b_{2}+b_{1}x+a_{2}y-a_{1}y-b_{1}a_{2}-b_{2}x\\
 & = & \left(-1\right)\times\left(\text{左辺}\right)
\end{eqnarray*}



\paragraph{6. (2) 別解}

この行列式はx,yの一次式。$\left|\begin{array}{ccc}
a_{1} & b_{1} & 1\\
a_{2} & b_{2} & 1\\
x & y & 1
\end{array}\right|=0$は直線。

$\left(x,y\right)=\left(a_{1},b_{1}\right)$を代入、$\left(a_{2},b_{2}\right)$を代入して計算すると、この二点を通ることが確かめられる。


\paragraph{6. (3)}

$\left|\begin{array}{ccc}
a_{1} & b_{1} & c_{1}\\
a_{2} & b_{2} & c_{2}\\
x & y & z
\end{array}\right|$はx,y,zの一次式→平面の方程式

$\left(x,y,z\right)=\left(0,0,0\right),\left(a_{1},b_{1},c_{1}\right),\left(a_{2},b_{2},c_{2}\right)$を代入して0になることを確かめる


\subsection{行列式}

\[
A=\left\{ a_{ij}\right\} _{ij}
\]


$\sigma\in\mathfrak{S}_{n}\text{について}$

\[
\sum_{\sigma\in\mathfrak{S}_{n}}\mathrm{sgn}\left(\sigma\right)a_{1\sigma\left(1\right)}a_{2\sigma\left(2\right)}\cdots a_{n\sigma(n)}
\]


を行列式といい、記号$\mathrm{det}\left(A\right)$と書く。


\subsubsection{いくつかの計算}

計算する回数は、

$n=3$のとき$3!=6$回

$n=4$のとき$4!=24$回

$n=5$のとき$5!=120$回

能率良いやり方が必要


\paragraph{命題5.13}

(1) $E_{n}$は単位行列

$\mathrm{det}(E_{n})=1$

(2) $A$: $\left(n\times n\right)$行列

→$^{t}A$ $t\left(n\times n\right)$行列

$\mathrm{det}\left(A\right)=\mathrm{det}\left(tA\right)$

証明

(1) クロネッカーのデルタ記号を用いて、

※ $\delta_{ij}=\begin{cases}
1 & (i=j)\\
0 & (i\neq j)
\end{cases}$

\[
E_{n}=\left(\delta_{ij}\right)_{ij}
\]


\[
dit\left(E_{n}\right)=\sum_{\sigma\in\mathfrak{S}_{n}}\mathrm{sgn}\left(\sigma\right)\delta_{1\sigma\left(1\right)}\delta_{2\sigma\left(2\right)}\cdots\delta_{n\sigma\left(n\right)}
\]


この和に寄与してくるのは、$1=\sigma\left(1\right),2=\sigma\left(2\right),\cdots,n=\sigma\left(n\right)$の時のみ。

$\therefore\sigma=e$の時のみ。

$\mathrm{sgn}\left(e\right)=1$なので、

\[
\mathrm{det}\left(E_{n}\right)=1
\]


(2) $A=\left(a_{ij}\right)_{ij}\rightarrow^{t}A=\left(a_{ij}\right)_{ij}$

\[
\mathrm{det}\left(^{t}A\right)=\sum_{\sigma\in\mathfrak{S}_{n}}\mathrm{sgn}\left(\sigma\right)a_{\sigma\left(1\right)1}a_{\sigma\left(2\right)2}\cdots a_{\sigma\left(n\right)n}
\]


掛け算を交換して、

\[
a_{\sigma\left(1\right)1}a_{\sigma\left(2\right)2}\cdots a_{\sigma\left(n\right)n}=a_{1\sigma^{-1}\left(1\right)}a_{2\sigma^{-1}\left(2\right)}a_{3\sigma^{-1}\left(3\right)}\cdots a_{n\sigma^{-1}\left(n\right)}
\]


なので、

\[
\mathrm{det}\left(^{t}A\right)=\sum_{\sigma\in\mathfrak{S}_{n}}\mathrm{sgn}\left(\sigma\right)a_{1\sigma^{-1}\left(1\right)}\cdots a_{n\sigma^{-1}\left(n\right)}
\]


実は$\mathrm{sgn}\left(\sigma\right)=\mathrm{sgn}\left(\sigma^{-1}\right)\in\left\{ 1,-1\right\} $

($\mathrm{sgn}\left(\sigma\right)\mathrm{sgn}\left(\sigma^{-1}\right)=\mathrm{sgn}\left(\sigma\sigma^{-1}\right)=\mathrm{sgn}\left(e\right)=1$より)

$\sigma$が$\mathfrak{S}_{n}$を一回ずつ動く

→$\sigma^{-1}$も$\mathfrak{S}_{n}$を一回ずつ動く

\[
\mathrm{det}\left(^{t}A\right)=\sum_{\sigma^{-1}\in\mathfrak{S}_{n}}\mathrm{sgn}\left(\sigma^{-1}\right)a_{1\sigma^{-1}\left(1\right)}\cdots a_{n\sigma^{-1}\left(n\right)}=\mathrm{det}\left(A\right)
\]


QED

例

\[
\left|\begin{array}{cc}
1 & 2\\
3 & 4
\end{array}\right|=1\times4-3\times2=-2
\]


\[
\left|\begin{array}{cc}
1 & 3\\
2 & 4
\end{array}\right|=1\times4-3\times2=-2
\]



\subsection{5.4 ブロック分割と行列式}

$n\geq1$、自然数$n=p+q\left(p,q\geq1\right)$

$A$の行、列ともに植えのnの分割に応じて分割する。


\paragraph{仮定}

この分割で$A_{21}$の部分の値が全て0になるものとする。

つまり

$A=\left(a_{ij}\right)_{ij}$とした時、$i\geq p+1,j\leq p$ならば、$a_{ij}=0$とする。


\paragraph{定理5.14}

\[
\mathrm{det}\left(A\right)=\mathrm{det}\left(A_{11}\right)\mathrm{det}\left(A_{22}\right)
\]


(注) $\left[1,p\right]$の置換全体を$\mathfrak{S}\left[1.p\right]$、$\left[p+1,n\right]$の置換全体を$\mathfrak{S}\left[p+1.n\right]$とおく。

\[
\mathrm{det}\left(A\right)=\sum_{\sigma\in\mathfrak{S}_{n}}\mathrm{sgn}\left(\sigma\right)a_{1\sigma\left(1\right)}\cdots a_{n\sigma\left(n\right)}
\]


$i\geq p+1,\sigma\left(i\right)\leq p$のとき、

\[
a_{i\sigma\left(i\right)}=0
\]


なので、和に寄与してくるのは、$i\geq p+1\Rightarrow\sigma\left(i\right)\geq p+1$となる$\sigma$のみ。

$\underbrace{\Longrightarrow}_{\text{全単射}}1\leq p\Rightarrow\sigma\left(i\right)\leq p$

\[
\sigma=\tau_{1}\times\tau_{2}
\]


※$\tau_{1}\in\mathfrak{S}\left[1,p\right],\tau_{2}\in\mathfrak{S}\left[p+1,n\right]$

この時

\[
\mathrm{sgn}\left(\sigma\right)=\mathrm{sgn}\left(\tau_{1}\right)\mathrm{sgn}\left(\tau_{2}\right)
\]


$\sigma$の動く範囲は

\[
\left\{ \left(\tau_{1},\tau_{2}\right)|\tau_{1}\in\mathfrak{S}\left[1,p\right],\tau_{2}\in\mathfrak{S}\left[p+1,n\right]\right\} 
\]


で、
\begin{eqnarray*}
\mathrm{det}\left(A\right) & = & \sum_{\begin{subarray}{c}
\tau_{1}\in\mathfrak{S}\left[1,p\right]\\
\tau_{2}\in\mathfrak{S}\left[p+1,n\right]
\end{subarray}}\mathrm{sgn}\left(\sigma\right)a_{1\tau_{1}\left(1\right)}\cdots a_{p\tau_{1}\left(p\right)}a_{p+1\tau_{2}\left(p+1\right)}\cdots a_{n\tau_{2}\left(n\right)}\\
 & = & \left(\sum_{\tau_{1}\in\mathfrak{S}\left[1,p\right]}\mathrm{sgn}\left(\tau_{1}\right)a_{1\tau_{1}\left(1\right)}\cdots a_{p\tau_{1}\left(p\right)}\right)\left(\sum_{\tau_{2}\in\mathfrak{S}\left[p+1,n\right]}\mathrm{sgn}\left(\tau_{2}\right)a_{p+1\tau_{2}\left(p+1\right)}\cdots a_{n\tau_{2}\left(n\right)}\right)\\
 & = & \mathrm{det}\left(A_{11}\right)\mathrm{det}\left(A_{22}\right)
\end{eqnarray*}


系

\[
A=\left(\begin{array}{cc}
A_{11} & 0\\
A_{21} & A_{22}
\end{array}\right)
\]


この時
\[
\mathrm{det}\left(A\right)=\mathrm{det}\left(A_{11}\right)\mathrm{det}\left(A_{22}\right)
\]


\begin{eqnarray*}
\mathrm{det}\left(A\right) & = & \mathrm{det}\left(^{t}\! A\right)\\
 & = & \mathrm{det}\left(\begin{array}{cc}
^{t}\! A_{11} & ^{t}\! A_{21}\\
0 & ^{t}\! A_{22}
\end{array}\right)\\
 & = & \mathrm{det}\left(^{t}\! A_{11}\right)\mathrm{det}\left(^{t}\! A_{22}\right)\\
 & = & \mathrm{det}\left(A_{11}\right)\mathrm{det}\left(A_{22}\right)
\end{eqnarray*}


Q.E.D.


\subsubsection{例}

$p=1,q=n-1$の場合は

\[
\mathrm{det}\left(A\right)=a_{11}\mathrm{det}\left(A_{22}\right)
\]



\subsection{5.5 多重線形性交代性}

A: $\left(n\times n\right)$行列

$A=\left(\boldsymbol{a}_{1},\boldsymbol{a}_{2},\cdots,\boldsymbol{a}_{n}\right)$と列ベクトル分解する。

\begin{eqnarray*}
\mathrm{det}\left(A\right) & = & \mathrm{det}\left(\left(\boldsymbol{a}_{1},\boldsymbol{a}_{2},\cdots,\boldsymbol{a}_{n}\right)\right)\\
 & = & \mathrm{det}\left(\boldsymbol{a}_{1},\boldsymbol{a}_{2},\cdots,\boldsymbol{a}_{n}\right)
\end{eqnarray*}


と表記する。


\subsubsection{命題5.19 (列に関する多重線形性)}

(1) $\boldsymbol{a}_{1},\boldsymbol{a}_{2},\cdots,\boldsymbol{a}_{n}$
n次元ベクトル($r\in\mathbb{R}$)

\[
\mathrm{det}\left(\boldsymbol{a}_{1},\boldsymbol{a}_{2},\cdots,r\boldsymbol{a}_{i},\cdots,\boldsymbol{a}_{n}\right)=r\mathrm{det}\left(\boldsymbol{a}_{1},\boldsymbol{a}_{2},\cdots,\boldsymbol{a}_{i},\cdots,\boldsymbol{a}_{n}\right)
\]


(2) $\boldsymbol{a}_{1},\boldsymbol{a}_{2},\cdots,\boldsymbol{a}_{i-1},\boldsymbol{a}_{i+1},\cdots,\boldsymbol{a}_{n},\boldsymbol{a}_{i}',\boldsymbol{a}_{i}''$
n次元ベクトル

\[
\mathrm{det}\left(\boldsymbol{a}_{1},\boldsymbol{a}_{2},\cdots,\boldsymbol{a}_{i}'+\boldsymbol{a}_{i}'',\cdots,\boldsymbol{a}_{n}\right)=\mathrm{det}\left(\boldsymbol{a}_{1},\boldsymbol{a}_{2},\cdots,\boldsymbol{a}_{i}',\cdots,\boldsymbol{a}_{n}\right)+\mathrm{det}\left(\boldsymbol{a}_{1},\boldsymbol{a}_{2},\cdots,\boldsymbol{a}_{i}'',\cdots,\boldsymbol{a}_{n}\right)
\]


(これをi列目に関する線形性という)


\subsubsection{証明}

(1)

\begin{eqnarray*}
\mathrm{det}\left(\boldsymbol{a}_{1},\boldsymbol{a}_{2},\cdots,r\boldsymbol{a}_{i},\cdots,\boldsymbol{a}_{n}\right) & = & \sum_{\sigma\in\mathfrak{S}_{n}}\mathrm{sgn}\left(\sigma\right)a_{1\sigma\left(1\right)}a_{2\sigma\left(2\right)}\cdots a_{n\sigma\left(n\right)}\\
 & = & \sum_{\sigma\in\mathfrak{S}_{n}}\mathrm{sgn}\left(\sigma\right)a_{\sigma\left(1\right)1}a_{\sigma\left(2\right)2}\cdots ra_{\sigma\left(i\right)i}\cdots a_{\sigma\left(n\right)n}\\
 & = & r\sum_{\sigma\in\mathfrak{S}_{n}}\mathrm{sgn}\left(\sigma\right)a_{\sigma\left(1\right)1}a_{\sigma\left(2\right)2}\cdots a_{\sigma\left(i\right)i}\cdots a_{\sigma\left(n\right)n}\\
 & = & r\mathrm{det}\left(\boldsymbol{a}_{1},\boldsymbol{a}_{2},\cdots,\boldsymbol{a}_{i},\cdots,\boldsymbol{a}_{n}\right)
\end{eqnarray*}


(2)

$\boldsymbol{a}_{i}=\boldsymbol{a}_{i}'+\boldsymbol{a}_{i}''$とおく

\begin{eqnarray*}
\mathrm{det}\left(\boldsymbol{a}_{1},\boldsymbol{a}_{2},\cdots,\boldsymbol{a}_{i}'+\boldsymbol{a}_{i}'',\cdots,\boldsymbol{a}_{n}\right) & = & \sum_{\sigma\in\mathfrak{S}_{n}}\mathrm{sgn}\left(\sigma\right)a_{1\sigma\left(1\right)}a_{2\sigma\left(2\right)}\cdots a_{n\sigma\left(n\right)}\\
 & = & \sum_{\sigma\in\mathfrak{S}_{n}}\mathrm{sgn}\left(\sigma\right)a_{1\sigma\left(1\right)}\cdots\left(a_{\sigma\left(i\right)i}'+a_{\sigma\left(i\right)i}''\right)\cdots a_{n\sigma\left(n\right)}\\
 & = & \sum_{\sigma\in\mathfrak{S}_{n}}\mathrm{sgn}\left(\sigma\right)a_{1\sigma\left(1\right)}\cdots a_{\sigma\left(i\right)i}'\cdots a_{n\sigma\left(n\right)}+\sum_{\sigma\in\mathfrak{S}_{n}}\mathrm{sgn}\left(\sigma\right)a_{1\sigma\left(1\right)}\cdots a_{\sigma\left(i\right)i}''\cdots a_{n\sigma\left(n\right)}\\
 & = & \mathrm{det}\left(\boldsymbol{a}_{1},\boldsymbol{a}_{2},\cdots,\boldsymbol{a}_{i}',\cdots,\boldsymbol{a}_{n}\right)+\mathrm{det}\left(\boldsymbol{a}_{1},\boldsymbol{a}_{2},\cdots,\boldsymbol{a}_{i}'',\cdots,\boldsymbol{a}_{n}\right)
\end{eqnarray*}



\subsubsection{例}

\begin{eqnarray*}
\left|\begin{array}{cc}
1 & 3+4\\
2 & 5+6
\end{array}\right| & = & \left|\begin{array}{cc}
1 & 3\\
2 & 5
\end{array}\right|+\left|\begin{array}{cc}
1 & 4\\
2 & 6
\end{array}\right|\\
-3 & = & \left(-1\right)+\left(-2\right)
\end{eqnarray*}



\subsubsection{命題5.12}

$i\leq j$とすると、

\[
\mathrm{det}\left(\boldsymbol{a}_{1},\cdots,\boldsymbol{a}_{j},\cdots,\boldsymbol{a}_{i},\cdots,\boldsymbol{a}_{n}\right)=-\mathrm{det}\left(\boldsymbol{a}_{1},\cdots,\boldsymbol{a}_{i},\cdots,\boldsymbol{a}_{j},\cdots,\boldsymbol{a}_{n}\right)
\]



\subsubsection{証明}

\[
\left(\text{左辺}\right)=\sum_{\sigma\in\mathfrak{S}_{n}}\mathrm{sgn}\left(\sigma\right)a_{\sigma\left(1\right)1}a_{\sigma\left(2\right)2}\cdots a_{\sigma\left(i\right)j}\cdots a_{\sigma\left(j\right)i}\cdots a_{\sigma\left(n\right)n}
\]


$\sigma\cdot\left(i\text{と}j\text{の互換}\right)=\sigma'$とおく。
\[
\sigma'\left(k\right)=\begin{cases}
j & \left(k=i\right)\\
i & \left(k=j\right)\\
k & \left(k\neq i,j\right)
\end{cases}
\]


\[
\mathrm{sgn}\left(\sigma'\right)=\mathrm{sgn}\left(\sigma\right)
\]


$\sigma$が$S_{n}$を動くと$\sigma^{-1}$も$S_{n}$を動く。

.

\begin{eqnarray*}
\left(\text{左辺}\right) & = & \sum_{\sigma\in\mathfrak{S}_{n}}-\mathrm{sgn}\left(\sigma\right)a_{\sigma\left(1\right)1}a_{\sigma\left(2\right)2}\cdots a_{\sigma\left(i\right)i}\cdots a_{\sigma\left(j\right)j}\cdots a_{\sigma\left(n\right)n}\\
 & = & -\mathrm{det}\left(\boldsymbol{a}_{1},\cdots,\boldsymbol{a}_{i},\cdots,\boldsymbol{a}_{j},\cdots,\boldsymbol{a}_{n}\right)
\end{eqnarray*}


\[
\left|\begin{array}{cc}
1 & 2\\
3 & 4
\end{array}\right|=-\left|\begin{array}{cc}
2 & 1\\
4 & 3
\end{array}\right|
\]



\subsubsection{定理}

(1)

\[
\mathrm{det}\left(\boldsymbol{a}_{1},\cdots,\boldsymbol{a}_{i},\cdots,\boldsymbol{a}_{i},\cdots,\boldsymbol{a}_{n}\right)=0
\]


(2)

\[
\mathrm{det}\left(\boldsymbol{a}_{1},\cdots,\boldsymbol{a}_{i},\cdots,\boldsymbol{a}_{j}+c\boldsymbol{a}_{i},\cdots,\boldsymbol{a}_{n}\right)=\mathrm{det}\left(\boldsymbol{a}_{1},\cdots,\boldsymbol{a}_{i},\cdots,\boldsymbol{a}_{j},\cdots,\boldsymbol{a}_{n}\right)
\]



\subsubsection{証明}

(1)

\[
\mathrm{det}\left(\boldsymbol{a}_{1},\cdots,\boldsymbol{a}_{i},\cdots,\boldsymbol{a}_{i},\cdots,\boldsymbol{a}_{n}\right)=-\mathrm{det}\left(\boldsymbol{a}_{1},\cdots,\boldsymbol{a}_{i},\cdots,\boldsymbol{a}_{i},\cdots,\boldsymbol{a}_{n}\right)
\]


移行して

\[
2\mathrm{det}\left(\boldsymbol{a}_{1},\cdots,\boldsymbol{a}_{i},\cdots,\boldsymbol{a}_{i},\cdots,\boldsymbol{a}_{n}\right)=0
\]


\[
\mathrm{det}\left(\boldsymbol{a}_{1},\cdots,\boldsymbol{a}_{i},\cdots,\boldsymbol{a}_{i},\cdots,\boldsymbol{a}_{n}\right)=0
\]


(2)

\begin{eqnarray*}
\mathrm{det}\left(\boldsymbol{a}_{1},\cdots,\boldsymbol{a}_{i},\cdots,\boldsymbol{a}_{j}+c\boldsymbol{a}_{i},\cdots,\boldsymbol{a}_{n}\right) & = & \mathrm{det}\left(\boldsymbol{a}_{1},\cdots,\boldsymbol{a}_{i},\cdots,\boldsymbol{a}_{j},\cdots,\boldsymbol{a}_{n}\right)+c\mathrm{det}\left(\boldsymbol{a}_{1},\cdots,\boldsymbol{a}_{i},\cdots,\boldsymbol{a}_{i},\cdots,\boldsymbol{a}_{n}\right)\\
 & = & \mathrm{det}\left(\boldsymbol{a}_{1},\cdots,\boldsymbol{a}_{i},\cdots,\boldsymbol{a}_{j},\cdots,\boldsymbol{a}_{n}\right)
\end{eqnarray*}



\subsubsection{例}

(1)

\[
\left|\begin{array}{cc}
1 & 1\\
2 & 2
\end{array}\right|=0
\]


(2)

\[
\left|\begin{array}{cc}
1 & 2\\
3 & 4
\end{array}\right|=\left|\begin{array}{cc}
1 & 2+3\times1\\
3 & 4+3\times3
\end{array}\right|=\left|\begin{array}{cc}
1 & 5\\
3 & 13
\end{array}\right|
\]


交代性とは、1つの列に他の列の何倍かを加えても行列式の値は変わらないこと。


\subsubsection{まとめ}

列に関する線形性、交代性から、行に関する線形性、交代性もいえる。

(A)(B)(C)の操作で要領よく計算できる。


\subsubsection{例題5.30}

\[
\left|\begin{array}{ccccc}
3 & 4 & 1 & 2 & -1\\
1 & 1 & -1 & 2 & 2\\
2 & 2 & 1 & -1 & 1\\
3 & 1 & -4 & 13 & 17\\
1 & 2 & 2 & -1 & -1
\end{array}\right|
\]


を求める。

行についての交代性、線形性を用いる。

$\left(1.2\right)$成分の1を一番上に持ってくる。

\[
\left|\begin{array}{ccccc}
3 & 4 & 1 & 2 & -1\\
1 & 1 & -1 & 2 & 2\\
2 & 2 & 1 & -1 & 1\\
3 & 1 & -4 & 13 & 17\\
1 & 2 & 2 & -1 & -1
\end{array}\right|=-\left|\begin{array}{ccccc}
1 & 1 & -1 & 2 & 2\\
3 & 4 & 1 & 2 & -1\\
2 & 2 & 1 & -1 & 1\\
3 & 1 & -4 & 13 & 17\\
1 & 2 & 2 & -1 & -1
\end{array}\right|
\]


一つの行の何倍かを他の行に加えても不変であることから

\[
-\left|\begin{array}{ccccc}
1 & 1 & -1 & 2 & 2\\
3 & 4 & 1 & 2 & -1\\
2 & 2 & 1 & -1 & 1\\
3 & 1 & -4 & 13 & 17\\
1 & 2 & 2 & -1 & -1
\end{array}\right|=-\left|\begin{array}{ccccc}
1 & 1 & -1 & 2 & 2\\
0 & 1 & 4 & -4 & -7\\
0 & 0 & 3 & -5 & -3\\
0 & -2 & -6 & 7 & 11\\
0 & 1 & 3 & -3 & -3
\end{array}\right|
\]


1+4のブロック分割より

\[
-\left|\begin{array}{ccccc}
1 & 1 & -1 & 2 & 2\\
0 & 1 & 4 & -4 & -7\\
0 & 0 & 3 & -5 & -3\\
0 & -2 & -6 & 7 & 11\\
0 & 1 & 3 & -3 & -3
\end{array}\right|=-\left|\begin{array}{cccc}
1 & 4 & -4 & 7\\
0 & 3 & -5 & -3\\
-2 & -6 & 7 & 11\\
1 & 3 & -3 & -3
\end{array}\right|
\]


一つの行の何倍かを他の行に加えても不変であることから

\[
-\left|\begin{array}{cccc}
1 & 4 & -4 & 7\\
0 & 3 & -5 & -3\\
-2 & -6 & 7 & 11\\
1 & 3 & -3 & -3
\end{array}\right|=-\left|\begin{array}{cccc}
1 & 4 & -4 & 7\\
0 & 3 & -5 & -3\\
0 & 2 & -1 & -3\\
0 & -1 & 1 & 4
\end{array}\right|
\]


1+3のブロック分割より

\[
-\left|\begin{array}{cccc}
1 & 4 & -4 & 7\\
0 & 3 & -5 & -3\\
0 & 2 & -1 & -3\\
0 & -1 & 1 & 4
\end{array}\right|=-\left|\begin{array}{ccc}
3 & -5 & -3\\
2 & -1 & -3\\
-1 & 1 & 4
\end{array}\right|
\]


これを計算して

\[
-\left|\begin{array}{ccc}
3 & -5 & -3\\
2 & -1 & -3\\
-1 & 1 & 4
\end{array}\right|=-19
\]


よって

\[
\left|\begin{array}{ccccc}
3 & 4 & 1 & 2 & -1\\
1 & 1 & -1 & 2 & 2\\
2 & 2 & 1 & -1 & 1\\
3 & 1 & -4 & 13 & 17\\
1 & 2 & 2 & -1 & -1
\end{array}\right|=-19
\]



\subsubsection{演習問題解説}


\paragraph{1. (1)\textmd{
\[
\left|\protect\begin{array}{cccc}
3 & 2 & 4 & 1\protect\\
5 & 3 & 2 & 3\protect\\
0 & 0 & 2 & 1\protect\\
0 & 0 & 3 & 5
\protect\end{array}\right|=\left|\protect\begin{array}{cc}
3 & 2\protect\\
5 & 3
\protect\end{array}\right|\left|\protect\begin{array}{cc}
2 & 1\protect\\
3 & 5
\protect\end{array}\right|=\left(-1\right)\times7=-7
\]
}}


\paragraph{1. (2)}

\begin{eqnarray*}
\left|\begin{array}{cccc}
3 & 0 & 4 & 0\\
2 & 3 & 1 & 3\\
1 & 0 & 2 & 0\\
3 & 2 & 4 & 5
\end{array}\right| & = & -\left|\begin{array}{cccc}
3 & 0 & 4 & 0\\
1 & 0 & 2 & 0\\
2 & 3 & 1 & 3\\
3 & 2 & 4 & 5
\end{array}\right|\\
 & = & \left|\begin{array}{cccc}
3 & 4 & 0 & 0\\
1 & 2 & 0 & 0\\
2 & 1 & 3 & 3\\
3 & 4 & 2 & 5
\end{array}\right|\\
 & = & \left|\begin{array}{cc}
3 & 4\\
1 & 2
\end{array}\right|\left|\begin{array}{cc}
3 & 3\\
2 & 5
\end{array}\right|\\
 & = & 2\times9\\
 & = & 18
\end{eqnarray*}



\paragraph{2. (1)}

\begin{eqnarray*}
\left|\begin{array}{cccc}
3 & 1 & 4 & 2\\
4 & 3 & 1 & 3\\
1 & 2 & 4 & 1\\
2 & 2 & 5 & 3
\end{array}\right| & = & -\left|\begin{array}{cccc}
1 & 2 & 4 & 1\\
4 & 3 & 1 & 3\\
3 & 1 & 4 & 2\\
2 & 2 & 5 & 3
\end{array}\right|\\
 & = & -\left|\begin{array}{cccc}
1 & 2 & 4 & 1\\
0 & -5 & -15 & -1\\
0 & -5 & -8 & -1\\
0 & -2 & -3 & 1
\end{array}\right|\\
 & = & -\left|\begin{array}{ccc}
-5 & -15 & -1\\
-5 & -8 & -1\\
-2 & -3 & 1
\end{array}\right|\\
 & = & \left|\begin{array}{ccc}
5 & 15 & 1\\
5 & 8 & 1\\
2 & 3 & -1
\end{array}\right|\\
 & = & 49
\end{eqnarray*}



\paragraph{2. (2)}

3回入れ替えが行われていることから

\[
\left|\begin{array}{cccc}
3 & 1 & 4 & 2\\
4 & 3 & 1 & 3\\
1 & 2 & 4 & 1\\
2 & 2 & 5 & 3
\end{array}\right|=\left|\begin{array}{cccc}
4 & 3 & 1 & 3\\
2 & 2 & 5 & 3\\
3 & 1 & 4 & 2\\
1 & 2 & 4 & 1
\end{array}\right|\times\left(-1\right)^{3}
\]


\[
x=-1
\]



\paragraph{3.}

\begin{eqnarray*}
\left|\begin{array}{ccccc}
1 & 1 & 2 & 3 & 2\\
2 & 3 & 6 & 7 & 5\\
2 & 3 & 8 & 8 & 6\\
1 & 3 & 7 & 7 & 5\\
2 & 3 & 7 & 8 & 7
\end{array}\right| & = & \left|\begin{array}{ccccc}
1 & 1 & 2 & 3 & 2\\
0 & 1 & 2 & 1 & 1\\
0 & 1 & 4 & 2 & 2\\
0 & 2 & 5 & 4 & 3\\
0 & 1 & 3 & 2 & 3
\end{array}\right|\\
 & = & \left|\begin{array}{cccc}
1 & 2 & 1 & 1\\
1 & 4 & 2 & 2\\
2 & 5 & 4 & 3\\
1 & 3 & 2 & 3
\end{array}\right|\\
 & = & \left|\begin{array}{cccc}
1 & 2 & 1 & 1\\
0 & 2 & 1 & 1\\
0 & 1 & 2 & 1\\
0 & 1 & 1 & 2
\end{array}\right|\\
 & = & \left|\begin{array}{ccc}
2 & 1 & 1\\
1 & 2 & 1\\
1 & 1 & 2
\end{array}\right|\\
 & = & 4
\end{eqnarray*}



\paragraph{4.}

\begin{eqnarray*}
S & = & \frac{1}{2}\left|\begin{array}{ccc}
1 & x_{1} & y_{1}\\
1 & x_{2} & y_{2}\\
1 & x_{3} & y_{3}
\end{array}\right|\\
 & = & \frac{1}{2}\left|\begin{array}{ccc}
1 & x_{1} & y_{1}\\
0 & x_{2}-x_{1} & y_{2}-y_{1}\\
0 & x_{3}-x_{1} & y_{3}-y_{1}
\end{array}\right|\\
 & = & \frac{1}{2}\left|\begin{array}{cc}
x_{2}-x_{1} & y_{2}-y_{1}\\
x_{3}-x_{1} & y_{3}-y_{1}
\end{array}\right|
\end{eqnarray*}


$v_{1}=\left(x_{1},y_{1}\right),v_{2}=\left(x_{2},y_{2}\right),v_{3}=\left(x_{3},y_{3}\right)$とおくと、

\[
S=\frac{1}{2}\times\left(\left(v_{2}-v_{1}\right),\left(v_{3}-v_{1}\right)\text{で作られる平行四辺形の面積}\right)
\]


となる。


\paragraph{5. (1)}

\[
f\left(x,y,z\right)=\left|\begin{array}{cccc}
1 & x_{1} & y_{1} & z_{1}\\
1 & x_{2} & y_{2} & z_{2}\\
1 & x_{3} & y_{3} & z_{3}\\
1 & x & y & z
\end{array}\right|
\]


とおく。

$f\left(x,y,z\right)$は、$x,y,z$の一次式になるので、$f\left(x,y,z\right)=0$は平面の方程式になる。

他方、
\begin{eqnarray*}
f\left(x_{1},y_{1},z_{1}\right) & = & \left|\begin{array}{cccc}
1 & x_{1} & y_{1} & z_{1}\\
1 & x_{2} & y_{2} & z_{2}\\
1 & x_{3} & y_{3} & z_{3}\\
1 & x_{1} & y_{1} & z_{1}
\end{array}\right|\\
 & = & 0\left(\because\text{交代性}\right)
\end{eqnarray*}


同様にして

\[
f\left(x_{2},y_{2},z_{2}\right)=0
\]


\[
f\left(x_{3},y_{3},z_{3}\right)=0
\]


よって、$f$は3点$P_{1},P_{2},P_{3}$を通る。

$P_{1},P_{2},P_{3}$は一直線上にないので、$f\left(x,y,z\right)=0$は3天を通る平面の方程式である。


\paragraph{5. (2)}

\begin{eqnarray*}
\left|\begin{array}{cccc}
1 & 1 & 1 & 1\\
1 & 1 & 2 & 3\\
1 & 2 & 3 & 6\\
1 & x & y & z
\end{array}\right| & = & \left|\begin{array}{cccc}
1 & 1 & 1 & 1\\
0 & 0 & 1 & 2\\
0 & 1 & 2 & 5\\
0 & x-1 & y-1 & z-1
\end{array}\right|\\
 & = & \left|\begin{array}{ccc}
0 & 1 & 2\\
1 & 2 & 5\\
x-1 & y-1 & z-1
\end{array}\right|\\
 & = & x+2y-z-2=0
\end{eqnarray*}


\rule[0.5ex]{1\columnwidth}{1pt}

\[
barAA=\left(\begin{array}{ccc}
5 & -3 & 1\\
-2 & 4 & -2\\
-1 & -1 & 1
\end{array}\right)\left(\begin{array}{ccc}
1 & 1 & 1\\
2 & 3 & 4\\
3 & 4 & 7
\end{array}\right)=\left(\begin{array}{ccc}
2 & 0 & 0\\
0 & 2 & 0\\
0 & 0 & 2
\end{array}\right)=dit\left(A\right)I_{3}
\]


\[
dit\left(A\right)=\left(\begin{array}{ccc}
1 & 1 & 1\\
2 & 3 & 4\\
3 & 4 & 7
\end{array}\right)=21+12+8-9-16-14=2
\]



\subsubsection{定理5.36}

$A$: $\left(n\times n\right)$行列

$\tilde{A}$を$A$の余因子行列とする。

\[
\tilde{A}A=\mathrm{det}\left(A\right)I_{n}=A\tilde{A}
\]



\paragraph{補題}

$A$: $\left(n\times n\right)$行列$=\left(\boldsymbol{a}_{1},\cdots,\boldsymbol{a}_{j},\cdots,\boldsymbol{a}_{n}\right)$

$\boldsymbol{e}_{i}=\left(\begin{array}{c}
0\\
\vdots\\
1\\
\vdots\\
0
\end{array}\right)$として、

\[
\mathrm{det}\left(\boldsymbol{a}_{1},\cdots,\boldsymbol{e}_{i},\cdots,\boldsymbol{a}_{n}\right)=\tilde{A}_{ij}
\]



\subsubsection{証明}

$\left(\boldsymbol{a}_{1},\cdots,\boldsymbol{e}_{i},\cdots,\boldsymbol{a}_{n}\right)$は、行について$\left(i-1\right)$回飛び越し、列について$\left(j-1\right)$回飛び越すので、

\begin{eqnarray*}
\mathrm{det}\left(\boldsymbol{a}_{1},\cdots,\boldsymbol{e}_{i},\cdots,\boldsymbol{a}_{n}\right) & = & \left(-1\right)^{i+j}\mathrm{det}\left(A_{ij}\right)\\
 & = & \tilde{A_{ij}}
\end{eqnarray*}


QED


\subsubsection{定理5.3.4 展開公式}


\paragraph{(1)}

任意の$j$について

\[
\sum_{i=1}^{n}a_{ij}\tilde{A_{ij}}=\mathrm{det}\left(A\right)
\]



\paragraph{(2)}

任意の$i$について
\[
\sum_{j=1}^{n}a_{ij}\tilde{A_{ij}}=\mathrm{det}\left(A\right)
\]



\subsubsection{証明}


\paragraph{(1)}

$A=\left(\boldsymbol{a}_{1},\boldsymbol{a}_{2},\cdots,\boldsymbol{a}_{j},\cdots,\boldsymbol{a}_{n}\right)$と書く。

\[
\boldsymbol{a}_{j}=\left(\begin{array}{c}
a_{1j}\\
a_{2j}\\
\vdots\\
a_{nj}
\end{array}\right)=a_{1j}\boldsymbol{e}_{1}+a_{2j}\boldsymbol{e}_{2}+\cdots+a_{nj}\boldsymbol{e}_{n}
\]


(ただし$\boldsymbol{e}_{i}$は$i$番目の値が1でその他の値が0の$\left(n\times1\right)$行列)

を(1)に代入して$j$列目に関する線形性

\begin{eqnarray*}
\mathrm{det}\left(A\right) & = & \mathrm{det}\left(\boldsymbol{a}_{1},\cdots,\sum_{i=1}^{n}a_{ij}\boldsymbol{e}_{i},\cdots,\boldsymbol{a}_{n}\right)\\
 & = & \sum_{i=1}^{n}a_{ij}\mathrm{det}\left(\boldsymbol{a}_{1},\cdots,\boldsymbol{e}_{i},\cdots\boldsymbol{a}_{n}\right)
\end{eqnarray*}


従って(1)が出る。

(2)は行と列の役割を変えて証明できる。


\subsubsection{系}


\paragraph{(1)}

$j\neq k$のとき

\[
\sum_{i=1}^{n}a_{ij}\tilde{A_{ik}}=0
\]



\paragraph{(2)}

$i\neq k$のとき

\[
\sum_{j=1}^{n}a_{ij}\tilde{A_{kj}}=0
\]



\subsubsection{証明}


\paragraph{(1)}

\[
\mathrm{det}\left(\boldsymbol{a}_{1},\cdots,\boldsymbol{a}_{j}\left(j\text{列目}\right),\cdots,\boldsymbol{a}_{j}\left(k\text{列目}\right),\cdots,\boldsymbol{a}_{n}\right)
\]


k列目に対して展開公式を用いて、

\[
=\sum_{i=1}^{n}a_{ij}\times\left(A'\text{の}\left(i,k\right)\text{余因子}\right)
\]


$A'$から$k$列目と$i$行目を取り去る$=A$の$\left(i,k\right)$余因子

\[
=\sum_{i=1}^{n}a_{ij}\tilde{A_{ik}}
\]


\[
\therefore\sum_{i=1}^{n}a_{ij}\tilde{A_{ik}}=0
\]


QED

(2)も同様。


\subsubsection{定理5.36}


\paragraph{(1)}

\[
\sum_{i=1}^{n}a_{ij}\tilde{A_{ik}}=\delta_{jk}\mathrm{det}\left(A\right)
\]



\paragraph{(2)}

\[
\sum_{j=1}^{n}a_{ij}\tilde{A_{kj}}=\delta_{ik}\mathrm{det}\left(A\right)
\]


特に


\paragraph{(1)}

\[
\tilde{A}A=\mathrm{det}\left(A\right)E_{n}
\]



\paragraph{(2)}

\[
A\tilde{A}=\mathrm{det}\left(A\right)E_{n}
\]



\subsubsection{系}

$\mathrm{det}\left(A\right)\neq0$のとき

\[
\left(\frac{1}{\mathrm{det}\left(A\right)}\tilde{A}\right)A=E_{n}=A\left(\frac{1}{\mathrm{det}\left(A\right)}\tilde{A}\right)
\]


なので、$A$は正則かつ

\[
A^{-1}=\frac{1}{\mathrm{det}\left(A\right)}\tilde{A}
\]


と逆行列が求められる。


\subsubsection{命題5.37}

$A$: $\left(n\times n\right)$行列とすると次は同値である。
\begin{enumerate}
\item $A$は正則行列
\item $\mathrm{det}\left(A\right)\neq0$
\item $AB=E_{n}$となる$B$がある。
\item $BA=E_{n}$となる$B$がある。
\end{enumerate}
(2)→(1)は先程の系である。

(1)→(3)は正則行列の条件の一部

(1)→(4)は正則行列の条件の一部

(3)→(2)、(4)→(2)は

$AB=E_{n}$の行列式をとる。

\[
\mathrm{det}\left(AB\right)=\mathrm{det}\left(E_{n}\right)=1
\]


他方
\[
\mathrm{det}\left(AB\right)=\mathrm{det}\left(A\right)\mathrm{det}\left(B\right)
\]


よって

\[
\mathrm{det}\left(A\right)\neq0
\]


によって証明される。


\subsubsection{命題5.38}

$A$: $n$次正方行列

→$A$は基本行列の積に書ける

(行基本変形は基本行列の左からの掛け算で表せる)

(注) これを用いて正則行列の逆行列は掃き出し法で必ず求まる。


\subsubsection{証明}

$A$: $\left(n\times n\right)$行列・正則とする。

行基本変形、列の入れ替えを用いて、$r+\left(n-r\right)$分割の$\left(\begin{array}{cc}
E_{r} & B\\
0 & 0
\end{array}\right)$の形になる

$A$に右、左から基本行列$P_{1},\cdots,P_{m}$$Q_{1},\cdots,Q_{l}$をかけて上の形になる、

\[
Q_{l}\cdots Q_{2}Q_{1}AP_{1}P_{2}\cdots P_{m}=\left(\begin{array}{cc}
E_{r} & B\\
0 & 0
\end{array}\right)
\]


基本行列は正則

\[
\mathrm{det\left(P_{i}\right)\neq0,}\mathrm{det}\left(Q_{j}\right)
\]


左辺の行列式は

\[
\mathrm{det}\left(Q_{l}\right)\mathrm{det}\left(Q_{l-1}\right)\cdots\mathrm{det}\left(Q_{1}\right)\mathrm{det}\left(A\right)\cdots\mathrm{det}\left(P_{m}\right)\neq0
\]


右辺の行列式は
\[
\begin{cases}
0 & \left(r<n\right)\\
1 & \left(r=n\right)
\end{cases}
\]


より
\[
r=n
\]


\[
\begin{array}{c}
Q_{l}\cdots Q_{1}AP_{1}\cdots P_{m}=E_{n}\\
Q_{l-1}\cdots Q_{1}AP_{1}\cdots P_{m}=Q_{l}^{-1}\\
\vdots
\end{array}
\]


と計算していくと
\[
A=Q_{1}^{-1}Q_{2}^{-1}\cdots Q_{l}^{-1}P_{m}^{-1}\cdots P_{1}^{-1}
\]


よって$A$は基本行列の積に書ける

QED


\subsection{5.8 ファンデルモンドの行列式}

$n\geqq2$自然数 $\lambda_{1},\cdots,\lambda_{n}\in\mathbb{R}$

$\left(n\times n\right)$行列$D\left(\lambda_{1},\lambda_{2},\cdots,\lambda_{n}\right)$を

\[
\left(\begin{array}{cccc}
1 & 1 & \cdots & 1\\
\lambda_{1} & \lambda_{2} & \cdots & \lambda_{n}\\
\lambda_{1}^{2} & \lambda_{2}^{2} & \cdots & \lambda_{n}^{2}\\
\vdots & \vdots &  & \vdots\\
\lambda_{1}^{n-1} & \lambda_{2}^{n-1} & \cdots & \lambda_{n}^{n-1}
\end{array}\right)
\]


と定める。


\subsubsection{定理5.39}

\[
\Delta\left(\lambda_{1},\cdots,\lambda_{n}\right)=\mathrm{det}\left(D\left(\lambda_{1},\cdots,\lambda_{n}\right)\right)
\]


よすると

\[
\Delta\left(\lambda_{1},\cdots,\lambda_{n}\right)=\left(\left(\lambda_{2}-\lambda_{1}\right)\left(\lambda_{3}-\lambda_{1}\right)\cdots\left(\lambda_{n}-\lambda_{1}\right)\right)\left(\left(\lambda_{3}-\lambda_{2}\right)\cdots\left(\lambda_{n}-\lambda_{2}\right)\right)\cdots\left(\lambda_{n}-\lambda_{n-1}\right)
\]


この右側の式のような値を差積といい、
\[
\prod_{1\leqq i<j\leqq n}\left(\lambda_{j}-\lambda_{i}\right)
\]


と書く。


\subsubsection{証明}

$n$に関する帰納法

$n=2$のとき

\[
\left|\begin{array}{cc}
1 & 1\\
\lambda_{1} & \lambda_{2}
\end{array}\right|=\lambda_{2}-\lambda_{1}
\]


$n-1$の場合に成立すると仮定して$n$のときに示す。

\[
D\left(\lambda_{1},\cdots,\lambda_{n}\right)
\]


今までの通り

\[
f\left(x\right)=\left(x-\lambda_{1}\right)\cdots\left(x-\lambda_{n-1}\right)
\]


\[
f\left(\lambda_{1}\right)=f\left(\lambda_{2}\right)=\cdots=f\left(\lambda_{n-1}\right)=0
\]


展開して

\[
f\left(x\right)=x^{n-1}+a_{1}x^{n-2}+\cdots+a_{n-2}x+a_{n-1}
\]


とする。

\begin{eqnarray*}
\left(\begin{array}{cccccc}
1 & 0 & \cdots & 0 & 0 & 0\\
0 & 1 & \cdots & 0 & 0 & 0\\
\vdots & \vdots &  & \vdots & \vdots & \vdots\\
0 & 0 & \cdots & 1 & 0 & 0\\
0 & 0 & \cdots & 0 & 1 & 0\\
a_{n-1} & a_{n-2} & \cdots & a_{2} & a_{1} & 1
\end{array}\right)\left(\begin{array}{cccc}
1 & 1 & \cdots & 1\\
\lambda_{1} & \lambda_{2} & \cdots & \lambda_{n}\\
\vdots & \vdots &  & \vdots\\
\lambda_{1}^{n-2} & \lambda_{2}^{n-2} & \cdots & \lambda_{n}^{n-2}\\
\lambda_{1}^{n-1} & \lambda_{2}^{n-1} & \cdots & \lambda_{n}^{n-1}
\end{array}\right) & = & \left(\begin{array}{ccccc}
 &  &  &  & \lambda_{1}\\
 &  & D\left(\lambda_{1},\cdots,\lambda_{n-1}\right) &  & \vdots\\
 &  &  &  & \lambda_{n}^{n-2}\\
\left(*1\right) & 0 & \cdots & 0 & \left(*2\right)
\end{array}\right)\\
 & = & f\left(\lambda_{1}\right)\\
 & = & 0
\end{eqnarray*}


ただし
\[
\left(*1\right)=a_{n-1}+\lambda_{1}a_{n-2}+\cdots+\lambda_{1}^{n-2}a_{1}+\lambda_{1}^{n-1}
\]


\begin{eqnarray*}
\left(*2\right) & = & a_{n-1}+\lambda_{n}a_{n-2}+\cdots+\lambda_{n}^{n-1}\\
 & = & f\left(\lambda_{n}\right)\\
 & = & \left(\lambda_{n}-\lambda_{1}\right)\left(\lambda_{n}-\lambda_{2}\right)\cdots\left(\lambda_{n}-\lambda_{n-1}\right)
\end{eqnarray*}


上野式より

\[
\mathrm{det}\left(M\right)\Delta\left(\lambda_{1},\cdots,\lambda_{n}\right)=\mathrm{det}\left(D\left(\lambda_{1},\cdots,\lambda_{n-1}\right)\right)\times\left(\lambda_{n}-\lambda_{1}\right)\left(\lambda_{n}-\lambda_{2}\right)\cdots\left(\lambda_{n}-\lambda_{n-1}\right)
\]


従って

\begin{eqnarray*}
\Delta\left(\lambda_{1},\cdots,\lambda_{n}\right) & = & \Delta\left(\lambda_{1},\cdots,\lambda_{n-1}\right)\left(\lambda_{n}-\lambda_{1}\right)\left(\lambda_{n}-\lambda_{n-1}\right)\\
 & = & \prod_{1\leqq i<j\leqq n-1}\left(\lambda_{j}-\lambda_{i}\right)\times\prod_{k=1}^{n-1}\left(\lambda_{n}-\lambda_{k}\right)\\
 & = & \prod_{1\leqq i<j\leqq n}\left(\lambda_{j}-\lambda_{i}\right)
\end{eqnarray*}


よって$n$の場合も成立。QED


\subsection{5.9 クラメルの公式}

$A=\left(a_{ij}\right)_{ij}$

$A$は正則とする。

$\left(\begin{array}{c}
b_{1}\\
\vdots\\
b_{n}
\end{array}\right)\in\mathbb{R}^{n}$とする

\[
\left(*\right)=\begin{cases}
a_{11}x_{1}+\cdots+a_{1n}x_{n}=b_{1}\\
a_{21}x_{1}+\cdots+a_{2n}x_{n}=b_{2}\\
\vdots\\
a_{n1}x_{1}+\cdots+a_{nn}x_{n}=b_{n}
\end{cases}
\]



\subsubsection{命題5.42}

$\left(*\right)$には解が存在してただ一つ


\subsubsection{証明}

$\left(*\right)$を行列を使って解く。$\boldsymbol{x}=\left(\begin{array}{c}
x_{1}\\
\vdots\\
x_{n}
\end{array}\right)$とおくと$A\boldsymbol{x}=\boldsymbol{b}$と書ける。

$A^{-1}$を左からかけて

\[
A^{-1}A\boldsymbol{x}=A^{-1}\boldsymbol{b}
\]


$\therefore\boldsymbol{x}=A^{-1}\boldsymbol{b}$となる。

逆に$A^{-1}\boldsymbol{b}=\boldsymbol{x}$とおけば$A\boldsymbol{x}=\boldsymbol{b}$を満たすので十分である。QED


\subsubsection{定理(クラメルの公式)}

今までの条件で$A=\left(\boldsymbol{a}_{1},\boldsymbol{a}_{2},\cdots,\boldsymbol{a}_{n}\right)$とおくと、

\[
\boldsymbol{x}_{i}=\frac{\mathrm{det}\left(\boldsymbol{a}_{1},\cdots,\boldsymbol{a}_{i-1},\boldsymbol{b},\boldsymbol{a}_{i+1},\cdots,\boldsymbol{a}_{2}\right)}{\mathrm{det}\left(A\right)}
\]



\subsubsection{例}

\[
\begin{cases}
2x+3y+7z=4\\
x-y+11z=13\\
3x+2y+z=8
\end{cases}
\]


\[
A=\left(\begin{array}{ccc}
2 & 3 & 7\\
1 & -1 & 11\\
3 & 2 & 1
\end{array}\right)
\]


ここで$z$だけ求めたいとき、

\[
z=\frac{\left|\begin{array}{ccc}
2 & 3 & 4\\
1 & -1 & 13\\
3 & 2 & 8
\end{array}\right|}{\left|\begin{array}{ccc}
2 & 3 & 7\\
1 & -1 & 11\\
3 & 2 & 1
\end{array}\right|}
\]


で求まる。


\subsubsection{証明}

$A=\left(\boldsymbol{a}_{1},\boldsymbol{a}_{2},\cdots,\boldsymbol{a}_{n}\right),x=\left(\begin{array}{c}
x_{1}\\
\vdots\\
x_{n}
\end{array}\right),A\boldsymbol{x}=\boldsymbol{b}$より、

\[
x_{1}\boldsymbol{a}_{1}+x_{2}\boldsymbol{a}_{2}+\cdots+x_{n}\boldsymbol{a}_{n}=\boldsymbol{b}
\]


ここで、

\begin{eqnarray*}
\mathrm{det}\left(\boldsymbol{a}_{1},\cdots,\boldsymbol{a}_{i-1},\boldsymbol{b},\boldsymbol{a}_{i+1},\cdots,\boldsymbol{a}_{n}\right) & = & \mathrm{det}\left(\boldsymbol{a}_{1},\cdots,\sum_{p=1}^{n}\boldsymbol{a}_{p}x_{p},\cdots,\boldsymbol{a}_{n}\right)\\
 & = & \sum_{p=1}^{n}x_{p}\mathrm{det}\left(\boldsymbol{a}_{1},\cdots,\boldsymbol{a}_{p}\left(i\text{番目}\right),\cdots,\boldsymbol{a}_{n}\right)\\
 & = & x_{i}\mathrm{det}\left(\boldsymbol{a}_{1},\cdots,\boldsymbol{a}_{i},\cdots,\boldsymbol{a}_{n}\right)
\end{eqnarray*}


($p+1$なる和は同じ行あるので寄与することはない。)
\[
\boldsymbol{x}_{i}=\frac{\mathrm{det}\left(\boldsymbol{a}_{1},\cdots,\boldsymbol{a}_{i-1},\boldsymbol{b},\boldsymbol{a}_{i+1},\cdots,\boldsymbol{a}_{2}\right)}{\mathrm{det}\left(A\right)}
\]


QED


\subsection{5.10 三次元行列式の幾何学的意味}

$n=2$→面積

$n=3$→体積(符号付き)

$v_{1}$と$v_{2}$がなす平行四辺形を$v_{3}$に沿って平行移動してできる立体

\[
v_{1},v_{2},v_{3}\text{のなす平行六面体の面積}\times\left(\pm1\right)=\mathrm{det}\left(v_{1},v_{2},v_{3}\right)
\]


符号は、$v_{1},v_{2},v_{3}$が右手の親指、人差し指、中指の関係(右手系)のとき$+1$

$v_{1},v_{2},v_{3}$が左手の親指、人差し指、中指の関係(左系)のとき$-1$


\subsubsection{外積}

$v_{1},v_{2}\in\mathbb{R}^{3}$空間ベクトル

\[
v_{1}\times v_{2}=\begin{cases}
\text{長さ}=v_{1}\text{と}v_{2}\text{のなす平行四辺形の面積}\\
\text{向き}=v_{1},v_{2}\text{に垂直で}v_{1},v_{2},v_{1}\times v_{2}\text{が右手系となるような向き}
\end{cases}
\]

\end{document}
