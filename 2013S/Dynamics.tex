%% LyX 2.0.6 created this file.  For more info, see http://www.lyx.org/.
%% Do not edit unless you really know what you are doing.
\documentclass[english]{article}
\usepackage[T1]{fontenc}
\usepackage[utf8]{inputenc}
\setlength{\parskip}{\smallskipamount}
\setlength{\parindent}{0pt}
\usepackage{textcomp}
\usepackage{amsmath}
\usepackage{amssymb}
\usepackage{esint}

\makeatletter

%%%%%%%%%%%%%%%%%%%%%%%%%%%%%% LyX specific LaTeX commands.
%% Because html converters don't know tabularnewline
\providecommand{\tabularnewline}{\\}

%%%%%%%%%%%%%%%%%%%%%%%%%%%%%% Textclass specific LaTeX commands.
\newenvironment{lyxlist}[1]
{\begin{list}{}
{\settowidth{\labelwidth}{#1}
 \setlength{\leftmargin}{\labelwidth}
 \addtolength{\leftmargin}{\labelsep}
 \renewcommand{\makelabel}[1]{##1\hfil}}}
{\end{list}}

\makeatother

\usepackage{babel}
\begin{document}
\[
m\frac{dv}{dt}=-mg+kv^{2}
\]


\[
m\frac{dv}{dt}=-mg-k\left|v\right|v
\]


下に動いているとすると

\[
m\frac{dv}{dt}=-mg+kv^{2}
\]


\[
v=-\alpha\frac{1-e^{-rt}}{1+e^{-rt}}
\]


ただし$\alpha=\sqrt{\frac{mg}{k}}$

\[
m\frac{dv}{dt}=-mg+kv^{2}
\]


\[
\frac{dv}{-mg+kv^{2}}=\frac{dt}{m}
\]


\[
\frac{dv}{-\alpha^{2}+v^{2}}=\frac{k}{m}dt
\]


\[
\frac{dv}{\left(v-\alpha\right)\left(v+\alpha\right)}=\frac{k}{m}dt
\]


\[
\int\left(\frac{1}{v-\alpha}-\frac{1}{v+\alpha}\right)dv=\int\frac{2\alpha k}{m}dt
\]


\[
\ln\left|\frac{v-\alpha}{v+\alpha}\right|=\frac{2\alpha k}{m}\cdot t+C
\]


\[
\left|\frac{v-\alpha}{v+\alpha}\right|=\frac{e^{\frac{2\alpha k}{m}\cdot t+C}}{1}
\]


\[
\frac{2v}{2\alpha}=\frac{1+e^{\frac{2\alpha k}{m}\cdot t+C}}{1-e^{\frac{2\alpha k}{m}\cdot t+C}}
\]


\[
v=\alpha\frac{1+K\exp\left(\frac{2\alpha k}{m}\cdot t+C\right)}{1-K\exp\left(\frac{2\alpha k}{m}\cdot t+C\right)}
\]



\section{エネルギー保存則}

U(ポテンシャルエナジー)+T(運動エネルギー)=E(一定)


\subsection{a. U: ポテンシャルエネルギー(位置)}

保存力$\mathbb{F}\left(x\right)\not\Leftrightarrow\text{例えば摩擦の}-kv$

\[
\mathbb{F}\left(x\right)=-mg
\]


\[
\mathbb{F}\left(x\right)=-kx
\]


\[
\mathbb{F}\left(x\right)=-k\left|x\right|^{n+1}x
\]


\[
\mathbb{F}\left(x\right)=-G\frac{mM}{x^{2}}
\]


\[
U=-\int F\left(x\right)dx
\]


\[
-\frac{dU}{dx}=F\left(x\right)
\]



\subsection{b. 運動エネルギー}

\[
T=\frac{1}{2}mv^{2}
\]



\section{2. エネルギー保存則⇔運動方程式}


\subsection{a. 運動方程式→エネルギー保存則}

\[
m\frac{d^{2}x}{dt^{2}}=F\left(x\right)
\]


\[
m\frac{dv}{dt}\frac{dx}{dt}dt=F\left(x\right)\mathrm{d}x
\]


\[
\int mv\mathrm{d}v=\int F\left(x\right)\mathrm{d}x
\]


\[
\frac{1}{2}mv^{2}=\int F\left(x\right)\mathrm{d}x=\mathrm{const}
\]



\subsection{b. エネルギー保存則→運動方程式}

\[
\frac{1}{2}mv^{2}+U\left(x\right)=const
\]


\[
mv\frac{dv}{dt}+\frac{dU}{dt}=0
\]


\[
mv\frac{dv}{dt}+\frac{dU}{dx}\frac{dx}{dt}=0
\]


\[
m\frac{dv}{dt}=-\frac{dU}{dx}
\]


単振動でエネルギー保存則

\[
m\frac{d^{2}x}{dt^{2}}=-kx
\]


\[
\rightarrow v=A\omega\cos\left(\omega t+\theta_{0}\right)
\]


\[
x=A\sin\left(\omega t+\theta_{0}\right)
\]


ただし$\omega=\sqrt{\frac{k}{m}}$

\[
U=\cdots
\]


\[
T=\cdots
\]


\[
U+T=\frac{1}{2}kA^{2}
\]



\section{3. エネルギー保存則で運動を理解する}


\subsection{a. 自由落下}

\[
U\left(x\right)=mgx
\]



\subsection{b. バネの単振動の場合}

\[
U\left(x\right)=kx^{2}
\]



\subsection{c. 一般化}

\[
U\left(x\right)=\int F\left(x\right)dx
\]


\[
E=U+T
\]


\[
T=E-U
\]



\subsection{d. 地球からの放出速度}

\[
F\left(x\right)=-G\frac{Mm}{x^{2}}
\]


\begin{eqnarray*}
U\left(x\right) & = & -\int^{x}\left(-G\frac{Mm}{x^{2}}\right)dx\\
 & = & -G\frac{Mm}{x}
\end{eqnarray*}


\[
E=U+T>0
\]


\[
\frac{1}{2}mv^{2}=-G\frac{Mm}{x}
\]


\[
v=\sqrt{2G\frac{M}{x}}
\]


$\frac{GM}{x^{2}}=g$とおくと

\begin{eqnarray*}
v & = & \sqrt{2gx}\\
 & = & \sqrt{2\times9.8\times6.4\times10^{6}}\\
 & = & 11\times10^{3}\\
 & = & 11\mathrm{km/s}
\end{eqnarray*}



\section{4. エネルギー保存則(の拡張)}

$U+T=E\left(t\right)$(一定でない)

ex. 摩擦を含む運動

$\frac{dE}{dt}$: 仕事率

$E\left(t_{2}\right)-E\left(t_{1}\right)$: 仕事


\subsection{エネルギー積分}

\[
E=\frac{1}{2}mv^{2}+U\left(x\right)=const
\]


\[
\sqrt{\frac{m}{2}}v=\sqrt{E-U}
\]


\[
v=\sqrt{\frac{2}{m}}\sqrt{\left(E-U\right)}
\]


\[
m\frac{d^{2}x}{dt^{2}}=-\frac{dt}{dx}+f
\]


\[
E=T+U
\]


\[
\frac{dE}{dt}\neq0
\]


\begin{eqnarray*}
\frac{dE}{dt} & = & \frac{d}{dt}\left(\frac{1}{2}mv^{2}+U\left(x\right)\right)\\
 & = & mv\frac{dv}{dt}+\frac{dU}{dt}\\
 & = & -mv\frac{dv}{dt}+\frac{dU}{dx}\cdot\frac{dx}{dt}\\
 & = & v\left(m\frac{dv}{dt}+\frac{dU}{dx}\right)\\
 & = & \frac{dE}{dt}\\
 & = & vf
\end{eqnarray*}


\[
E\left(t_{2}\right)-E\left(t_{1}\right)=\int_{t_{1}}^{t_{2}}vfdt
\]



\subsubsection{エネルギー積分}

\[
E\left(x\right)=\frac{1}{2}mv^{2}+U\left(x\right)\longrightarrow x=f\left(t\right)
\]


\[
\sqrt{\frac{1}{2}mv}=\sqrt{E\left(x\right)-U\left(x\right)}
\]


\[
\sqrt{\frac{1}{2}m}\cdot\frac{dx}{dt}=\sqrt{E\left(x\right)-U\left(x\right)}
\]


\[
\int_{x_{0}}^{x}\sqrt{\frac{m}{2}}\frac{1}{\sqrt{E\left(x\right)-U\left(x\right)}}dx=\int_{t_{0}}^{t}dt=t-t_{0}
\]


\[
t=\sqrt{\frac{m}{2}}\int_{x_{0}}^{x}\frac{1}{\sqrt{E\left(x\right)-U\left(x\right)}}dx+t_{0}
\]


\[
t=g\left(x\right)\Rightarrow x=g^{-1}\left(t\right)
\]


\[
x=\sqrt{\frac{2E}{k}}\sin\left(\sqrt{\frac{k}{m}}\left(t-U_{0}\right)\right)
\]



\subsection{数学の準備}


\subsubsection{偏微分~一つの変数方向の微分}

1変数 
\[
\frac{df\left(x\right)}{dx}=\lim_{\Delta\rightarrow0}\frac{\Delta f}{\Delta x}
\]


2変数 
\begin{eqnarray*}
\frac{\partial f\left(x,y\right)}{\partial x} & = & \lim_{\Delta\rightarrow0}\frac{\Delta f}{\Delta x}\\
 & = & \lim_{\Delta\rightarrow0}\frac{f\left(x+\Delta x,y\right)-f\left(x,y\right)}{\Delta x}
\end{eqnarray*}


\[
\frac{\partial f\left(x,y\right)}{\partial y}=\lim_{\Delta\rightarrow0}\frac{\Delta f}{\Delta y}
\]


ex)

\[
f\left(x,y\right)=x^{2}y+xy+4x+5y+6
\]


\[
\frac{\partial f}{\partial x}=2xy+y+4
\]


\[
\frac{\partial f}{\partial y}=x^{2}+x+5
\]


極値を求めたい

$\frac{\partial f}{\partial x}=\frac{\partial f}{\partial y}=0$ (極値であるための必要条件)


\subsubsection{全微分}

$x$方向に$\mathrm{d}x$,$y$方向に$\mathrm{d}y$だけ増加した時の$f$の増分。

\begin{eqnarray*}
d'f & = & \lim_{\Delta\rightarrow0}\left(f\left(x+\Delta x,y+\Delta y\right)-f\left(x,y\right)\right)\\
 & = & \frac{\partial f}{\partial x}dx+\frac{\partial f}{\partial y}dy\\
 & = & \lim_{\Delta\rightarrow0}\left(\frac{f\left(x+\Delta x,y+\Delta y\right)-f\left(x,y+\Delta y\right)}{\Delta x}\Delta x+\frac{f\left(x,y+\Delta y\right)-f\left(x,y\right)}{\Delta y}\Delta y\right)
\end{eqnarray*}



\subsubsection{合成関数の微分}

1変数 
\[
\frac{dU\left(x\right)}{dx}=\frac{dx}{dt}\cdot\frac{dU}{dx}
\]


2変数 
\begin{eqnarray*}
\frac{dU\left(x,y\right)}{dt} & = & \lim_{\Delta\rightarrow0}\frac{\Delta U}{\Delta t}\\
 & = & \lim_{\Delta\rightarrow0}\left(\frac{\Delta U_{x}}{\Delta t}+\frac{\Delta U_{y}}{\Delta t}\right)\\
 & = & \lim_{\Delta\rightarrow0}\left(\frac{\Delta U_{x}}{\Delta x}\cdot\frac{\Delta x}{\Delta t}+\frac{\Delta U_{y}}{\Delta y}\cdot\frac{\Delta y}{\Delta t}\right)\\
 & = & \frac{\partial U}{\partial x}\cdot\frac{dx}{dt}+\frac{\partial U}{\partial y}\cdot\frac{dy}{dt}
\end{eqnarray*}



\subsection{空間内の位置などベクトル表現}

位置 
\[
\left(x,y,z\right)=\mathbf{r}
\]


速度 
\[
\left(\frac{dx}{dt},\frac{dy}{dt},\frac{dz}{dt}\right)=\frac{\mathrm{d}\mathbf{r}}{\mathrm{d}t}
\]


加速度 
\[
\left(\frac{d^{2}x}{dt^{2}},\frac{d^{2}y}{dt^{2}},\frac{d^{2}z}{dt^{2}}\right)=\frac{d^{2}\mathbf{r}}{dt^{2}}
\]


力 
\[
\left(F_{x},F_{y},F_{z}\right)=\mathbf{F}
\]


運動方程式 
\[
F_{x}=m\frac{d^{2}x}{dt^{2}},F_{y}=m\frac{d^{2}y}{dt^{2}},F_{z}=m\frac{d^{2}z}{dt^{2}}\Rightarrow\mathbf{F}=m\frac{d^{2}\mathbf{r}}{dt^{2}}
\]


\[
m\left(\begin{array}{c}
\frac{dv_{x}}{dt}\\
\frac{dv_{y}}{dt}\\
\frac{dv_{z}}{dt}
\end{array}\right)=-mg\left(\begin{array}{c}
0\\
0\\
1
\end{array}\right)
\]


\[
\Rightarrow m\frac{d\mathbf{v}}{dt}=-mg\hat{\mathbf{z}}
\]


\[
\mathbf{v}\left(t\right)=-g\hat{\boldsymbol{z}}tL+\mathbf{v}_{0}
\]


\[
\mathbf{r}\left(t\right)=-\frac{1}{2}g\hat{\mathbf{z}}t^{2}+\mathbf{v}_{0}t+\mathbf{r}_{0}
\]


1次元 
\[
\mathbf{F}\left(x\right)=\frac{-dU}{dx}
\]


3次元 
\[
F_{x}\left(x,y,z\right)=-\frac{\partial U}{\partial x},F_{y}\left(x,y,z\right)=-\frac{\partial U}{\partial y},F_{z}\left(x,y,z\right)=-\frac{\partial U}{\partial z}
\]


ex. 万有引力

\[
\mathbf{r}=\left(\begin{array}{c}
x\\
y\\
z
\end{array}\right)
\]


\begin{eqnarray*}
\mathbf{F} & = & -G\frac{Mm}{r^{2}}\hat{\mathbf{r}}\\
 & = & -G\frac{Mm}{r^{2}}\cdot\frac{1}{r}\left(\begin{array}{c}
x\\
y\\
z
\end{array}\right)
\end{eqnarray*}
 (地球の中心を原点とした)

\begin{eqnarray*}
F_{x} & = & -G\frac{Mm}{r^{2}}\frac{x}{r}\\
 & = & -G\frac{Mm}{r^{2}}\frac{x}{r}
\end{eqnarray*}


\[
U\left(x,y,z\right)=-G\frac{Mm}{r}
\]


※$r=\sqrt{x^{2}+y^{2}+z^{2}}$

\[
F_{x}=-\frac{\partial U}{\partial x}=\ldots(\text{各自求めてみよ})
\]


\[
\mathrm{grd}U\triangleq\left(\frac{\partial U}{\partial x},\frac{\partial U}{\partial y},\frac{\partial U}{\partial z}\right)
\]


\[
\boldsymbol{F}=-\mathrm{grad}U
\]


\[
\nabla\triangleq\left(\frac{\partial}{\partial x},\frac{\partial}{\partial y},\frac{\partial}{\partial z}\right)
\]


\[
\nabla U\triangleq\left(\frac{\partial U}{\partial x},\frac{\partial U}{\partial y},\frac{\partial U}{\partial z}\right)
\]



\section{ラグランジュの方程式}


\subsection{最小作用の原理}


\subsubsection{作用}

\[
S\left(x\left(t\right)\right)=\int_{t_{1}}^{t_{2}}L\left(x\left(t\right),\dot{x}\left(t\right)\right)\mathrm{d}t
\]


※$\dot{x}\left(t\right)\triangleq\frac{\mathrm{d}x}{\mathrm{d}t}$


\subsubsection{ラグランジアン}

\begin{eqnarray*}
L & \triangleq & T-U\\
 & = & \frac{1}{2}m\dot{x}^{2}-U\left(\dot{x}\right)
\end{eqnarray*}



\subsubsection{汎関数}

汎関数とは、関数が変数となっている関数のことである。

⇔通常の関数$f\left(x\right)$(変数は数字)

ex)

\[
F\left(f\right)\triangleq\int_{0}^{1}\left(f\left(t\right)\right)^{2}\mathrm{d}t
\]


$f=t$とすると、
\[
F\left(f\right)=\int_{0}^{1}t^{2}\mathrm{d}t=\frac{1}{3}
\]


$f=t^{2}$とすると、
\[
F\left(f\right)=\int_{0}^{1}t^{4}\mathrm{d}t=\frac{1}{5}
\]


c.f.)

通常の関数
\[
f\left(t\right)\triangleq t^{2}
\]


において、

$t=1$のとき、

\[
f\left(t\right)=1
\]


$t=2$のとき、

\[
f\left(t\right)=4
\]


ex)

放物運動を考える。$t=0$のとき$x=0$、$t=T$のとき$x=0$とする。

$x\left(t\right)=At\left(t-T\right)$として始めてみよう。

\begin{eqnarray*}
S & = & \int_{0}^{T}L\left(x\left(t\right),x\left(t\right)\right)\mathrm{d}t\\
 & = & \int_{0}^{T}\left(\frac{1}{2}m\left(2At-AT\right)^{2}-mgAt\left(t-T\right)\right)\mathrm{d}t
\end{eqnarray*}


$S$を最小にするような$x\left(t\right)$を求める。(つまり、Aを求める)

\[
x\left(t\right)=-\frac{g}{2}t\left(t-T\right)
\]


\[
A=-\frac{g}{2}
\]


これは各自求めてみよ。


\subsubsection{変分}

$f\left(x\right)$を最小にする$x$を求める時、$\frac{\mathrm{d}f}{\mathrm{d}x}=0\left(\Leftrightarrow\lim_{x\rightarrow0}\Delta f=0\right)$は必要条件となる。

このとき$\Delta f$は$\Delta x$増加した時の$f$の増分となり、これを変分という。

\begin{eqnarray*}
\Delta S\left(x_{0}\right) & = & \int_{t_{1}}^{t_{2}}\left\{ L\left(x_{0}+\Delta x,\dot{x}+\Delta\dot{x}\right)-L\left(x_{0},x_{0}\right)\right\} \mathrm{d}t\\
 & = & \int_{t_{1}}^{t_{2}}\Delta L\left(x_{0},\dot{x}_{0}\right)\mathrm{d}t
\end{eqnarray*}


※$f\left(x,y\right)$の全微分とは、
\[
\mathrm{d}f=\frac{\partial f}{\partial x}\mathrm{d}x+\frac{\partial f}{\partial y}\mathrm{d}y
\]


である。

\[
\Delta f\left(x_{0},y_{0}\right)=\frac{\partial f}{\partial x}|_{x_{0}}\Delta x+\frac{\partial f}{\partial y}|_{y_{0}}\Delta y
\]


を用いて

\[
\Delta S\left(x_{0}\right)=\int_{t_{1}}^{t_{2}}\left\{ \frac{\partial L}{\partial x}|_{x_{0}}\Delta x+\frac{\partial L}{\partial\dot{x}}|_{\dot{x}_{0}}\Delta\dot{x}_{0}\right\} 
\]


$\Delta\dot{x}=\frac{\mathrm{d}}{\mathrm{d}t}\left(\Delta x\right)$より、第二項は

\[
\int_{t_{1}}^{t_{2}}\frac{\partial L}{\partial\dot{x}}|_{\dot{x}_{0}}\frac{\mathrm{d}}{\mathrm{d}t}\left(\Delta x\right)\mathrm{d}t=\left[\frac{\partial L}{\partial\dot{x}}|_{\dot{x}_{0}}\Delta x\right]_{t_{1}}^{t_{2}}-\int_{t_{1}}^{t_{2}}\frac{\mathrm{d}}{\mathrm{d}t}\left(\frac{\partial L}{\partial\dot{x}}\right)\Delta x\mathrm{d}t
\]


※$\int fg'\mathrm{d}t=fg-\int f'g\mathrm{d}t$

よって

\[
\Delta S\left(x_{0}\right)=\int_{t_{1}}^{t_{2}}\left(\frac{\partial L}{\partial x}|_{x_{0}}-\frac{\mathrm{d}}{\mathrm{d}t}\left(\frac{\partial L}{\partial\dot{x}}\right)\right)\Delta x\mathrm{d}t\left(=0\right)
\]


よって
\[
\frac{\mathrm{d}}{\mathrm{d}t}\left(\frac{\partial L}{\partial\dot{x}}\right)-\frac{\partial L}{\partial x}=0
\]


ラグランジュの方程式が導出された。


\subsection{ニュートンの運動方程式への変換}

直交座標系において

\begin{eqnarray*}
L\left(x,\dot{x}\right) & = & T-U\\
 & = & \frac{1}{2}mv^{2}-U\left(x\right)
\end{eqnarray*}


これと

\[
\frac{\mathrm{d}}{\mathrm{d}t}\left(\frac{\partial L}{\partial\dot{x}}\right)-\frac{\partial L}{\partial x}=0
\]


より

\[
\frac{\mathrm{d}}{\mathrm{d}t}\left(mv\right)+\frac{\partial U}{\partial x}=0
\]


\[
m\frac{\mathrm{d}v}{\mathrm{d}t}=-\frac{\partial U}{\partial x}
\]


ニュートンの運動方程式に変形された。

ラグランジュの運動方程式は座標系にかかわらず適用出来るため、ニュートンの運動方程式よりも拡張性が高い。


\subsection{極座標への適用}

極座標$\left(r,\theta\right)$

※$x=r\cos\theta,y=r\sin\theta$

位置$\left(r,\theta\right)$

速度$\left(v_{r},v_{\theta}\right)$

\[
v_{r}=\frac{\mathrm{d}r}{\mathrm{d}t}
\]
$v_{\theta}=\frac{\mathrm{d}\theta}{\mathrm{d}r}$ではない。

\begin{eqnarray*}
v_{\theta} & = & \lim_{\Delta t\rightarrow0}\frac{r\Delta\theta}{\Delta t}\\
 & = & r\frac{\mathrm{d}\theta}{\mathrm{d}t}
\end{eqnarray*}


保存力$\left(F_{r},F_{\theta}\right)$

\[
F_{r}=-\frac{\partial U}{\partial r}
\]


$F_{\theta}=-\frac{\partial U}{\partial\theta}$ではない。

\begin{eqnarray*}
F_{r} & = & -\frac{\partial U}{\partial\left(r\theta\right)}\\
 & = & -\frac{1}{r}\frac{\partial U}{\partial\theta}
\end{eqnarray*}



\subsection{極座標系の運動方程式}

\begin{eqnarray*}
T & = & \frac{1}{2}m\left(vr^{2}+v_{\theta}^{2}\right)\\
 & = & \frac{1}{2}m\left(\dot{r}^{2}+r^{2}\dot{\theta}^{2}\right)
\end{eqnarray*}


\[
L=\frac{1}{2}m\left(\dot{r}^{2}+r^{2}\dot{\theta}^{2}\right)-U\left(r,\theta\right)
\]


ラグランジュ方程式

\[
\frac{\mathrm{d}}{\mathrm{d}t}\left(\frac{\partial L}{\partial\dot{L}}\right)-\frac{\partial L}{\partial r}=0
\]
\[
\frac{\mathrm{d}}{\mathrm{d}t}\left(\frac{\partial L}{\partial\dot{\theta}}\right)-\frac{\partial L}{\partial\theta}=0
\]


より、

\[
\frac{\mathrm{d}}{\mathrm{d}t}\left(m\dot{r}\right)-mr\dot{\theta}^{2}+\frac{\partial U}{\partial r}=0
\]


\[
\frac{\mathrm{d}}{\mathrm{d}t}\left(mr^{2}\dot{\theta}^{2}\right)+\frac{\partial U}{\partial\theta}=0
\]


移項して、

\[
\frac{\mathrm{d}}{\mathrm{d}t}\left(m\dot{r}^{2}\right)=mr\dot{\theta}^{2}-\frac{\partial U}{\partial r}
\]
\[
\frac{\mathrm{d}}{\mathrm{d}t}\left(mr^{2}\dot{\theta}\right)=-\frac{\partial U}{\partial\theta}
\]


これが極座標系の運動方程式である。

$mr\dot{\theta}^{2}$はいわゆる見かけの力である。

$m\dot{r}$は運動量である。

$mr^{2}\dot{\theta}$は角運動量である。


\section{第十回 運動量・角運動量/剛体の運動}


\subsection{角運動量ベクトル}


\paragraph{復習}

\[
P_{w}\triangleq\frac{\partial L}{\partial w}
\]


\[
L=T-U=\frac{1}{2}m\left(\dot{r}^{2}+r^{2}\dot{\theta}^{2}\right)-U
\]


角運動量
\begin{eqnarray*}
l & = & mr^{2}\dot{\theta}\\
 & = & mr\dot{\theta}\cdot r\\
 & = & mv_{\theta}\cdot r\\
 & = & P_{\theta}\cdot r\\
 & = & P\cdot r\sin\phi
\end{eqnarray*}

\begin{lyxlist}{00.00.0000}
\item [{
\[
l=\left|\boldsymbol{r}\times\boldsymbol{P}\right|
\]
}]~
\item [{角運動量ベクトル$\boldsymbol{L}=\boldsymbol{r}\times\boldsymbol{P}$}]~
\item [{※$\left|\boldsymbol{a}\times\boldsymbol{b}\right|=\left|\boldsymbol{a}\right|\left|\boldsymbol{b}\right|\sin\phi$}]~
\end{lyxlist}

\subsection{力のモーメント}

運動方程式: $\frac{\mathrm{d}\left(\text{運動量}\right)}{\mathrm{d}t}=\text{なんとか}$

狭義の運動量: $\frac{\mathrm{d}\boldsymbol{p}}{\mathrm{d}t}=\boldsymbol{F}$

角運動量についても同様に

\begin{eqnarray*}
\frac{\mathrm{d}\boldsymbol{L}}{\mathrm{d}t} & = & \frac{\mathrm{d}}{\mathrm{d}t}\left(\boldsymbol{r}\times\boldsymbol{p}\right)\\
 & = & \frac{\mathrm{d}\boldsymbol{r}}{\mathrm{d}t}\times\boldsymbol{p}+\boldsymbol{r}\times\frac{\mathrm{d}\boldsymbol{p}}{\mathrm{d}t}\\
 & = & \boldsymbol{r}\times\boldsymbol{F}
\end{eqnarray*}


\[
\frac{\mathrm{d}\boldsymbol{L}}{\mathrm{d}t}=\boldsymbol{r}\times\boldsymbol{F}
\]


この式の右辺を力のモーメント$\boldsymbol{N}$という。小学校で習った(距離)\texttimes{}(力)のようなもの。

ここでラグランジュ方程式

\[
L=\frac{1}{2}m\left(\dot{r}^{2}+r^{2}\dot{\theta}^{2}\right)-U
\]


\[
\frac{\mathrm{d}}{\mathrm{d}t}\left(\frac{\partial L}{\partial\dot{\theta}}\right)-\frac{\partial L}{\partial\theta}=0
\]


\begin{eqnarray*}
\frac{\mathrm{d}}{\mathrm{d}t}\left(\frac{\partial L}{\partial\dot{\theta}}\right) & = & -\frac{\partial U}{\partial\theta}\\
 & = & -\frac{\partial x}{\partial\theta}\frac{\partial U}{\partial x}-\frac{\partial y}{\partial\theta}\frac{\partial U}{\partial y}\\
 & = & -r\sin\theta\cdot F_{x}+r\cos\theta\cdot F_{y}\\
 & = & -yF_{x}+xF_{y}\\
 & = & \left|\boldsymbol{r}\times\boldsymbol{F}\right|
\end{eqnarray*}



\subsection{複数の質点系の全運動量と全角運動量}

$N$: 質点の個数

$m_{i}$: 質点iの質量

$\boldsymbol{r}$: 質点iの位置ベクトル

$M\triangleq\sum_{i}m_{i}$

$\boldsymbol{R}$: 重心の位置ベクトル$\triangleq\frac{1}{M}\sum m_{i}\boldsymbol{r}_{i}$


\paragraph{全運動量}

\[
\sum_{i}m_{i}\frac{\mathrm{d}^{2}\boldsymbol{r}_{i}}{\mathrm{d}t^{2}}=\sum_{i}\left(\boldsymbol{F}_{i}+\sum_{j}\boldsymbol{F}_{ij}\right)
\]


すべての質点について足すということ。

\[
M\frac{\mathrm{d}^{2}\boldsymbol{R}}{\mathrm{d}t^{2}}=\sum_{i}\boldsymbol{F}_{i}
\]



\paragraph{全角運動量$L_{\mathrm{all}}$}

\[
\frac{\mathrm{d}\boldsymbol{L}_{i}}{\mathrm{d}t}=\boldsymbol{r}_{i}\times\boldsymbol{F}_{i}+\sum_{j}\left(\boldsymbol{r}_{i}\times\boldsymbol{F}_{ij}\right)
\]


\begin{eqnarray*}
\frac{\mathrm{d}\boldsymbol{L}_{\mathrm{all}}}{\mathrm{d}t} & = & \sum_{i}\left(\boldsymbol{r}_{i}\times\boldsymbol{F}_{i}\right)+\sum_{i}\sum_{j}\left(\boldsymbol{r}_{i}\times\boldsymbol{F}_{ij}\right)\\
 & = & \sum_{i}\left(\boldsymbol{r}_{i}\times\boldsymbol{F}_{i}\right)+\sum_{i>j}\left(\boldsymbol{r}_{i}-\boldsymbol{r}_{j}\right)\times\boldsymbol{F}_{ij}
\end{eqnarray*}


\[
\boldsymbol{F}_{ij}=-\boldsymbol{F}_{ji}
\]


\[
\boldsymbol{P}_{\mathrm{all}}=\sum_{i}m_{i}\boldsymbol{v}_{i}=M\boldsymbol{V}
\]


\[
\boldsymbol{L}_{\mathrm{all}}=\boldsymbol{R}\times M\boldsymbol{V}+\sum_{i}\left(\tilde{\boldsymbol{r}}_{i}\times m_{i}\tilde{\boldsymbol{v}}_{i}\right)
\]


\[
\boldsymbol{r}_{i}=\boldsymbol{R}+\tilde{\boldsymbol{r}}_{i}
\]


\[
\boldsymbol{v}_{i}=\boldsymbol{V}+\tilde{\boldsymbol{v}}_{i}
\]


\begin{eqnarray*}
\boldsymbol{L}_{\mathrm{all}} & = & \sum_{i}\left\{ \boldsymbol{r}_{i}\times m_{i}\boldsymbol{v}_{i}\right\} \\
 & = & \sum_{i}\left\{ \left(\boldsymbol{R}+\tilde{\boldsymbol{r}}_{i}\right)\times m_{i}\left(\boldsymbol{V}+\tilde{\boldsymbol{v}}_{i}\right)\right\} 
\end{eqnarray*}


式途中

\[
\sum m_{i}\tilde{\boldsymbol{r}}_{i}=0
\]


\[
\sum_{_{i}}m_{i}\frac{\mathrm{d}\tilde{\boldsymbol{r}}_{i}}{\mathrm{d}t}=\sum m_{i}\tilde{\boldsymbol{v}}_{i}=0
\]


に注意。


\subsection{単振り子と剛体振り子}

剛体: 大きさがある、形が変わらない。→回転を考える。位置だけではない。

長さ$l$の単振子において

\[
T=\frac{1}{2}mv^{2}=\frac{1}{2}ml^{2}\theta^{2}
\]


\[
U=mgl\left(1-\cos\theta\right)
\]


\[
\frac{\mathrm{d}}{\mathrm{d}t}\left(\frac{\partial L}{\partial\dot{\theta}}\right)-\frac{\partial L}{\partial\theta}=0
\]


\[
\frac{\mathrm{d}}{\mathrm{d}t}\left(ml^{2}\dot{\theta}\right)\sim+Mgl\sin\theta=0
\]


\[
\frac{\mathrm{d}}{\mathrm{d}t}\left(ml^{2}\dot{\theta}\right)=-mgl\sin\theta
\]



\paragraph{単振動}

\[
m\frac{\mathrm{d}^{2}x}{\mathrm{d}t^{2}}=-kx
\]


\[
m\frac{\mathrm{d}^{2}\theta}{\mathrm{d}t^{2}}=-k\theta
\]



\paragraph{単振子}

運動エネルギー
\begin{eqnarray*}
T & = & \frac{1}{2}mv^{2}\\
 & = & \frac{1}{2}ml^{2}\dot{\theta}^{2}
\end{eqnarray*}


ポテンシャル
\[
U=mgl\left(1-\cos\theta\right)
\]


ラグランジェ方程式

\[
\frac{\mathrm{d}}{\mathrm{d}t}\left(\frac{\partial L}{\partial\dot{\theta}}\right)-\frac{\partial L}{\partial\theta}=0
\]


\[
\frac{\mathrm{d}}{\mathrm{d}t}\left(ml^{2}\dot{\theta}\right)=-mgl\sin\theta
\]


\[
m\frac{\mathrm{d}}{\mathrm{d}t}\left(l^{2}\dot{\theta}\right)\doteqdot-mgl\theta
\]


\[
m\frac{\mathrm{d}\theta}{\mathrm{d}t^{2}}=-mgl\theta
\]



\paragraph{剛体振り子}

密度: $\rho\left(x,y,z\right)$

\begin{eqnarray*}
T_{\Delta} & = & \frac{1}{2}\rho\left(x,y.z\right)\Delta x\Delta y\Delta z\cdot v^{2}\\
 & = & \frac{1}{2}\rho\left(x.y.z\right)\Delta x\Delta y\Delta z\cdot r^{2}\dot{\theta}^{2}
\end{eqnarray*}


\begin{eqnarray*}
T & = & \frac{1}{2}\iiint\left(x^{2}+y^{2}\right)\rho\left(x,y,z\right)\mathrm{d}x\mathrm{d}y\mathrm{d}z\cdot\dot{\theta}^{2}\\
 & = & \frac{1}{2}I\dot{\theta}^{2}
\end{eqnarray*}


\[
U_{\Delta}=\rho\left(x,y,z\right)\Delta x\Delta y\Delta zgr\left(1-\cos\theta\right)
\]


\begin{eqnarray*}
U & = & \iiint\sqrt{x^{2}+y^{2}}\rho\left(x,y.z\right)\mathrm{d}x\mathrm{d}y\mathrm{d}z\cdot g\left(1-\cos\theta\right)\\
 & = & Mgl_{G}\left(1-\cos\theta\right)
\end{eqnarray*}


ただし

\[
\begin{cases}
l_{G}=\frac{1}{M}\iiint\sqrt{x^{2}+y^{2}}\rho\left(x,y.z\right)\mathrm{d}x\mathrm{d}y\mathrm{d}z\\
M=\iiint\rho\left(x,y.z\right)\mathrm{d}x\mathrm{d}y\mathrm{d}z
\end{cases}
\]


\begin{eqnarray*}
L & = & T-U\\
 & = & \frac{1}{2}I\dot{\theta}^{2}-Mgl_{G}\left(1-\cos\theta\right)
\end{eqnarray*}


\[
\frac{\mathrm{d}}{\mathrm{d}t}\left(\frac{\partial L}{\partial\dot{\theta}}\right)=\frac{\partial L}{\partial\theta}
\]


\[
\frac{\mathrm{d}}{\mathrm{d}t}\left(I\dot{\theta}\right)=-Mgl_{G}\sin\theta
\]


\begin{tabular}{|c|c|c|c|c|c|c|c|}
\hline 
 & 質量 & 重心の位置 & 慣性モーメント & 運動エネルギー & ポテンシャル & 角運動量 & 力のモーメント\tabularnewline
\hline 
\hline 
単振子 & $m$ & $l$ & $ml^{2}$ & $\frac{1}{2}ml^{2}\dot{\theta}^{2}$ & $mgl\left(1-\cos\theta\right)$ & $ml^{2}\dot{\theta}^{2}$ & $-mgl\sin\theta$\tabularnewline
\hline 
剛体振り子 & $M$ & $l_{G}$x & $I$ & $\frac{1}{2}I\dot{\theta}^{2}$ & $Mgl_{G}\left(1-\cos\theta\right)$ & $I\dot{\theta}$ & $-Mgl_{G}\sin\theta$\tabularnewline
\hline 
\end{tabular}

※慣性モーメントの定義$l=\iiint\left(x^{2}+y^{2}\right)\rho\left(x,y.z\right)\mathrm{d}x\mathrm{d}y\mathrm{d}z$

剛体振り子の角運動量は運動量$m\dot{x}$に対応していると考えることができる。


\subsection{慣性モーメント}

長さ$l$重さ$M$の棒を一端で支え振り子とする。

\[
\rho=\frac{M}{l}
\]


\begin{eqnarray*}
I & = & \int_{0}^{l}x^{2}\frac{M}{l}\mathrm{d}x\\
 & = & \frac{1}{3}Ml^{2}
\end{eqnarray*}


※$ml^{2}\rightarrow ml_{G}^{2}=\frac{1}{4}ml^{2}$としてはいけない

長さ$l$重さ$M$の棒を端点から$l_{1}$の点で支え振り子とする。

\begin{eqnarray*}
I & = & \int_{0}^{l_{1}}x^{2}\frac{M}{l}\mathrm{d}x+\int_{0}^{l-l_{1}}x^{2}\frac{M}{l}\mathrm{d}x\\
 & = & \frac{M}{3l}\left(l_{1}^{3}-\left(l-l_{1}\right)^{3}\right)\\
 & = & \frac{M}{3l}\left(l^{3}-3l^{2}l_{1}+3ll_{1}^{2}\right)\\
 & = & M\left(\frac{l^{2}}{3}-ll_{1}+l_{1}^{2}\right)
\end{eqnarray*}



\subsection{円柱の慣性モーメント}

極座標系で考える。円柱の半径を$a$として

\[
\rho=\frac{M}{\pi a^{2}}
\]


\begin{eqnarray*}
I & = & \iint r^{2}\cdot\rho\mathrm{d}r\cdot r\mathrm{d}\theta\\
 & = & \int2\pi r^{3}\rho\mathrm{d}r\\
 & = & \int_{0}^{a}2\pi r^{2}\frac{M}{\pi a^{2}}\mathrm{d}r\\
 & = & \frac{Ma^{2}}{2}
\end{eqnarray*}



\subsection{円筒の慣性モーメント}

円筒の半径を$a$として、

\begin{eqnarray*}
I & = & \iint r^{2}\cdot\rho\mathrm{d}r\cdot r\mathrm{d}\theta\\
 & = & Ma^{2}
\end{eqnarray*}



\subsection{宿題}
\begin{enumerate}
\item $a\times b$の長方形の形の板の慣性モーメントを求めよ
\item H型の板の慣性モーメントを求めよ
\item 円板を直径を軸として回転させる時の慣性モーメントを求めよ。
\end{enumerate}

\subsection{すべらずに坂を転がる剛体}

傾度$\alpha$の斜面で半径の円柱を転がす。

斜面にそって$x$軸を、斜面に鉛直に$y$軸を取り、同時に極座標系$\left(r,\varphi\right)$をとる。また円柱の回転角度を$\theta$とする。

微小部分の$x$方向の速度は
\[
r\dot{\theta}\sin\varphi+\dot{x}
\]


$\dot{x}$は円柱の移動する速さ$\left(=a\dot{\theta}\right)$

微小部分の$y$方向の速度は
\[
r\dot{\theta}\cos\varphi
\]


\[
T_{\Delta}=\frac{1}{2}\rho\left\{ \left(r\dot{\theta}\sin\varphi+\dot{x}\right)^{2}+\left(r\dot{\theta}\cos\varphi\right)^{2}\right\} r\Delta\varphi\cdot\Delta r\cdot\Delta z
\]


\begin{eqnarray*}
T & = & \frac{1}{2}\iiint\rho\left(r^{2}\dot{\theta}^{2}+2\dot{x}r\dot{\theta}\sin\varphi+\dot{x}^{2}\right)r\mathrm{d}\varphi\mathrm{d}r\mathrm{d}z\\
 & = & \frac{1}{2}\int\rho r^{2}\dot{\theta}^{2}\mathrm{d}t\mathrm{d}\varphi\mathrm{d}z+\frac{1}{2}\int\dot{x}^{2}\rho r\mathrm{d}r\mathrm{d}\phi\\
 & = & \frac{1}{2}I\dot{\theta}^{2}+\frac{1}{2}M\dot{x}^{2}\text{(置換)}\\
 & = & \frac{1}{2}I\frac{\dot{x}^{2}}{a^{2}}+\frac{1}{2}M\dot{x}^{2}\\
 & = & \frac{1}{2}\left(M+\frac{I}{a^{2}}\right)\dot{x}^{2}
\end{eqnarray*}


\[
U=-Mgx\sin\alpha
\]


\[
\left(M+\frac{I}{a^{2}}\right)\dot{x}=Mg\sin\alpha
\]


各自求めてみよ。


\subsection{まとめ}


\paragraph{円筒}

\[
I=Ma^{2}
\]


\[
2M\ddot{x}=Mg\sin\alpha
\]



\paragraph{円柱}

\[
I=\frac{Ma^{2}}{2}
\]


\[
\frac{3}{2}M\ddot{x}=Mg\sin\alpha
\]



\subsection{おまけ}

重心から$d$離れた点$O'$を回転軸にした時の慣性モーメントは

\[
I=I_{G}+Md^{2}
\]


となる。ただし、$I_{G}$は重心周りの慣性モーメント


\subsubsection{証明}

重心$G$を原点にして座標系を考える。

$O'\left(X,Y\right)$として、

\[
I=\int\left\{ \left(x-X\right)^{2}+\left(y-Y\right)^{2}\right\} \rho\mathrm{d}x\mathrm{d}y\mathrm{d}z
\]


続きを宿題とする。各自求めてみよ。


\subsection{滑りながら転がる剛体←先週のものを一般化}

(すべらない場合$a\dot{\theta}=\dot{x}$)

斜面を転がる円柱、半径$a$、摩擦力$F$、傾斜角$\alpha$、斜面に沿った座標を$x$とする。

$a\dot{\theta}=\dot{x}$のとき$\left(M+\frac{I}{a^{2}}\right)\ddot{x}=Mg\sin\alpha$

\[
M\frac{\mathrm{d}^{2}x}{\mathrm{d}t^{2}}=Mg\sin\alpha-F(\text{並進})
\]


\[
\frac{\mathrm{d}}{\mathrm{d}t}\left(\text{角運動量}\right)=\text{力のモーメント}
\]


\[
\Rightarrow\frac{\mathrm{d}}{\mathrm{d}t}\left(I\dot{\theta}\right)=Fa(\text{回転})
\]


\[
M\ddot{x}=Mg\sin\alpha-\mu'Mg\cos\alpha
\]


\[
\therefore\ddot{x}=g\left(\sin\alpha-\mu'g\cos\alpha\right)
\]


\[
I\ddot{\theta}=\mu'Mg\cos\alpha\cdot a
\]


\[
a\ddot{\theta}=\mu'Mg\frac{a^{2}}{I}\cos\alpha
\]


すべらないとき

\[
a\dot{\theta}=\dot{x}
\]


\[
g\left(\sin\alpha-\mu'g\cos\alpha\right)=\mu'Mg\cos\alpha\frac{a^{2}}{I}
\]


\[
g\cos\left(\tan-\mu'\frac{Ma^{2}+I}{I}\right)=0
\]


すべるとき

\[
a\dot{\theta}<\dot{x}
\]


\[
g\cos\left(\tan\alpha-\mu'\frac{Ma^{2}+I}{I}\right)>0
\]


速度

$a\dot{\theta}=\dot{x}$になるまでの間

\[
\begin{cases}
M\ddot{x}=-F & \left(\text{並進}\right)\rightarrow\text{減速}\\
I\ddot{\theta}=Fa & \left(\text{回転}\right)\rightarrow\text{加速}
\end{cases}
\]


$a\dot{\theta}=\dot{x}$になったとき

\[
\begin{cases}
M\ddot{x}=0\\
I\ddot{\theta}=0
\end{cases}\rightarrow\text{等速直線運動}
\]



\subsection{剛体の静力学}

\[
M\frac{\mathrm{d}^{2}x}{\mathrm{d}t^{2}}=\boldsymbol{F}\:(1)
\]


\[
\frac{\mathrm{d}\boldsymbol{L}}{\mathrm{d}t}=\boldsymbol{N}\left(=\boldsymbol{r}\times\boldsymbol{F}\right)\:(2)
\]


$\boldsymbol{L}$: 角運動量ベクトル

$\boldsymbol{N}$: 力のモーメント

壁に立てかけた棒、傾斜を$\theta$、床との摩擦力を$f$、壁からの垂直抗力を$N_{1}$、床からの垂直抗力を$N_{2}$とする。

(1)より、($x$方向、$y$方向の釣り合い式)

\[
\begin{cases}
Mg-N_{2}=0 & y\text{方向}\\
N_{1}-f=0 & x\text{方向}
\end{cases}
\]


(2)より、力のモーメント=0

\[
\frac{1}{2}lMg\sin\theta-N_{1}\cos\theta=0
\]


\[
f=\frac{Mg}{2}\tan\theta
\]



\subsection{補足}

前回空洞の円柱と中身が詰まった円柱が斜面を転がるときの違いを説明したが、空洞の円柱の中に液体が詰まっている場合を考えてみよう。

「ジャックと豆の木は存在するか」。豆の木が存在したとしてそれはどこまで伸びていくか。宇宙まで伸ばせるような構造物は力学的に存在しうるか。c.f.
宇宙エレベーター


\subsection{試験について}

教科書は何でもいいが、書いてあることを自分のなかで一回は咀嚼して考えること。
\end{document}
